\documentclass[12pt,a4paper]{article}

% ----------- Packages -----------
\usepackage{amsmath, amssymb, amsthm} % Math symbols & theorems
\usepackage{enumitem} % Better lists
\usepackage{geometry} % Page layout
\usepackage{fancyhdr} % Header/footer
\usepackage{tikz}     % Diagrams
\usepackage{hyperref} % Clickable references

% ----------- Page Setup -----------
\geometry{margin=1in}
\setlength{\parskip}{0.5em}
\setlength{\parindent}{0pt}
\pagestyle{fancy}
\fancyhf{}
% ----------- Header/Footer -----------
\lhead{MAT121 -- Analysis I}
\chead{Exercise sheet 6}
\rhead{Stefan du Toit}
\rfoot{\thepage}

% ----------- Theorem Environments -----------
\newtheorem{theorem}{Theorem}[section]
\newtheorem{lemma}[theorem]{Lemma}
\newtheorem{proposition}[theorem]{Proposition}
\newtheorem{corollary}[theorem]{Corollary}

\theoremstyle{definition}
\newtheorem{definition}[theorem]{Definition}
\newtheorem{example}[theorem]{Example}
\newtheorem{exercise}{Exercise}[section]
\newcommand{\C}{\mathbb{C}}

\theoremstyle{remark}
\newtheorem*{remark}{Remark}

\newenvironment{solution}{\begin{proof}[Solution]}{\end{proof}}



% ==========================================
\begin{document}

\section*{1}

\subsection*{(a)}

\textbf{Series:} $\sum_{k=0}^{\infty} \frac{(-1)^k}{2^k}$

\begin{solution}
We apply the Leibniz criterion (alternating series test) to show convergence.

The series has the form $\sum_{k=0}^{\infty} (-1)^k a_k$ where $a_k = \frac{1}{2^k}$.

We need to verify three conditions:

\textbf{1. Alternating signs:} The factor $(-1)^k$ ensures alternating signs.  

\textbf{2. Terms decreasing in absolute value:} We have $|a_k| = \frac{1}{2^k}$.
For $k \geq 0$: $|a_{k+1}| = \frac{1}{2^{k+1}} = \frac{1}{2} \cdot \frac{1}{2^k} = \frac{1}{2} |a_k| < |a_k|$

Therefore $\{|a_k|\}$ is monotone decreasing.  

\textbf{3. Terms approach zero:} $\lim_{k \to \infty} |a_k| = \lim_{k \to \infty} \frac{1}{2^k} = 0$  

Since all three conditions of the Leibniz criterion are satisfied, the series $\sum_{k=0}^{\infty} \frac{(-1)^k}{2^k}$ converges.

\end{solution}

\subsection*{(b)}

\textbf{Series:} $\sum_{k=1}^{\infty} \frac{1}{4k^2 - 1}$

\begin{solution}
We use the Direct Comparison Test to show convergence.

\textbf{Step 1: Choose comparison series}
We know that $\sum_{k=1}^{\infty} \frac{1}{k^2}$ converges (p-series with $p = 2 > 1$).

We will compare our series with $\sum_{k=1}^{\infty} \frac{1}{3k^2}$, which also converges since it equals $\frac{1}{3}\sum_{k=1}^{\infty} \frac{1}{k^2}$.

\textbf{Step 2: Establish the inequality}
For $k \geq 1$, we need to show that $\frac{1}{4k^2-1} \leq \frac{1}{3k^2}$.

This is equivalent to showing $3k^2 \leq 4k^2 - 1$, or $1 \leq k^2$.

Since $k \geq 1$, we have $k^2 \geq 1$, so:
$$4k^2 - 1 \geq 4k^2 - k^2 = 3k^2$$

Taking reciprocals (and flipping the inequality):
$$\frac{1}{4k^2-1} \leq \frac{1}{3k^2}$$

\textbf{Step 3: Apply the Direct Comparison Test}
Since:
\begin{itemize}
\item $\sum_{k=1}^{\infty} \frac{1}{3k^2}$ converges, and
\item $0 < \frac{1}{4k^2-1} \leq \frac{1}{3k^2}$ for all $k \geq 1$
\end{itemize}

By the Direct Comparison Test (Theorem 3.5.4), the series $\sum_{k=1}^{\infty} \frac{1}{4k^2-1}$ converges.
\end{solution}

\subsection*{(c)}

\textbf{Series:} $\sum_{k=1}^{\infty} \frac{k^4}{3^k}$

\begin{solution}
We use the Ratio Test to show convergence. This series has the form of a polynomial in the numerator and an exponential in the denominator, which suggests the Ratio Test.

\textbf{Step 1: Apply the Ratio Test}
Let $a_k = \frac{k^4}{3^k}$. We compute:
$$\left|\frac{a_{k+1}}{a_k}\right| = \left|\frac{\frac{(k+1)^4}{3^{k+1}}}{\frac{k^4}{3^k}}\right| = \frac{(k+1)^4}{3^{k+1}} \cdot \frac{3^k}{k^4} = \frac{(k+1)^4}{k^4} \cdot \frac{3^k}{3^{k+1}} = \frac{(k+1)^4}{k^4} \cdot \frac{1}{3}$$

\textbf{Step 2: Simplify the ratio}
$$\left|\frac{a_{k+1}}{a_k}\right| = \frac{1}{3} \cdot \left(\frac{k+1}{k}\right)^4 = \frac{1}{3} \cdot \left(1 + \frac{1}{k}\right)^4$$

\textbf{Step 3: Take the limit}
$$\lim_{k \to \infty} \left|\frac{a_{k+1}}{a_k}\right| = \lim_{k \to \infty} \frac{1}{3} \cdot \left(1 + \frac{1}{k}\right)^4 = \frac{1}{3} \cdot (1 + 0)^4 = \frac{1}{3}$$

\textbf{Step 4: Apply the Ratio Test criterion}
Since $\lim_{k \to \infty} \left|\frac{a_{k+1}}{a_k}\right| = \frac{1}{3} < 1$, the series $\sum_{k=1}^{\infty} \frac{k^4}{3^k}$ converges.

\end{solution}

\subsection*{(d)}

\textbf{Series:} $\sum_{k=1}^{\infty} \frac{k!}{k^k}$

\begin{solution}
We use the Ratio Test to show convergence. This series involves factorials, which makes the Ratio Test a natural choice.

\textbf{Step 1: Set up the ratio}
Let $a_k = \frac{k!}{k^k}$. We compute:
$$\left|\frac{a_{k+1}}{a_k}\right| = \frac{\frac{(k+1)!}{(k+1)^{k+1}}}{\frac{k!}{k^k}} = \frac{(k+1)! \cdot k^k}{k! \cdot (k+1)^{k+1}}$$

\textbf{Step 2: Simplify using factorial properties}
Since $(k+1)! = (k+1) \cdot k!$:
$$\frac{(k+1)! \cdot k^k}{k! \cdot (k+1)^{k+1}} = \frac{(k+1) \cdot k! \cdot k^k}{k! \cdot (k+1)^{k+1}} = \frac{(k+1) \cdot k^k}{(k+1)^{k+1}}$$

\textbf{Step 3: Simplify the powers}
Since $(k+1)^{k+1} = (k+1) \cdot (k+1)^k$:
$$\frac{(k+1) \cdot k^k}{(k+1)^{k+1}} = \frac{(k+1) \cdot k^k}{(k+1) \cdot (k+1)^k} = \frac{k^k}{(k+1)^k} = \left(\frac{k}{k+1}\right)^k$$

\textbf{Step 4: Rewrite for limit calculation}
$$\left(\frac{k}{k+1}\right)^k = \left(\frac{1}{1 + \frac{1}{k}}\right)^k = \frac{1}{\left(1 + \frac{1}{k}\right)^k}$$

\textbf{Step 5: Take the limit}
Using the fundamental limit $\lim_{k \to \infty} \left(1 + \frac{1}{k}\right)^k = e$:
$$\lim_{k \to \infty} \left|\frac{a_{k+1}}{a_k}\right| = \lim_{k \to \infty} \frac{1}{\left(1 + \frac{1}{k}\right)^k} = \frac{1}{e}$$

\textbf{Step 6: Apply the Ratio Test criterion}
Since $\lim_{k \to \infty} \left|\frac{a_{k+1}}{a_k}\right| = \frac{1}{e} \approx 0.368 < 1$, the series $\sum_{k=1}^{\infty} \frac{k!}{k^k}$ converges.

\end{solution}

\section*{2}

\subsection*{(a)}

\textbf{Problem:} Let $(a_n)_{n \in \mathbb{N}}$ be a sequence and $(a_{n_k})_{k \in \mathbb{N}}$ be a subsequence. Show by counterexample: If $\sum_{n \in \mathbb{N}} a_n$ converges, it does not generally follow that $\sum_{k \in \mathbb{N}} a_{n_k}$ converges.

\begin{solution}
We construct a counterexample using the alternating harmonic series.

\textbf{Step 1: Define the sequence}
Consider the alternating harmonic series:
$$\sum_{n=1}^{\infty} a_n = \sum_{n=1}^{\infty} \frac{(-1)^{n+1}}{n} = 1 - \frac{1}{2} + \frac{1}{3} - \frac{1}{4} + \frac{1}{5} - \frac{1}{6} + \cdots$$

This series converges by the Leibniz criterion (alternating series test):
\begin{itemize}
\item Terms alternate in sign: $(-1)^{n+1}$
\item $\frac{1}{n} \to 0$ as $n \to \infty$
\item $\frac{1}{n+1} < \frac{1}{n}$ (decreasing)
\end{itemize}

\textbf{Step 2: Choose an appropriate subsequence}
Define the subsequence by taking only the positive terms:
$$n_k = 2k-1 \text{ for } k = 1, 2, 3, \ldots$$

This gives us $n_1 = 1, n_2 = 3, n_3 = 5, n_4 = 7, \ldots$ (all odd indices).

\textbf{Step 3: Examine the subsequence series}
The corresponding subsequence series is:
$$\sum_{k=1}^{\infty} a_{n_k} = \sum_{k=1}^{\infty} a_{2k-1} = \sum_{k=1}^{\infty} \frac{(-1)^{2k}}{2k-1} = \sum_{k=1}^{\infty} \frac{1}{2k-1}$$

This becomes:
$$\frac{1}{1} + \frac{1}{3} + \frac{1}{5} + \frac{1}{7} + \frac{1}{9} + \cdots$$

\textbf{Step 4: Show the subsequence series diverges}
This is the series of reciprocals of odd numbers. We can show it diverges by comparison with the harmonic series:

Since $\frac{1}{2k-1} > \frac{1}{2k}$ for all $k \geq 1$, we have:
$$\sum_{k=1}^{\infty} \frac{1}{2k-1} > \sum_{k=1}^{\infty} \frac{1}{2k} = \frac{1}{2} \sum_{k=1}^{\infty} \frac{1}{k}$$

Since $\sum_{k=1}^{\infty} \frac{1}{k}$ diverges (harmonic series), the series $\sum_{k=1}^{\infty} \frac{1}{2k-1}$ also diverges.

\textbf{Conclusion:}
We have shown that:
\begin{itemize}
\item $\sum_{n=1}^{\infty} a_n = \sum_{n=1}^{\infty} \frac{(-1)^{n+1}}{n}$ converges
\item $\sum_{k=1}^{\infty} a_{n_k} = \sum_{k=1}^{\infty} \frac{1}{2k-1}$ diverges
\end{itemize}

Therefore, convergence of the original series does not imply convergence of every subsequence series.
\end{solution}

\subsection*{(b)}

\textbf{Problem:} Prove: If $\sum_{n \in \mathbb{N}} a_n$ converges absolutely, then $\sum_{k \in \mathbb{N}} a_{n_k}$ also converges absolutely.

\begin{solution}
We prove this using the Monotone Bounded Theorem (Satz 3.2.2).

\textbf{Given:} $\sum_{n=1}^{\infty} a_n$ converges absolutely, so $\sum_{n=1}^{\infty} |a_n|$ converges.

\textbf{To prove:} $\sum_{k=1}^{\infty} a_{n_k}$ converges absolutely, i.e., $\sum_{k=1}^{\infty} |a_{n_k}|$ converges.

\textbf{Step 1: Define partial sums}
Let $S_N = \sum_{n=1}^{N} |a_n|$ and $T_K = \sum_{k=1}^{K} |a_{n_k}|$

Since $\sum_{n=1}^{\infty} |a_n|$ converges, we have $\lim_{N \to \infty} S_N = S < \infty$.

\textbf{Step 2: Show $(T_K)$ is bounded above}
Since $(n_k)$ is a strictly increasing subsequence with $n_1 < n_2 < n_3 < \cdots$, each term $|a_{n_k}|$ appears in the partial sum $S_{n_K}$ of the absolutely convergent series.

Therefore:
$$T_K = |a_{n_1}| + |a_{n_2}| + \cdots + |a_{n_K}| \leq S_{n_K} \leq S < \infty$$

So the sequence $(T_K)$ is bounded above by $S$.

\textbf{Step 3: Show $(T_K)$ is monotone increasing}
Since all terms $|a_{n_k}| \geq 0$:
$$T_{K+1} = T_K + |a_{n_{K+1}}| \geq T_K$$

Therefore $(T_K)$ is monotone increasing.

\textbf{Step 4: Apply the Monotone Bounded Theorem}
Since $(T_K)$ is monotone increasing and bounded above, it converges:
$$\lim_{K \to \infty} T_K = \lim_{K \to \infty} \sum_{k=1}^{K} |a_{n_k}| \text{ exists and is finite}$$

This means $\sum_{k=1}^{\infty} |a_{n_k}|$ converges.

\textbf{Step 5: Conclude absolute convergence}
Since $\sum_{k=1}^{\infty} |a_{n_k}|$ converges, the series $\sum_{k=1}^{\infty} a_{n_k}$ converges absolutely.

\end{solution}

\subsection*{(c)}

\textbf{Task:} Using part (b), compute the sum of the series in Problem 1(a):
$$\sum_{k=0}^{\infty} \frac{(-1)^k}{2^k}.$$

\begin{solution}
By part (b), if a series is absolutely convergent, then any subsequence series is also absolutely convergent and we may split the series into disjoint subsequences and sum them. Since
$$\sum_{k=0}^{\infty} \left|\frac{(-1)^k}{2^k}\right| = \sum_{k=0}^{\infty} \frac{1}{2^k} $$
converges, the original series is absolutely convergent. Split it into the even and the odd subsequences:
\[
\sum_{k=0}^{\infty} \frac{(-1)^k}{2^k}
= \sum_{m=0}^{\infty} \frac{1}{2^{2m}} \; + \; \sum_{m=0}^{\infty} \frac{(-1)}{2^{2m+1}}.
\]
Both sums are geometric:
\[
\sum_{m=0}^{\infty} \frac{1}{2^{2m}} = \sum_{m=0}^{\infty} \left(\frac{1}{4}\right)^m = \frac{1}{1-\frac{1}{4}} = \frac{4}{3},\qquad
\sum_{m=0}^{\infty} \frac{(-1)}{2^{2m+1}} = -\frac{1}{2} \sum_{m=0}^{\infty} \left(\frac{1}{4}\right)^m = -\frac{1}{2}\cdot \frac{1}{1-\frac{1}{4}} = -\frac{2}{3}.
\]
Adding the two convergent subsequence sums yields
\[
\sum_{k=0}^{\infty} \frac{(-1)^k}{2^k} = \frac{4}{3} - \frac{2}{3} = \frac{2}{3}.
\]
\end{solution}

\section*{3}

\textbf{Problem:} Let $(x_n)_{n\in\mathbb{N}}$ be a real sequence with $L := \limsup_{n\to\infty} x_n > -\infty$. Show that for every $\varepsilon>0$ there are at most finitely many indices $k$ such that
$$x_k \ge L + \varepsilon.$$

\begin{solution}

\textbf{Solution via tail suprema.}
Recall that $\displaystyle L = \lim_{n\to\infty} s_n$ where $s_n := \sup_{k\ge n} x_k$ and that $(s_n)$ is decreasing. Fix $\varepsilon>0$ and set $\delta := \varepsilon/2$. Since $s_n\to L$, there exists $N$ such that for all $n\ge N$ we have
$$ s_n < L + \delta = L + \varepsilon/2. $$

Assume by contradiction that there are infinitely many $k$ with $x_k \ge L+\varepsilon$. Then we can choose some $k\ge N$ with this property. But for any $k\ge N$ we also have
$$ x_k \le s_k \le s_N < L + \varepsilon/2, $$
which contradicts $x_k \ge L+\varepsilon$. Hence only finitely many indices can satisfy $x_k \ge L+\varepsilon$.

\end{solution}

\section*{4}

\textbf{Problem:} Let $(a_n)_{n\in\mathbb{N}}$ be a sequence of nonnegative real numbers with $\sum_{n=1}^{\infty} a_n < \infty$. Show that the series
$$\sum_{n=1}^{\infty} \frac{\sqrt{a_n}}{n}$$
converges.

\begin{solution}
Set $x_n := \sqrt{a_n}$ and $y_n := \frac{1}{n}$. By the Cauchy--Schwarz inequality for series,
\[
\sum_{n=1}^{\infty} x_n y_n \;\le\; \Big(\sum_{n=1}^{\infty} x_n^2\Big)^{\!1/2} \; \Big(\sum_{n=1}^{\infty} y_n^2\Big)^{\!1/2}.
\]
Since $x_n^2 = a_n$ and $y_n^2 = 1/n^2$, we obtain
\[
\sum_{n=1}^{\infty} \frac{\sqrt{a_n}}{n} \;\le\; \Big(\sum_{n=1}^{\infty} a_n\Big)^{\!1/2}\; \Big(\sum_{n=1}^{\infty} \frac{1}{n^2}\Big)^{\!1/2} < \infty,
\]
because $\sum a_n$ converges by assumption and $\sum 1/n^2$ converges (p-series with $p=2$). Hence $\sum_{n=1}^{\infty} \frac{\sqrt{a_n}}{n}$ converges absolutely.

\end{solution}

\end{document}