\documentclass[12pt,a4paper]{article}

% ----------- Packages -----------
\usepackage{amsmath, amssymb, amsthm} % Math symbols & theorems
\usepackage{enumitem} % Better lists
\usepackage{geometry} % Page layout
\usepackage{fancyhdr} % Header/footer
\usepackage{tikz}     % Diagrams
\usepackage{hyperref} % Clickable references

% ----------- Page Setup -----------
\geometry{margin=1in}
\setlength{\parskip}{0.5em}
\setlength{\parindent}{0pt}
\pagestyle{fancy}
\fancyhf{}
% ----------- Header/Footer -----------
\lhead{MAT121 -- Analysis I}
\chead{Exercise sheet 5}
\rhead{Stefan du Toit}
\rfoot{\thepage}

% ----------- Theorem Environments -----------
\newtheorem{theorem}{Theorem}[section]
\newtheorem{lemma}[theorem]{Lemma}
\newtheorem{proposition}[theorem]{Proposition}
\newtheorem{corollary}[theorem]{Corollary}

\theoremstyle{definition}
\newtheorem{definition}[theorem]{Definition}
\newtheorem{example}[theorem]{Example}
\newtheorem{exercise}{Exercise}[section]
\newcommand{\C}{\mathbb{C}}

\theoremstyle{remark}
\newtheorem*{remark}{Remark}

\newenvironment{solution}{\begin{proof}[Solution]}{\end{proof}}



% ==========================================
\begin{document}

\section*{1}

\subsection*{(a)}
\begin{proof}
Let
\[
a_n:=(1+(-1)^n)\frac{n+1}{n}+(-1)^n.
\]
Split by parity:
\[
(-1)^n=
\begin{cases}
1, & n\ \text{even},\\
-1,& n\ \text{odd}.
\end{cases}
\]
Hence
\[
a_n=
\begin{cases}
2\frac{n+1}{n}+1=3+\frac{2}{n}, & n\ \text{even},\\[4pt]
-1, & n\ \text{odd}.
\end{cases}
\]
Therefore the even subsequence satisfies $a_{2k}=3+\frac{1}{k}\to 3$, and the odd subsequence is constant $a_{2k+1}=-1$.
Thus the set of accumulation points is $\{-1,\,3\}$, realized by the subsequences $(a_{2k+1})$ and $(a_{2k})$, respectively.
\end{proof}

\subsection*{(b)}
\begin{proof}
Let $u_n:=\sqrt[n]{n}-1>0$. Since $\sqrt[n]{n}\to 1$, we have $u_n\to 0$. Fix $r\in(0,1)$, e.g. $r=\tfrac12$. Then there exists $N$ such that $0<u_n<r$ for all $n\ge N$. Hence, for $n\ge N$,
\[
0\le (\sqrt[n]{n}-1)^n=u_n^{\,n}\le r^{\,n}\xrightarrow[n\to\infty]{}0.
\]
By the squeeze theorem, $(\sqrt[n]{n}-1)^n\to 0$.
\end{proof}

%%%%%%

\section*{2}

\subsection*{(a)}
\begin{proof}
Define the sequence $(x_n)$ by
\[
x_1:=7,\qquad x_2:=-5,\qquad
x_{2k}:=2-\frac{1}{k}\ (k\ge 2),\qquad
x_{2k+1}:=\frac{1}{k}\ (k\ge 1).
\]
Then the sequence is bounded with $-5\le x_n\le 7$.

- Supremum and infimum:
\[
\sup_{n\in\mathbb{N}} x_n=7\quad(\text{attained at }n=1),\qquad
\inf_{n\in\mathbb{N}} x_n=-5\quad(\text{attained at }n=2).
\]

- Limsup: For even indices with $k\ge 2$ we have $x_{2k}=2-\frac{1}{k}$ and hence $x_{2k}\to 2$ with $x_{2k}<2$ for all $k$. All odd terms satisfy $x_{2k+1}\le 1$. Therefore, for every $n$,
\[
\sup_{m\ge n} x_m\le 2.
\]
Given $\varepsilon>0$, choose $K$ so that $\frac{1}{K}<\varepsilon$. If $n\le 2K$ then
\[
\sup_{m\ge n} x_m \ge x_{2K} = 2-\frac{1}{K} > 2-\varepsilon.
\]
Thus $\limsup_{n\to\infty} x_n = 2$.

- Liminf: For $k\ge 1$ we have $x_{2k+1}=\frac{1}{k}$ and hence $x_{2k+1}\to 0$, while $x_m\ge 0$ for all $m\ge 3$. Let $\varepsilon>0$ and pick $K$ with $\frac{1}{K}<\varepsilon$. For any $n\ge 3$ choose $j\ge \max\!\big\{K,\lceil n/2\rceil\big\}$ and set $m:=2j+1\ge n$. Then
\[
x_m=\frac{1}{j}\le \frac{1}{K}<\varepsilon,
\]
so $0\le \inf_{m\ge n} x_m \le \varepsilon$ for all sufficiently large $n$. Hence $\liminf_{n\to\infty} x_n = 0$.

All four numbers are finite and pairwise distinct:
\[
\inf x_n=-5,\qquad \liminf x_n=0,\qquad \limsup x_n=2,\qquad \sup x_n=7.
\]
\end{proof}

\subsection*{(b)}
\begin{proof}
Let $(x_n)$ and $(y_n)$ be bounded sequences. For each $N\in\mathbb{N}$ define the tail suprema
\[
s_N:=\sup\{x_n:\ n\ge N\},\qquad
t_N:=\sup\{y_n:\ n\ge N\},\qquad
u_N:=\sup\{x_n+y_n:\ n\ge N\}.
\]
By definition of $\limsup$ for bounded sequences,
\[
\limsup_{n\to\infty}x_n=\lim_{N\to\infty}s_N,\qquad
\limsup_{n\to\infty}y_n=\lim_{N\to\infty}t_N,\qquad
\limsup_{n\to\infty}(x_n+y_n)=\lim_{N\to\infty}u_N.
\]

Fix $N$. For every $n\ge N$ we have $x_n\le s_N$ and $y_n\le t_N$, hence
\[
x_n+y_n\le s_N+t_N\qquad\text{for all }n\ge N.
\]
Taking the supremum over all $n\ge N$ yields
\[
u_N=\sup\{x_n+y_n:\ n\ge N\}\ \le\ s_N+t_N\qquad\text{for every }N.
\]
Now pass to the limit $N\to\infty$:
\[
\limsup_{n\to\infty}(x_n+y_n)
=\lim_{N\to\infty}u_N
\le \lim_{N\to\infty}(s_N+t_N)
=\lim_{N\to\infty}s_N+\lim_{N\to\infty}t_N
=\limsup_{n\to\infty}x_n+\limsup_{n\to\infty}y_n.
\]
This proves
\[
\limsup_{n\to\infty}(x_n+y_n)\ \le\ \limsup_{n\to\infty}x_n\ +\ \limsup_{n\to\infty}y_n.
\]
\end{proof}

\subsection*{(c)}
\begin{proof}
Construct bounded sequences $(x_n)$ and $(y_n)$ by
\[
x_n=\begin{cases}
1,& n\ \text{even},\\
0,& n\ \text{odd},
\end{cases}
\qquad
y_n=\begin{cases}
0,& n\ \text{even},\\
1,& n\ \text{odd}.
\end{cases}
\]
Then $\limsup x_n=1$ and $\limsup y_n=1$. Moreover $x_n+y_n=1$ for all $n$, hence
\[
\limsup_{n\to\infty}(x_n+y_n)=1\ <\ 1+1=\limsup_{n\to\infty}x_n+\limsup_{n\to\infty}y_n.
\]
\end{proof}


%%%%%%%%%%%%%%%%
\section*{3}

\begin{proof}
Let $(x_n)_{n\in\mathbb{N}}\subset\mathbb{R}^k$ be bounded, where
\[
x_n=(x_{n,1},\dots,x_{n,k}),\qquad
\|x_n\|=\Big(\sum_{l=1}^k |x_{n,l}|^2\Big)^{1/2}.
\]
Then there exists $M>0$ such that $\|x_n\|\le M$ for all $n$, hence for each $l$,
\[
|x_{n,l}|\le \|x_n\|\le M\qquad(n\in\mathbb{N}),
\]
so each scalar sequence $(x_{n,l})_{n\in\mathbb{N}}$ is bounded.

\medskip
\noindent
\textbf{Diagonal extraction}
We build a nested family of index subsequences $(n_{r,j})_{j\in\mathbb{N}}$ for $r=1,\dots,k$ with
\[
n_{1,1}<n_{1,2}<\cdots,\quad
n_{2,1}<n_{2,2}<\cdots\ \text{ and }\ (n_{2,j})_j\subset (n_{1,j})_j,\ \dots,\ 
(n_{k,j})_j\subset (n_{k-1,j})_j,
\]
such that for each $r=1,\dots,k$ there exists $L_r\in\mathbb{R}$ with
\[
x_{n_{r,j},\,r}\ \longrightarrow\ L_r\qquad (j\to\infty),
\]
and, because we always pass to subsequences, also
\[
x_{n_{r,j},\,t}\ \longrightarrow\ L_t\qquad\text{for every }t=1,\dots,r.
\]

- Step $r=1$: Apply Bolzano--Weierstrass in $\mathbb{R}$ to $(x_{n,1})_{n}$ to get a strictly increasing $(n_{1,j})_{j}$ with $x_{n_{1,j},1}\to L_1$.

- Inductive step: Suppose $(n_{r-1,j})_{j}$ is chosen with $x_{n_{r-1,j},t}\to L_t$ for $t=1,\dots,r-1$. The sequence $(x_{n_{r-1,j},r})_{j}$ is bounded, so by Bolzano--Weierstrass there exists a strictly increasing map $j\mapsto m_j$ such that
\[
x_{n_{r-1,m_j},\,r}\ \longrightarrow\ L_r.
\]
Define the next index subsequence by $n_{r,j}:=n_{r-1,m_j}$. Then $(n_{r,j})_j$ is strictly increasing, $(n_{r,j})_j\subset (n_{r-1,j})_j$, and for $t=1,\dots,r-1$ we still have $x_{n_{r,j},t}\to L_t$ (subsequence of a convergent sequence), while $x_{n_{r,j},r}\to L_r$ by construction.

After $k$ steps we have $(n_{k,j})_{j}$ such that, for every $t=1,\dots,k$,
\[
x_{n_{k,j},\,t}\ \longrightarrow\ L_t\qquad (j\to\infty).
\]
Let $L:=(L_1,\dots,L_k)\in\mathbb{R}^k$.

\medskip
\noindent
\textbf{Vector convergence.}
Fix $\varepsilon>0$. For each $t=1,\dots,k$, choose $J_t$ so that
\[
|x_{n_{k,j},\,t}-L_t|<\frac{\varepsilon}{\sqrt{k}}\qquad\text{for all } j\ge J_t.
\]
Let $J:=\max\{J_1,\dots,J_k\}$. Then for $j\ge J$,
\[
\|x_{n_{k,j}}-L\|
=\Big(\sum_{t=1}^k |x_{n_{k,j},\,t}-L_t|^2\Big)^{1/2}
\le \Big(\sum_{t=1}^k \frac{\varepsilon^2}{k}\Big)^{1/2}
=\varepsilon.
\]
Hence $x_{n_{k,j}}\to L$.

\medskip
\noindent
\textbf{Conclusion.}
We have produced a convergent subsequence $(x_{n_{k,j}})_j$ of $(x_n)$. Therefore every bounded sequence in $\mathbb{R}^k$ has an accumulation point.
\end{proof}
%%%%%%%%%%%%%



%=========================================

\section*{4}

\begin{proof}
Let $q\in[0,1)$ and suppose
\[
|a_{n+1}-a_n| \le q\,|a_n-a_{n-1}|\qquad\text{for all } n\ge 2.
\]

Fix $n\in\mathbb{N}$ and $p\in\mathbb{N}$. Use the telescoping identity
\[
a_{n+p}-a_n
= (a_{n+1}-a_n)+(a_{n+2}-a_{n+1})+\cdots+(a_{n+p}-a_{n+p-1})
= \sum_{j=n}^{n+p-1}(a_{j+1}-a_j),
\]
hence, by the triangle inequality,
\begin{equation*}
|a_{n+p}-a_n|
\le \sum_{j=n}^{n+p-1} |a_{j+1}-a_j|.
\end{equation*}

For each $j\ge 2$, iterating the recursive bound gives
\[
|a_{j+1}-a_j|
\le q\,|a_j-a_{j-1}|
\le q^2\,|a_{j-1}-a_{j-2}|
\le \cdots \le q^{\,j-1}\,|a_2-a_1|.
\]
Substituting into the previous estimate yields
\[
|a_{n+p}-a_n|
\le |a_2-a_1|\sum_{j=n}^{n+p-1} q^{\,j-1}
= |a_2-a_1|\,q^{\,n-1}\sum_{t=0}^{p-1} q^{\,t}
= |a_2-a_1|\,q^{\,n-1}\,\frac{1-q^{\,p}}{1-q}
\le \frac{|a_2-a_1|}{1-q}\,q^{\,n-1}.
\]

If $q=0$, then $|a_{n+1}-a_n|=0$ for all $n\ge 2$, so $(a_n)$ is eventually constant and hence Cauchy. If $0<q<1$, let $\varepsilon>0$ and choose $N$ such that
\[
\frac{|a_2-a_1|}{1-q}\,q^{\,N-1}<\varepsilon.
\]
For any $m>n\ge N$ (write $m=n+p$ with $p\ge 1$), the bound above gives
\[
|a_m-a_n|=|a_{n+p}-a_n|<\varepsilon.
\]
Therefore $(a_n)$ is a Cauchy sequence.
\end{proof}

\end{document}