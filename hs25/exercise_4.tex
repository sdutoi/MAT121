\documentclass[12pt,a4paper]{article}

% ----------- Packages -----------
\usepackage{amsmath, amssymb, amsthm} % Math symbols & theorems
\usepackage{enumitem} % Better lists
\usepackage{geometry} % Page layout
\usepackage{fancyhdr} % Header/footer
\usepackage{tikz}     % Diagrams
\usepackage{hyperref} % Clickable references

% ----------- Page Setup -----------
\geometry{margin=1in}
\setlength{\parskip}{0.5em}
\setlength{\parindent}{0pt}
\pagestyle{fancy}
\fancyhf{}
% ----------- Header/Footer -----------
\lhead{MAT121 -- Analysis I}
\chead{Exercise sheet 4}
\rhead{Stefan du Toit}
\rfoot{\thepage}

% ----------- Theorem Environments -----------
\newtheorem{theorem}{Theorem}[section]
\newtheorem{lemma}[theorem]{Lemma}
\newtheorem{proposition}[theorem]{Proposition}
\newtheorem{corollary}[theorem]{Corollary}

\theoremstyle{definition}
\newtheorem{definition}[theorem]{Definition}
\newtheorem{example}[theorem]{Example}
\newtheorem{exercise}{Exercise}[section]
\newcommand{\C}{\mathbb{C}}

\theoremstyle{remark}
\newtheorem*{remark}{Remark}

\newenvironment{solution}{\begin{proof}[Solution]}{\end{proof}}



% ==========================================
\begin{document}

\section*{1}
\subsection*{(a)}
\begin{proof}
Let \(a_n:=\sqrt{1+\frac{1}{n}}\). Then
\[
a_n-1=\frac{\frac{1}{n}}{\sqrt{1+\frac{1}{n}}+1}.
\]
For \(n\ge1\), the denominator is \(\ge1\), hence
\[
0< a_n-1 \le \frac{1}{n}\xrightarrow[n\to\infty]{}0.
\]
Thus \(a_n\to 1\).
\end{proof}

\subsection*{(b)}
\begin{proof}
Let \(b_n:=\dfrac{3n^p+5n}{17n^3+11n-7}\) for \(p\in\{2,3,4\}\). Divide numerator and denominator by \(n^3\):
\[
b_n=\frac{3n^{p-3}+5n^{-2}}{17+11n^{-2}-7n^{-3}}.
\]
\begin{itemize}
  \item \(p=2\): \(b_n=\dfrac{3n^{-1}+5n^{-2}}{17+11n^{-2}-7n^{-3}}\to 0\).
  \item \(p=3\): \(b_n=\dfrac{3+5n^{-2}}{17+11n^{-2}-7n^{-3}}\to \dfrac{3}{17}\).
  \item \(p=4\): \(b_n=\dfrac{3n+5n^{-2}}{17+11n^{-2}-7n^{-3}}\).
    For \(n\) large, \(17+11n^{-2}-7n^{-3}\le 19\), hence
    \[
    b_n\ge \frac{3n}{19}\xrightarrow[n\to\infty]{}+\infty,
    \]
    so \(b_n\) diverges to \(+\infty\).
\end{itemize}
\[\boxed{
\lim_{n\to\infty} b_n =
\begin{cases}
0, & p=2,\\[6pt]
\frac{3}{17}, & p=3,\\[6pt]
+\infty, & p=4.
\end{cases}}\]
\end{proof}

\subsection*{(c)}
\begin{proof}
\[
c_n:=\frac{n!}{(n+1)!-n!}
=\frac{n!}{n!\big((n+1)-1\big)}
=\frac{1}{n}\xrightarrow[n\to\infty]{}0.
\]
\end{proof}

\subsection*{(d)}
% Part (d): d_n = (2 + 3/n)^{11}
\begin{proof}
Let \(d_n := \left(2+\frac{3}{n}\right)^{11}\) for \(n\in\mathbb{N}_+\).

\textbf{Lower bound.}
Since \(\frac{3}{n}>0\) for all \(n\),
\[
2+\frac{3}{n} \;>\; 2
\quad\Rightarrow\quad
d_n \;=\; \left(2+\frac{3}{n}\right)^{11} \;\ge\; 2^{11}.
\]
Thus \((d_n)\) is bounded below by \(2^{11}\).

\textbf{Monotonicity.}
Because \(\frac{3}{n+1}<\frac{3}{n}\), we have
\[
2+\frac{3}{n+1} \;<\; 2+\frac{3}{n}.
\]
The map \(x\mapsto x^{11}\) is strictly increasing on \((0,\infty)\), hence
\[
d_{n+1} \;=\; \left(2+\frac{3}{n+1}\right)^{11}
\;<\; \left(2+\frac{3}{n}\right)^{11} \;=\; d_n.
\]
Therefore \((d_n)\) is strictly decreasing.

\textbf{Convergence and identification (no roots needed).}
Since \(\frac{3}{n}\to 0\), we have \(2+\frac{3}{n}\to 2\).
By continuity of \(x\mapsto x^{11}\),
\[
d_n=\left(2+\frac{3}{n}\right)^{11}\xrightarrow[n\to\infty]{}2^{11}.
\]
Thus
\[
\boxed{\displaystyle \lim_{n\to\infty}\left(2+\frac{3}{n}\right)^{11}=2^{11}.}
\]
\end{proof}

%%%%%%

\subsection*{(e)}
\begin{proof}

Let
\[
e_n:=\sqrt{n+1}-\sqrt{n}.
\]
Multiplying numerator and denominator by the conjugate gives
\[
e_n
= \frac{(\sqrt{n+1}-\sqrt{n})(\sqrt{n+1}+\sqrt{n})}{\sqrt{n+1}+\sqrt{n}}
= \frac{1}{\sqrt{n+1}+\sqrt{n}}.
\]
Hence, for all $n\in\mathbb{N}$,
\[
0<e_n=\frac{1}{\sqrt{n+1}+\sqrt{n}}
\le \frac{1}{2\sqrt{n}},
\]
since $\sqrt{n+1}+\sqrt{n}\ge 2\sqrt{n}$.

To see that $1/\sqrt{n}\to 0$, let $\varepsilon>0$ and choose $N>\varepsilon^{-2}$. Then for all $n\ge N$,
\[
0<\frac{1}{\sqrt{n}}\le\frac{1}{\sqrt{N}}<\varepsilon.
\]
Thus $1/\sqrt{n}\to 0$, and therefore $1/(2\sqrt{n})\to 0$.

By the squeeze theorem,
\[
0\le e_n\le \frac{1}{2\sqrt{n}}\xrightarrow[n\to\infty]{}0,
\]
so
\[
\boxed{\displaystyle \lim_{n\to\infty}\big(\sqrt{n+1}-\sqrt{n}\big)=0.}
\]


\end{proof}


%%%%%%%

\section*{2}

\subsection*{(a)}
\begin{proof}
Define
\[
a_n:=i^{n(n+1)}\left(\frac{n^5}{5^n}+1\right).
\]
Since $|i^{n(n+1)}|=1$, we have
\[
|a_n|=\left|\frac{n^5}{5^n}+1\right|.
\]

First we show \(\displaystyle\frac{n^5}{5^n}\to 0\). Observe
\[
\left(\frac{n^5}{5^n}\right)^{1/n}= \frac{n^{5/n}}{5}\xrightarrow[n\to\infty]{} \frac{1}{5}<1,
\]
hence \(\dfrac{n^5}{5^n}\to 0\) (root test / standard fact that any polynomial is dominated by an exponential).

Therefore there exists \(N\) such that for all \(n\ge N\) we have \(\dfrac{n^5}{5^n}<1\). For these \(n\),
\[
|a_n| \le \frac{n^5}{5^n}+1 < 2.
\]
Let \(M_0:=\max\{|a_1|,\dots,|a_{N-1}|\}\) and set \(M:=\max\{M_0,2\}\). Then for all \(n\in\mathbb{N}\),
\[
|a_n|\le M,
\]
so the sequence \((a_n)\) is bounded in \(\mathbb{C}\).

By the Bolzano--Weierstrass theorem every bounded sequence in \(\mathbb{C}\cong\mathbb{R}^2\) has at least one accumulation point. Hence \((a_n)\) has at least one accumulation point.

\[
\boxed{\text{The sequence }(a_n)\text{ is bounded and therefore has at least one accumulation point.}}
\]
\end{proof}

\subsection*{(b)}
\begin{proof}
We first identify the accumulation points. Since $n(n+1)$ is always even, its residue modulo $4$ is
\[
n(n+1)\equiv
\begin{cases}
0 \pmod 4, & n\equiv 0,3 \pmod 4,\\
2 \pmod 4, & n\equiv 1,2 \pmod 4.
\end{cases}
\]
Therefore
\[
i^{n(n+1)}=
\begin{cases}
1, & n\equiv 0,3 \pmod 4,\\
-1, & n\equiv 1,2 \pmod 4.
\end{cases}
\]
With $c_n:=\dfrac{n^5}{5^n}\to 0$ we can write $a_n=\pm(1+c_n)$, so any limit of $(a_n)$ must be either $1$ or $-1$.

For each accumulation point we now provide an explicit convergent subsequence:
- To $1$: take the subsequence along $n=4k$ (or $n=4k+3$). Then
\[
a_{4k}=1+\frac{(4k)^5}{5^{4k}}\xrightarrow[k\to\infty]{}1.
\]
- To $-1$: take the subsequence along $n=4k+1$ (or $n=4k+2$). Then
\[
a_{4k+1}=-\!\left(1+\frac{(4k+1)^5}{5^{4k+1}}\right)\xrightarrow[k\to\infty]{}-1.
\]

Hence the set of accumulation points of $(a_n)$ is $\{1,-1\}$, and for each of them we have exhibited a subsequence converging to it.
\end{proof}

%%%%%

\section*{3}
Define
\[
g:\C\to \C,\qquad
g(z)=
\begin{cases}
\dfrac{z}{|z|}, & z\neq 0,\\[6pt]
1, & z=0.
\end{cases}
\]

\begin{enumerate}[label=(\alph*)]
\item Show that $|g(z)|=1$ for all $z\in\C$.

Case $z\neq 0$:
\[
|g(z)|
= \left| \frac{z}{|z|} \right|
= \frac{|z|}{\,|\,|z|\,|}
= \frac{|z|}{|z|}
= 1.
\]
Case $z=0$:
\[
|g(0)| = |1| = 1.
\]
Therefore $|g(z)|=1$ for all $z\in\C$.

\item Let $a_n:=\dfrac{1+i}{n}$ for $n\in\mathbb{N}\setminus\{0\}$. Does
$\displaystyle \lim_{n\to\infty} g(a_n) = g\!\left(\lim_{n\to\infty} a_n\right)$ hold?

First,
\[
\lim_{n\to\infty} a_n = \lim_{n\to\infty} \frac{1+i}{n} = 0,
\qquad\Rightarrow\qquad
g\!\left(\lim_{n\to\infty} a_n\right) = g(0)=1.
\]
Next, for each $n$ we have
\[
|a_n|
= \left| \frac{1+i}{n} \right|
= \frac{1}{n}|1+i|
= \frac{\sqrt{2}}{n}.
\]
Hence, for all $n$,
\[
g(a_n)
= \frac{a_n}{|a_n|}
= \frac{(1+i)/n}{(\sqrt{2})/n}
= \frac{1+i}{\sqrt{2}}.
\]
Thus the sequence $(g(a_n))$ is constant, so
\[
\lim_{n\to\infty} g(a_n)
= \frac{1+i}{\sqrt{2}}
\neq 1
= g\!\left(\lim_{n\to\infty} a_n\right).
\]
Therefore the equality does not hold, and $g$ is not continuous at $0$.
\end{enumerate}


%%%%

\section*{4}

\begin{enumerate}[label=(\alph*)]

\item Assume the subsequences converge:
\[
x_{2k}\to A,\qquad x_{2k+1}\to B,\qquad x_{5k}\to C.
\]
The common subsequence $x_{10k}$ belongs to both $(x_{2k})$ and $(x_{5k})$, hence it must converge to both $A$ and $C$, so $A=C$.
Likewise $x_{10k+5}=x_{5(2k+1)}$ is a common subsequence of $(x_{2k+1})$ and $(x_{5k})$, hence $B=C$.
Thus all three limits coincide; write $A=B=C=:L$.

To show $x_n\to L$, let $\varepsilon>0$.
Since $x_{2k}\to L$, there exists $K_e$ such that $|x_{2k}-L|<\varepsilon$ for all $k\ge K_e$.
Since $x_{2k+1}\to L$, there exists $K_o$ such that $|x_{2k+1}-L|<\varepsilon$ for all $k\ge K_o$.
Set
\[
N:=\max\{2K_e,\;2K_o+1\}.
\]
If $n\ge N$ is even, then $n=2k$ with $k\ge K_e$, so $|x_n-L|<\varepsilon$.
If $n\ge N$ is odd, then $n=2k+1$ with $k\ge K_o$, so $|x_n-L|<\varepsilon$.
Hence $x_n\to L$.

\item Let $(x_n)$ be convergent with $x_n\to L$, and let $\phi:\mathbb{N}\to\mathbb{N}$ be bijective.
Given $\varepsilon>0$, choose $N$ such that $|x_n-L|<\varepsilon$ for all $n\ge N$.
Consider
\[
S:=\{k\in\mathbb{N}:\ \phi(k)\le N\}.
\]
Because $\phi$ is bijective, $S$ is finite; define $K:=\max S$ (if $S=\varnothing$, take $K:=0$).
For any $k>K$ we have $\phi(k)>N$, hence $|x_{\phi(k)}-L|<\varepsilon$.
Therefore $x_{\phi(k)}\to L$; a bijective reindexing (permutation) preserves convergence.

\end{enumerate}

%%%%%%%%%%%%%%%%%

\section*{5}

Let $(a_n)_{n\in\mathbb{N}}$ be a sequence and let $(b_k)_{k\in\mathbb{N}}=(a_{n_k})_{k\in\mathbb{N}}$ be a subsequence with strictly increasing indices $(n_k)$ such that $\mathbb{N}\setminus\{n_k:k\in\mathbb{N}\}$ is finite. Then there exists $N_0$ with
\[
\{n\in\mathbb{N}: n\ge N_0\}\subseteq \{n_k:k\in\mathbb{N}\}.
\]
Equivalently, for some $K_0$ we have $\{n_k:k\ge K_0\}=\{n\in\mathbb{N}:n\ge N_0\}$.

\begin{enumerate}[label=(\alph*)]
\item $\alpha$ is an accumulation point of $(a_n)$ iff $\alpha$ is an accumulation point of $(b_k)$.

“Only if”: If $\alpha$ is an accumulation point of $(a_n)$, there exists a subsequence $(a_{m_j})$ with $a_{m_j}\to\alpha$.
Since only finitely many indices are omitted from $\{n_k\}$, all sufficiently large $m_j$ lie in $\{n_k\}$; discarding finitely many terms we get a tail $(a_{m_j})_{j\ge j_0}$ which is also a subsequence of $(b_k)$ and still converges to $\alpha$. Hence $\alpha$ is an accumulation point of $(b_k)$.

“If”: Trivial, because $(b_k)$ is a subsequence of $(a_n)$; any accumulation point of $(b_k)$ is an accumulation point of $(a_n)$.

\item $a=\lim_{n\to\infty} a_n$ iff $a=\lim_{k\to\infty} b_k$.

“$\Rightarrow$”: Suppose $a_n\to a$. Given $\varepsilon>0$, choose $N$ such that $|a_n-a|<\varepsilon$ for all $n\ge N$.
Pick $K$ with $n_K\ge N$; then for all $k\ge K$,
\[
|b_k-a|=|a_{n_k}-a|<\varepsilon,
\]
so $b_k\to a$.

“$\Leftarrow$”: Suppose $b_k\to a$. Because only finitely many indices are omitted, choose $N_0$ such that every $n\ge N_0$ belongs to $\{n_k\}$.
Given $\varepsilon>0$, choose $K$ such that $|a_{n_k}-a|<\varepsilon$ for all $k\ge K$ and set $N:=\max\{N_0,n_K\}$.
If $n\ge N$, then $n\in\{n_k\}$ and $n\ge n_K$, hence $n=n_s$ for some $s\ge K$, and therefore
\[
|a_n-a|=|a_{n_s}-a|<\varepsilon.
\]
Thus $a_n\to a$.

\end{enumerate}

\end{document}