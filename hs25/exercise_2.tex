\documentclass[12pt,a4paper]{article}

% ----------- Packages -----------
\usepackage{amsmath, amssymb, amsthm} % Math symbols & theorems
\usepackage{enumitem} % Better lists
\usepackage{geometry} % Page layout
\usepackage{fancyhdr} % Header/footer
\usepackage{tikz}     % Diagrams
\usepackage{hyperref} % Clickable references

% ----------- Page Setup -----------
\geometry{margin=1in}
\setlength{\parskip}{0.5em}
\setlength{\parindent}{0pt}
\pagestyle{fancy}
\fancyhf{}
% ----------- Header/Footer -----------
\lhead{MAT121 -- Analysis I}
\chead{Exercise sheet 2}
\rhead{Stefan du Toit}
\rfoot{\thepage}

% ----------- Theorem Environments -----------
\newtheorem{theorem}{Theorem}[section]
\newtheorem{lemma}[theorem]{Lemma}
\newtheorem{proposition}[theorem]{Proposition}
\newtheorem{corollary}[theorem]{Corollary}

\theoremstyle{definition}
\newtheorem{definition}[theorem]{Definition}
\newtheorem{example}[theorem]{Example}
\newtheorem{exercise}{Exercise}[section]

\theoremstyle{remark}
\newtheorem*{remark}{Remark}

\newenvironment{solution}{\begin{proof}[Solution]}{\end{proof}}



% ==========================================
\begin{document}

\section*{1}
\subsection*{(a)}
Let $c \ge 0$. We claim: $|a| \le c \iff -c \le a \le c$.

Proof. By the definition of absolute value, for every $a$ we have
\[
-|a| \le a \le |a|.
\]
(Indeed, if $a \ge 0$ then $|a|=a$; if $a<0$ then $|a|=-a$.)

($\Rightarrow$) If $|a| \le c$, then
\[
-c \le -|a| \le a \le |a| \le c,
\]
so $-c \le a \le c$.

($\Leftarrow$) If $-c \le a \le c$, then:
- If $a \ge 0$, $|a|=a \le c$.
- If $a \le 0$, then $0 \le -a \le c$, hence $|a|=-a \le c$.

Thus $|a| \le c \iff -c \le a \le c$.

\subsection*{(b) Triangle inequality}
From (a) with $c=|a|$ and $c=|b|$ we have
\[
-|a|\le a\le |a|,\qquad -|b|\le b\le |b|.
\]
Adding these inequalities gives
\[
-(|a|+|b|)\le a+b\le |a|+|b|.
\]
Since $|a|+|b|\ge0$, applying (a) again (with $x=a+b$ and $c=|a|+|b|$) yields
\[
|a+b|\le |a|+|b|.
\]

\subsection*{(c) Reverse triangle inequality}
Apply (b) to $a=(a-b)+b$:
\[
|a|\le |a-b|+|b|\quad\Rightarrow\quad |a|-|b|\le |a-b|.
\]
Swapping $a$ and $b$ gives
\[
|b|\le |a-b|+|a|\quad\Rightarrow\quad |b|-|a|\le |a-b|.
\]
Combining both we get
\[
||a|-|b||\le |a-b|.
\]

\subsection*{(d) Triangle inequality with a midpoint}
Note that $a-c=(a-b)+(b-c)$. By (b),
\[
|a-c|=|(a-b)+(b-c)|\le |a-b|+|b-c|.
\]



%%%%%%%%%%%


\section*{2}
Determine the infimum and supremum of the set
\[
M := \left\{ \frac{m-n}{m+n} : m,n\in \mathbb{N},\; n\ne 0 \right\}.
\]

\begin{proof}[Solution]
Fix $m,n\in\mathbb{N}$ with $n\ge 1$ (so $m+n>0$). Set
\[
x := \frac{m-n}{m+n}.
\]
Then the following identities give immediate bounds:
\[
1 - x \,=\, 1 - \frac{m-n}{m+n} \,=\, \frac{2n}{m+n} \,>\, 0 \quad\Rightarrow\quad x<1,
\]
and
\[
x + 1 \,=\, \frac{m-n}{m+n} + 1 \,=\, \frac{2m}{m+n} \,\ge\, 0 \quad\Rightarrow\quad x\ge -1,
\]
with equality $x=-1$ iff $m=0$. Hence for all admissible $m,n$ we have
\[
-1 \;\le\; \frac{m-n}{m+n} \;<\; 1.
\]

\medskip
\noindent\textit{Supremum.} For every $x\in M$ we just saw $x<1$, so $1$ is an upper bound. To see it is the least upper bound, fix $\varepsilon>0$ and let $n=1$, $m\in\mathbb{N}$ large. Then
\[
\frac{m-1}{m+1} \,=\, 1 - \frac{2}{m+1} \,>\, 1 - \varepsilon
\]
whenever $m+1 > 2/\varepsilon$. Thus elements of $M$ come arbitrarily close to $1$ from below, and therefore
\[
\sup M = 1.
\]
Moreover, no element equals $1$ since $n\ge 1$ forces $1-x=2n/(m+n)>0$.

\medskip
\noindent\textit{Infimum.} From $x\ge -1$ we have $-1$ is a lower bound. If $0\in\mathbb{N}$ is allowed (as is common in this course and consistent with the condition $n\ne 0$), choose $m=0$ and any $n\ge 1$ to get $x=-1\in M$. Hence
\[
\inf M = -1,\quad \text{and }\min M = -1 \text{ is attained (at }m=0\text{)}.
\]
If, alternatively, one uses $\mathbb{N}=\{1,2,\dots\}$, then $m\ge 1$ and equality $x=-1$ is not attained. However, taking $m=1$ and $n\to\infty$ gives
\[
\frac{1-n}{1+n} \,=\, -1 + \frac{2}{1+n} \searrow -1,
\]
so still $\inf M=-1$ but there is no minimum.
\end{proof}


%%%%%%%%%%%%%
\section*{3}
Let $A\subset\mathbb{R}$ with $\alpha:=\inf A>0$, and set $A^{-1}:=\{1/a:\ a\in A\}$. We prove $\sup(A^{-1})=1/\alpha$ by the “no smaller upper bound” method.

\textbf{Step 1: $1/\alpha$ is an upper bound of $A^{-1}$.}
For every $a\in A$ we have $\alpha\le a$ and $\alpha>0$. Since inversion is order-reversing on $(0,\infty)$,
\[
\frac{1}{a}\le \frac{1}{\alpha}.
\]
Hence $1/\alpha$ is an upper bound of $A^{-1}$.

\textbf{Step 2: No smaller number is an upper bound.}
Let $u<1/\alpha$. 
\begin{itemize}
  \item If $u\le 0$, pick any $a\in A$; then $1/a>0\ge u$, so $u$ is not an upper bound.
  \item If $u>0$, then $u<1/\alpha$ implies $1/u>\alpha$. Define $\varepsilon:=1/u-\alpha>0$. By the definition of $\inf$, there exists $a\in A$ with
  \[
  \alpha\le a<\alpha+\varepsilon=\frac{1}{u}.
  \]
  As $a>0$, inversion reverses inequalities:
  \[
  \frac{1}{a}>\frac{1}{\alpha+\varepsilon}=u.
  \]
  Thus there is an element of $A^{-1}$ exceeding $u$, so $u$ is not an upper bound.
\end{itemize}

Since $1/\alpha$ is an upper bound and every smaller number fails to be one, $1/\alpha=\sup(A^{-1})$.

%%%%%%%%%%

\section*{4}
\subsection*{(a)}
Yes. $\mathbb{N}$ and $\mathbb{Q}$ are equinumerous (both are countably infinite).

Construct an explicit enumeration of $\mathbb{Q}$.
List $0$ first. For each integer $s=2,3,4,\dots$, traverse the diagonal
\[
\{(p,q)\in\mathbb{N}_{\ge1}^2:\; p+q=s\}
\]
in increasing $p$, and append to the list the two rationals
\[
\frac{p}{q}\quad\text{and}\quad -\frac{p}{q}
\]
but only when $\gcd(p,q)=1$ (i.e., the fraction is in lowest terms).

This produces an infinite sequence
\[
0,\; 1,-1,\; \tfrac12,-\tfrac12,\; 2,-2,\; \tfrac13,-\tfrac13,\; \tfrac23,-\tfrac23,\; \dots
\]
in which:
- Surjectivity: Every $r\in\mathbb{Q}$ has a unique reduced form $r=\pm\,\frac{m}{n}$ with $m\in\mathbb{N}_{\ge0}$, $n\in\mathbb{N}_{\ge1}$, $\gcd(m,n)=1$. Then $(m,n)$ lies on the diagonal $m+n=s$, so $r$ appears in the list.
- Injectivity: We only list reduced fractions, and each sign is listed once, so no rational appears twice.

Let $f:\mathbb{N}\to\mathbb{Q}$ map $k$ to the $k$-th term of this sequence. Then $f$ is bijective, proving $\mathbb{N}\sim\mathbb{Q}$.

\subsection*{(b)}
Let $[a,b],[c,d]\subset\mathbb{R}$.

Case 1: $a<b$ and $c<d$ (non-degenerate intervals). Define the affine map
\[
f:[a,b]\to[c,d],\qquad f(x):=c+\frac{d-c}{\,b-a\,}\,(x-a).
\]
Then $f(a)=c$ and $f(b)=d$, and $f$ is strictly increasing (hence injective).
For any $x\in[a,b]$ we can write $x=a+t(b-a)$ with $t\in[0,1]$, so
\[
f(x)=c+t(d-c)=(1-t)c+td\in[c,d],
\]
showing $f$ is surjective as well. The inverse is the affine map
\[
f^{-1}(y)=a+\frac{b-a}{\,d-c\,}\,(y-c),\qquad y\in[c,d].
\]
Therefore $f$ is a bijection; the intervals are equinumerous.

Case 2: $a=b$ and $c=d$ (both singletons). The map $f(a)=c$ is a bijection.

Remark (edge case): If exactly one interval is a singleton (e.g., $a=b$ but $c<d$), then the intervals are not equinumerous, since one has cardinality $1$ and the other has the cardinality of the continuum.

%%%%%%%

\subsection*{(c)}
Yes. $(0,1)$ is equinumerous with $\mathbb{R}$.

Define
\[
f:(0,1)\longrightarrow\mathbb{R},\qquad f(x):=\tan\!\big(\pi(x-\tfrac12)\big).
\]
Then $f$ is strictly increasing because $f'(x)=\pi\sec^2\!\big(\pi(x-\tfrac12)\big)>0$ on $(0,1)$, hence injective. Moreover,
\[
\lim_{x\to 0^+} f(x)=\tan\!\big(-\tfrac{\pi}{2}\big)=-\infty,
\qquad
\lim_{x\to 1^-} f(x)=\tan\!\big(\tfrac{\pi}{2}\big)=+\infty,
\]
so $f$ is surjective onto $\mathbb{R}$. The inverse map is explicit:
\[
f^{-1}(y)=\tfrac12+\frac{1}{\pi}\arctan(y),\qquad y\in\mathbb{R}.
\]
Thus $f$ is a bijection and $(0,1)\sim\mathbb{R}$ (both have cardinality of the continuum).

%%%%%%%

\section*{5}
\subsection*{(a)}

Let $a\in\mathbb{R}$ with $a>0$ and $p\in\mathbb{N}$. Set
\[
B:=\{x\in\mathbb{R}:\ x\ge 0,\ x^{p}\le a\}.
\]
We show that $B$ is bounded above.

If $a\ge 1$, then for every $x\ge a$ we have $x^p\ge x\ge a$ (since $p\ge1$ and $x\mapsto x^p$ is increasing on $[0,\infty)$), hence $x\notin B$. Therefore $a$ is an upper bound of $B$.

If $0<a<1$, then for every $x\ge 1$ we have $x^p\ge 1>a$, so $x\notin B$. Hence $1$ is an upper bound of $B$.

Thus in all cases $B$ is bounded above, with the explicit bound
\[
\sup B\le M,\qquad M:=\max\{1,a\}.
\]
%%%%%


\subsection*{5(b)}
Let $B=\{x\in\mathbb{R}:\ x\ge 0,\ x^p\le a\}$ with $a>0$, $p\in\mathbb{N}$, and let $c:=\sup B$. We prove $c^p=a$.

\textbf{Claim (hint).} For every $p\in\mathbb{N}$ and every $\varepsilon$ with $0<\varepsilon<c$ there exists a constant $R=R(p,c)\ge0$ (depending only on $p$ and $c$, not on $\varepsilon$) such that
\[
(c+\varepsilon)^p\le c^p+\varepsilon R,\qquad (c-\varepsilon)^p\ge c^p-\varepsilon R.
\tag{$\star$}
\]
\emph{Proof of the claim (by induction on $p$).}
For $p=1$, take $R=1$, and equalities hold. Suppose $(\star)$ holds for $p$ with some $R_p=R(p,c)$. Then for $p+1$,
\[
(c+\varepsilon)^{p+1}=(c+\varepsilon)^p(c+\varepsilon)
\le (c^p+\varepsilon R_p)(c+\varepsilon)
= c^{p+1}+\varepsilon(R_pc+c^p)+\varepsilon^2R_p.
\]
Since $0<\varepsilon<c$, we have $\varepsilon^2R_p\le \varepsilon\,c\,R_p$. Hence
\[
(c+\varepsilon)^{p+1}\le c^{p+1}+\varepsilon\big(R_pc+c^p+cR_p\big)
= c^{p+1}+\varepsilon R_{p+1},
\]
with $R_{p+1}:=c^p+2cR_p$, which depends only on $p+1$ and $c$. Similarly,
\begin{align}
(c-\varepsilon)^{p+1}
&=(c-\varepsilon)^p(c-\varepsilon)\\
&\ge (c^p-\varepsilon R_p)(c-\varepsilon)\\
&= c^{p+1}-\varepsilon(R_pc+c^p)+\varepsilon^2R_p\\
&\ge c^{p+1}-\varepsilon(c^p+cR_p)\\
&= c^{p+1}-\varepsilon R_{p+1}',
\end{align}
where $R_{p+1}':=c^p+cR_p\le c^p+2cR_p=:R_{p+1}$. Enlarging the constant if necessary, the same $R_{p+1}$ works for both inequalities. This completes the induction and proves the claim.

\medskip
\textbf{Step 1: $c^p\ge a$.}
By definition of $c=\sup B$, for every $\varepsilon\in(0,c)$ we have $c-\varepsilon\in B$ (otherwise $c-\varepsilon$ would be an upper bound of $B$ smaller than $c$). Hence
\[
(c-\varepsilon)^p\le a.
\]
Using $(\star)$,
\[
c^p-\varepsilon R\le (c-\varepsilon)^p\le a
\quad\Rightarrow\quad
c^p\le a+\varepsilon R.
\]
Letting $\varepsilon\downarrow0$ yields $c^p\le a$. Wait—this gives the reverse inequality. So we instead argue as follows: since $c$ is an upper bound of $B$, every $x\in B$ satisfies $x\le c$ and hence $x^p\le c^p$, from which $\sup\{x^p:x\in B\}\le c^p$. But $\{x^p:x\in B\}=[0,a]$ by definition of $B$, so $a\le c^p$. Thus $c^p\ge a$.

\textbf{Step 2: $c^p\le a$.}
For every $\varepsilon>0$, the number $c+\varepsilon$ is not an upper bound of $B$; hence there exists $x_\varepsilon\in B$ with $x_\varepsilon>c-\tfrac{\varepsilon}{2}$. Then
\[
c< x_\varepsilon+\tfrac{\varepsilon}{2}\le c+\varepsilon,
\]
so by monotonicity on $[0,\infty)$,
\[
c^p<(x_\varepsilon+\tfrac{\varepsilon}{2})^p\le (c+\varepsilon)^p.
\]
Using $(\star)$,
\[
c^p<(c+\varepsilon)^p\le c^p+\varepsilon R,
\]
hence $0<(c+\varepsilon)^p-c^p\le \varepsilon R$. Letting $\varepsilon\downarrow0$ gives
\[
\lim_{\varepsilon\downarrow0}(c+\varepsilon)^p=c^p.
\]
But since $x_\varepsilon\in B$, we have $x_\varepsilon^p\le a$, and from $x_\varepsilon>c-\tfrac{\varepsilon}{2}$ and $(\star)$ (applied with $\varepsilon/2$) we get
\[
c^p-\tfrac{\varepsilon}{2}R \le (c-\tfrac{\varepsilon}{2})^p
\le x_\varepsilon^p\le a.
\]
Letting $\varepsilon\downarrow0$ yields $c^p\le a$.

\medskip
Combining Steps 1 and 2 we conclude $c^p=a$.

\end{document}