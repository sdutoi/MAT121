\documentclass[12pt,a4paper]{article}

% ----------- Packages -----------
\usepackage{amsmath, amssymb, amsthm} % Math symbols & theorems
\usepackage{enumitem} % Better lists
\usepackage{geometry} % Page layout
\usepackage{fancyhdr} % Header/footer
\usepackage{tikz}     % Diagrams
\usepackage{hyperref} % Clickable references

% ----------- Page Setup -----------
\geometry{margin=1in}
\setlength{\parskip}{0.5em}
\setlength{\parindent}{0pt}
\pagestyle{fancy}
\fancyhf{}
% ----------- Header/Footer -----------
\lhead{MAT121 -- Analysis I}
\chead{Exercise sheet 6}
\rhead{Stefan du Toit}
\rfoot{\thepage}

% ----------- Theorem Environments -----------
\newtheorem{theorem}{Theorem}[section]
\newtheorem{lemma}[theorem]{Lemma}
\newtheorem{proposition}[theorem]{Proposition}
\newtheorem{corollary}[theorem]{Corollary}

\theoremstyle{definition}
\newtheorem{definition}[theorem]{Definition}
\newtheorem{example}[theorem]{Example}
\newtheorem{exercise}{Exercise}[section]
\newcommand{\C}{\mathbb{C}}

\theoremstyle{remark}
\newtheorem*{remark}{Remark}

\newenvironment{solution}{\begin{proof}[Solution]}{\end{proof}}



% ==========================================
\begin{document}


\section*{Practice Problems}

\subsection*{Problem 1}
Untersuchen Sie die folgenden Reihen auf Konvergenz:

(a) $\sum_{n=1}^{\infty} \frac{(-1)^n}{\sqrt{n}}$

(b) $\sum_{n=1}^{\infty} \frac{n^2}{2^n + n^3}$

(c) $\sum_{n=1}^{\infty} \frac{2^n \cdot n!}{(n+2)!}$

(d) $\sum_{n=2}^{\infty} \frac{1}{n(\ln n)^2}$

\subsection*{Problem 2}
Untersuchen Sie die folgenden Reihen auf Konvergenz:

(a) $\sum_{k=1}^{\infty} \frac{(-1)^k \cdot k}{k^2 + 1}$

(b) $\sum_{k=1}^{\infty} \frac{1}{k^2 + k}$

(c) $\sum_{k=1}^{\infty} \frac{(2k)!}{(k!)^2 \cdot 4^k}$

(d) $\sum_{k=1}^{\infty} \left(\frac{k}{k+1}\right)^{k^2}$

\section*{Solutions}

\subsection*{Problem 1}

\subsubsection*{(a) $\sum_{n=1}^{\infty} \frac{(-1)^n}{\sqrt{n}}$}

\begin{solution}
We apply the Leibniz criterion (alternating series test).

The series has the form $\sum_{n=1}^{\infty} (-1)^n a_n$ where $a_n = \frac{1}{\sqrt{n}}$.

We need to verify three conditions:

\textbf{1. Alternating signs:} The factor $(-1)^n$ ensures alternating signs.  

\textbf{2. Terms decreasing in absolute value:} We have $|a_n| = \frac{1}{\sqrt{n}}$.
For $n \geq 1$: $|a_{n+1}| = \frac{1}{\sqrt{n+1}} < \frac{1}{\sqrt{n}} = |a_n|$

Therefore $\{|a_n|\}$ is monotone decreasing.  

\textbf{3. Terms approach zero:} $\lim_{n \to \infty} |a_n| = \lim_{n \to \infty} \frac{1}{\sqrt{n}} = 0$  

Since all three conditions of the Leibniz criterion are satisfied, the series $\sum_{n=1}^{\infty} \frac{(-1)^n}{\sqrt{n}}$ converges.
\end{solution}

\subsubsection*{(b) $\sum_{n=1}^{\infty} \frac{n^2}{2^n + n^3}$}

\begin{solution}
We use the Direct Comparison Test.

\textbf{Step 1: Choose comparison series}
We compare with $\sum_{n=1}^{\infty} \frac{n^2}{2^n}$, which we will show converges using the ratio test.

For $\sum_{n=1}^{\infty} \frac{n^2}{2^n}$, let $b_n = \frac{n^2}{2^n}$:
$$\left|\frac{b_{n+1}}{b_n}\right| = \frac{\frac{(n+1)^2}{2^{n+1}}}{\frac{n^2}{2^n}} = \frac{(n+1)^2}{n^2} \cdot \frac{2^n}{2^{n+1}} = \left(\frac{n+1}{n}\right)^2 \cdot \frac{1}{2}$$

$$\lim_{n \to \infty} \left|\frac{b_{n+1}}{b_n}\right| = \lim_{n \to \infty} \left(1 + \frac{1}{n}\right)^2 \cdot \frac{1}{2} = 1 \cdot \frac{1}{2} = \frac{1}{2} < 1$$

So $\sum \frac{n^2}{2^n}$ converges by the ratio test.

\textbf{Step 2: Establish the inequality}
For $n \geq 1$: $2^n + n^3 > 2^n$, therefore:
$$\frac{n^2}{2^n + n^3} < \frac{n^2}{2^n}$$

\textbf{Step 3: Apply the Direct Comparison Test}
Since $\sum_{n=1}^{\infty} \frac{n^2}{2^n}$ converges and $0 < \frac{n^2}{2^n + n^3} < \frac{n^2}{2^n}$ for all $n \geq 1$, by the Direct Comparison Test, the series $\sum_{n=1}^{\infty} \frac{n^2}{2^n + n^3}$ converges.
\end{solution}

\subsubsection*{(c) $\sum_{n=1}^{\infty} \frac{2^n \cdot n!}{(n+2)!}$}

\begin{solution}
We first simplify the general term and then apply the Ratio Test.

\textbf{Step 1: Simplify the general term}
$$\frac{2^n \cdot n!}{(n+2)!} = \frac{2^n \cdot n!}{(n+2)(n+1) \cdot n!} = \frac{2^n}{(n+2)(n+1)}$$

\textbf{Step 2: Apply the Ratio Test}
Let $a_n = \frac{2^n}{(n+2)(n+1)}$. We compute:
$$\left|\frac{a_{n+1}}{a_n}\right| = \frac{\frac{2^{n+1}}{(n+3)(n+2)}}{\frac{2^n}{(n+2)(n+1)}} = \frac{2^{n+1} \cdot (n+2)(n+1)}{2^n \cdot (n+3)(n+2)}$$

\textbf{Step 3: Simplify the ratio}
$$\left|\frac{a_{n+1}}{a_n}\right| = \frac{2^{n+1}}{2^n} \cdot \frac{(n+2)(n+1)}{(n+3)(n+2)} = 2 \cdot \frac{n+1}{n+3}$$

\textbf{Step 4: Take the limit}
$$\lim_{n \to \infty} \left|\frac{a_{n+1}}{a_n}\right| = \lim_{n \to \infty} 2 \cdot \frac{n+1}{n+3} = 2 \cdot \lim_{n \to \infty} \frac{1 + \frac{1}{n}}{1 + \frac{3}{n}} = 2 \cdot \frac{1}{1} = 2$$

\textbf{Step 5: Apply the Ratio Test criterion}
Since $\lim_{n \to \infty} \left|\frac{a_{n+1}}{a_n}\right| = 2 > 1$, the series $\sum_{n=1}^{\infty} \frac{2^n \cdot n!}{(n+2)!}$ diverges.

\textbf{Insight:} Even though the factorial $(n+2)!$ grows very rapidly, the exponential $2^n$ in the numerator combined with the polynomial factor dominates, causing divergence.
\end{solution}

\subsection*{Problem 2}

\subsubsection*{(a) $\sum_{k=1}^{\infty} \frac{(-1)^k \cdot k}{k^2 + 1}$}

\begin{solution}
We apply the Leibniz criterion (alternating series test).

The series has the form $\sum_{k=1}^{\infty} (-1)^k a_k$ where $a_k = \frac{k}{k^2 + 1}$.

We need to verify three conditions:

\textbf{1. Alternating signs:} The factor $(-1)^k$ ensures alternating signs.

\textbf{2. Terms approach zero:} 
$$\lim_{k \to \infty} |a_k| = \lim_{k \to \infty} \frac{k}{k^2 + 1} = \lim_{k \to \infty} \frac{1}{k + \frac{1}{k}} = \frac{1}{\infty + 0} = 0$$

\textbf{3. Terms decreasing in absolute value:} We need to show that $\frac{k+1}{(k+1)^2 + 1} < \frac{k}{k^2 + 1}$ for $k \geq 1$.

Cross multiplying (both denominators are positive):
$$(k+1)(k^2 + 1) < k[(k+1)^2 + 1]$$

Expanding the left side: $(k+1)(k^2 + 1) = k^3 + k^2 + k + 1$

Expanding the right side: $k[(k+1)^2 + 1] = k[k^2 + 2k + 2] = k^3 + 2k^2 + 2k$

We need: $k^3 + k^2 + k + 1 < k^3 + 2k^2 + 2k$

Simplifying: $k^2 + k + 1 < 2k^2 + 2k$, which gives us $1 < k^2 + k = k(k+1)$

For $k \geq 1$: $k(k+1) \geq 1 \cdot 2 = 2 > 1$, so the inequality holds.

Therefore $\{|a_k|\}$ is monotone decreasing.

Since all three conditions of the Leibniz criterion are satisfied, the series $\sum_{k=1}^{\infty} \frac{(-1)^k \cdot k}{k^2 + 1}$ converges.
\end{solution}

\subsubsection*{(b) $\sum_{k=1}^{\infty} \frac{1}{k^2 + k}$}

\begin{solution}
We first try the Ratio Test, which turns out to be inconclusive, then use the Direct Comparison Test.

\textbf{Attempt 1: Ratio Test}
Let $a_k = \frac{1}{k^2 + k}$. We compute:
$$\left|\frac{a_{k+1}}{a_k}\right| = \frac{\frac{1}{(k+1)^2 + (k+1)}}{\frac{1}{k^2 + k}} = \frac{k^2 + k}{(k+1)^2 + (k+1)}$$

Simplifying the denominator: $(k+1)^2 + (k+1) = k^2 + 2k + 1 + k + 1 = k^2 + 3k + 2$

So: $$\left|\frac{a_{k+1}}{a_k}\right| = \frac{k^2 + k}{k^2 + 3k + 2} = \frac{k(k+1)}{(k+1)(k+2)} = \frac{k}{k+2}$$

Taking the limit:
$$\lim_{k \to \infty} \left|\frac{a_{k+1}}{a_k}\right| = \lim_{k \to \infty} \frac{k}{k+2} = \lim_{k \to \infty} \frac{1}{1 + \frac{2}{k}} = \frac{1}{1} = 1$$

Since the limit equals 1, the Ratio Test is inconclusive.

\textbf{Attempt 2: Direct Comparison Test}
We compare with the convergent series $\sum_{k=1}^{\infty} \frac{1}{k^2}$ (p-series with $p = 2 > 1$).

For $k \geq 1$: $k^2 + k > k^2$, therefore:
$$\frac{1}{k^2 + k} < \frac{1}{k^2}$$

Since $\sum_{k=1}^{\infty} \frac{1}{k^2}$ converges and $0 < \frac{1}{k^2 + k} < \frac{1}{k^2}$ for all $k \geq 1$, by the Direct Comparison Test, the series $\sum_{k=1}^{\infty} \frac{1}{k^2 + k}$ converges.

\textbf{Alternative approach:} We could also use partial fractions: $\frac{1}{k^2 + k} = \frac{1}{k(k+1)} = \frac{1}{k} - \frac{1}{k+1}$, giving a telescoping series with sum 1.
\end{solution}

\end{document}