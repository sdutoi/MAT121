\documentclass[12pt,a4paper]{article}

% ----------- Packages -----------
\usepackage{amsmath, amssymb, amsthm} % Math symbols & theorems
\usepackage{enumitem} % Better lists
\usepackage{geometry} % Page layout
\usepackage{fancyhdr} % Header/footer
\usepackage{tikz}     % Diagrams
\usepackage{hyperref} % Clickable references
\usepackage{mathrsfs} % Fancy math fonts

% ----------- Page Setup -----------
\geometry{margin=1in}
\setlength{\parskip}{0.5em}
\setlength{\parindent}{0pt}
\pagestyle{fancy}
\fancyhf{}
% ----------- Header/Footer -----------
\lhead{MAT121 -- Analysis I}
\chead{Exercise sheet 2}
\rhead{Stefan du Toit}
\rfoot{\thepage}

% ----------- Theorem Environments -----------
\newtheorem{theorem}{Theorem}[section]
\newtheorem{lemma}[theorem]{Lemma}
\newtheorem{proposition}[theorem]{Proposition}
\newtheorem{corollary}[theorem]{Corollary}

\theoremstyle{definition}
\newtheorem{definition}[theorem]{Definition}
\newtheorem{example}[theorem]{Example}
\newtheorem{exercise}{Exercise}[section]

\theoremstyle{remark}
\newtheorem*{remark}{Remark}

% ----------- Custom Environments -----------
\newenvironment{solution}{\begin{proof}[Solution]}{\end{proof}}



% ==========================================
\begin{document}

\section*{1}
\subsection*{(a)}

We want to prove that for $0 \le k < (n-1)/2$, the inequality $\binom{n}{k} < \binom{n}{k+1}$ holds.
Since the binomial coefficients are positive for $n \ge 0$, this is equivalent to showing that their ratio is greater than 1:
\[
\frac{\binom{n}{k+1}}{\binom{n}{k}} > 1
\]
We transform the ratio:
\begin{align*}
\frac{\binom{n}{k+1}}{\binom{n}{k}} &= \frac{\frac{n!}{(k+1)!(n-k-1)!}}{\frac{n!}{k!(n-k)!}} \\
&= \frac{n!}{(k+1)!(n-k-1)!} \cdot \frac{k!(n-k)!}{n!} \\
&= \frac{k!(n-k)!}{(k+1)!(n-k-1)!} \\
&= \frac{k! \cdot (n-k) \cdot (n-k-1)!}{(k+1) \cdot k! \cdot (n-k-1)!} \\
&= \frac{n-k}{k+1}
\end{align*}
Now we must show that $\frac{n-k}{k+1} > 1$, given the condition $k < \frac{n-1}{2}$.
\begin{align*}
k &< \frac{n-1}{2} \\
2k &< n-1 \\
2k+1 &< n \\
k+1 &< n-k \\
1 &< \frac{n-k}{k+1} \quad (\text{since } k+1 > 0)
\end{align*}

Thus, we have shown that $\frac{\binom{n}{k+1}}{\binom{n}{k}} > 1$, from which it follows that $\binom{n}{k} < \binom{n}{k+1}$. $\blacksquare$

\subsection*{(b)}


We want to prove the identity $\sum_{i=0}^{k} \binom{n+i}{i} = \binom{n+1+k}{k}$ by induction on $k$.

\paragraph{Base Case:} We test the case for $k=0$.
\begin{itemize}
    \item Left-Hand Side (LHS): $\sum_{i=0}^{0} \binom{n+i}{i} = \binom{n+0}{0} = \binom{n}{0} = 1$.
    \item Right-Hand Side (RHS): $\binom{n+1+0}{0} = \binom{n+1}{0} = 1$.
\end{itemize}
Since LHS = RHS, the base case holds.

\paragraph{Inductive Hypothesis:} We assume that the formula holds for an arbitrary but fixed integer $k \ge 0$:
\[
\sum_{i=0}^{k} \binom{n+i}{i} = \binom{n+1+k}{k}
\]

\paragraph{Inductive Step:} We must show that the formula also holds for $k+1$. The goal is to prove:
\[
\sum_{i=0}^{k+1} \binom{n+i}{i} = \binom{n+1+(k+1)}{k+1} = \binom{n+k+2}{k+1}
\]
We start with the left-hand side, split the sum, and apply the inductive hypothesis:
\begin{align*}
\sum_{i=0}^{k+1} \binom{n+i}{i} &= \left( \sum_{i=0}^{k} \binom{n+i}{i} \right) + \binom{n+(k+1)}{k+1} \\
&\overset{\text{Hyp.}}{=} \binom{n+1+k}{k} + \binom{n+k+1}{k+1}
\end{align*}
Now we use Pascal's Identity, which states that $\binom{m}{r} + \binom{m}{r+1} = \binom{m+1}{r+1}$.
Letting $m=n+k+1$ and $r=k$, we get:
\begin{align*}
\binom{n+k+1}{k} + \binom{n+k+1}{k+1} &= \binom{(n+k+1)+1}{k+1} \\
&= \binom{n+k+2}{k+1}
\end{align*}
This is the right-hand side of the statement for $k+1$. Thus, the inductive step is complete.

By the principle of mathematical induction, the identity holds for all integers $k \ge 0$. $\blacksquare$

\section*{2}
\subsection*{(a)}

\textbf{Statement:} For all $a, b \in K$ with $a \neq 0$, the equation $a \cdot x + b = 0$ has exactly one solution.

\textbf{Proof:}

We need to prove two things: existence and uniqueness of the solution.

\textbf{Existence:}
Since $K$ is a field and $a \neq 0$, the multiplicative inverse $a^{-1}$ exists in $K$.

To find a solution, we solve for $x$:
\begin{align}
a \cdot x + b &= 0\\
a \cdot x &= -b\\
a^{-1} \cdot (a \cdot x) &= a^{-1} \cdot (-b)\\
(a^{-1} \cdot a) \cdot x &= -a^{-1}b\\
1 \cdot x &= -a^{-1}b\\
x &= -a^{-1}b
\end{align}

Let's verify this is indeed a solution:
$$a \cdot (-a^{-1}b) + b = -(a \cdot a^{-1}) \cdot b + b = -1 \cdot b + b = -b + b = 0$$

\textbf{Uniqueness:}
Suppose $x_1$ and $x_2$ are both solutions to $a \cdot x + b = 0$.

Then:
\begin{align}
a \cdot x_1 + b &= 0 \implies a \cdot x_1 = -b\\
a \cdot x_2 + b &= 0 \implies a \cdot x_2 = -b
\end{align}

Therefore: $a \cdot x_1 = a \cdot x_2$

Since $a \neq 0$, we can multiply both sides by $a^{-1}$:
$$a^{-1} \cdot (a \cdot x_1) = a^{-1} \cdot (a \cdot x_2)$$

Using associativity:
$$(a^{-1} \cdot a) \cdot x_1 = (a^{-1} \cdot a) \cdot x_2$$

Since $a^{-1} \cdot a = 1$:
$$x_1 = x_2$$

Therefore, the solution is unique: $x = -a^{-1}b$.

\subsection*{(b)}

\textbf{Statement:} For all $a \in K$ with $a \neq 0$, we have $a^2 > 0$.

\textbf{Proof:}

Since $K$ is an ordered field, we have the trichotomy property: for any element $a \in K$, exactly one of the following holds:
\begin{itemize}
\item $a > 0$ (a is positive)
\item $a = 0$ (a is zero)
\item $a < 0$ (a is negative)
\end{itemize}

Since we're given that $a \neq 0$, we have two cases to consider:

\textbf{Case 1:} $a > 0$

If $a > 0$, then by the axiom for ordered fields (if $x > 0$ and $y > 0$, then $xy > 0$), we have:
$$a \cdot a = a^2 > 0$$

\textbf{Case 2:} $a < 0$

If $a < 0$, then $-a > 0$ (since the additive inverse reverses the inequality).

By the axiom for ordered fields, since both $a < 0$ and $a < 0$, their product satisfies:
$$a \cdot a = a^2 > 0$$

This follows from the general principle that the product of two negative elements in an ordered field is positive.

\textbf{Conclusion:} In both cases, $a^2 > 0$ for any $a \neq 0$.


\subsection*{(c)}

\textbf{Statement:} For all $x, y \in K$ with $x, y > 0$, we have $x^2 > y^2$ if and only if $x > y$.

\textbf{Proof:}

We need to prove both directions of the if-and-only-if statement.

\textbf{Direction 1: ($\Rightarrow$) If $x^2 > y^2$, then $x > y$}

We prove this by contradiction. Assume $x^2 > y^2$ but $x \leq y$.

Since $x, y > 0$, we have two subcases:

\textbf{Subcase 1a:} $x = y$

If $x = y$, then $x^2 = y^2$, which contradicts $x^2 > y^2$.

\textbf{Subcase 1b:} $x < y$

If $x < y$, then since both $x, y > 0$, we can apply the ordering axiom for positive elements: if $0 < x < y$, then $x \cdot x < y \cdot y$ (since multiplying inequalities by positive numbers preserves the inequality).

This gives us $x^2 < y^2$, which contradicts our assumption that $x^2 > y^2$.

Since both subcases lead to contradictions, we must have $x > y$.

\textbf{Direction 2: ($\Leftarrow$) If $x > y$, then $x^2 > y^2$}

Assume $x > y$ where $x, y > 0$.

Since $x > y > 0$, we can apply the ordering axiom for ordered fields: if $a > b > 0$ and $c > d > 0$, then $ac > bd$.

Setting $a = c = x$ and $b = d = y$, we get:
$$x \cdot x > y \cdot y$$

Therefore: $x^2 > y^2$.

\textbf{Conclusion:} We have proven both directions, so $x^2 > y^2$ if and only if $x > y$ for positive $x, y$.


\section*{3}

\textbf{Problem:} We consider the subset 
$$A = \left\{(-1)^n \left(5 - \frac{3}{n+1}\right) + 1 : n \in \mathbb{N}\right\}$$
of $\mathbb{Q}$.

\begin{enumerate}
\item[(a)] Show that $A$ is bounded.
\item[(b)] Calculate $\sup(A)$ and $\inf(A)$.
\end{enumerate}

\subsection*{(a)}

\textbf{Statement:} Show that $A$ is bounded.

\textbf{Solution:}

To show that $A$ is bounded, we need to find real numbers $M$ and $m$ such that $m \leq a \leq M$ for all $a \in A$.

First, let's analyze the general term of $A$:
$$a_n = (-1)^n \left(5 - \frac{3}{n+1}\right) + 1$$

Let's simplify this expression:
\begin{align}
a_n &= (-1)^n \left(5 - \frac{3}{n+1}\right) + 1\\
&= (-1)^n \left(\frac{5(n+1) - 3}{n+1}\right) + 1\\
&= (-1)^n \left(\frac{5n + 2}{n+1}\right) + 1
\end{align}

Now we need to consider two cases based on the parity of $n$:

\textbf{Case 1:} $n$ is even (so $(-1)^n = 1$)
\begin{align}
a_n &= \frac{5n + 2}{n+1} + 1\\
&= \frac{5n + 2 + n + 1}{n+1}\\
&= \frac{6n + 3}{n+1}\\
&= \frac{6(n+1) - 3}{n+1}\\
&= 6 - \frac{3}{n+1}
\end{align}

We established that: $0 < \frac{3}{n+1} \leq \frac{3}{2}$

Now we have the expression: $a_n = 6 - \frac{3}{n+1}$

To find the bounds for $a_n$, we need to see what happens when we subtract $\frac{3}{n+1}$ from $6$.

\textbf{Key insight:} When we subtract a positive number from $6$, the result is less than $6$.

Since $0 < \frac{3}{n+1} \leq \frac{3}{2}$, let's see what happens to $6 - \frac{3}{n+1}$:

\textbf{Upper bound:}
\begin{itemize}
\item The smallest value of $\frac{3}{n+1}$ occurs when $n$ is largest
\item As $n \to \infty$, we have $\frac{3}{n+1} \to 0$
\item So $6 - \frac{3}{n+1}$ approaches $6$ from below
\item Therefore: $6 - \frac{3}{n+1} < 6$
\end{itemize}

\textbf{Lower bound:}
\begin{itemize}
\item The largest value of $\frac{3}{n+1}$ is $\frac{3}{2}$ (when $n = 1$)
\item So the smallest value of $6 - \frac{3}{n+1}$ is: $6 - \frac{3}{2} = 6 - 1.5 = 4.5$
\item Therefore: $6 - \frac{3}{n+1} \geq 4.5$
\end{itemize}

\textbf{Final result:}
$$4.5 \leq 6 - \frac{3}{n+1} < 6$$

Which means: $4.5 \leq a_n < 6$

\textbf{Case 2:} $n$ is odd (so $(-1)^n = -1$)
\begin{align}
a_n &= -\frac{5n + 2}{n+1} + 1\\
&= \frac{-(5n + 2) + n + 1}{n+1}\\
&= \frac{-4n - 1}{n+1}\\
&= -\frac{4(n+1) - 3}{n+1}\\
&= -4 + \frac{3}{n+1}
\end{align}

Since $n \geq 1$, we have $0 < \frac{3}{n+1} \leq \frac{3}{2}$.

Therefore: $-4 < -4 + \frac{3}{n+1} \leq -4 + \frac{3}{2} = -2.5$, so $-4 < a_n \leq -2.5$.

\textbf{Conclusion:}
\begin{itemize}
\item When $n$ is even: $4.5 \leq a_n < 6$
\item When $n$ is odd: $-4 < a_n \leq -2.5$
\end{itemize}

Therefore, all elements of $A$ satisfy: $-4 < a_n < 6$.

We can choose $m = -4$ and $M = 6$ as bounds, so $A$ is bounded with $-4 < a \leq 6$ for all $a \in A$.

\subsection*{(b)}

\textbf{Statement:} Calculate $\sup(A)$ and $\inf(A)$.

\textbf{Solution:}

From part (a), we found that:
\begin{itemize}
\item When $n$ is even: $a_n = 6 - \frac{3}{n+1}$, so $4.5 \leq a_n < 6$
\item When $n$ is odd: $a_n = -4 + \frac{3}{n+1}$, so $-4 < a_n \leq -2.5$
\end{itemize}

Let's analyze each case more carefully to find the supremum and infimum.

\textbf{Finding $\sup(A)$}

\textbf{Case 1: Even $n$}

For even $n$, we have $a_n = 6 - \frac{3}{n+1}$.

Let's look at specific values:
\begin{align}
n = 2: \quad a_2 &= 6 - \frac{3}{3} = 6 - 1 = 5\\
n = 4: \quad a_4 &= 6 - \frac{3}{5} = 6 - 0.6 = 5.4\\
n = 6: \quad a_6 &= 6 - \frac{3}{7} \approx 5.57
\end{align}

As $n \to \infty$ (even), $a_n \to 6$.

\textbf{Case 2: Odd $n$}

For odd $n$, we have $a_n = -4 + \frac{3}{n+1}$.

\begin{align}
n = 1: \quad a_1 &= -4 + \frac{3}{2} = -4 + 1.5 = -2.5\\
n = 3: \quad a_3 &= -4 + \frac{3}{4} = -4 + 0.75 = -3.25
\end{align}

As $n \to \infty$ (odd), $a_n \to -4$.

\textbf{Conclusion for $\sup(A)$:}

The largest values in $A$ come from the even case, and they approach $6$ as $n \to \infty$, but never reach $6$.

For any $\varepsilon > 0$, we can find an even $n$ large enough so that $6 - \frac{3}{n+1} > 6 - \varepsilon$.
But $6 \notin A$ since $6 - \frac{3}{n+1} < 6$ for all finite $n$.

Therefore: $\sup(A) = 6$.

\textbf{Finding $\inf(A)$}

From our analysis:
\begin{itemize}
\item The odd case gives: $-4 < a_n \leq -2.5$
\item The even case gives: $4.5 \leq a_n < 6$
\end{itemize}

The smallest values in $A$ come from the odd case, and they approach $-4$ as $n \to \infty$, but never reach $-4$.

For any $\varepsilon > 0$, we can find an odd $n$ large enough so that $-4 + \frac{3}{n+1} < -4 + \varepsilon$.
But $-4 \notin A$ since $-4 + \frac{3}{n+1} > -4$ for all finite $n$.

Therefore: $\inf(A) = -4$.

\textbf{Final Answer:}

$$\sup(A) = 6 \quad \text{and} \quad \inf(A) = -4$$

\textbf{Note:} Both the supremum and infimum are not achieved by any element in $A$ (they are not maximum or minimum values), but they are the least upper bound and greatest lower bound respectively.

\section*{4}

\textbf{Problem:} Let $A$ and $B$ be subsets of $\mathbb{Q}$. We use the notation: $A + B = \{a + b : a \in A, b \in B\}$.

\begin{enumerate}
\item[(a)] Assume that $\max(A)$ and $\max(B)$ exist. Prove that $\max(A + B)$ also exists and
$$\max(A + B) = \max(A) + \max(B)$$
\item[(b)] Assume that $\sup(A)$ and $\sup(B)$ exist. Prove that $\sup(A + B)$ also exists and
$$\sup(A + B) = \sup(A) + \sup(B)$$
\end{enumerate}

\subsection*{(a)}

\textbf{Statement:} Assume that $\max(A)$ and $\max(B)$ exist. Prove that $\max(A + B)$ also exists and $\max(A + B) = \max(A) + \max(B)$.

\textbf{Proof:}

Since $\max(A)$ and $\max(B)$ exist, let's denote:
\begin{itemize}
\item $\alpha = \max(A)$
\item $\beta = \max(B)$
\end{itemize}

This means:
\begin{itemize}
\item $\alpha \in A$ and $\alpha \geq a$ for all $a \in A$
\item $\beta \in B$ and $\beta \geq b$ for all $b \in B$
\end{itemize}

We need to prove two things:
\begin{enumerate}
\item $\max(A + B)$ exists
\item $\max(A + B) = \alpha + \beta$
\end{enumerate}

\textbf{Step 1: Show that $\alpha + \beta \in A + B$}

Since $\alpha \in A$ and $\beta \in B$, by definition of $A + B$, we have:
$$\alpha + \beta \in A + B$$

\textbf{Step 2: Show that $\alpha + \beta$ is an upper bound for $A + B$}

Let $c$ be any element of $A + B$. Then $c = a + b$ for some $a \in A$ and $b \in B$.

Since $\alpha = \max(A)$, we have $a \leq \alpha$.
Since $\beta = \max(B)$, we have $b \leq \beta$.

Adding these inequalities:
$$a + b \leq \alpha + \beta$$

Therefore: $c \leq \alpha + \beta$

This shows that $\alpha + \beta$ is an upper bound for all elements in $A + B$.

\textbf{Step 3: Conclude that $\alpha + \beta = \max(A + B)$}

From Steps 1 and 2:
\begin{itemize}
\item $\alpha + \beta \in A + B$ (so it's achieved)
\item $\alpha + \beta \geq c$ for all $c \in A + B$ (so it's an upper bound)
\end{itemize}

Therefore, $\alpha + \beta$ is the maximum element of $A + B$.

\textbf{Conclusion:}
$$\max(A + B) = \alpha + \beta = \max(A) + \max(B)$$

\subsection*{(b)}

\textbf{Statement:} Assume that $\sup(A)$ and $\sup(B)$ exist. Prove that $\sup(A + B)$ also exists and $\sup(A + B) = \sup(A) + \sup(B)$.

\textbf{Proof:}

Since $\sup(A)$ and $\sup(B)$ exist, let's denote:
\begin{itemize}
\item $\sigma = \sup(A)$
\item $\tau = \sup(B)$
\end{itemize}

We need to prove two things:
\begin{enumerate}
\item $\sup(A + B)$ exists
\item $\sup(A + B) = \sigma + \tau$
\end{enumerate}

\textbf{Step 1: Show that $A + B$ is bounded above}

Let $c$ be any element of $A + B$. Then $c = a + b$ for some $a \in A$ and $b \in B$.

Since $\sigma = \sup(A)$, we have $a \leq \sigma$ for all $a \in A$.
Since $\tau = \sup(B)$, we have $b \leq \tau$ for all $b \in B$.

Adding these inequalities:
$$a + b \leq \sigma + \tau$$

Therefore: $c \leq \sigma + \tau$ for all $c \in A + B$

This shows that $A + B$ is bounded above by $\sigma + \tau$, so $\sup(A + B)$ exists.

\textbf{Step 2: Show that $\sigma + \tau \leq \sup(A + B)$}

We need to show that $\sigma + \tau$ is the least upper bound. We'll prove that any number smaller than $\sigma + \tau$ cannot be an upper bound for $A + B$.

Let $\varepsilon > 0$ be arbitrary. We want to find an element in $A + B$ that is greater than $(\sigma + \tau) - \varepsilon$.

Since $\sigma = \sup(A)$, there exists $a \in A$ such that $a > \sigma - \frac{\varepsilon}{2}$.
Since $\tau = \sup(B)$, there exists $b \in B$ such that $b > \tau - \frac{\varepsilon}{2}$.

Then $a + b \in A + B$ and:
$$a + b > \left(\sigma - \frac{\varepsilon}{2}\right) + \left(\tau - \frac{\varepsilon}{2}\right) = \sigma + \tau - \varepsilon$$

This shows that for any $\varepsilon > 0$, there exists an element in $A + B$ greater than $(\sigma + \tau) - \varepsilon$.

Therefore, $\sigma + \tau$ is the least upper bound of $A + B$.

\textbf{Step 3: Conclude that $\sigma + \tau = \sup(A + B)$}

From Steps 1 and 2:
\begin{itemize}
\item $A + B$ is bounded above by $\sigma + \tau$
\item $\sigma + \tau$ is the least such upper bound
\end{itemize}

Therefore: $\sup(A + B) = \sigma + \tau = \sup(A) + \sup(B)$


\end{document}