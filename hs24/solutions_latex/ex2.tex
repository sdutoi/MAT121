\documentclass[12pt,a4paper]{article}

% ----------- Packages -----------
\usepackage{amsmath, amssymb, amsthm} % Math symbols & theorems
\usepackage{enumitem} % Better lists
\usepackage{geometry} % Page layout
\usepackage{fancyhdr} % Header/footer
\usepackage{tikz}     % Diagrams
\usepackage{hyperref} % Clickable references
\usepackage{mathrsfs} % Fancy math fonts

% ----------- Page Setup -----------
\geometry{margin=1in}
\setlength{\parskip}{0.5em}
\setlength{\parindent}{0pt}
\pagestyle{fancy}
\fancyhf{}
% ----------- Header/Footer -----------
\lhead{MAT121 -- Analysis I}
\chead{Exercise sheet 2}
\rhead{Stefan du Toit}
\rfoot{\thepage}

% ----------- Theorem Environments -----------
\newtheorem{theorem}{Theorem}[section]
\newtheorem{lemma}[theorem]{Lemma}
\newtheorem{proposition}[theorem]{Proposition}
\newtheorem{corollary}[theorem]{Corollary}

\theoremstyle{definition}
\newtheorem{definition}[theorem]{Definition}
\newtheorem{example}[theorem]{Example}
\newtheorem{exercise}{Exercise}[section]

\theoremstyle{remark}
\newtheorem*{remark}{Remark}

% ----------- Custom Environments -----------
\newenvironment{solution}{\begin{proof}[Solution]}{\end{proof}}



% ==========================================
\begin{document}

\section*{1}
\subsection*{(a)}

We want to prove that for $0 \le k < (n-1)/2$, the inequality $\binom{n}{k} < \binom{n}{k+1}$ holds.
Since the binomial coefficients are positive for $n \ge 0$, this is equivalent to showing that their ratio is greater than 1:
\[
\frac{\binom{n}{k+1}}{\binom{n}{k}} > 1
\]
We transform the ratio:
\begin{align*}
\frac{\binom{n}{k+1}}{\binom{n}{k}} &= \frac{\frac{n!}{(k+1)!(n-k-1)!}}{\frac{n!}{k!(n-k)!}} \\
&= \frac{n!}{(k+1)!(n-k-1)!} \cdot \frac{k!(n-k)!}{n!} \\
&= \frac{k!(n-k)!}{(k+1)!(n-k-1)!} \\
&= \frac{k! \cdot (n-k) \cdot (n-k-1)!}{(k+1) \cdot k! \cdot (n-k-1)!} \\
&= \frac{n-k}{k+1}
\end{align*}
Now we must show that $\frac{n-k}{k+1} > 1$, given the condition $k < \frac{n-1}{2}$.
\begin{align*}
k &< \frac{n-1}{2} \\
2k &< n-1 \\
2k+1 &< n \\
k+1 &< n-k \\
1 &< \frac{n-k}{k+1} \quad (\text{since } k+1 > 0)
\end{align*}

Thus, we have shown that $\frac{\binom{n}{k+1}}{\binom{n}{k}} > 1$, from which it follows that $\binom{n}{k} < \binom{n}{k+1}$. $\blacksquare$

\subsection*{(b)}


We want to prove the identity $\sum_{i=0}^{k} \binom{n+i}{i} = \binom{n+1+k}{k}$ by induction on $k$.

\paragraph{Base Case:} We test the case for $k=0$.
\begin{itemize}
    \item Left-Hand Side (LHS): $\sum_{i=0}^{0} \binom{n+i}{i} = \binom{n+0}{0} = \binom{n}{0} = 1$.
    \item Right-Hand Side (RHS): $\binom{n+1+0}{0} = \binom{n+1}{0} = 1$.
\end{itemize}
Since LHS = RHS, the base case holds.

\paragraph{Inductive Hypothesis:} We assume that the formula holds for an arbitrary but fixed integer $k \ge 0$:
\[
\sum_{i=0}^{k} \binom{n+i}{i} = \binom{n+1+k}{k}
\]

\paragraph{Inductive Step:} We must show that the formula also holds for $k+1$. The goal is to prove:
\[
\sum_{i=0}^{k+1} \binom{n+i}{i} = \binom{n+1+(k+1)}{k+1} = \binom{n+k+2}{k+1}
\]
We start with the left-hand side, split the sum, and apply the inductive hypothesis:
\begin{align*}
\sum_{i=0}^{k+1} \binom{n+i}{i} &= \left( \sum_{i=0}^{k} \binom{n+i}{i} \right) + \binom{n+(k+1)}{k+1} \\
&\overset{\text{Hyp.}}{=} \binom{n+1+k}{k} + \binom{n+k+1}{k+1}
\end{align*}
Now we use Pascal's Identity, which states that $\binom{m}{r} + \binom{m}{r+1} = \binom{m+1}{r+1}$.
Letting $m=n+k+1$ and $r=k$, we get:
\begin{align*}
\binom{n+k+1}{k} + \binom{n+k+1}{k+1} &= \binom{(n+k+1)+1}{k+1} \\
&= \binom{n+k+2}{k+1}
\end{align*}
This is the right-hand side of the statement for $k+1$. Thus, the inductive step is complete.

By the principle of mathematical induction, the identity holds for all integers $k \ge 0$. $\blacksquare$

\section*{2}
\subsection*{(a)}

\textbf{Statement:} For all $a, b \in K$ with $a \neq 0$, the equation $a \cdot x + b = 0$ has exactly one solution.

\textbf{Proof:}

We need to prove two things: existence and uniqueness of the solution.

\textbf{Existence:}
Since $K$ is a field and $a \neq 0$, the multiplicative inverse $a^{-1}$ exists in $K$.

To find a solution, we solve for $x$:
\begin{align}
a \cdot x + b &= 0\\
a \cdot x &= -b\\
a^{-1} \cdot (a \cdot x) &= a^{-1} \cdot (-b)\\
(a^{-1} \cdot a) \cdot x &= -a^{-1}b\\
1 \cdot x &= -a^{-1}b\\
x &= -a^{-1}b
\end{align}

Let's verify this is indeed a solution:
$$a \cdot (-a^{-1}b) + b = -(a \cdot a^{-1}) \cdot b + b = -1 \cdot b + b = -b + b = 0$$

\textbf{Uniqueness:}
Suppose $x_1$ and $x_2$ are both solutions to $a \cdot x + b = 0$.

Then:
\begin{align}
a \cdot x_1 + b &= 0 \implies a \cdot x_1 = -b\\
a \cdot x_2 + b &= 0 \implies a \cdot x_2 = -b
\end{align}

Therefore: $a \cdot x_1 = a \cdot x_2$

Since $a \neq 0$, we can multiply both sides by $a^{-1}$:
$$a^{-1} \cdot (a \cdot x_1) = a^{-1} \cdot (a \cdot x_2)$$

Using associativity:
$$(a^{-1} \cdot a) \cdot x_1 = (a^{-1} \cdot a) \cdot x_2$$

Since $a^{-1} \cdot a = 1$:
$$x_1 = x_2$$

Therefore, the solution is unique: $x = -a^{-1}b$.

\subsection*{(b)}

\textbf{Statement:} For all $a \in K$ with $a \neq 0$, we have $a^2 > 0$.

\textbf{Proof:}

Since $K$ is an ordered field, we have the trichotomy property: for any element $a \in K$, exactly one of the following holds:
\begin{itemize}
\item $a > 0$ (a is positive)
\item $a = 0$ (a is zero)
\item $a < 0$ (a is negative)
\end{itemize}

Since we're given that $a \neq 0$, we have two cases to consider:

\textbf{Case 1:} $a > 0$

If $a > 0$, then by the axiom for ordered fields (if $x > 0$ and $y > 0$, then $xy > 0$), we have:
$$a \cdot a = a^2 > 0$$

\textbf{Case 2:} $a < 0$

If $a < 0$, then $-a > 0$ (since the additive inverse reverses the inequality).

By the axiom for ordered fields, since both $a < 0$ and $a < 0$, their product satisfies:
$$a \cdot a = a^2 > 0$$

This follows from the general principle that the product of two negative elements in an ordered field is positive.

\textbf{Conclusion:} In both cases, $a^2 > 0$ for any $a \neq 0$.


\subsection*{(c)}

\textbf{Statement:} For all $x, y \in K$ with $x, y > 0$, we have $x^2 > y^2$ if and only if $x > y$.

\textbf{Proof:}

We need to prove both directions of the if-and-only-if statement.

\textbf{Direction 1: ($\Rightarrow$) If $x^2 > y^2$, then $x > y$}

We prove this by contradiction. Assume $x^2 > y^2$ but $x \leq y$.

Since $x, y > 0$, we have two subcases:

\textbf{Subcase 1a:} $x = y$

If $x = y$, then $x^2 = y^2$, which contradicts $x^2 > y^2$.

\textbf{Subcase 1b:} $x < y$

If $x < y$, then since both $x, y > 0$, we can apply the ordering axiom for positive elements: if $0 < x < y$, then $x \cdot x < y \cdot y$ (since multiplying inequalities by positive numbers preserves the inequality).

This gives us $x^2 < y^2$, which contradicts our assumption that $x^2 > y^2$.

Since both subcases lead to contradictions, we must have $x > y$.

\textbf{Direction 2: ($\Leftarrow$) If $x > y$, then $x^2 > y^2$}

Assume $x > y$ where $x, y > 0$.

Since $x > y > 0$, we can apply the ordering axiom for ordered fields: if $a > b > 0$ and $c > d > 0$, then $ac > bd$.

Setting $a = c = x$ and $b = d = y$, we get:
$$x \cdot x > y \cdot y$$

Therefore: $x^2 > y^2$.

\textbf{Conclusion:} We have proven both directions, so $x^2 > y^2$ if and only if $x > y$ for positive $x, y$.


\section*{3}

\textbf{Problem:} We consider the subset 
$$A = \left\{(-1)^n \left(5 - \frac{3}{n+1}\right) + 1 : n \in \mathbb{N}\right\}$$
of $\mathbb{Q}$.

\begin{enumerate}
\item[(a)] Show that $A$ is bounded.
\item[(b)] Calculate $\sup(A)$ and $\inf(A)$.
\end{enumerate}

\subsection*{(a)}

\textbf{Statement:} Show that $A$ is bounded.

\textbf{Solution:}

To show that $A$ is bounded, we need to find real numbers $M$ and $m$ such that $m \leq a \leq M$ for all $a \in A$.

First, let's analyze the general term of $A$:
$$a_n = (-1)^n \left(5 - \frac{3}{n+1}\right) + 1$$

Let's simplify this expression:
\begin{align}
a_n &= (-1)^n \left(5 - \frac{3}{n+1}\right) + 1\\
&= (-1)^n \left(\frac{5(n+1) - 3}{n+1}\right) + 1\\
&= (-1)^n \left(\frac{5n + 2}{n+1}\right) + 1
\end{align}

Now we need to consider two cases based on the parity of $n$:

\textbf{Case 1:} $n$ is even (so $(-1)^n = 1$)
\begin{align}
a_n &= \frac{5n + 2}{n+1} + 1\\
&= \frac{5n + 2 + n + 1}{n+1}\\
&= \frac{6n + 3}{n+1}\\
&= \frac{6(n+1) - 3}{n+1}\\
&= 6 - \frac{3}{n+1}
\end{align}

Since $n \in \mathbb{N}$, we have $n \geq 1$, so $n+1 \geq 2$, which means $0 < \frac{3}{n+1} \leq \frac{3}{2}$.

Therefore: $6 - \frac{3}{2} \leq 6 - \frac{3}{n+1} < 6$, so $4.5 \leq a_n < 6$.

\textbf{Case 2:} $n$ is odd (so $(-1)^n = -1$)
\begin{align}
a_n &= -\frac{5n + 2}{n+1} + 1\\
&= \frac{-(5n + 2) + n + 1}{n+1}\\
&= \frac{-4n - 1}{n+1}\\
&= -\frac{4(n+1) - 3}{n+1}\\
&= -4 + \frac{3}{n+1}
\end{align}

Since $n \geq 1$, we have $0 < \frac{3}{n+1} \leq \frac{3}{2}$.

Therefore: $-4 < -4 + \frac{3}{n+1} \leq -4 + \frac{3}{2} = -2.5$, so $-4 < a_n \leq -2.5$.

\textbf{Conclusion:}
\begin{itemize}
\item When $n$ is even: $4.5 \leq a_n < 6$
\item When $n$ is odd: $-4 < a_n \leq -2.5$
\end{itemize}

Therefore, all elements of $A$ satisfy: $-4 < a_n < 6$.

We can choose $m = -4$ and $M = 6$ as bounds, so $A$ is bounded with $-4 < a \leq 6$ for all $a \in A$.

\end{document}