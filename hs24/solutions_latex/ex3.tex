\documentclass[12pt,a4paper]{article}

% ----------- Packages -----------
\usepackage{amsmath, amssymb, amsthm} % Math symbols & theorems
\usepackage{enumitem} % Better lists
\usepackage{geometry} % Page layout
\usepackage{fancyhdr} % Header/footer
\usepackage{tikz}     % Diagrams
\usepackage{hyperref} % Clickable references
\usepackage{mathrsfs} % Fancy math fonts

% ----------- Page Setup -----------
\geometry{margin=1in}
\setlength{\parskip}{0.5em}
\setlength{\parindent}{0pt}
\pagestyle{fancy}
\fancyhf{}
% ----------- Header/Footer -----------
\lhead{MAT121 -- Analysis I}
\rhead{Stefan du Toit}
\rfoot{\thepage}

% ----------- Theorem Environments -----------
\newtheorem{theorem}{Theorem}[section]
\newtheorem{lemma}[theorem]{Lemma}
\newtheorem{proposition}[theorem]{Proposition}
\newtheorem{corollary}[theorem]{Corollary}

\theoremstyle{definition}
\newtheorem{definition}[theorem]{Definition}
\newtheorem{example}[theorem]{Example}
\newtheorem{exercise}{Exercise}[section]

\theoremstyle{remark}
\newtheorem*{remark}{Remark}

% ----------- Custom Environments -----------
\newenvironment{solution}{\begin{proof}[Solution]}{\end{proof}}



% ==========================================
\begin{document}

\section*{Exercise sheet 3}

\section*{2}
\begin{solution}
To prove that $a$ is the supremum of the set $S = \{a^{(n)} : n \in \mathbb{N}\}$, we must show two conditions are met:
\begin{enumerate}
    \item $a$ is an upper bound for $S$.
    \item $a$ is the least upper bound for $S$.
\end{enumerate}
\end{solution}

\begin{proof}
\
\paragraph{1. Show that $a$ is an upper bound.}
An upper bound is a value greater than or equal to every element in the set. We need to show that $a^{(n)} \le a$ for all $n \in \mathbb{N}$.

The value of the number $a$ can be written as the sum of an infinite series:
\[ a = a_0 + \sum_{i=1}^{\infty} \frac{a_i}{10^i} \]
The value of the $n$-th term of the sequence, $a^{(n)}$, is a finite sum:
\[ a^{(n)} = a_0 + \sum_{i=1}^{n} \frac{a_i}{10^i} \]
Let's consider the difference between $a$ and $a^{(n)}$:
\[ a - a^{(n)} = \left(a_0 + \sum_{i=1}^{\infty} \frac{a_i}{10^i}\right) - \left(a_0 + \sum_{i=1}^{n} \frac{a_i}{10^i}\right) = \sum_{i=n+1}^{\infty} \frac{a_i}{10^i} \]
Since the digits $a_i$ are all non-negative ($0 \le a_i \le 9$), the sum $\sum_{i=n+1}^{\infty} \frac{a_i}{10^i}$ must be greater than or equal to zero.
\[ a - a^{(n)} \ge 0 \implies a \ge a^{(n)} \]
This holds for all $n \in \mathbb{N}$, so $a$ is an upper bound for the set $\{a^{(n)}\}$.

\paragraph{2. Show that $a$ is the least upper bound.}
We will use a proof by contradiction. Assume there exists another upper bound, $b$, such that $b < a$. Our goal is to show that this assumption leads to a contradiction by finding an element in the set, $a^{(N)}$, that is greater than $b$.

If $b < a$, then the difference $a - b$ is a positive real number. Let $\varepsilon = a - b > 0$.

We need to find an $N$ such that $a^{(N)} > b$. This is equivalent to showing $a^{(N)} > a - \varepsilon$.
This inequality can be rewritten as $\varepsilon > a - a^{(N)}$.

From part 1, we know that $a - a^{(n)} = \sum_{i=n+1}^{\infty} \frac{a_i}{10^i}$. We can find an upper bound for this difference:
\[ a - a^{(n)} = \sum_{i=n+1}^{\infty} \frac{a_i}{10^i} \le \sum_{i=n+1}^{\infty} \frac{9}{10^i} \]
This is a geometric series with first term $9/10^{n+1}$ and common ratio $1/10$. The sum is:
\[ \sum_{i=n+1}^{\infty} \frac{9}{10^i} = \frac{9/10^{n+1}}{1 - 1/10} = \frac{9/10^{n+1}}{9/10} = \frac{1}{10^n} \]
So, we have the inequality $a - a^{(n)} \le \frac{1}{10^n}$.

Now, we want to find an $N$ such that $\varepsilon > a - a^{(N)}$. If we can find an $N$ such that $\varepsilon > \frac{1}{10^N}$, then we will have:
\[ \varepsilon > \frac{1}{10^N} \ge a - a^{(N)} \]
which proves our goal.

By the Archimedean Property of the real numbers, for any $\varepsilon > 0$, we can always find a natural number $N$ large enough such that $10^N > 1/\varepsilon$, which is equivalent to $1/10^N < \varepsilon$.

So, for our $\varepsilon = a - b$, there exists an $N$ such that $1/10^N < a - b$.
This leads to the chain of inequalities:
\[ a - a^{(N)} \le \frac{1}{10^N} < a - b \]
From $a - a^{(N)} < a - b$, we can conclude that $-a^{(N)} < -b$, which means $a^{(N)} > b$.

This contradicts our initial assumption that $b$ is an upper bound for the set $\{a^{(n)}\}$, because we have found an element $a^{(N)}$ in the set that is larger than $b$.
Therefore, no number smaller than $a$ can be an upper bound.

Since $a$ is an upper bound and is also the least upper bound, it is the supremum of the set.
\end{proof}



\section*{3}

\subsection*{(a) Proof that the intersection is non-empty}

\begin{proof}
Let the sequence of nested intervals be $I_n = [a_n, b_n]$ for $n \in \mathbb{N}$. The condition $I_{n+1} \subset I_n$ means that $[a_{n+1}, b_{n+1}] \subset [a_n, b_n]$ for all $n$. This implies two properties for the sequences of endpoints:
\begin{itemize}
    \item The sequence of left endpoints, $\{a_n\}$, is non-decreasing: $a_1 \le a_2 \le \dots \le a_n \le \dots$
    \item The sequence of right endpoints, $\{b_n\}$, is non-increasing: $b_1 \ge b_2 \ge \dots \ge b_n \ge \dots$
\end{itemize}

Let's consider the set of all left endpoints, $A = \{a_n : n \in \mathbb{N}\}$.
The set $A$ is non-empty since the intervals are non-empty. The set $A$ is also bounded above. For any $a_n \in A$, we know that $a_n \le b_n$. Since the sequence $\{b_n\}$ is non-increasing, $b_n \le b_1$ for all $n$. Thus, $a_n \le b_1$ for all $n$, which means $b_1$ is an upper bound for $A$.

By the Completeness Axiom (or Supremum Property) of the real numbers, any non-empty set that is bounded above has a least upper bound (supremum). Let $x = \sup A$.

We will now show that $x$ lies in every interval $I_n$. To do this, we must show that $a_n \le x \le b_n$ for all $n \in \mathbb{N}$.

\begin{enumerate}
    \item \textbf{Show $a_n \le x$:} By the definition of a supremum, $x$ is an upper bound for the set $A$. Therefore, $a_n \le x$ for all $a_n \in A$. This part is satisfied.

    \item \textbf{Show $x \le b_n$:} We will show that any given $b_n$ is an upper bound for the set $A$. Let's fix an arbitrary $n$. We must show that $a_m \le b_n$ for all $m \in \mathbb{N}$.
    \begin{itemize}
        \item \textit{Case 1: $m \le n$.} Since $\{a_k\}$ is non-decreasing, $a_m \le a_n$. We also know $a_n \le b_n$. Thus, $a_m \le b_n$.
        \item \textit{Case 2: $m > n$.} Since the intervals are nested, $I_m \subset I_n$. This implies $a_n \le a_m \le b_m \le b_n$. Thus, $a_m \le b_n$.
    \end{itemize}
    In both cases, $a_m \le b_n$ for all $m$. This means that any $b_n$ is an upper bound for the set $A$. Since $x$ is the \emph{least} upper bound (supremum) of $A$, it must be less than or equal to any other upper bound. Therefore, $x \le b_n$.
\end{enumerate}

Since we have shown that $a_n \le x \le b_n$ for all $n$, it follows that $x \in [a_n, b_n] = I_n$ for all $n$. This means $x \in \bigcap_{n=1}^{\infty} I_n$, and the intersection is non-empty.
\end{proof}

\subsection*{(b) Proof that the intersection is a single point}

\begin{proof}
We are given the additional condition that for any $\varepsilon > 0$, there exists an $N$ such that the length of the interval $I_N$, $b_N - a_N$, is less than $\varepsilon$. This is equivalent to stating that $\lim_{n\to\infty}(b_n - a_n) = 0$.

From part (a), we know there is at least one point $x$ in the intersection. We will now prove by contradiction that there cannot be more than one point.

Assume there are two distinct points, $x_1$ and $x_2$, in the intersection $\bigcap_{n=1}^{\infty} I_n$. Without loss of generality, assume $x_1 < x_2$.

Since both points are in every interval, it must be that for all $n \in \mathbb{N}$:
\[ a_n \le x_1 < x_2 \le b_n \]
From this, we can deduce that the distance between $x_1$ and $x_2$ is less than or equal to the length of any interval:
\[ x_2 - x_1 \le b_n - a_n \quad \text{for all } n. \]
Let $\varepsilon = x_2 - x_1$. Since $x_1 \neq x_2$, we have $\varepsilon > 0$.

According to the given condition, for this value of $\varepsilon$, there must exist some interval $I_N$ such that its length is less than $\varepsilon$. That is, there exists an $N$ such that:
\[ b_N - a_N < \varepsilon = x_2 - x_1 \]
We now have two conflicting statements:
\begin{enumerate}
    \item $x_2 - x_1 \le b_N - a_N$ (from our assumption that both points are in $I_N$)
    \item $b_N - a_N < x_2 - x_1$ (from the problem's given condition)
\end{enumerate}
This is a contradiction. Therefore, our initial assumption that there are two distinct points in the intersection must be false.

Since the intersection contains at least one point (from part a) and at most one point, it must contain exactly one point.
\end{proof}


\section*{4}
\subsection*{(a)}
\begin{align*}
& (x - y) \left( \sum_{i=0}^{n-1} x^{n-1-i}y^i \right) \\
&= x \left( \sum_{i=0}^{n-1} x^{n-1-i}y^i \right) - y \left( \sum_{i=0}^{n-1} x^{n-1-i}y^i \right) \\
&= x(x^{n-1} + x^{n-2}y + \dots + y^{n-1}) - y(x^{n-1} + x^{n-2}y + \dots + y^{n-1}) \\
&= (x^n + x^{n-1}y + x^{n-2}y^2 + \dots + xy^{n-1}) - (x^{n-1}y + x^{n-2}y^2 + \dots + y^n) \\
&= x^n + (x^{n-1}y - x^{n-1}y) + (x^{n-2}y^2 - x^{n-2}y^2) + \dots + (xy^{n-1} - xy^{n-1}) - y^n \\
&= x^n - y^n \quad \blacksquare
\end{align*}

\subsection*{(b)}


\end{document}