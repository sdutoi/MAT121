\documentclass[12pt,a4paper]{article}

% ----------- Packages -----------
\usepackage{amsmath, amssymb, amsthm} % Math symbols & theorems
\usepackage{enumitem} % Better lists
\usepackage{geometry} % Page layout
\usepackage{fancyhdr} % Header/footer
\usepackage{tikz}     % Diagrams
\usepackage{hyperref} % Clickable references
\usepackage{mathrsfs} % Fancy math fonts

% ----------- Page Setup -----------
\geometry{margin=1in}
\setlength{\parskip}{0.5em}
\setlength{\parindent}{0pt}
\pagestyle{fancy}
\fancyhf{}
% ----------- Header/Footer -----------
\lhead{MAT121 -- Analysis I}
\chead{Exercise sheet 5}
\rhead{Stefan du Toit}
\rfoot{\thepage}

% ----------- Theorem Environments -----------
\newtheorem{theorem}{Theorem}[section]
\newtheorem{lemma}[theorem]{Lemma}
\newtheorem{proposition}[theorem]{Proposition}
\newtheorem{corollary}[theorem]{Corollary}

\theoremstyle{definition}
\newtheorem{definition}[theorem]{Definition}
\newtheorem{example}[theorem]{Example}
\newtheorem{exercise}{Exercise}[section]

\theoremstyle{remark}
\newtheorem*{remark}{Remark}

% ----------- Custom Environments -----------
\newenvironment{solution}{\begin{proof}[Solution]}{\end{proof}}



% ==========================================
\begin{document}

\section*{1}


\textbf{Problem:} Given: $a_n \to a$ (where $a > 0$) and $p \in \mathbb{N}$
Prove: $\sqrt[p]{a_n} \to \sqrt[p]{a}$

\textbf{Solution using the hint from Problem 4(a):}

\textbf{Hint:} For all $x, y \in \mathbb{R}$ and $n \in \mathbb{N}$:
$$x^n - y^n = (x - y) \cdot \left(\sum_{i=0}^{n-1} x^{n-1-i}y^i\right)$$

\textbf{Step 1: Set up the problem}

Let $x = \sqrt[p]{a_n}$ and $y = \sqrt[p]{a}$. We want to show that $|\sqrt[p]{a_n} - \sqrt[p]{a}| \to 0$ as $n \to \infty$.

\textbf{Step 2: Apply the hint}

Using the hint with $n = p$:
$$(\sqrt[p]{a_n})^p - (\sqrt[p]{a})^p = (\sqrt[p]{a_n} - \sqrt[p]{a}) \cdot \left(\sum_{i=0}^{p-1} (\sqrt[p]{a_n})^{p-1-i}(\sqrt[p]{a})^i\right)$$

Since $(\sqrt[p]{a_n})^p = a_n$ and $(\sqrt[p]{a})^p = a$:
$$a_n - a = (\sqrt[p]{a_n} - \sqrt[p]{a}) \cdot \left(\sum_{i=0}^{p-1} (\sqrt[p]{a_n})^{p-1-i}(\sqrt[p]{a})^i\right)$$

\textbf{Step 3: Solve for the difference}

Rearranging:
$$\sqrt[p]{a_n} - \sqrt[p]{a} = \frac{a_n - a}{\sum_{i=0}^{p-1} (\sqrt[p]{a_n})^{p-1-i}(\sqrt[p]{a})^i}$$

\textbf{Step 4: Bound the denominator}

Since $a_n \to a$ and $a > 0$, for sufficiently large $n$, we have $a_n$ bounded away from 0. 
Specifically, for large enough $n$: $\frac{a}{2} < a_n < \frac{3a}{2}$.

Taking $p$-th roots: $\sqrt[p]{\frac{a}{2}} < \sqrt[p]{a_n} < \sqrt[p]{\frac{3a}{2}}$.

Since $\sqrt[p]{a} > 0$, each term in the sum $\sum_{i=0}^{p-1} (\sqrt[p]{a_n})^{p-1-i}(\sqrt[p]{a})^i$ is bounded below by a positive constant.

The smallest possible value occurs when $\sqrt[p]{a_n} = \sqrt[p]{\frac{a}{2}}$:
$$\sum_{i=0}^{p-1} (\sqrt[p]{a_n})^{p-1-i}(\sqrt[p]{a})^i \geq \sum_{i=0}^{p-1} \left(\sqrt[p]{\frac{a}{2}}\right)^{p-1-i}(\sqrt[p]{a})^i = C > 0$$

where $C$ is a positive constant independent of $n$.

\textbf{Step 5: Apply the limit}

Taking absolute values:
$$|\sqrt[p]{a_n} - \sqrt[p]{a}| = \frac{|a_n - a|}{\left|\sum_{i=0}^{p-1} (\sqrt[p]{a_n})^{p-1-i}(\sqrt[p]{a})^i\right|} \leq \frac{|a_n - a|}{C}$$

Since $a_n \to a$, we have $|a_n - a| \to 0$.

Therefore: $|\sqrt[p]{a_n} - \sqrt[p]{a}| \leq \frac{|a_n - a|}{C} \to \frac{0}{C} = 0$

This proves $\sqrt[p]{a_n} \to \sqrt[p]{a}$.

\textbf{Key Insight:} The factorization formula from the hint allows us to express the difference of $p$-th roots in terms of the difference of the original sequences, transforming the problem from dealing with roots to dealing with the convergent sequence $a_n \to a$.

\section*{2}
\subsection*{(a)}
\textbf{Problem:} Calculate $\lim_{n \to \infty} \frac{2n^2 + (-1)^n}{11n^2 + 3n + 1}$

\textbf{Solution:}

\textbf{Step 1: Analyze the oscillating term}

The term $(-1)^n$ oscillates between $-1$ and $+1$:
\begin{itemize}
\item When $n$ is even: $(-1)^n = 1$
\item When $n$ is odd: $(-1)^n = -1$
\end{itemize}

So the numerator becomes $2n^2 + 1$ or $2n^2 - 1$ depending on $n$.

\textbf{Step 2: Factor out the highest power}

Since we're looking at the limit as $n \to \infty$, we factor out $n^2$ from both numerator and denominator:

\textbf{Numerator:} $2n^2 + (-1)^n = n^2\left(2 + \frac{(-1)^n}{n^2}\right)$

\textbf{Denominator:} $11n^2 + 3n + 1 = n^2\left(11 + \frac{3}{n} + \frac{1}{n^2}\right)$

\textbf{Step 3: Simplify the fraction}

$$\lim_{n \to \infty} \frac{2n^2 + (-1)^n}{11n^2 + 3n + 1} = \lim_{n \to \infty} \frac{n^2\left(2 + \frac{(-1)^n}{n^2}\right)}{n^2\left(11 + \frac{3}{n} + \frac{1}{n^2}\right)}$$

Cancel $n^2$:
$$= \lim_{n \to \infty} \frac{2 + \frac{(-1)^n}{n^2}}{11 + \frac{3}{n} + \frac{1}{n^2}}$$

\textbf{Step 4: Evaluate the limit of each term}

As $n \to \infty$:
\begin{itemize}
\item $\frac{(-1)^n}{n^2} \to 0$ (since $\left|\frac{(-1)^n}{n^2}\right| = \frac{1}{n^2} \to 0$)
\item $\frac{3}{n} \to 0$
\item $\frac{1}{n^2} \to 0$
\end{itemize}

\textbf{Step 5: Apply the limit}

$$\lim_{n \to \infty} \frac{2 + \frac{(-1)^n}{n^2}}{11 + \frac{3}{n} + \frac{1}{n^2}} = \frac{2 + 0}{11 + 0 + 0} = \frac{2}{11}$$

\textbf{Answer:} $\boxed{\frac{2}{11}}$

%%%

\subsection*{(b)}
\textbf{Problem:} Calculate $\lim_{n \to \infty} \frac{a^n + 3b^n}{2a^n + 5b^n}$ (for $a, b > 0$)

\textbf{Solution:}

Since we are given that $a, b > 0$ but their relationship is not specified, we must consider all possible cases.

\textbf{Case 1: $a > b$}

When $a > b > 0$, we have $\frac{b}{a} < 1$, so $\left(\frac{b}{a}\right)^n \to 0$ as $n \to \infty$.

Factor out $a^n$ from both numerator and denominator:
$$\frac{a^n + 3b^n}{2a^n + 5b^n} = \frac{a^n\left(1 + 3\left(\frac{b}{a}\right)^n\right)}{a^n\left(2 + 5\left(\frac{b}{a}\right)^n\right)} = \frac{1 + 3\left(\frac{b}{a}\right)^n}{2 + 5\left(\frac{b}{a}\right)^n}$$

Taking the limit:
$$\lim_{n \to \infty} \frac{1 + 3\left(\frac{b}{a}\right)^n}{2 + 5\left(\frac{b}{a}\right)^n} = \frac{1 + 3 \cdot 0}{2 + 5 \cdot 0} = \frac{1}{2}$$

\textbf{Case 2: $b > a$}

When $b > a > 0$, we have $\frac{a}{b} < 1$, so $\left(\frac{a}{b}\right)^n \to 0$ as $n \to \infty$.

Factor out $b^n$ from both numerator and denominator:
$$\frac{a^n + 3b^n}{2a^n + 5b^n} = \frac{b^n\left(\left(\frac{a}{b}\right)^n + 3\right)}{b^n\left(2\left(\frac{a}{b}\right)^n + 5\right)} = \frac{\left(\frac{a}{b}\right)^n + 3}{2\left(\frac{a}{b}\right)^n + 5}$$

Taking the limit:
$$\lim_{n \to \infty} \frac{\left(\frac{a}{b}\right)^n + 3}{2\left(\frac{a}{b}\right)^n + 5} = \frac{0 + 3}{2 \cdot 0 + 5} = \frac{3}{5}$$

\textbf{Case 3: $a = b$}

When $a = b$, we have:
$$\frac{a^n + 3a^n}{2a^n + 5a^n} = \frac{4a^n}{7a^n} = \frac{4}{7}$$

Therefore:
$$\lim_{n \to \infty} \frac{4a^n}{7a^n} = \frac{4}{7}$$

\textbf{Final Answer:}

$$\lim_{n \to \infty} \frac{a^n + 3b^n}{2a^n + 5b^n} = \begin{cases}
\frac{1}{2} & \text{if } a > b \\
\frac{3}{5} & \text{if } b > a \\
\frac{4}{7} & \text{if } a = b
\end{cases}$$

%%%%

\subsection*{(c)}
\textbf{Problem:} Calculate $\lim_{n \to \infty} n\left(\sqrt{4n^2 + 1} - 2n\right)$

\textbf{Hint:} $x^2 - y^2 = (x - y)(x + y)$

\textbf{Solution:}

\textbf{Step 1: Identify the problem}

As $n \to \infty$, we have:
\begin{itemize}
\item $n \to \infty$
\item $\sqrt{4n^2 + 1} - 2n \to \infty - \infty$ (indeterminate form)
\end{itemize}

This gives us the indeterminate form $\infty \cdot (\infty - \infty)$, which we cannot evaluate directly.

\textbf{Step 2: Apply conjugate multiplication}

Multiply and divide by the conjugate of $(\sqrt{4n^2 + 1} - 2n)$:

$$n\left(\sqrt{4n^2 + 1} - 2n\right) = n\left(\sqrt{4n^2 + 1} - 2n\right) \cdot \frac{\sqrt{4n^2 + 1} + 2n}{\sqrt{4n^2 + 1} + 2n}$$

\textbf{Step 3: Apply the difference of squares formula}

Using the hint $x^2 - y^2 = (x - y)(x + y)$ with $x = \sqrt{4n^2 + 1}$ and $y = 2n$:

$$(\sqrt{4n^2 + 1} - 2n)(\sqrt{4n^2 + 1} + 2n) = (\sqrt{4n^2 + 1})^2 - (2n)^2$$

$$= (4n^2 + 1) - 4n^2 = 1$$

\textbf{Step 4: Simplify the expression}

$$n\left(\sqrt{4n^2 + 1} - 2n\right) = n \cdot \frac{1}{\sqrt{4n^2 + 1} + 2n} = \frac{n}{\sqrt{4n^2 + 1} + 2n}$$

\textbf{Step 5: Factor out $n$ from the denominator}

$$\sqrt{4n^2 + 1} = \sqrt{n^2(4 + \frac{1}{n^2})} = n\sqrt{4 + \frac{1}{n^2}}$$

Therefore:
$$\frac{n}{\sqrt{4n^2 + 1} + 2n} = \frac{n}{n\sqrt{4 + \frac{1}{n^2}} + 2n} = \frac{n}{n\left(\sqrt{4 + \frac{1}{n^2}} + 2\right)}$$

Cancel $n$:
$$= \frac{1}{\sqrt{4 + \frac{1}{n^2}} + 2}$$

\textbf{Step 6: Evaluate the limit}

As $n \to \infty$:
$$\frac{1}{n^2} \to 0$$

Therefore:
$$\lim_{n \to \infty} \frac{1}{\sqrt{4 + \frac{1}{n^2}} + 2} = \frac{1}{\sqrt{4 + 0} + 2} = \frac{1}{\sqrt{4} + 2} = \frac{1}{2 + 2} = \frac{1}{4}$$

\textbf{Answer:} $\boxed{\frac{1}{4}}$

%%%%

\subsection*{(d)}
\textbf{Problem:} Calculate $\lim_{n \to \infty} \left(\frac{1}{1 \cdot 2} + \frac{1}{2 \cdot 3} + \cdots + \frac{1}{(n-1) \cdot n}\right)$

\textbf{Hint:} $\frac{1}{k(k+1)} = \frac{1}{k} - \frac{1}{k+1}$

\textbf{Solution:}

\textbf{Step 1: Apply the hint to each term}

Using the given identity $\frac{1}{k(k+1)} = \frac{1}{k} - \frac{1}{k+1}$, we can rewrite each term:

\begin{align}
\frac{1}{1 \cdot 2} &= \frac{1}{1} - \frac{1}{2}\\
\frac{1}{2 \cdot 3} &= \frac{1}{2} - \frac{1}{3}\\
\frac{1}{3 \cdot 4} &= \frac{1}{3} - \frac{1}{4}\\
&\vdots\\
\frac{1}{(n-1) \cdot n} &= \frac{1}{n-1} - \frac{1}{n}
\end{align}

\textbf{Step 2: Write out the sum}

$$\sum_{k=1}^{n-1} \frac{1}{k(k+1)} = \sum_{k=1}^{n-1} \left(\frac{1}{k} - \frac{1}{k+1}\right)$$

Expanding this sum:
$$= \left(\frac{1}{1} - \frac{1}{2}\right) + \left(\frac{1}{2} - \frac{1}{3}\right) + \left(\frac{1}{3} - \frac{1}{4}\right) + \cdots + \left(\frac{1}{n-1} - \frac{1}{n}\right)$$

\textbf{Step 3: Observe the telescoping pattern}

When we expand the sum, most terms cancel:
$$= \frac{1}{1} - \frac{1}{2} + \frac{1}{2} - \frac{1}{3} + \frac{1}{3} - \frac{1}{4} + \cdots + \frac{1}{n-1} - \frac{1}{n}$$

The cancellation pattern:
\begin{itemize}
\item $-\frac{1}{2}$ from the first term cancels with $+\frac{1}{2}$ from the second term
\item $-\frac{1}{3}$ from the second term cancels with $+\frac{1}{3}$ from the third term
\item And so on...
\end{itemize}

\textbf{Step 4: Simplify after cancellation}

After all the intermediate terms cancel, only the first and last terms remain:
$$= \frac{1}{1} - \frac{1}{n} = 1 - \frac{1}{n}$$

\textbf{Step 5: Take the limit}

$$\lim_{n \to \infty} \left(1 - \frac{1}{n}\right) = 1 - \lim_{n \to \infty} \frac{1}{n} = 1 - 0 = 1$$

\textbf{Answer:} $\boxed{1}$

\textbf{Key Insight:} The partial fraction decomposition transforms the sum into a telescoping series where consecutive terms cancel, leaving only the boundary terms. This technique is powerful for evaluating sums involving products in denominators.

%%%%%

\section*{3}
\subsection*{(a)}

\begin{enumerate}
    \item Let 
    \[
        A_n = \frac{1}{n}(a_1 + \cdots + a_n).
    \]
    \begin{enumerate}
        \item Show that if $a_n \to a$, then also $A_n \to a$. \\
        Hint: the sequence $(a_n)$ is bounded.
    \end{enumerate}
\end{enumerate}

Let \(A_n=\frac{1}{n}\sum_{k=1}^n a_k\). Assume \(a_n \to a\). Set \(b_k:=a_k-a\). Then \(b_k \to 0\) and, by convergence, \((b_k)\) is bounded: there exists \(M>0\) with \(|b_k|\le M\) for all \(k\).

Fix \(\varepsilon>0\). Choose \(m\) such that \(|b_k|<\varepsilon/2\) for all \(k\ge m+1\). For any \(n\ge m\),
\[
|A_n-a|
= \left|\frac{1}{n}\sum_{k=1}^n b_k\right|
\le \frac{1}{n}\sum_{k=1}^m |b_k| + \frac{1}{n}\sum_{k=m+1}^n |b_k|
\le \frac{S_m}{n} + \frac{n-m}{n}\cdot \frac{\varepsilon}{2}
\le \frac{S_m}{n} + \frac{\varepsilon}{2},
\]
where \(S_m:=\sum_{k=1}^m |b_k|\) is a fixed finite number (finite sum of bounded terms).

Pick \(N \ge \max\{m,\, 2S_m/\varepsilon\}\). Then for all \(n\ge N\),
\[
|A_n-a| \le \frac{S_m}{n} + \frac{\varepsilon}{2} \le \frac{\varepsilon}{2} + \frac{\varepsilon}{2} = \varepsilon.
\]
Hence \(A_n \to a\).

\emph{Use of the hint.} We used that \((a_n)\) (hence \((b_k)\)) is bounded to ensure \(S_m\) is finite and to control the initial finite block.

\subsection*{(b)}
\begin{enumerate}
  \item Does the converse hold as well (i.e.\ $A_n \to a \implies a_n \to a$)?
\end{enumerate}

The converse is false in general. Let $a_n = (-1)^n$. Then
\[
A_n \;=\; \frac{1}{n}\sum_{k=1}^n (-1)^k
\;=\;
\begin{cases}
0, & \text{if $n$ is even},\\[2pt]
-\,\frac{1}{n}, & \text{if $n$ is odd}.
\end{cases}
\]
Hence $A_n \to 0$ while $(a_n)$ does not converge, so $A_n \to a$ does not imply $a_n \to a$.

\textit{Remark.} With extra assumptions (e.g., $(a_n)$ monotone), convergence of $(A_n)$ forces $(a_n)$ to converge to the same limit.


%%%%%

\section*{4}
Let $a_n=\dfrac{10^n}{n!}$.

\textbf{Key ratio (with full cancellation).} For all $n\ge 1$,
\[
\frac{a_{n+1}}{a_n}
=\frac{\dfrac{10^{\,n+1}}{(n+1)!}}{\dfrac{10^{\,n}}{n!}}
=\frac{10^{\,n+1}}{(n+1)!}\cdot\frac{n!}{10^{\,n}}
=\frac{10^{\,n+1}}{10^{\,n}}\cdot\frac{n!}{(n+1)\,n!}
=10\cdot\frac{1}{n+1}
=\frac{10}{n+1}.
\]

\textbf{Eventually decreasing.} Since $\frac{10}{n+1}<1$ for all $n\ge 10$, we have $a_{n+1}<a_n$ for $n\ge 10$.

\textbf{Bounded.} The sequence is positive. Let $M:=\max\{a_1,\dots,a_{10}\}$. For $n\ge 10$, $a_n\le a_{10}\le M$ by monotonicity; for $1\le n\le 10$, $a_n\le M$ by definition. Hence $(a_n)$ is bounded.

\textbf{Limit is $0$ (geometric comparison explained).} Choose $N=19$. Then for all $n\ge N$,
\[
\frac{a_{n+1}}{a_n}=\frac{10}{n+1}\le \frac12
\quad\Longrightarrow\quad
a_{n+1}\le \frac12\,a_n.
\]
Iterating gives
\[
a_{N+1}\le \tfrac12 a_N,\quad
a_{N+2}\le \tfrac12 a_{N+1}\le \big(\tfrac12\big)^2 a_N,\ \ldots,\
a_{N+k}\le \big(\tfrac12\big)^k a_N\quad(k\ge 0).
\]
Thus $(a_{N+k})$ is dominated by the geometric sequence $a_N\big(\tfrac12\big)^k$ with ratio $r=\tfrac12$.

\emph{Why this implies $a_n\to 0$.} For $0<r<1$, the geometric series $\sum_{k=0}^{\infty} r^k$ converges (its sum is $1/(1-r)$), and a necessary condition for convergence of any series is that its terms tend to $0$. Hence $r^k\to 0$, so in particular $\big(\tfrac12\big)^k\to 0$. By the comparison above,
\[
0\le a_{N+k}\le a_N\big(\tfrac12\big)^k\to 0,
\]
which forces $a_{N+k}\to 0$, and therefore $a_n\to 0$.

\textit{Conclusion.} $(a_n)$ is bounded, strictly decreasing for all $n\ge 10$, and $\lim_{n\to\infty} a_n=0$.


%%%%%%%

\section*{5}
\subsection*{(a)}
Let $(a_n)$ and $(b_n)$ be sequences with $a_n>0$ and $b_n>0$. Set
\[
\alpha:=\limsup_{n\to\infty} a_n,\qquad
\beta :=\limsup_{n\to\infty} b_n \quad (\in[0,\infty]).
\]
If $\alpha=\infty$ or $\beta=\infty$, then the right-hand side is $\infty$ and the inequality is trivial. Hence assume $\alpha,\beta<\infty$.

Define the tail sups
\[
s_n:=\sup_{k\ge n} a_k,\qquad t_n:=\sup_{k\ge n} b_k,\qquad
u_n:=\sup_{k\ge n}(a_k b_k).
\]
Standard facts about the limsup give
\[
s_n\searrow \alpha,\qquad t_n\searrow \beta,\qquad
\limsup_{n\to\infty}(a_n b_n)=\lim_{n\to\infty} u_n.
\]
By positivity, for every $k\ge n$,
\[
a_k b_k \le \big(\sup_{j\ge n} a_j\big)\big(\sup_{j\ge n} b_j\big)=s_n\,t_n,
\]
hence $u_n=\sup_{k\ge n}(a_k b_k)\le s_n t_n$ for all $n$. Taking limits,
\[
\limsup_{n\to\infty}(a_n b_n)
=\lim_{n\to\infty} u_n
\le \lim_{n\to\infty} (s_n t_n)
= \big(\lim_{n\to\infty}s_n\big)\big(\lim_{n\to\infty}t_n\big)
=\alpha\,\beta.
\]
Here we used that $(s_n)$ and $(t_n)$ are monotone decreasing and bounded below by $0$, hence converge, and for convergent sequences the limit of the product equals the product of the limits. Therefore,
\[
\limsup_{n\to\infty}(a_n b_n)\le \big(\limsup_{n\to\infty} a_n\big)\big(\limsup_{n\to\infty} b_n\big).
\]

\textit{Remark.} Positivity is used in the estimate $u_n\le s_n t_n$; without $a_n,b_n\ge 0$ this simple supremum bound can fail.

\end{document}