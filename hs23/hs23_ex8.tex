\documentclass[12pt,a4paper]{article}

% ----------- Packages -----------
\usepackage{amsmath, amssymb, amsthm} % Math symbols & theorems
\usepackage{amssymb} % ensures \setminus is defined
\usepackage{enumitem} % Better lists
\usepackage{geometry} % Page layout
\usepackage{fancyhdr} % Header/footer
\usepackage{tikz}     % Diagrams
\usepackage{graphicx} % Images
\usepackage{hyperref} % Clickable references

% ----------- Page Setup -----------
\geometry{margin=1in}
\setlength{\parskip}{0.5em}
\setlength{\parindent}{0pt}
\pagestyle{fancy}
\fancyhf{}
% ----------- Header/Footer -----------
\lhead{MAT121 -- Analysis I}
\chead{hs23 Ex. 8}
\rhead{Stefan du Toit}
\rfoot{\thepage}

% ----------- Theorem Environments -----------
\newtheorem{theorem}{Theorem}[section]
\newtheorem{lemma}[theorem]{Lemma}
\newtheorem{proposition}[theorem]{Proposition}
\newtheorem{corollary}[theorem]{Corollary}

\theoremstyle{definition}
\newtheorem{definition}[theorem]{Definition}
\newtheorem{example}[theorem]{Example}
\newtheorem{exercise}{Exercise}[section]
\newcommand{\C}{\mathbb{C}}
\newcommand{\norm}[1]{\left\lVert#1\right\rVert}
\newcommand{\R}{\mathbb{R}}
\newcommand{\Z}{\mathbb{Z}}


\theoremstyle{remark}
\newtheorem*{remark}{Remark}

\newenvironment{solution}{\begin{proof}[Solution]}{\end{proof}}

% ==========================================
\begin{document}

\section*{1}
\subsection*{(i)}

\noindent \textbf{1 a) i)} \quad $\forall x \in \mathbb{R} \setminus \{1\}$ we have
\[
\frac{x^p - 1}{x^q - 1} 
= \frac{(x-1)(x^{p-1} + x^{p-2} + \dots + x + 1)}{(x-1)(x^{q-1} + x^{q-2} + \dots + x + 1)} 
= \frac{x^{p-1} + x^{p-2} + \dots + x + 1}{x^{q-1} + x^{q-2} + \dots + x + 1}
\]

\noindent Since the right-hand side is continuous at $x=1$, we conclude that
\[
\lim_{x \to 1} \frac{x^p - 1}{x^q - 1} = \frac{p}{q}.
\]

\subsection*{(ii)}

\textbf{Problem:} Calculate $\lim_{x \to 0} x\left(1 + \frac{4}{x^2}\right)^{\frac{1}{2}}$.

\subsection*{Step-by-Step Simplification}
First, we rewrite the term inside the square root fraction using a common denominator:
\[
f(x) = x \sqrt{1 + \frac{4}{x^2}} = x \sqrt{\frac{x^2 + 4}{x^2}}
\]

Using the rule $\sqrt{\frac{a}{b}} = \frac{\sqrt{a}}{\sqrt{b}}$, we separate the numerator and denominator:
\[
f(x) = x \frac{\sqrt{x^2 + 4}}{\sqrt{x^2}}
\]

\noindent \textbf{Important:} Recall that $\sqrt{x^2} = |x|$. Therefore:
\[
f(x) = \frac{x}{|x|} \sqrt{x^2 + 4}
\]

\subsection*{Calculating One-Sided Limits}

\subsubsection*{1. Right-hand limit ($x \to 0^+$)}
For $x > 0$, we have $|x| = x$.
\[
\lim_{x \to 0^+} \frac{x}{x} \sqrt{x^2 + 4} = \lim_{x \to 0^+} 1 \cdot \sqrt{x^2 + 4} = \sqrt{4} = 2
\]

\subsubsection*{2. Left-hand limit ($x \to 0^-$)}
For $x < 0$, we have $|x| = -x$.
\[
\lim_{x \to 0^-} \frac{x}{-x} \sqrt{x^2 + 4} = \lim_{x \to 0^-} (-1) \cdot \sqrt{x^2 + 4} = -\sqrt{4} = -2
\]

\subsection*{Conclusion}
Since the left-hand limit ($-2$) and the right-hand limit ($2$) are not equal:
\[
\lim_{x \to 0^-} f(x) \neq \lim_{x \to 0^+} f(x)
\]
The limit $\lim_{x \to 0} x\left(1 + \frac{4}{x^2}\right)^{\frac{1}{2}}$ \textbf{does not exist}.

\subsection*{(iii)}

\textbf{Problem:} Calculate $\lim_{x \to \infty} \left( \sqrt{(x+a)(x+b)} - x \right)$.

\subsection*{Step-by-Step Calculation}

Direct substitution leads to the form $\infty - \infty$. We multiply and divide by the conjugate expression $\sqrt{(x+a)(x+b)} + x$.

Let $L$ be the limit:
\[
L = \lim_{x \to \infty} \frac{\left(\sqrt{(x+a)(x+b)} - x\right)\left(\sqrt{(x+a)(x+b)} + x\right)}{\sqrt{(x+a)(x+b)} + x}
\]

Using the difference of squares formula $(A-B)(A+B) = A^2 - B^2$ in the numerator:
\[
L = \lim_{x \to \infty} \frac{(x+a)(x+b) - x^2}{\sqrt{(x+a)(x+b)} + x}
\]

Next, we expand the term $(x+a)(x+b)$ in the numerator:
\[
(x+a)(x+b) = x^2 + ax + bx + ab = x^2 + (a+b)x + ab
\]

Substitute this back into the limit and simplify the numerator:
\begin{align*}
L &= \lim_{x \to \infty} \frac{(x^2 + (a+b)x + ab) - x^2}{\sqrt{x^2 + (a+b)x + ab} + x} \\
&= \lim_{x \to \infty} \frac{(a+b)x + ab}{\sqrt{x^2 + (a+b)x + ab} + x}
\end{align*}

To evaluate the limit at infinity, we factor out the dominant term $x$ from both the numerator and the denominator.
Recall that for $x > 0$, $\sqrt{x^2} = x$. We factor $x^2$ out of the square root in the denominator:

\begin{align*}
L &= \lim_{x \to \infty} \frac{x(a+b + \frac{ab}{x})}{\sqrt{x^2(1 + \frac{a+b}{x} + \frac{ab}{x^2})} + x} \\
&= \lim_{x \to \infty} \frac{x(a+b + \frac{ab}{x})}{x\sqrt{1 + \frac{a+b}{x} + \frac{ab}{x^2}} + x} \\
&= \lim_{x \to \infty} \frac{x(a+b + \frac{ab}{x})}{x\left(\sqrt{1 + \frac{a+b}{x} + \frac{ab}{x^2}} + 1\right)}
\end{align*}

Cancel $x$ from the numerator and denominator:
\[
L = \lim_{x \to \infty} \frac{a+b + \frac{ab}{x}}{\sqrt{1 + \frac{a+b}{x} + \frac{ab}{x^2}} + 1}
\]

Now, we take the limit as $x \to \infty$. All terms with $x$ in the denominator approach $0$:
\[
L = \frac{a+b + 0}{\sqrt{1 + 0 + 0} + 1} = \frac{a+b}{1 + 1}
\]

\subsection*{Conclusion}
\[
\lim_{x \to \infty} \left( \sqrt{(x+a)(x+b)} - x \right) = \frac{a+b}{2}
\]

\subsection*{(b)}

\textbf{Problem:} Determine $\alpha, \beta \in \mathbb{R}$ such that the function $f(x)$ is continuous on $\mathbb{R}$.
\[
f(x) = \begin{cases} 
x^2 - \alpha x + \beta & \text{for } x \le -1 \\ 
(\alpha + \beta)x & \text{for } -1 \le x \le 1 \\ 
x^2 + \alpha x - \beta & \text{for } x \ge 1 
\end{cases}
\]

\subsection*{Condition for Continuity}
Since polynomials are continuous everywhere, we only need to ensure continuity at the transition points $x = -1$ and $x = 1$.
For $f(x)$ to be continuous, the limit from the left must equal the limit from the right at these points.

\subsection*{1. Continuity at $x = -1$}
We equate the limit from the left ($x \le -1$) and the limit from the right ($x \ge -1$):
\begin{align*}
\lim_{x \to -1^-} f(x) &= \lim_{x \to -1^+} f(x) \\
(-1)^2 - \alpha(-1) + \beta &= (\alpha + \beta)(-1) \\
1 + \alpha + \beta &= -\alpha - \beta
\end{align*}
Rearranging terms to form an equation:
\[
2\alpha + 2\beta = -1 \quad \dots \text{(Equation I)}
\]

\subsection*{2. Continuity at $x = 1$}
We equate the limit from the left ($x \le 1$) and the limit from the right ($x \ge 1$):
\begin{align*}
\lim_{x \to 1^-} f(x) &= \lim_{x \to 1^+} f(x) \\
(\alpha + \beta)(1) &= (1)^2 + \alpha(1) - \beta \\
\alpha + \beta &= 1 + \alpha - \beta
\end{align*}
Subtracting $\alpha$ from both sides:
\[
\beta = 1 - \beta
\]
\[
2\beta = 1 \implies \beta = \frac{1}{2}
\]

\subsection*{3. Solving for $\alpha$}
Substitute $\beta = \frac{1}{2}$ into (Equation I):
\begin{align*}
2\alpha + 2\left(\frac{1}{2}\right) &= -1 \\
2\alpha + 1 &= -1 \\
2\alpha &= -2 \\
\alpha &= -1
\end{align*}

\subsection*{Conclusion}
The function $f(x)$ is continuous on the entire real axis for the values:
\[
\alpha = -1, \quad \beta = \frac{1}{2}
\]


\section*{4}
\subsection*{(a)}
\begin{solution}
Let $\epsilon > 0$. We aim to show that $(f(x_n))_{n \in \mathbb{N}}$ is a Cauchy sequence. That is, we must find an $N \in \mathbb{N}$ such that for all $n, m \geq N$, $|f(x_n) - f(x_m)| < \epsilon$.

Since $f$ is uniformly continuous on $X$, there exists a $\delta > 0$ such that for all $x, y \in X$:
\[
|x - y| < \delta \implies |f(x) - f(y)| < \epsilon
\]
Since $(x_n)_{n \in \mathbb{N}}$ is a Cauchy sequence in $X$, there exists an $N \in \mathbb{N}$ such that for all $n, m \geq N$:
\[
|x_n - x_m| < \delta
\]
Now, consider arbitrary $n, m \geq N$. We have $|x_n - x_m| < \delta$. By the uniform continuity condition stated above, this implies:
\[
|f(x_n) - f(x_m)| < \epsilon
\]
Since this holds for any $\epsilon > 0$, the sequence $(f(x_n))_{n \in \mathbb{N}}$ is a Cauchy sequence.
\end{solution}

\subsection*{(b)}
\begin{solution}
Let $x_0$ be an accumulation point of the sequence $(x_n)_{n \in \mathbb{N}} \subset X$. By definition, there exists a subsequence $(x_{n_k})_{k \in \mathbb{N}}$ such that $\lim_{k \to \infty} x_{n_k} = x_0$.

Since $(x_{n_k})$ is a convergent sequence in $\mathbb{R}$, it is a Cauchy sequence. From part (a), we know that since $f$ is uniformly continuous, the image sequence $(f(x_{n_k}))_{k \in \mathbb{N}}$ is also a Cauchy sequence in $\mathbb{R}$. Because $\mathbb{R}$ is complete, this sequence converges. Let $y = \lim_{k \to \infty} f(x_{n_k})$.

We define the extension $g: X \cup \{x_0\} \to \mathbb{R}$ as:
\[
g(x) = \begin{cases} 
f(x) & \text{if } x \in X \\
y & \text{if } x = x_0 
\end{cases}
\]
We now show that $g$ is continuous at $x_0$. (Note: Continuity on $X$ is inherited from $f$).
Let $\epsilon > 0$. Since $f$ is uniformly continuous on $X$, there exists a $\delta > 0$ such that for all $u, v \in X$:
\[
|u - v| < \delta \implies |f(u) - f(v)| < \frac{\epsilon}{2}
\]
Since $x_{n_k} \to x_0$ and $f(x_{n_k}) \to y$, we can choose a sufficiently large index $K$ such that:
\[
|x_{n_K} - x_0| < \frac{\delta}{2} \quad \text{and} \quad |f(x_{n_K}) - y| < \frac{\epsilon}{2}
\]
Now, consider any $x \in X$ with $|x - x_0| < \frac{\delta}{2}$. By the triangle inequality:
\[
|x - x_{n_K}| \leq |x - x_0| + |x_0 - x_{n_K}| < \frac{\delta}{2} + \frac{\delta}{2} = \delta
\]
Using the uniform continuity condition, we have $|f(x) - f(x_{n_K})| < \frac{\epsilon}{2}$.
Finally, we bound the distance from $g(x)$ to the limit $y$:
\[
|g(x) - g(x_0)| = |f(x) - y| \leq |f(x) - f(x_{n_K})| + |f(x_{n_K}) - y| < \frac{\epsilon}{2} + \frac{\epsilon}{2} = \epsilon
\]
Thus, $\lim_{x \to x_0} g(x) = g(x_0)$, which proves that $g$ is continuous at $x_0$.
\end{solution}

\subsection*{(b from handwritten solution)}
\begin{solution}
Let $(a_n)_{n \in \mathbb{N}} \subset X$ be a sequence such that $\lim_{n \to \infty} a_n = x_0$. Since $(a_n)$ converges, it is a Cauchy sequence. By part (a), $(f(a_n))_{n \in \mathbb{N}}$ is also a Cauchy sequence. Since $\mathbb{R}$ is complete, the limit $y := \lim_{n \to \infty} f(a_n)$ exists. We want to show that the extension
\[
\bar{f}(x) := \begin{cases} 
f(x) & x \in X \\ 
\lim_{n \to \infty} f(a_n) & x = x_0 
\end{cases}
\]
is continuous on $X \cup \{x_0\}$.

Let $(b_n)_{n \in \mathbb{N}} \subset X$ be another sequence such that $\lim_{n \to \infty} b_n = x_0$. We want to show that $\lim_{n \to \infty} f(b_n) = \lim_{n \to \infty} f(a_n)$. If this holds, then $f$ is continuous by the sequential criterion for continuity.

Let $c_n$ be the interleaved sequence defined as:
\[
(c_n)_{n \in \mathbb{N}} := (a_1, b_1, a_2, b_2, \dots)
\]
For any $\epsilon > 0$, there exists $N \in \mathbb{N}$ such that $|a_n - x_0| < \epsilon$ and $|b_n - x_0| < \epsilon$ for all $n \geq N$.
If we choose an index $n \geq 2N$, then the corresponding index for the original sequences is at least $N$ (specifically $n/2$ if even, or $(n+1)/2$ if odd).
\begin{itemize}
    \item For $n$ even, $|c_n - x_0| = |b_{n/2} - x_0| < \epsilon$.
    \item For $n$ odd, $|c_n - x_0| = |a_{(n+1)/2} - x_0| < \epsilon$.
\end{itemize}
Therefore, $c_n \to x_0$. As shown in part (a), since $(c_n)$ converges, $\lim_{n \to \infty} f(c_n)$ exists.
Since $(f(a_n))$ and $(f(b_n))$ are subsequences of $(f(c_n))$, they must converge to the same limit.
\[
\lim_{n \to \infty} f(a_n) = \lim_{n \to \infty} f(b_n) = \lim_{n \to \infty} f(c_n) = \bar{f}(x_0)
\]
Thus, the extension is continuous.
\end{solution}
\end{document}