\documentclass[12pt]{article}
\usepackage[utf8]{inputenc}
\usepackage[T1]{fontenc}
\usepackage{amsmath,amssymb,amsthm}
\usepackage{geometry}
\geometry{margin=1in}
\usepackage{enumitem}
\setlist{nosep}

% This chapter translation: Original German "Die Exponentialfunktion" (The Exponential Function)
% Numbering (Definition/Satz) preserved exactly as in source. Do NOT renumber.

\begin{document}

\section*{4.5 The Exponential Function}

\paragraph{Definition 4.5.1} By
\[
	\exp(z) := e^{z} := \sum_{n=0}^{\infty} \frac{z^{n}}{n!},\quad
	\cos(z) := \sum_{n=0}^{\infty} (-1)^n \frac{z^{2n}}{(2n)!},\quad
	\sin(z) := \sum_{n=0}^{\infty} (-1)^n \frac{z^{2n+1}}{(2n+1)!}
\]
we define three functions from $\mathbb C$ to $\mathbb C$, called the exponential, cosine, and sine.

\paragraph{Theorem 4.5.2}
\begin{enumerate}[label=({\arabic*})]
	\item The power series in Def.~4.5.1 all have radius of convergence $+\infty$.
	\item $\exp$, $\sin$, $\cos$ are real-valued on $\mathbb R$.
	\item $e^{z_1 + z_2} = e^{z_1}\cdot e^{z_2}$.
	\item $e^{z} \ne 0$ for all $z \in \mathbb C$, and $e^{-z} = \dfrac{1}{e^{z}}$.
	\item $e^{iz} = \cos(z) + i\,\sin(z)$.
	\item $\cos$ is an even function, i.e. $\cos(-z) = \cos(z)$ for all $z \in \mathbb C$; $\sin$ is an odd function, i.e. $\sin(-z) = -\sin(z)$ for all $z \in \mathbb C$.
	\item For all $z \in \mathbb C$,
	\[
		\cos(z) = \tfrac{1}{2}\bigl(e^{iz} + e^{-iz}\bigr),\qquad
		\sin(z) = \tfrac{1}{2i}\bigl(e^{iz} - e^{-iz}\bigr).
	\]
	\item For all $x \in \mathbb R$,
	\[
		\cos(x) = \operatorname{Re}(e^{ix}),\qquad \sin(x) = \operatorname{Im}(e^{ix}).
	\]
	\item $\exp$, $\cos$, and $\sin$ are continuous.
\end{enumerate}

\emph{Proof.}
\begin{enumerate}[label=({\arabic*})]
	\item Use the root test for the exponential series\footnote{For the exponential series one can also use the ratio test more directly; the root test applies analogously, and the same argument works for $\sin$ and $\cos$.}:
	For a power series $\sum_{n=0}^{\infty} a_n z^n$ the radius of convergence is
	\[
		\rho 
		:= \frac{1}{\limsup\limits_{n\to\infty} \sqrt[n]{|a_n|}}
		= \lim_{n\to\infty} \sqrt[n]{\,n!\,}
	\]
	here with $a_n = 1/n!$. Using $n! \ge (n/2)^{\,n/2}$ we get
	\[
		\rho \ge \lim_{n\to\infty} \sqrt{\tfrac{n}{2}} = +\infty.
	\]
	Hence the radius of convergence is infinite.
		\item Clear.
		\item See Example 3.5.15.
		\item See Example 3.5.15.
		\item We have
		\[
			e^{iz} = \sum_{n=0}^{\infty} \frac{(iz)^n}{n!}
			= \sum_{k=0}^{\infty} (-1)^k \frac{z^{2k}}{(2k)!}
				+ i \sum_{k=0}^{\infty} (-1)^k \frac{z^{2k+1}}{(2k+1)!}
			= \cos(z) + i\,\sin(z),
		\]
		where we separated even and odd powers (the exponential series converges absolutely, so terms may be regrouped).
		\item Trivial (consider the terms $z^{2n}$ and $z^{2n+1}$).
		\item Using (5) and (6),
		\[
				frac{1}{2}\bigl(e^{iz}+e^{-iz}\bigr)
				\stackrel{(5)}{=} \tfrac{1}{2}\bigl(\cos(z)+i\sin(z) + \cos(-z)+i\sin(-z)\bigr)
				\stackrel{(6)}{=} \cos z,
		\]
		and similarly
		\[
				frac{1}{2i}\bigl(e^{iz}-e^{-iz}\bigr)
				\stackrel{(5)}{=} \tfrac{1}{2i}\bigl(\cos(z)+i\sin(z) - \cos(-z)-i\sin(-z)\bigr)
				\stackrel{(6)}{=} \sin z.
		\]
		\item Follows from (2) and (5).
		\item Use (cf. Example 4.3.14): there exists $\rho>0$ small enough such that
		\[
			\left|\frac{e^{z}-1}{z}\right| < 2 \qquad \text{for all } |z|<\rho.
		\]
		Let $z_0\in\mathbb C$ and $\varepsilon>0$. Choose $\delta := \min\{\rho, \tfrac{\varepsilon}{2|e^{z_0}|}\}$. Then for $|z-z_0|<\delta$ we have
		\[
			|e^{z} - e^{z_0}| = |e^{z_0}(e^{z-z_0}-1)| \le |e^{z_0}|\,\left|\frac{e^{z-z_0}-1}{z-z_0}\right|\,|z-z_0|
			\le |e^{z_0}|\cdot 2\delta \le \varepsilon.
		\]
		Hence $e^{z}$ is continuous. For $\cos$ use $\cos(z)=\tfrac{1}{2}(e^{iz}+e^{-iz})$ and the fact that the compositions of continuous functions $e^{z}$ and $z\mapsto\pm i z$ are continuous. The same argument yields continuity of $\sin$.
\end{enumerate}

	\hfill$\blacksquare$

	\paragraph{Theorem 4.5.3 (Addition theorems)} For all $z,w\in\mathbb C$ it holds:
	\begin{enumerate}[label=({\arabic*})]
		\item $\cos(z\pm w) = \cos z\,\cos w \mp \sin z\,\sin w$\\
					$\sin(z\pm w) = \sin z\,\cos w \pm \cos z\,\sin w$.
		\item $\sin z - \sin w = 2\cos\bigl(\tfrac{z+w}{2}\bigr)\,\sin\bigl(\tfrac{z-w}{2}\bigr)$\\
					$\cos z - \cos w = -2\sin\bigl(\tfrac{z+w}{2}\bigr)\,\sin\bigl(\tfrac{z-w}{2}\bigr)$.
		\item $\cos^{2}z + \sin^{2}z = 1$.
	\end{enumerate}

	\emph{Proof.}
	\begin{enumerate}[label=({\arabic*})]
		\item Express $\cos,\sin$ via the exponential function:
		\[
			\cos z\,\cos w \mp \sin z\,\sin w
			= \tfrac{1}{4}\Bigl\{(e^{iz}+e^{-iz})(e^{iw}+e^{-iw})
				\mp \tfrac{1}{i}(e^{iz}-e^{-iz})\,\tfrac{1}{i}(e^{iw}-e^{-iw})\Bigr\}
			= \tfrac{1}{4}\bigl\{2e^{i(z+w)} + 2e^{-i(z+w)}\bigr\}.
		\]
		The remaining identities in (1) are obtained analogously.
		\item Set $u = \tfrac{z+w}{2}$ and $v = \tfrac{z-w}{2}$. Using (1) we get
		\[
			\sin(u+v) - \sin(u-v)
			= (\sin u\,\cos v + \cos u\,\sin v) - (\sin u\,\cos v - \cos u\,\sin v)
			= 2\cos u\,\sin v.
		\]
		\item Put $w = z$ in (1). Then
		\[
			\cos^{2}z + \sin^{2}z = \cos(z-z) = \cos(0) = 1.
		\]
	\end{enumerate}
	\hfill $\blacksquare$

	From Theorem 4.5.12 it follows: $e^{z} = e^{z + 2k\pi i}$ for all $k\in\mathbb Z$. If a function $f: \mathbb K \to \mathbb K$ admits a $p\in\mathbb K\setminus\{0\}$ with $f(x+p)=f(x)$ for all $x\in\mathbb K$, we call $f$ $p$-periodic. Hence the exponential function is $2\pi i$-periodic.

	\paragraph{Theorem 4.5.13} For every $a\in\mathbb R$ the map
	\[
		\operatorname{ixp} : [a, a+2\pi[\ \to\ S^{1}
	\]
	is bijective.

	\emph{Proof.}
	\begin{enumerate}[label=({\arabic*})]
		\item Injectivity:
		\[
			\operatorname{ixp}(x) = \operatorname{ixp}(y) \iff e^{ix} = e^{iy}
				\iff e^{i(x-y)} = 1 \iff x-y = 2\pi k,\ k\in\mathbb Z.
		\]
		For all $x,y\in[a,a+2\pi[$ we have $|x-y|<2\pi$, hence $x=y$.
		\item Surjectivity: Let $z\in S^{1}$. By Theorem 4.5.10 there exists $x\in\mathbb R$ with $\operatorname{ixp}(x)=z$. Write $x = a + 2k\pi + r$ with $r\in[0,2\pi[$. Then
		\[
			z = \operatorname{ixp}(x) = e^{i2k\pi}\,e^{i(a+r)} = \operatorname{ixp}(a+r),
		\]
		so $a+r\in[a,a+2\pi[$.
	\end{enumerate}
	\hfill $\blacksquare$

	\paragraph{Theorem 4.5.14}
	\begin{enumerate}[label=({\arabic*})]
		\item $\cos(z+2k\pi)=\cos z$ and $\sin(z+2k\pi)=\sin z$. Thus cosine and sine are $2\pi$-periodic.
		\item $\cos z = 0 \iff z = \tfrac{\pi}{2} + k\pi$, $k\in\mathbb Z$; and $\sin z = 0 \iff z = k\pi$, $k\in\mathbb Z$.
		\item $\sin(x)>0$ for all $x\in]0,\pi[$, and $\sin(x)$ is strictly increasing on $[0,\tfrac{\pi}{2}]$.
		\item $\cos(z+\pi) = -\cos z$ and $\sin(z+\pi) = -\sin z$ for all $z\in\mathbb C$.
		\item $\sin\bigl(\tfrac{\pi}{2} - z\bigr) = \cos z$ and $\cos\bigl(\tfrac{\pi}{2} - z\bigr) = \sin z$ for all $z\in\mathbb C$.
		\item $\cos(\mathbb R) = \sin(\mathbb R) = [-1,1]$.
	\end{enumerate}

	\emph{Proof.} Items (1), (2), (4), (5), (6) are left as exercises. For (3):
	From the proof of Theorem 4.5.10, part (3), we have $\sin(x) > 0$ for all $x\in]0,1[$. From part (2) of that theorem it follows that $\sin(x)\ne 0$ for all $x\in]0,\pi[$. By continuity of $\sin$ (Intermediate Value Theorem, Theorem 4.4.7) we conclude $\sin(x) > 0$ on $]0,\pi[$.

	For monotonicity: arguing as in the proof for $\sin$, from $\cos(0)=1$ and the alternating series estimates one obtains $\cos x > 0$ for all $x\in]0,\tfrac{\pi}{2}[$. If $0\le x<y\le \tfrac{\pi}{2}$, then by Theorem 4.5.3 (addition theorems)
	\[
		\sin y - \sin x = 2\cos\Bigl(\frac{x+y}{2}\Bigr)\,\sin\Bigl(\frac{y-x}{2}\Bigr) > 0,
	\]
	which shows that $\sin$ is strictly increasing on $[0,\tfrac{\pi}{2}]$.
	\hfill $\blacksquare$

	Let $z \in \mathbb C$, $z = x + iy$ with $x,y \in \mathbb R$. Then
	\[
		e^{z} = e^{x+iy} = e^{x}\cdot e^{iy},
	\]
	and the behavior of $\exp$ is determined by the behavior of $\exp : \mathbb R \to \mathbb R$ and $\exp : i\mathbb R \to \mathbb R$.

	We first consider the real exponential function.

	\paragraph{Theorem 4.5.4 R}
	\begin{enumerate}[label=({\arabic*})]
		\item $0 < e^{x} < 1$ for all $x<0$; $\ e^{x} > 1$ for all $x>0$; and $e^{0} = 1$.
		\item $\exp : \mathbb R \to \mathbb R_{>0}$ is strictly monotonically increasing.
		\item $\displaystyle\lim_{x\to\infty} x^{-n} e^{x} = \infty$ for every $n\in\mathbb N$. Moreover, $\displaystyle\lim_{x\to -\infty} e^{x} = 0$.
	\end{enumerate}

	\emph{Proof.}
	\begin{enumerate}[label=({\arabic*})]
		\item We have, for $x>0$,
		\[
			e^{x} = \sum_{n=0}^{\infty} \frac{1}{n!} x^{n} = 1 + \sum_{n=1}^{\infty} \frac{1}{n!} x^{n} > 1,
		\]
		and for $x<0$,
		\[
			e^{x} = e^{-(-x)} = \frac{1}{e^{-x}} \in (0,1).
		\]
		\item For $y>x$ it holds
		\[
			e^{y} = e^{x}\,e^{y-x} > e^{x},
		\]
		because $y-x>0$ and (1) implies $e^{y-x}>1$.
		\item For $n\in\mathbb N$ and $x>0$,
		\[
			x^{-n} e^{x} \ge x^{-n}\,\frac{x^{n+1}}{(n+1)!} = \frac{x}{(n+1)!} \xrightarrow[x\to\infty]{} \infty.
		\]
		Finally,
		\[
			\lim_{x\to-\infty} e^{x} = \lim_{y=-x\to\infty} e^{-y} = \lim_{y\to\infty} \frac{1}{e^{y}} = 0.
		\]
	\end{enumerate}
	\hfill $\blacksquare$

	\paragraph{Theorem 4.5.15} For every $a\in\mathbb R$ the map
	\[
		\exp : \mathbb R + i\,[a, a+2\pi[\ \to\ \mathbb C\setminus\{0\}
	\]
	is bijective.

	\emph{Proof.}
	\begin{enumerate}[label=({\arabic*})]
		\item Injectivity: Let $z = x+iy$ and $w = u+iv$ with $x,y,u,v\in\mathbb R$. Then
		\[
			e^{z} = e^{w}
			\iff e^{x+iy} = e^{u+iv}
			\iff e^{x-u} = e^{i(v-y)}
			\iff e^{x-u} = |e^{i(y-v)}| = 1 \implies x-u = 0
		\]
		and hence $e^{i(y-v)} = 1 \implies y-v = 2\pi k$ with $k\in\mathbb Z$. Since $|y-v|<2\pi$, we get $y=v$.
		\item Surjectivity: Let $z\in\mathbb C\setminus\{0\}$. Then $|z|>0$ and we can write
		\[
			z = |z|\cdot \frac{z}{|z|}, \qquad \frac{z}{|z|} \in S^{1}.
		\]
		By Theorem 4.5.13 there exists a unique $y\in[a,a+2\pi[$ with $e^{iy} = z/|z|$. Moreover, since $\exp: \mathbb R\to \mathbb R_{>0}$ is bijective, there is a unique $x\in\mathbb R$ with $e^{x} = |z|$. Therefore
		\[
			e^{x+iy} = e^{x}\,e^{iy} = |z|\cdot \frac{z}{|z|} = z.
		\]
	\end{enumerate}
	\hfill $\blacksquare$

	\paragraph{Definition 4.5.5} The inverse function of the real exponential function is called the (natural) logarithm and is denoted by $\ln$.
	\[
		\ln := (\exp|_{\mathbb R})^{-1} : ]0,\infty[\ \to\ \mathbb R.
	\]

	\paragraph{Theorem 4.5.6}
	\begin{enumerate}[label=({\arabic*})]
		\item The logarithm is continuous and strictly increasing, with
		\[
			\lim_{x\to\infty} \ln(x) = +\infty,\qquad
			\lim_{x\to 0^{+}} \ln(x) = -\infty.
		\]
		\item For all $x,y \in \mathbb R_{>0}$ it holds
		\[
			\ln(x\cdot y) = \ln(x) + \ln(y),\qquad
			\ln\!\left(\frac{x}{y}\right) = \ln(x) - \ln(y).
		\]
	\end{enumerate}

	\emph{Proof.}
	\begin{enumerate}[label=({\arabic*})]
		\item Follows from Theorem 4.4.8.
		\item Set $a := \ln(x)$ and $b := \ln(y)$. Then
		\[
			x\cdot y = e^{a} \cdot e^{b} = e^{a+b} \implies \ln(x\cdot y) = a+b = \ln(x) + \ln(y).
		\]
		The quotient rule follows analogously: $\ln\bigl(\tfrac{x}{y}\bigr) = \ln(x) - \ln(y)$.
	\end{enumerate}
	\hfill $\blacksquare$

	\paragraph{Extension of the power function}
	Let $a>0$. Up to now $a^{x}$ was defined for $x = \tfrac{p}{q}$ with $p\in\mathbb Z$, $q\in\mathbb N_{\ge 1}$.\\
	Goal: Define $a^{x}$ for all $x\in\mathbb R$ so that the definition agrees with the previous one for $x\in\mathbb Q$.

	Write $a = e^{\ln a}$. Then:
	\begin{enumerate}[label=({\arabic*})]
		\item For $x\in\mathbb N$ it holds
		\[
			a^{x} = (e^{\ln a})^{x} = e^{\underbrace{\ln a + \cdots + \ln a}_{x\text{-times}}} = e^{x\,\ln a}.
		\]
		\item For $x\in\mathbb N$ it holds
		\[
			a^{-x} = \frac{1}{a^{x}} = \frac{1}{e^{x\ln a}} = e^{-x\ln a}.
		\]
		\item For $x\in\mathbb N_{\ge 1}$ set $y := e^{\frac{1}{x}\ln a}$. Then
		\[
			y^{x} = \Bigl(e^{\frac{1}{x}\ln a}\Bigr)^{x} = e^{\ln a} = a,\quad\text{i.e.}\quad a^{1/x} = \sqrt[x]{a} = e^{\frac{1}{x}\ln a}.
		\]
		Let $x = \tfrac{p}{q}$ with $p\in\mathbb Z$, $q\in\mathbb N_{\ge 1}$. Then
		\[
			a^{x} = \Bigl(e^{\frac{1}{q}\ln a}\Bigr)^{p} = e^{\frac{p}{q}\,\ln a}.
		\]
	\end{enumerate}
	In summary, with the “previous” definition we have for all $x\in\mathbb Q$:
	\[
		a^{x} = e^{x\,\ln a}.
	\]

	\paragraph{Definition 4.5.7} For $a>0$ and $x\in\mathbb R$, set
	\[
		a^{x} := e^{x\ln a}.
	\]

	\paragraph{Theorem 4.5.8} Let $a,b\in\mathbb R_{>0}$ and $x,y\in\mathbb R$. Then:
	\begin{enumerate}[label=({\arabic*})]
		\item $a^{x} a^{y} = a^{x+y}$, and $\displaystyle \frac{a^{x}}{a^{y}} = a^{x-y}$.
		\item $a^{x} b^{x} = (ab)^{x}$.
		\item $(a^{x})^{y} = a^{xy}$.
		\item $\ln(a^{x}) = x\,\ln(a)$.
	\end{enumerate}

	\emph{Proof.} Exercise.\hfill $\blacksquare$

	\paragraph{Theorem 4.5.9} For $\alpha\in\mathbb R_{>0}$ it holds:
	\[
		\lim_{x\to\infty} x^{-\alpha}\,\ln(x) = 0,\qquad
		\lim_{x\to 0^{+}} x^{\alpha}\,\ln(x) = 0.
	\]

	\emph{Proof.} As $x\to\infty$ we set $y := \ln x \to \infty$. Then
	\[
		\lim_{x\to\infty} x^{-\alpha}\,\ln(x)
		 = \frac{1}{\alpha} \lim_{y\to\infty} \frac{\alpha y}{e^{\alpha y}}
		 = \frac{1}{\alpha} \lim_{z\to\infty} \frac{z}{e^{z}} = 0
	\]
	by Theorem~4.5.4(3). Similarly,
	\[
		\lim_{x\to 0^{+}} x^{\alpha}\,\ln(x)
		 = \lim_{x\to 0^{+}} \Bigl(\frac{1}{x}\Bigr)^{-\alpha} \ln\!\Bigl(\frac{1}{x}\Bigr)
		 = - \lim_{y\to\infty} y^{-\alpha}\,\ln y = 0.
	\]
	\hfill $\blacksquare$

	Next we investigate the behavior of the exponential function on $i\mathbb R \subset \mathbb C$ and, for brevity, introduce the notation
	\[
		\operatorname{ixp} : \mathbb R \to \mathbb C,\qquad \operatorname{ixp}(x) := e^{ix}.
	\]

	\paragraph{Theorem 4.5.10} $\operatorname{ixp}(\mathbb R) = S^{1} := \{ z\in\mathbb C : |z| = 1\}$.

	\emph{Proof.}
	\begin{enumerate}[label=({\arabic*})]
		\item The inclusion $\operatorname{ixp}(\mathbb R) \subset S^{1}$ follows from
		\[
			|\operatorname{ixp}(x)| = |e^{ix}| = |\cos x + i\sin x| = \sqrt{\cos^{2}x + \sin^{2}x} = 1.
		\]
		\item Let $I := (-1,1)$. We show: the function $\sin : I \to \mathbb R$ is strictly increasing.

		Using the power series of $\sin$ we obtain for $u,v\in I$:
		\[
			\sin(u) - \sin(v)
				= \sum_{n=0}^{\infty} \frac{(-1)^n}{(2n+1)!}\,(u^{2n+1}-v^{2n+1})
				= \sum_{n=0}^{\infty} (u-v)\,\frac{(-1)^n}{(2n+1)!}\,\sum_{j=0}^{2n} u^{j} v^{2n-j}
				=: (u-v)\,(1+R(u,v)). \tag{$*$}
		\]
		For $u,v\in I$ we have the bound
		\[
			|R(u,v)| \le \sum_{n=1}^{\infty} \frac{1}{(2n+1)!(2n+1)}
				\le \sum_{n=1}^{\infty} \frac{1}{(2n)!}
				< \sum_{n=0}^{\infty} \frac{1}{2^{2n+1}}
				= \tfrac{1}{2} \sum_{n=0}^{\infty} \Bigl(\tfrac{1}{4}\Bigr)^{n}
				= \tfrac{1}{2}\,\frac{1}{1-1/4} = \tfrac{2}{3}.
		\]
		Hence $|1+R(u,v)| > 1/3$, and by ($*$) the sign of $\sin(u)-\sin(v)$ equals the sign of $u-v$. Thus $\sin$ is strictly increasing on $I$.

		\item We now show that from (2) it follows that $I$ is mapped injectively by $\operatorname{ixp}$ onto an arc $\tilde y \subset S^{1}$ that passes through $1$ from below to above.

		Since $\sin$ is strictly increasing and continuous on $I$, its image $J := \sin(I)$ is an open interval. Clearly $\operatorname{ixp}(0) = e^{0} = 1 \in \tilde y$. Because $\sin(-x) = -\sin(x)$ and $I$ is symmetric, we have
		\[
			J = (-\alpha,\alpha) \quad \text{for some } \alpha \in (0,1).
		\]
		For $0<x<1$, the series of $\sin$ and $\cos$ are alternating; as in the proof of Theorem 3.4.8 (Leibniz criterion) this yields
		\[
			\cos(x) > 1 - \frac{x^{2}}{2} > 0,
			\qquad 0 < x\Bigl(1 - \frac{x^{2}}{6}\Bigr) < \sin(x) < x, \quad x\in(0,1). \tag{$**$}
		\]
		Hence $\cos x>0$ and $\sin x$ changes sign only at $0$, so $\operatorname{ixp}$ maps $I$ injectively onto an open arc of $S^{1}$ going from below to above through $1$.

		\item We show that $I$ can be shrunk to a symmetric interval $[-\lambda,\lambda]$ such that $\operatorname{ixp}:[-\lambda,\lambda]$ traces exactly a quarter-circle arc with apex at $1$.

		From ($**$) we get
		\[
			\sin\!\Bigl(\tfrac{1}{\sqrt{2}}\Bigr) < \tfrac{1}{\sqrt{2}},
			\qquad
			\sin\!\Bigl(\tfrac{4}{5}\Bigr) > \tfrac{4}{5}\Bigl(1 - \tfrac{16}{6\cdot 25}\Bigr)
				> \tfrac{4}{5}\Bigl(1 - \tfrac{16}{144}\Bigr) = \tfrac{32}{45}
				= \frac{\sqrt{1024}}{\sqrt{2025}} > \sqrt{\tfrac{1}{2}}.
		\]
		By monotonicity and continuity of $\sin$ on $I$ there exists a unique $\lambda\in(1/\sqrt{2},\,4/5)$ with $\sin \lambda = 1/\sqrt{2}$. Thus $\operatorname{ixp}([ -\lambda,\lambda])$ is precisely the quarter-arc of $S^{1}$ centered at $1$.
			\item Continuity of $\sin$ ensures the existence of such a $\lambda\in(1/\sqrt{2},\,4/5)$ (intermediate value theorem) with $\sin\lambda=1/\sqrt{2}$. From ($**$) we also get
			\[
				\cos\lambda = +\sqrt{1-\sin^{2}\!\lambda} = \sqrt{\tfrac{1}{2}},
			\]
			hence
			\[
				e^{\pm i\lambda} = \cos\lambda \pm i\sin\lambda = \tfrac{1}{\sqrt{2}}(1 \pm i).
			\]
			\item We claim: $\operatorname{ixp}$ maps the interval $[0,8\lambda]$ bijectively onto $S^{1}$ (up to the identification $\operatorname{ixp}(0)=\operatorname{ixp}(8\lambda)=1$).

			Preparation:
			\[
				e^{i\,2\lambda} = (e^{i\lambda})^{2} = \Bigl(\tfrac{1}{\sqrt{2}}(1+i)\Bigr)^{2} = \tfrac{1}{2}(1-1+2i) = i,
				\quad e^{i\,4\lambda} = -1,\quad e^{i\,8\lambda} = 1.
			\]
			We already showed that $\operatorname{ixp} : [-\lambda,\lambda] \to y$ (the quarter arc of $S^{1}$ with apex $1$) is bijective. For $t\in[\lambda,3\lambda]$ we compute
			\[
				\operatorname{ixp}(t+2\lambda) = e^{i(t+2\lambda)} = e^{it}\,e^{i\,2\lambda} = i\,\operatorname{ixp}(t) =: T(\operatorname{ixp}(t)),
			\]
			where $T(z):=iz$ is the rotation of the complex plane by $90^\circ$ counterclockwise about the origin. Hence
			\[
				\operatorname{ixp}([\alpha+2\lambda,\beta+2\lambda]) = T\bigl(\operatorname{ixp}([\alpha,\beta])\bigr).
			\]
			Iterating this $90^\circ$ rotation shows that the intervals
			\[
				[-\lambda,\lambda],\quad [\lambda,3\lambda],\quad [3\lambda,5\lambda],\quad [5\lambda,7\lambda],\quad [7\lambda,9\lambda]
			\]
			are mapped bijectively onto the quarter arcs of $S^{1}$ with apices $1,i,-1,-i,1$, respectively. In particular, $\operatorname{ixp}$ maps $[0,8\lambda]$ bijectively onto $S^{1}$ (with the exception that $0$ and $8\lambda$ both map to $1$).
	\end{enumerate}

	\paragraph{Theorem 4.5.11} The set
	\[
		M := \{\, x \in \mathbb R_{>0} : \operatorname{ixp}(x) = 1 \,\}
	\]
	has a positive minimum. We define
	\[
		\pi := \tfrac{1}{2}\,\min M.
	\]

	\emph{Proof.} In the proof of Theorem 4.5.10 (part (5)) we showed that the map
	\[
		\operatorname{ixp} : ]0,8\lambda] \to S^{1}
	\]
	is bijective. Hence there exists an inverse
	\[
		\phi := (\operatorname{ixp}|_{]0,8\lambda]})^{-1} : S^{1} \to ]0,8\lambda].
	\]
	It follows that $x := \phi(1)$ lies in $]0,8\lambda]$ and is uniquely determined. Consequently there is no smaller $y\in\mathbb R_{>0}$ with $\operatorname{ixp}(y)=1$. We denote this value by
	\[
		x = 2\pi.
	\]
	Since $1 = e^{i2\pi} = \operatorname{ixp}(8\lambda)$ (by part (5) of the proof of Theorem 4.5.10) and $8\lambda \in ]0,8\lambda]$, we have $\phi(1) = 8\lambda$ and therefore
	\[
		\pi = 4\lambda,\qquad \lambda = \tfrac{\pi}{4}.
	\]
	From this we obtain the familiar values
	\[
		e^{i\pi/2} = i,\qquad e^{i\pi} = -1,\qquad e^{2i\pi} = 1.
	\]
	\hfill $\blacksquare$

	\paragraph{Theorem 4.5.12}
	\begin{enumerate}[label=({\arabic*})]
		\item $e^{z} = 1$ if and only if $z \in 2\pi i\,\mathbb Z$, i.e., $z = 2\pi i\,k$ for $k\in\mathbb Z$.
		\item $e^{z} = -1$ if and only if $z \in \pi i + 2\pi i\,\mathbb Z$, i.e., $z = (2k+1)\pi i$ for $k\in\mathbb Z$.
	\end{enumerate}

	\emph{Proof.}
	\begin{enumerate}[label=({\arabic*})]
		\item “$\Leftarrow$”: $e^{2\pi i k} = (e^{2\pi i})^{k} = 1^{k} = 1$.\\
					“$\Rightarrow$”: Let $z = x + iy$, $x,y\in\mathbb R$, and suppose $e^{z} = 1$. Then
					\[
						1 = |e^{z}| = |e^{x+iy}| = |e^{x}|\,|e^{iy}| = e^{x} \implies x = 0.
					\]
					Write $y = 2k\pi + r$ with $r\in[0,2\pi[$. Then
					\[
						1 = e^{i2\pi k} e^{ir} \implies e^{ir} = 1 \implies r = 0.
					\]
					Hence $y = 2k\pi$ and $z = 2\pi i\,k$.
		\item “$\Leftarrow$”: $e^{(2k+1)\pi i} = e^{\pi i} e^{2k\pi i} = (-1)\cdot 1 = -1$.\\
					“$\Rightarrow$”: If $-1 = e^{z}$, then $1 = e^{2z}$, and by (1) we get $2z = 2k\pi i$, hence $z = k\pi i$. In particular, when $k$ is even, $z = k\pi i$ does not solve $e^{z} = -1$; therefore $k$ must be odd and $z = (2k+1)\pi i$.
	\end{enumerate}
	\hfill $\blacksquare$

	From this we obtain the polar-coordinate representation of complex numbers.

	\paragraph{Theorem 4.5.16} For every $z\in\mathbb C\setminus\{0\}$ there exists a unique $\alpha\in[0,2\pi[$ such that
	\[
		z = |z|\,e^{i\alpha}.
	\]
	We call $\alpha$ the argument of $z$; write $\alpha := \arg(z)$.

	\paragraph{Theorem 4.5.17} For every $a\in\mathbb C\setminus\{0\}$ the equation $z^{n}=a$ has exactly $n$ distinct solutions:
	\[
		z_k = |a|^{1/n}\,\exp\!\left(\,\frac{i(\arg(a)+2\pi k)}{n}\,\right),\qquad k=0,1,\dots,n-1.
	\]

	\emph{Proof.} Let $z$ be a solution of $z^{n}=a$ (clearly $z\ne 0$). Use polar coordinates
	\[
		a = |a|e^{i\arg(a)},\qquad z = r\,e^{i\phi},\quad r\in\mathbb R_{>0},\ \phi\in[0,2\pi[.
	\]
	Then
	\[
		z^{n} = r^{n} e^{in\phi} = |a|e^{i\arg(a)}
		\iff\begin{cases}
			r^{n}=|a|,\\
			e^{in\phi} = e^{i\arg(a)}.
		\end{cases}
	\]
	The first condition gives $r = |a|^{1/n}$. The second says $e^{i(n\phi-\arg(a))}=1$, hence $n\phi-\arg(a)=2\pi k$ and
	\[
		\phi = \tfrac{1}{n}\bigl(\arg(a)+2\pi k\bigr),\qquad k\in\mathbb Z.
	\]
	Distinct solutions arise exactly for $k=0,1,\dots,n-1$.
	\hfill $\blacksquare$

\end{document}
