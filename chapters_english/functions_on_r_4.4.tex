\documentclass[12pt]{article}
% -------------------- Packages --------------------
\usepackage[utf8]{inputenc}
\usepackage[T1]{fontenc}
\usepackage[english]{babel}
\usepackage{amsmath,amssymb,amsthm}
\usepackage{mathtools}
\usepackage{geometry}
\usepackage{enumitem}
\usepackage{xcolor}
\usepackage[unicode]{hyperref}
\hypersetup{unicode=true,hypertexnames=false}
% Avoid enumitem negative labelwidth warnings by setting a safe leftmargin
\setlist[enumerate]{leftmargin=*,labelsep=0.5em}
% Soften line-breaking to avoid overfull hboxes in long headings/paragraphs
\emergencystretch=2em
\usepackage{microtype}

\geometry{margin=2.5cm}

% -------------------- Metadata --------------------
\iffalse
		itle{Analysis I \& II \newline\small (English Translation Draft)}
\fi
% (replaced malformed title line below)
\csname title\endcsname{Analysis I \& II \newline\small (English Translation Draft)}
\author{Based on original German script (private use)}
\date{\today}

% -------------------- Macros --------------------
\newcommand{\N}{\mathbb{N}}
\newcommand{\Z}{\mathbb{Z}}
\newcommand{\Q}{\mathbb{Q}}
\newcommand{\R}{\mathbb{R}}
\newcommand{\C}{\mathbb{C}}
\newcommand{\K}{\mathbb{K}} % used to denote either R or C
\newcommand{\eps}{\varepsilon}
\newcommand{\dd}{\,\mathrm{d}}
\newcommand{\todo}[1]{\textcolor{red}{[TODO: #1]}}
% Some earlier notes used \odo by mistake; keep it compiling by aliasing:
\newcommand{\odo}[1]{\todo{#1}}

% -------------------- Manual Numbering Helpers --------------------
% Use these when you want to preserve the original numbering from the German
% script. They deliberately avoid coupling to LaTeX's counters.
% Pattern:
%   \NumberedDefinition{<chapter.section.item>}{<Title>}{<Body>}
% If you later want cross-references, add an explicit \label inside #3 and
% reference it normally with \ref. The visual number won't auto-sync if you
% renumber—update the macro argument manually.
% Translator note: Keep using numbered theorems/remarks/examples exactly as in
% the German script (e.g., 4.1.1, 4.1.2, ...). Prefer these helpers where you
% need to mirror source numbering during translation.
\newcommand{\NumberedDefinition}[3]{% #1 number, #2 title, #3 body
\paragraph*{Definition #1 ( #2 ).} #3\par}
\newcommand{\NumberedTheorem}[3]{% #1 number, #2 title, #3 body
\paragraph*{Theorem #1 ( #2 ).} #3\par}
\newcommand{\NumberedProposition}[3]{% #1 number, #2 title, #3 body
\paragraph*{Proposition #1 ( #2 ).} #3\par}
\newcommand{\NumberedLemma}[3]{% #1 number, #2 title, #3 body
\paragraph*{Lemma #1 ( #2 ).} #3\par}
\newcommand{\NumberedCorollary}[3]{% #1 number, #2 title, #3 body
\paragraph*{Corollary #1 ( #2 ).} #3\par}
\newcommand{\NumberedRemark}[3]{% #1 number, #2 title, #3 body
\paragraph*{Remark #1 ( #2 ).} #3\par}
\newcommand{\NumberedExample}[3]{% #1 number, #2 title, #3 body
\paragraph*{Example #1 ( #2 ).} #3\par}
\begin{document}


\subsection{Functions on \texorpdfstring{$\R$}{R} (Funktionen in \texorpdfstring{$\R$}{R})}

In the following, let $I\subset\R$ always be a nonempty interval.

\NumberedDefinition{4.4.1}{Monotonicity}{A function $f:I\to\R$ is called
\begin{enumerate}[label={(\arabic*)}, leftmargin=*]
	\item monotone increasing,
	\item monotone decreasing,
	\item strictly increasing,
	\item strictly decreasing,
\end{enumerate}
if for all $x,y\in I$ with $x<y$ we have, respectively,
\begin{enumerate}[label={(\arabic*)}, leftmargin=*]
	\item $f(x)\le f(y)$,
	\item $f(x)\ge f(y)$,
	\item $f(x)< f(y)$,
	\item $f(x)> f(y)$.
\end{enumerate}
A function that is either (strictly) decreasing or (strictly) increasing is called (strictly) monotone.}

\NumberedExample{4.4.2}{}{{\leavevmode}
\begin{enumerate}[label={(\arabic*)}, leftmargin=*]
	\item $f(x):=x^{2}$ is strictly decreasing on $(-\infty,0]$ and strictly increasing on $[0,\infty)$. If $0\in \mathring I$ (i.e. $0$ is an interior point of $I$), then $f$ is not monotone on $I$.
	\item The floor function $f(x):=\lfloor x\rfloor$ is monotone increasing.
	\item The function
		\[
			f(x):=\begin{cases}
				1, & x\in\Q,\\
				0, & x\in\R\setminus\Q,
			\end{cases}
		\]
		is not monotone.
\end{enumerate}}

\NumberedTheorem{4.4.3}{One-sided limits of monotone functions}{Let $f:I\to\R$ be monotone and set $\alpha:=\inf I$, $\beta:=\sup I$ (possibly infinite). Then the one-sided limits
\[
	f(\alpha+0):=\lim_{x\to\alpha+0} f(x)=\lim_{\substack{x\to\alpha\\ x>\alpha}} f(x),
	\qquad
	f(\beta-0):=\lim_{x\to\beta-0} f(x)=\lim_{\substack{x\to\beta\\ x<\beta}} f(x)
\]
exist in $\R$, and we have
\[
	f(\alpha+0)=
	\begin{cases}
		\inf\{f(x): x\in I\setminus\{\alpha\}\}, & \text{if $f$ is increasing},\\[0.3em]
		\sup\{f(x): x\in I\setminus\{\alpha\}\}, & \text{if $f$ is decreasing},
	\end{cases}
\]
and
\[
	f(\beta-0)=
	\begin{cases}
		\sup\{f(x): x\in I\setminus\{\beta\}\}, & \text{if $f$ is increasing},\\[0.3em]
		\inf\{f(x): x\in I\setminus\{\beta\}\}, & \text{if $f$ is decreasing}.
	\end{cases}
\]
}
\paragraph{Proof.} Consider the increasing case and set
\[
	b:=\sup\{f(x): x\in I\setminus\{\beta\}\}\in\overline{\R}.
\]
Let $(x_n)\subset [\alpha,\beta)$ with $x_n\to\beta$ and $x_n<\beta$ for all $n\in\N$. Since $b$ is the supremum, for every $\eta<b$ there exists $\xi\in[\alpha,\beta)$ with $f(\xi)\ge\eta$ and an index $n_{\xi}$ such that $x_n\ge\xi$ for all $n\ge n_{\xi}$. Because $f$ is increasing,
\[
	\eta\le f(\xi)\le f(x_n)\le b \qquad (n\ge n_{\xi}).
\]
As $\eta<b$ was arbitrary, it follows that $\lim_{n\to\infty} f(x_n)=b$. Hence $f(\beta-0)$ exists and equals $b$. The remaining statements are shown analogously. \qed

\NumberedDefinition{4.4.4}{Jump discontinuity}{Let $D\subset\R$, $D\ne\emptyset$, and let $Y$ be a normed vector space. A function $f:D\to Y$ is said to have a jump at a point $x_0\in\R$ with
\[
	x_0\in \overline{D\cap(-\infty,x_0)}\ \cap\ \overline{D\cap(x_0,\infty)}
\]
if the one-sided limits
\[
	\lim_{x\to x_0\pm 0} f(x) = f(x_0\pm 0)
\]
exist and are different.}

\NumberedExample{4.4.5}{}{For the function $f(x):=\lfloor x\rfloor$, every $k\in\Z$ is a jump point, since
\[
	f(k+0)-f(k-0)=1.
\]
}

Theorem~4.4.3 shows that for every monotone function $f:I\to\R$ and every $x_0$ in the interior of $I$, the one-sided limits $f(x_0\pm 0)$ exist. Example~4.4.5 shows that there are functions with countably many jump discontinuities. The next theorem shows that monotone functions can have at most countably many jumps.

\NumberedTheorem{4.4.6}{}{Let $f:I\to\R$ be monotone. Then $f$ is continuous except at at most countably many jump points.}
\paragraph{Proof.} Assume $f$ is increasing. Define the set of jump points
\[
	M := \{\, x\in \mathring I : f(x+0)\ne f(x-0)\,\}.
\]
We claim that $M$ is countable. Idea: construct an injective map $q:M\to\Q$. For $x\in M$ we have $f(x-0)<f(x+0)$, so by density of $\Q$ there exists $z\in\Q$ with
\[
	z\in\bigl( f(x-0),\, f(x+0) \bigr).
\]
Define $q(x):=z$. We show $q$ is injective. If $x,y\in M$ with $x<y$, then by monotonicity
\[
	q(x) < f(x+0) \le f(y-0) < q(y).
\]
Hence $q(x)\ne q(y)$ whenever $x\ne y$, so $q$ is injective. Since $\Q$ is countable, $M$ is countable. \qed

\NumberedTheorem{4.4.7}{Intermediate Value Theorem}{Let $f:[a,b]\to\R$ be continuous with $f(a)<0<f(b)$. Then $f$ has a zero $\xi\in(a,b)$, i.e. $f(\xi)=0$.}
\paragraph{Proof.} (Interval bisection.) Construct a sequence of nested intervals $[a_k,b_k]$ recursively by
\[
	[a_0,b_0]:=[a,b], \qquad
	[a_{k+1},b_{k+1}]:=
	\begin{cases}
		\bigl[a_k,\, \tfrac{a_k+b_k}{2}\bigr], & \text{if } f\bigl( \tfrac{a_k+b_k}{2} \bigr) > 0,\\[0.4em]
		\bigl[ \tfrac{a_k+b_k}{2},\, b_k \bigr], & \text{if } f\bigl( \tfrac{a_k+b_k}{2} \bigr) \le 0,
	\end{cases}
	\qquad k\ge 0.
\]
Then: (1) $(a_k)$ is monotonically increasing and bounded above by $b$; (2) $(b_k)$ is monotonically decreasing and bounded below by $a$; (3) for all $k\in\N$, $b_k-a_k=2^{-k}(b-a)$. By the nested-interval argument (cf. Bolzano–Weierstrass, Theorem~3.2.5), there exists a common limit $\xi\in[a,b]$. By construction, for all $k\ge0$ we have
\[
	f(a_k) \le 0 \le f(b_k).
\]
By continuity of $f$,
\[
	f(\xi) = \lim_{k\to\infty} f(a_k) \le 0 \le \lim_{k\to\infty} f(b_k) = f(\xi),
\]
hence $f(\xi)=0$ and $\xi\in(a,b)$ since $f(a)<0<f(b)$. \qed

We distinguish three types of intervals: closed $I=[a,b]$, open $I=(a,b)$, and half-open $I=(a,b]$ or $I=[a,b)$.

\NumberedTheorem{4.4.8}{}{Let $f:I\to\R$ be continuous and strictly monotone increasing (respectively, strictly monotone decreasing). Then $J:=f(I)$ is an interval of the same type as $I$, and $f$ maps $I$ bijectively onto $J$. The inverse map $f^{-1}:J\to I$ is continuous and strictly monotone increasing (respectively, strictly monotone decreasing).}
\paragraph{Proof.} We consider the strictly increasing case; the decreasing case is analogous. Set $\alpha:=\inf I$, $\beta:=\sup I$ (in $\overline{\R}$). By Theorem~4.4.3, the one-sided limits
\[
	a:=f(\alpha+0), \qquad b:=f(\beta-0)
\]
exist (in $\overline{\R}$). We first show
\[
	(a,b)\subset J:=f(I)\subset [a,b].
\]
The right inclusion is clear by the definition of $a$ and $b$. For the left inclusion, let $a<\eta<b$. Because $b$ is the supremum of $J$, there exist $t_2\in (\alpha,\beta)$ with $f(t_2)>\eta$ and $t_1\in(\alpha,\beta)$ with $f(t_1)<\eta$. Since $f$ is strictly increasing, $t_1<t_2$ and hence $f(t_1)<\eta<f(t_2)$. By continuity (Intermediate Value Theorem, Theorem~4.4.7) there exists $\tau\in[t_1,t_2]\subset(\alpha,\beta)$ with $f(\tau)=\eta$. Thus $\eta\in J$, so $(a,b)\subset J$. Hence $J$ is an interval.

Strict monotonicity implies that $J$ contains its infimum $a$ if and only if $I$ contains $\alpha$, and $J$ contains its supremum $b$ if and only if $I$ contains $\beta$. Therefore $J$ has the same endpoint type as $I$.

For bijectivity: strict monotonicity yields injectivity of $f$ on $I$, and surjectivity holds by definition of $J=f(I)$. Hence $f:I\to J$ is bijective.

For the inverse $g:=f^{-1}:J\to I$ we first show strict monotonicity: let $y_1<y_2$ in $J$ and assume towards a contradiction that $x_1:=g(y_1)\ge g(y_2)=:x_2$. Then
\[
	y_1=f(g(y_1))\ge f(g(y_2))=y_2,
\]
contradiction. Thus $g$ is strictly increasing.

It remains to prove continuity of $g$. Fix $y_0\in J$. Consider a sequence $(y_n)\subset J$ with $y_n\nearrow y_0$; set $x_n:=g(y_n)$. Then $(x_n)$ is increasing and bounded above by $g(y_0)$, hence $x_n\to\bar x\le g(y_0)$. If $\bar x< x_0:=g(y_0)$, then by continuity of $f$,
\[
	y_0 = \lim_{n\to\infty} y_n = \lim_{n\to\infty} f(x_n) = f(\bar x) < f(x_0)=y_0,
\]
which is impossible. Thus $g(y_0-0)=g(y_0)$. An analogous argument with $y_n\searrow y_0$ shows $g(y_0+0)=g(y_0)$. Hence $g$ is continuous at $y_0$. \qed

\NumberedExample{4.4.9}{}{Let $n\ge 2$ and define $f: \R_{\ge0}\to\R_{\ge0}$ by $f(x):=x^{n}$. We show $f$ is strictly increasing. For $0\le x<y$,
\[
	f(y)-f(x)=y^{n}-x^{n}=y^{n}\Bigl(1-\bigl(\tfrac{x}{y}\bigr)^{n}\Bigr)
	= y^{n}\Bigl(1-\tfrac{x}{y}\Bigr)\sum_{j=0}^{n-1} \Bigl(\tfrac{x}{y}\Bigr)^{\,n-1-j} > 0,
\]
since $y>0$, $0\le x/y<1$, and the sum is positive. Moreover, because $0\le f(x)=x^{n}\le x$ for $x<1$, we have $f(0+0)=0$, and since $f(x)=x^{n}\ge x$ for $x>1$, we have $f(+\infty-0)=+\infty$.

Thus $f:\R_{\ge0}\to\R_{\ge0}$ is bijective, continuous, and strictly increasing. By Theorem~4.4.8 it has a continuous, strictly increasing inverse $g: \R_{\ge0}\to\R_{\ge0}$. We write
\[
	\sqrt[n]{\,x\,} := g(x) := f^{-1}(x).
\]
}
\end{document}
