\documentclass[12pt,a4paper]{article}

% -- Packages --
\usepackage{amsmath, amssymb, amsthm} % Math symbols & theorems
\usepackage{enumitem} % Better lists
\usepackage{geometry} % Page layout
\usepackage{fancyhdr} % Header/footer
\usepackage{tikz}     % Diagrams
\usepackage{hyperref} % Clickable references
\usepackage{mathrsfs} % Fancy math fonts

% -- Page Setup --
\geometry{margin=1in}
\setlength{\parskip}{0.5em}
\setlength{\parindent}{0pt}
\pagestyle{fancy}
\fancyhf{}
% -- Header/Footer --
\lhead{MAT121 -- Analysis I}
\chead{Exercise sheet 1}
\rhead{Stefan du Toit}
\rfoot{\thepage}

% -- Theorem Environments --
\newtheorem{theorem}{Theorem}[section]
\newtheorem{lemma}[theorem]{Lemma}
\newtheorem{proposition}[theorem]{Proposition}
\newtheorem{corollary}[theorem]{Corollary}

\theoremstyle{definition}
\newtheorem{definition}[theorem]{Definition}
\newtheorem{example}[theorem]{Example}
\newtheorem{exercise}{Exercise}[section]

\theoremstyle{remark}
\newtheorem*{remark}{Remark}

% -- Custom Environments --
\newenvironment{solution}{\begin{proof}[Solution]}{\end{proof}}



% ==========================================
\begin{document}

\section*{1}
\subsection*{(b)}


Given
\begin{itemize}
    \item initial value: $a_0=1$,
    \item recurrence (for all $n\ge 0$):
    $$a_{n+1}=a_n+a_{n-1}+\cdots+a_1+a_0+1=\sum_{k=0}^{n} a_k + 1,$$
\end{itemize}
find a closed formula for $a_n$.

This is a ``cumulative'' recurrence: the next term equals the sum of all previous terms plus a constant.

\begin{solution}
\textbf{How to see the trick:}

Define partial sums
$$S_n:=\sum_{k=0}^n a_k.$$
Then the recurrence is simply
$$a_{n+1}=S_n+1.$$
But also
$$a_n=S_{n-1}+1.$$
If you subtract these two equations, the constants cancel and you expose the last added term:
$$a_{n+1}-a_n=(S_n+1)-(S_{n-1}+1)=S_n-S_{n-1}=a_n.$$
This is the key ``difference of consecutive cumulative sums = last term'' idea.

So you get a first-order linear recurrence with constant coefficient:
$$a_{n+1}=2a_n \quad (n\ge 0).$$

With $a_0=1$, the solution is immediate:
$$a_n=2^n.$$

\textbf{Why this works (intuition):}
\begin{itemize}
    \item $a_{n+1}$ is ``all the previous stuff'' plus $1$.
    \item $a_n$ is ``all the stuff before that'' plus $1$.
    \item Their difference is exactly the most recent piece, which is $a_n$ itself.
    \item So each step doubles the sequence value.
\end{itemize}

\textbf{Verification (first terms):}
\begin{align}
a_0 &= 1,\\
a_1 &= a_0+1=2,\\
a_2 &= a_1+a_0+1=2+1+1=4,\\
a_3 &= 4+2+1+1=8,
\end{align}
matching $2^n$.

\textbf{Alternative route via partial sums:}

From $a_{n+1}=S_n+1$,
$$S_{n+1}=S_n+a_{n+1}=2S_n+1,\qquad S_0=a_0=1.$$
Solve $S_{n+1}=2S_n+1$:
\begin{itemize}
    \item homogeneous part $C\cdot 2^n$, particular solution $-1$,
    \item $S_n=2^{n+1}-1$ (from $S_0=1$),
\end{itemize}
so $a_{n+1}=S_n+1=2^{n+1}$ and hence $a_n=2^n$.

\textbf{Generalization and consistency check:}

If the problem were $a_{n+1}=\sum_{k=0}^n a_k + c$ with $a_0=\alpha$, the same subtraction gives $a_{n+1}=2a_n$ (the constant cancels), so $a_n=2^n\alpha$. For this to be consistent with $a_1=a_0+c$, you must have $c=\alpha$ (as in our problem: $c=1$, $\alpha=1$).

\textbf{Takeaway pattern:}
\begin{itemize}
    \item For recurrences of the form ``next = sum of previous + constant,'' try subtracting consecutive equations; you'll often reduce to a simple first-order linear recurrence.
    \item Once you have $a_{n+1}=r a_n$ and an initial value, the solution is $a_n=a_0\, r^n$. Here, $r=2$ and $a_0=1$, so $a_n=2^n$.
\end{itemize}
\end{solution}



\section*{3}
\subsection*{(a)i}
\textbf{Claim.} In an ordered field $K$, if $x<y$ and $a>0$, then $ax<ay$.

\textbf{Key order axioms used (standard for an ordered field):}
\begin{enumerate}
    \item Trichotomy / transitivity (not needed explicitly here).
    \item Translation invariance: $x<y \implies x+z < y+z$ for all $z$.
    \item Positivity is closed under multiplication: if $0<b$ and $0<c$, then $0<bc$.
\end{enumerate}

\begin{proof}
Given $x<y$. Subtract $x$ from both sides (translation invariance with $z=-x$):
$$x<y \;\Longrightarrow\; 0< y - x.$$

Given also $a>0$. Using closure of the positive set under multiplication:
$$0 < a \quad\text{and}\quad 0< y-x \;\Longrightarrow\; 0 < a(y-x).$$

Expand:
$$0 < a(y-x) = ay - ax.$$

Thus $ay - ax > 0$, which by definition of the order means $ax < ay$.

$$\boxed{ax < ay}$$

That's all. (The essential move is turning $x<y$ into a positive difference and then using that multiplying by a positive element preserves ``greater than zero.'') Let me know when you want part (a) ii).
\end{proof}

\subsection*{(a)ii}

\textbf{Claim.} In an ordered field \(K\), if  
\[
0 \le x < y, \qquad 0 < x' < y',
\]
then
\[
0 \le x x' \le y y'.
\]

\begin{proof}
\textbf{Step 1: Non‑negativity of the product.}  
We have \(0 \le x\) and \(0 < x'\). The product of two non‑negative elements is non‑negative, so
\[
0 \le x x'.
\]

\textbf{Step 2: First comparison \(x x' < y x'\).}  
From \(x<y\) and \(x'>0\), part (i) (order preserved by multiplying with a positive element) gives
\[
x x' < y x'.
\]

\textbf{Step 3: Second comparison \(y x' < y y'\).}  
From \(x'<y'\) and \(y>0\) (since \(0 \le x < y\) implies \(y>0\)), again by part (i),
\[
y x' < y y'.
\]

\textbf{Step 4: Chain the inequalities.}  
Combining:
\[
0 \le x x' < y x' < y y'.
\]
Thus in particular \(0 \le x x' \le y y'\).

(Strict inequalities appear internally; the statement uses \(\le\) on the right just to allow for the general pattern—here equality on the right cannot occur because \(x'<y'\).)

\[
\boxed{0 \le x x' \le y y'}
\]
\end{proof}

\subsection*{(b)}

\textbf{Goal.} Show: If \(K\) is an ordered field, then there exists a polynomial with coefficients in \(K\) that has no root in \(K\). Hence \(K\) is not algebraically closed.

\textbf{Key property of ordered fields.} In any ordered field \(K\):

\begin{enumerate}
    \item \(1 > 0\).
    \item If \(x \neq 0\), then \(x^2 > 0\). (Because \(x > 0 \implies x \cdot x > 0\); if \(x < 0\), then \(-x > 0\) and \((-x)^2 = x^2 > 0\).)
    \item If \(x^2 \ge 0\) for all \(x\), then \(x^2 + 1 > 0\) for all \(x\) (since \(1 > 0\)).
\end{enumerate}

Thus no square is negative, and in particular \(-1\) cannot be a square.

\textbf{The obstructing polynomial.} Consider
\[
p(t) = t^2 + 1 \in K[t].
\]

Suppose for contradiction \(p\) has a root \(a \in K\). Then
\[
a^2 + 1 = 0 \quad\Longrightarrow\quad a^2 = -1.
\]
But from the order properties above, \(a^2 \ge 0\) for all \(a\), while \(-1 < 0\). Impossible.

Therefore \(t^2 + 1\) has no root in \(K\). So \(K\) fails the definition of being algebraically closed (“every nonconstant polynomial has a root”).

\textbf{Conclusion.} Every ordered field admits a polynomial (already degree \(2\)) without a root, namely \(t^2 + 1\). Hence no ordered field is algebraically closed.

\[
\boxed{\text{If } K \text{ is ordered, then } K \text{ is not algebraically closed.}}
\]

(For orientation: \(\mathbb{R}\) is ordered but not algebraically closed; its algebraic closure is \(\mathbb{C}\), which cannot be ordered consistently with the field operations because \(i^2 + 1 = 0\) would force \(i^2 = -1 < 0\), contradicting the “squares are nonnegative” property.)

%%%%%%%%%%%%%

\section*{3}
\subsection*{(a)}

\textbf{Goal.} Show: If \(A,B\subseteq \mathbb{R}\) with \(A\subseteq B\) and both \(\sup A,\sup B\) exist (i.e. \(A\) is nonempty \& bounded above; \(B\) bounded above), then
\[
\sup A \;\le\; \sup B.
\]



\subsection*{Definitions recalled}

- An \emph{upper bound} of a set \(S\subseteq \mathbb{R}\) is a number \(U\) with \(s\le U\) for all \(s\in S\).
- The \emph{supremum} \(\sup S\) is the least upper bound: (i) it is an upper bound, and (ii) any smaller number fails to be an upper bound.



\subsection*{Intuitive picture}

Because \(A\) sits inside \(B\), anything that bounds \(B\) from above also bounds \(A\). The \emph{least} upper bound of \(A\) cannot exceed the least upper bound of the larger set \(B\).



\subsection*{Formal proof}

\begin{enumerate}
    \item Since \(B\) is bounded above, let \(M=\sup B\). By definition \(b\le M\) for all \(b\in B\).  
       Because \(A\subseteq B\), every \(a\in A\) also satisfies \(a\le M\). Thus \(M\) is an upper bound of \(A\).

    \item Let \(m=\sup A\). Because \(m\) is the \emph{least} upper bound of \(A\) and \(M\) is an upper bound of \(A\), we must have \(m\le M\); i.e.
       \[
       \sup A \le \sup B.
       \]
\end{enumerate}

That is the entire argument.



\subsection*{(Optional) Contradiction style (epsilon form)}

Assume for contradiction \(\sup A > \sup B\). Set \(\varepsilon = \frac{\sup A - \sup B}{2} >0\). Then \(\sup B + \varepsilon < \sup A\). By property of supremum of \(A\), there exists \(a\in A\) with
\[
a > \sup A - \varepsilon = \sup B + (\sup A - \sup B) - \varepsilon = \sup B + \varepsilon.
\]
Hence \(a > \sup B\). But \(a\in A\subseteq B\), so \(a\in B\) contradicts that \(\sup B\) is an upper bound of \(B\). Therefore \(\sup A \le \sup B\).



\subsection*{Edge cases / remarks}

\begin{itemize}
    \item If \(A=\varnothing\): standard real analysis usually \emph{does not} define \(\sup\varnothing\). Some contexts set \(\sup\varnothing=-\infty\); then the inequality is trivially true.
    \item If \(B\) is not bounded above, \(\sup B\) does not exist (or is \(+\infty\) in extended reals). In the extended reals, you’d write \(\sup B=+\infty\) and the inequality still holds.
    \item The result generalizes: in any partially ordered set where suprema exist, inclusion implies monotonicity of the supremum operator.
\end{itemize}



\[
\boxed{\ A\subseteq B \text{ and } \sup A,\sup B \text{ exist } \Longrightarrow \sup A \le \sup B.\ }
\]

\subsection*{(b)}

\textbf{Claim.} For a nonempty \(A\subseteq \mathbb{R}\),
\[
\inf(-A) = -\sup(A), \qquad \sup(-A)= -\inf(A),
\]
where \(-A=\{-a: a\in A\}\). In particular the answer to (b) is
\[
\boxed{\inf(-A)= -\sup(A)}.
\]

\begin{proof}
Let \(M=\sup(A)\). Then \(a\le M\) for all \(a\in A\). Multiplying both sides by \(-1\) reverses the inequality: \(-a \ge -M\) for all \(a\in A\). Thus, \(-M\) is a lower bound of \(-A\), so \(\inf(-A)\le -M\).

To show the reverse inequality, let \(\varepsilon>0\). Since \(M=\sup(A)\), there exists \(a_\varepsilon\in A\) with \(M-\varepsilon < a_\varepsilon \le M\). Negating gives
\[
-M < -a_\varepsilon < -M + \varepsilon.
\]
So for any \(\varepsilon>0\), there is an element of \(-A\) greater than \(-M\) but less than \(-M+\varepsilon\). Thus, no number strictly less than \(-M\) can be a lower bound of \(-A\), so \(-M\) is the greatest lower bound:
\[
\inf(-A) = -M = -\sup(A).
\]
\end{proof}

\textbf{Remarks.}
\begin{itemize}
    \item If \(A\) is unbounded above, then \(\sup(A)=+\infty\) (in the extended reals), so \(\inf(-A) = -\infty\), which matches the fact that \(-A\) is unbounded below.
    \item The dual identity \(\sup(-A) = -\inf(A)\) follows similarly.
\end{itemize}


\section*{5}
\subsection*{(a)}
\textbf{Set:}
\[
X = [0,1) = \{x\in\mathbb{R}\mid 0\le x < 1\}.
\]

\textbf{1. Boundedness}

\begin{itemize}
    \item \emph{Lower bound:} $0$ (and in fact any $L\le 0$ also works). For every $x\in X$, $x\ge 0$, so $X$ is bounded below.
    \item \emph{Upper bound:} $1$ (and any $U\ge 1$ also works). For every $x\in X$, $x<1$, so $X$ is bounded above.
\end{itemize}
Thus $X$ is bounded both below and above.

\textbf{2. Infimum}

\emph{Claim:} $\inf X = 0$.

\begin{itemize}
    \item (\emph{Lower bound}) As noted, $0\le x$ for all $x\in X$, so $0$ is a lower bound.
    \item (\emph{Greatest such}) Let $\varepsilon>0$. Then $\varepsilon/2 >0$ and $\varepsilon/2 < 1$, hence $\varepsilon/2\in X$ and $\varepsilon/2 < \varepsilon$. This shows no positive number can be a lower bound. Therefore no number larger than $0$ is a lower bound, so $0$ is the greatest lower bound.
\end{itemize}
Because $0\in X$, the infimum is attained; hence
\[
\min X = 0.
\]

\textbf{3. Supremum}

\emph{Claim:} $\sup X = 1$.

\begin{itemize}
    \item (\emph{Upper bound}) For all $x\in X$, $x<1$, so $1$ is an upper bound.
    \item (\emph{Least such}) Let $M<1$. Set $x=\dfrac{1+M}{2}$. Then
    \[
    M < \frac{1+M}{2} < 1,
    \]
    so $x\in X$ and $x>M$. Thus $M$ cannot be an upper bound. Hence no number smaller than $1$ is an upper bound, making $1$ the least upper bound.
\end{itemize}
Since $1\notin X$, the supremum is \emph{not} attained; there is no maximum:
\[
\max X \text{ does not exist.}
\]

\textbf{4. Summary}

\[
\inf X = 0 = \min X,\qquad \sup X = 1,\qquad \text{no maximum.}
\]

So $X$ is bounded below and above; minimum exists ($0$), maximum does not (because the endpoint $1$ is excluded).

\subsection*{(b)}

\textbf{Set:}
\[
X = \left\{ \frac{1}{n} \;\middle|\; n \in \mathbb{N},\; n > 0 \right\} = \left\{ 1,\, \tfrac{1}{2},\, \tfrac{1}{3},\, \tfrac{1}{4},\, \dots \right\}.
\]

\textbf{1. Boundedness}

\begin{itemize}
    \item \emph{Upper bound:} $1$. Clearly $\frac{1}{n} \le 1$ for all $n \ge 1$. So $X$ is bounded above (and any $U \ge 1$ is an upper bound).
    \item \emph{Lower bound:} $0$. All terms are positive, so $0 \le \frac{1}{n}$. Thus $X$ is bounded below (and any $L \le 0$ is also a lower bound).
\end{itemize}

\textbf{2. Supremum}

\emph{Claim:} $\sup X = 1$.

\begin{itemize}
    \item $1$ is an upper bound (as above).
    \item If $M < 1$, pick $n = 1$. Then $1 \in X$ and $1 > M$, so $M$ cannot be an upper bound.
\end{itemize}
Hence no number smaller than $1$ is an upper bound.

Because $1 \in X$, the supremum is attained:
\[
\max X = 1.
\]

\textbf{3. Infimum}

\emph{Claim:} $\inf X = 0$.

\begin{itemize}
    \item $0$ is a lower bound: $\frac{1}{n} > 0$ for all $n$.
    \item \emph{Minimality:} Let $p > 0$. Choose $n > \frac{1}{p}$ (possible since $\mathbb{N}$ is unbounded). Then $\frac{1}{n} < p$ and $\frac{1}{n} \in X$. So $p$ fails to be a lower bound. Because this works for every positive $p$, no positive number is a lower bound; thus $0$ is the greatest lower bound.
\end{itemize}

Alternatively, for any $\varepsilon > 0$, choose $n > \frac{1}{\varepsilon}$ so $0 < \frac{1}{n} < \varepsilon$. This shows elements of $X$ get arbitrarily close to $0$ from above.

Since $0 \notin X$, the infimum is not attained:
\[
\text{no minimum (no } \min X\text{).}
\]

\textbf{4. Summary}

\[
\inf X = 0 \quad (\text{not attained}),\qquad \sup X = 1 = \max X,\qquad \text{no minimum.}
\]

\emph{Key pattern:} A decreasing sequence $1/n$ converging to $0$ has supremum equal to its first term (if included) and infimum equal to the limit (if the limit is not included, no minimum).
\end{document}