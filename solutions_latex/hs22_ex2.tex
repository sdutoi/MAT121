\documentclass[12pt,a4paper]{article}

% -- Packages --
\usepackage{amsmath, amssymb, amsthm} % Math symbols & theorems
\usepackage{enumitem} % Better lists
\usepackage{geometry} % Page layout
\usepackage{fancyhdr} % Header/footer
\usepackage{tikz}     % Diagrams
\usepackage{hyperref} % Clickable references
\usepackage{mathrsfs} % Fancy math fonts

% -- Page Setup --
\geometry{margin=1in}
\setlength{\parskip}{0.5em}
\setlength{\parindent}{0pt}
\pagestyle{fancy}
\fancyhf{}
% -- Header/Footer --
\lhead{MAT121 -- Analysis I}
\chead{Exercise sheet 1}
\rhead{Stefan du Toit}
\rfoot{\thepage}

% -- Theorem Environments --
\newtheorem{theorem}{Theorem}[section]
\newtheorem{lemma}[theorem]{Lemma}
\newtheorem{proposition}[theorem]{Proposition}
\newtheorem{corollary}[theorem]{Corollary}

\theoremstyle{definition}
\newtheorem{definition}[theorem]{Definition}
\newtheorem{example}[theorem]{Example}
\newtheorem{exercise}{Exercise}[section]

\theoremstyle{remark}
\newtheorem*{remark}{Remark}

% -- Custom Environments --
\newenvironment{solution}{\begin{proof}[Solution]}{\end{proof}}



% ==========================================
\begin{document}

\section*{1}



\section*{2}

Throughout this section we regard $\mathbb{Q}$ as an ordered field. Thus, for a set $A\subseteq\mathbb{Q}$, the statement "$\sup_{\mathbb{Q}}(A)$ exists" means: there is some $s\in\mathbb{Q}$ that is an upper bound of $A$ (in the usual order) and for every rational upper bound $u\in\mathbb{Q}$ we have $s\le u$.

\subsection*{(a)}
\[A 
	= \left\{\dfrac{n}{n+1} : n\in\mathbb{N}\right\}.\]
We claim that case (ii) holds: $\sup(A)$ exists in $\mathbb{Q}$ but $\sup(A)\notin A$, and in fact $\sup(A)=1$.

\emph{Upper bound.} For every $n\in\mathbb{N}$, $\dfrac{n}{n+1}<1$, hence $1$ is a rational upper bound of $A$.

\emph{Least among rational upper bounds.} Let $q\in\mathbb{Q}$ with $q<1$. Then $\dfrac{q}{1-q}$ is a real number, so by the Archimedean property there exists $n\in\mathbb{N}$ with $n>\dfrac{q}{1-q}$. Multiplying by $1-q>0$ gives $(1-q)n>q$, i.e. $n>q(n+1)$. Dividing by $n+1>0$ yields
\[
	\frac{n}{n+1} > q.
\]
Hence $q$ is \emph{not} an upper bound of $A$. Therefore no rational number $<1$ can be an upper bound, so $1$ is the least rational upper bound: $\sup(A)=1$. Clearly $1\notin A$. Thus (ii) is true for (a).

\subsection*{(b)}
\[A 
	= \{\, t\in\mathbb{Q} : t^3-2t-1<0\,\}.\]
Factor the cubic:
\[
 t^3-2t-1 = (t+1)\bigl(t^2-t-1\bigr),
\]
whose real zeros are $t=-1$ and $t=\dfrac{1\pm\sqrt{5}}{2}$. Thus
\[
 \{ t\in\mathbb{R} : t^3-2t-1<0 \} = (-\infty,-1) \;\cup\; \Bigl(\tfrac{1-\sqrt{5}}{2}, \tfrac{1+\sqrt{5}}{2}\Bigr).
\]
Intersecting with $\mathbb{Q}$ gives $A$. In $\mathbb{R}$ the least upper bound of $A$ is
\[\varphi = \frac{1+\sqrt{5}}{2},\]
which is irrational. Since we are taking suprema in the ordered field $\mathbb{Q}$, there is no smallest \emph{rational} upper bound of $A$: for any $u\in\mathbb{Q}$ with $u<\varphi$ there are rationals $t\in A$ with $u<t<\varphi$ (density of $\mathbb{Q}$), so $u$ fails to be an upper bound; and if $u\in\mathbb{Q}$ with $u>\varphi$, then $u$ is an upper bound but not the least among rational upper bounds. Consequently, $\sup(A)$ does \emph{not} exist in $\mathbb{Q}$. Hence case (i) holds for (b).

\subsection*{(c)}
\[A 
	= \bigl\{\, \alpha\in\mathbb{Q} : x^2-x+\alpha \text{ has a rational root }\bigr\}.\]
The quadratic $x^2-x+\alpha$ has a rational (indeed real) root iff its discriminant is a rational square with $\,1-4\alpha\ge 0\,$, i.e.,
\[
 1-4\alpha = r^2 \quad \text{for some } r\in\mathbb{Q}_{\ge 0}.
\]
Equivalently,
\[
 \alpha = \frac{1-r^2}{4}, \qquad r\in\mathbb{Q}_{\ge 0}.
\]
Thus $A = \bigl\{\tfrac{1-r^2}{4} : r\in\mathbb{Q}_{\ge 0}\bigr\} \subset (-\infty,\tfrac14]$ and, since squares of rationals are dense in $[0,\infty)$, the set $A$ is dense from below up to $\tfrac14$. In particular $\tfrac14\in A$ (discriminant $0$, double root $x=\tfrac12\in\mathbb{Q}$), and it is an upper bound of $A$.

Moreover, if $u<\tfrac14$ then $1-4u>0$; choosing $r\in\mathbb{Q}$ with $0<r^2<1-4u$ yields $\alpha=(1-r^2)/4\in A$ with $u<\alpha\le \tfrac14$. Hence no rational $<\tfrac14$ can be an upper bound. Therefore
\[
 \sup(A) = \tfrac14\in A,
\]
so case (iii) holds for (c).

%%%%%%%

\section*{3}
Determine $\sup(A)$ for the given set $A\subset\mathbb{R}$.

\subsection*{(a)}
Let
\[
A=\{x\in\mathbb{R} : x^3-2x-1<0\}.
\]
Set $f(x)=x^3-2x-1$. Then
\[
f(x)=(x+1)\bigl(x^2-x-1\bigr),
\]
and the real zeros are
\[
x=-1,\qquad x=\frac{1\pm\sqrt{5}}{2}=: \varphi_{-}\ (<-1<\varphi_{-})\ ,\ \varphi_{+}\ (>0).
\]
Since $f$ is continuous and changes sign at each simple root, a sign chart shows
\[
\{x\in\mathbb{R}: f(x)<0\}=(-\infty,-1)\ \cup\ (\varphi_{-},\,\varphi_{+}).
\]
In particular, $\varphi_{+}$ is an upper bound of $A$, and for every $u<\varphi_{+}$ the interval 
$(\max\{u,\varphi_{-}\},\,\varphi_{+})\subset A$, so $u$ is not an upper bound. Therefore
\[
\sup(A)=\varphi_{+}=\frac{1+\sqrt{5}}{2}.
\]

\subsection*{(b)}
Let
\[
A=\bigl\{\,x\in\mathbb{R}:\ x\ \text{has a decimal expansion}\ 0.d_1d_2d_3\ldots\ \text{with}\ d_i\in\{0,1,2,3\}\ \text{for all }i\,\bigr\}.
\]
Every $x\in A$ uses only digits $\le 3$, hence
\[
x\le 0.3333\ldots=\frac{1}{3}.
\]
Thus $\tfrac{1}{3}$ is an upper bound of $A$, and in fact $\tfrac{1}{3}\in A$ via the expansion $0.\overline{3}$ (all digits equal to $3$).

Moreover, the truncations
\[
x_n:=0.\underbrace{33\ldots 3}_{n\ \text{digits}}
=\sum_{k=1}^n \frac{3}{10^k}
=\frac{1-10^{-n}}{3}
\nearrow \frac{1}{3}
\]
lie in $A$ and approach $\tfrac{1}{3}$ from below. Hence every $u<\tfrac{1}{3}$ fails to be an upper bound. Therefore
\[
\sup(A)=\frac{1}{3}\in A.
\]

\end{document}