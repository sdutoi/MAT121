\documentclass[12pt,a4paper]{article}

% ----------- Packages -----------
\usepackage{amsmath, amssymb, amsthm} % Math symbols & theorems
\usepackage{enumitem} % Better lists
\usepackage{geometry} % Page layout
\usepackage{fancyhdr} % Header/footer
\usepackage{tikz}     % Diagrams
\usepackage{hyperref} % Clickable references
\usepackage{mathrsfs} % Fancy math fonts

% ----------- Page Setup -----------
\geometry{margin=1in}
\setlength{\parskip}{0.5em}
\setlength{\parindent}{0pt}
\pagestyle{fancy}
\fancyhf{}
% ----------- Header/Footer -----------
\lhead{MAT121 -- Analysis I}
\chead{Exercise sheet 6}
\rhead{Stefan du Toit}
\rfoot{\thepage}

% ----------- Theorem Environments -----------
\newtheorem{theorem}{Theorem}[section]
\newtheorem{lemma}[theorem]{Lemma}
\newtheorem{proposition}[theorem]{Proposition}
\newtheorem{corollary}[theorem]{Corollary}

\theoremstyle{definition}
\newtheorem{definition}[theorem]{Definition}
\newtheorem{example}[theorem]{Example}
\newtheorem{exercise}{Exercise}[section]

\theoremstyle{remark}
\newtheorem*{remark}{Remark}

% ----------- Custom Environments -----------
\newenvironment{solution}{\begin{proof}[Solution]}{\end{proof}}



% ==========================================
\begin{document}

\section*{3}
\subsection*{(a)}
% Problem (a)
Compute the limit
\[
\lim_{x\to 2}\frac{x^{3}-2x-4}{x^{2}-4}.
\]

% Solution
Direct substitution gives \(0/0\), so factor:
\[
x^{2}-4=(x-2)(x+2),\qquad
x^{3}-2x-4=(x-2)(x^{2}+2x+2).
\]
Then
\begin{align*}
\lim_{x\to 2}\frac{x^{3}-2x-4}{x^{2}-4}
&=\lim_{x\to 2}\frac{(x-2)(x^{2}+2x+2)}{(x-2)(x+2)}\\
&=\lim_{x\to 2}\frac{x^{2}+2x+2}{x+2}\\
&=\frac{2^{2}+2\cdot 2+2}{2+2}
=\frac{10}{4}
=\frac{5}{2}.
\end{align*}
Therefore, the limit equals \(\dfrac{5}{2}\).

\subsection*{(b)}
Evaluate the limit
\[
\lim_{x\to 0}\frac{x^{2}-x}{|x|}.
\]
Factor the numerator:
\[
\frac{x^{2}-x}{|x|}=\frac{x(x-1)}{|x|}=\frac{x}{|x|}(x-1).
\]

Right-hand limit (\(x\to 0^{+}\)):
\[
x>0 \;\Rightarrow\; |x|=x \;\Rightarrow\; \frac{x}{|x|}=1
\quad\Longrightarrow\quad
\frac{x^{2}-x}{|x|}=x-1 \xrightarrow[x\to 0^{+}]{} -1.
\]

Left-hand limit (\(x\to 0^{-}\)):
\[
x<0 \;\Rightarrow\; |x|=-x \;\Rightarrow\; \frac{x}{|x|}=-1
\quad\Longrightarrow\quad
\frac{x^{2}-x}{|x|}=-1\,(x-1) = -x+1 \xrightarrow[x\to 0^{-}]{} 1.
\]

Since
\[
\lim_{x\to 0^{+}}\frac{x^{2}-x}{|x|} = -1
\quad\text{and}\quad
\lim_{x\to 0^{-}}\frac{x^{2}-x}{|x|} = 1,
\]
the two-sided limit does not exist.

\[
\boxed{\text{Limit does not exist}}
\]

\subsection*{(c)}
Evaluate
\[
\lim_{x\to +\infty}\frac{\lfloor x\rfloor}{x}.
\]

Start from the defining inequalities for the floor function:
\[
\lfloor x \rfloor \le x < \lfloor x \rfloor + 1.
\]
Divide through by \(x>0\):
\[
\frac{\lfloor x \rfloor}{x} \le 1 < \frac{\lfloor x \rfloor + 1}{x}.
\]
Split the last term:
\[
\frac{\lfloor x \rfloor}{x} \le 1 < \frac{\lfloor x \rfloor}{x} + \frac{1}{x}.
\]
Subtract \(\tfrac{1}{x}\) from the rightmost inequality and combine:
\[
1 - \frac{1}{x} \le \frac{\lfloor x \rfloor}{x} \le 1.
\]

Take limits as \(x\to +\infty\):
\[
\lim_{x\to +\infty}\left(1-\frac{1}{x}\right)=1,
\qquad
\lim_{x\to +\infty} 1 = 1.
\]
By the Squeeze Theorem,
\[
\boxed{\lim_{x\to +\infty}\frac{\lfloor x\rfloor}{x}=1 }.
\]


%%%%%%%%%%%

\section*{4}
\subsection*{(a)}
Given $f:\mathbb{R}\to(0,+\infty)$ with $\displaystyle\lim_{x\to 0} f(x)=2$. Compute
\[
\lim_{x\to 0}\bigl(f(x)+f(5x)\bigr).
\]

Since $x\to 0 \implies 5x\to 0$, we have
\[
\lim_{x\to 0} f(5x)=\lim_{u\to 0} f(u)=2.
\]
Using the sum law for limits:
\[
\lim_{x\to 0}\bigl(f(x)+f(5x)\bigr)=\lim_{x\to 0} f(x)+\lim_{x\to 0} f(5x)=2+2=4.
\]

\[
\boxed{4}
\]



%%%%

\subsection*{(b)}
Given $f:\mathbb{R}\to(0,+\infty)$ and
\[
\lim_{x\to 0}\Bigl(f(x)+\frac{1}{f(x)}\Bigr)=2.
\]
Goal: find $\displaystyle \lim_{x\to 0} f(x)$.

Let $(x_n)$ be any sequence with $x_n\to 0$. Since $f(x_n)>0$, by compactness of closed bounded intervals we may extract a subsequence (still denoted) with $f(x_n)\to L$ for some $L>0$ (if the whole sequence already converges we are done; otherwise take a convergent subsequence—positivity plus the given limit will force boundedness near 0).

Because $g(t)=t+1/t$ is continuous on $(0,\infty)$,
\[
2=\lim_{n\to\infty} \Bigl(f(x_n)+\frac{1}{f(x_n)}\Bigr)=\lim_{n\to\infty} g\bigl(f(x_n)\bigr)=g(L)=L+\frac{1}{L}.
\]
Thus
\[
L+\frac{1}{L}=2 \;\Longrightarrow\; L^2-2L+1=0 \;\Longrightarrow\; (L-1)^2=0 \;\Longrightarrow\; L=1.
\]

Every subsequential limit equals $1$, hence the full limit exists and equals $1$:
\[
\boxed{\lim_{x\to 0} f(x)=1}.
\]

\end{document}