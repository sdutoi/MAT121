\documentclass[12pt,a4paper]{article}

% -- Packages --
\usepackage{amsmath, amssymb, amsthm} % Math symbols & theorems
\usepackage{enumitem} % Better lists
\usepackage{geometry} % Page layout
\usepackage{fancyhdr} % Header/footer
\usepackage{tikz}     % Diagrams
\usepackage{hyperref} % Clickable references
\usepackage{mathrsfs} % Fancy math fonts

% -- Page Setup --
\geometry{margin=1in}
\setlength{\parskip}{0.5em}
\setlength{\parindent}{0pt}
\pagestyle{fancy}
\fancyhf{}
% -- Header/Footer --
\lhead{MAT121 -- Analysis I}
\chead{Exercise sheet 1}
\rhead{Stefan du Toit}
\rfoot{\thepage}

% -- Theorem Environments --
\newtheorem{theorem}{Theorem}[section]
\newtheorem{lemma}[theorem]{Lemma}
\newtheorem{proposition}[theorem]{Proposition}
\newtheorem{corollary}[theorem]{Corollary}

\theoremstyle{definition}
\newtheorem{definition}[theorem]{Definition}
\newtheorem{example}[theorem]{Example}
\newtheorem{exercise}{Exercise}[section]

\theoremstyle{remark}
\newtheorem*{remark}{Remark}

% -- Custom Environments --
\newenvironment{solution}{\begin{proof}[Solution]}{\end{proof}}



% ==========================================
\begin{document}

\section*{1}
\subsection*{(a) Binet’s formula}
\begin{proof}
We are given $F_0=0$, $F_1=1$, and for $n\ge 2$,
\[
F_n \;=\; F_{n-1}+F_{n-2}.
\]
We will solve this linear recurrence via its characteristic equation and determine the constants from the initial conditions.

\paragraph{Step 1: Characteristic equation.}
Look for solutions of the form $u_n=r^n$. Substituting into $u_n=u_{n-1}+u_{n-2}$ gives
\[
r^n \;=\; r^{n-1}+r^{n-2}
\;\Longleftrightarrow\;
r^2 \;=\; r+1
\;\Longleftrightarrow\;
r^2-r-1=0.
\]
The two roots are
\[
\varphi:=\frac{1+\sqrt{5}}{2},\qquad
\psi:=\frac{1-\sqrt{5}}{2}.
\]
Hence every sequence of the form
\[
u_n \;=\; A\,\varphi^n + B\,\psi^n
\]
satisfies the same recurrence.

\paragraph{Step 2: Fit the initial conditions.}
Impose $u_0=F_0=0$ and $u_1=F_1=1$:
\[
u_0=A\varphi^0+B\psi^0=A+B=0 \quad\Longrightarrow\quad B=-A,
\]
\[
u_1=A\varphi+B\psi=A(\varphi-\psi)=1.
\]
Compute
\[
\varphi-\psi=\frac{1+\sqrt{5}}{2}-\frac{1-\sqrt{5}}{2}=\sqrt{5},
\]
so
\[
A=\frac{1}{\varphi-\psi}=\frac{1}{\sqrt{5}},
\qquad
B=-\frac{1}{\sqrt{5}}.
\]

\paragraph{Step 3: Closed form and conclusion.}
Therefore
\[
u_n \;=\; \frac{1}{\sqrt{5}}\bigl(\varphi^n-\psi^n\bigr).
\]
Since $u_n$ satisfies the Fibonacci recurrence and matches the initial values $F_0,F_1$, by uniqueness of solutions to a second-order linear recurrence (or by a short induction) we have $u_n=F_n$ for all $n\in\mathbb{N}$. Hence
\[
F_n \;=\; \frac{1}{\sqrt{5}}
\left(
\left(\frac{1+\sqrt{5}}{2}\right)^{\!n}
-
\left(\frac{1-\sqrt{5}}{2}\right)^{\!n}
\right).
\qedhere
\end{proof}


\end{document}