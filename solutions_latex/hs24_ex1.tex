\documentclass[12pt,a4paper]{article}

% ----------- Packages -----------
\usepackage{amsmath, amssymb, amsthm} % Math symbols & theorems
\usepackage{enumitem} % Better lists
\usepackage{geometry} % Page layout
\usepackage{fancyhdr} % Header/footer
\usepackage{tikz}     % Diagrams
\usepackage{hyperref} % Clickable references
\usepackage{mathrsfs} % Fancy math fonts

% ----------- Page Setup -----------
\geometry{margin=1in}
\setlength{\parskip}{0.5em}
\setlength{\parindent}{0pt}
\pagestyle{fancy}
\fancyhf{}
% ----------- Header/Footer -----------
\lhead{MAT121 -- Analysis I}
\chead{Exercise sheet 1}
\rhead{Stefan du Toit}
\rfoot{\thepage}

% ----------- Theorem Environments -----------
\newtheorem{theorem}{Theorem}[section]
\newtheorem{lemma}[theorem]{Lemma}
\newtheorem{proposition}[theorem]{Proposition}
\newtheorem{corollary}[theorem]{Corollary}

\theoremstyle{definition}
\newtheorem{definition}[theorem]{Definition}
\newtheorem{example}[theorem]{Example}
\newtheorem{exercise}{Exercise}[section]

\theoremstyle{remark}
\newtheorem*{remark}{Remark}

% ----------- Custom Environments -----------
\newenvironment{solution}{\begin{proof}[Solution]}{\end{proof}}



% ==========================================
\begin{document}

\section*{1}

\textbf{Problem:} Let $A, B, C$ be sets. Show that:
$$(A \setminus B) \cup (A \setminus C) = A \setminus (B \cap C)$$

\textbf{Proof:}

To prove set equality, we'll show both directions: every element of the left side is in the right side, and vice versa.

Let $x$ be an arbitrary element. We'll show:
$$x \in (A \setminus B) \cup (A \setminus C) \iff x \in A \setminus (B \cap C)$$

\textbf{Direction 1: ($\Rightarrow$) If $x \in (A \setminus B) \cup (A \setminus C)$, then $x \in A \setminus (B \cap C)$}

Assume $x \in (A \setminus B) \cup (A \setminus C)$.

This means $x \in (A \setminus B)$ or $x \in (A \setminus C)$ (or both).

\textbf{Case 1:} $x \in (A \setminus B)$

Then $x \in A$ and $x \notin B$.
Since $x \notin B$, we have $x \notin (B \cap C)$.
Therefore $x \in A$ and $x \notin (B \cap C)$, so $x \in A \setminus (B \cap C)$.

\textbf{Case 2:} $x \in (A \setminus C)$

Then $x \in A$ and $x \notin C$.
Since $x \notin C$, we have $x \notin (B \cap C)$.
Therefore $x \in A$ and $x \notin (B \cap C)$, so $x \in A \setminus (B \cap C)$.

In both cases, $x \in A \setminus (B \cap C)$.

\textbf{Direction 2: ($\Leftarrow$) If $x \in A \setminus (B \cap C)$, then $x \in (A \setminus B) \cup (A \setminus C)$}

Assume $x \in A \setminus (B \cap C)$.

This means $x \in A$ and $x \notin (B \cap C)$.

Since $x \notin (B \cap C)$, by De Morgan's law, we have $x \notin B$ or $x \notin C$ (or both).

\textbf{Case 1:} $x \notin B$

Since $x \in A$ and $x \notin B$, we have $x \in (A \setminus B)$.
Therefore $x \in (A \setminus B) \cup (A \setminus C)$.

\textbf{Case 2:} $x \notin C$

Since $x \in A$ and $x \notin C$, we have $x \in (A \setminus C)$.
Therefore $x \in (A \setminus B) \cup (A \setminus C)$.

In both cases, $x \in (A \setminus B) \cup (A \setminus C)$.

\textbf{Conclusion:}

Since we've shown both directions, we have:
$$(A \setminus B) \cup (A \setminus C) = A \setminus (B \cap C)$$

This is a distributive law for set difference over intersection.

\section*{2}


\textbf{Problem:} Let $X$ and $Y$ be sets and $f : X \to Y$ be a function. Prove that:

\begin{enumerate}
\item[(a)] For every $A \subseteq X$, $f^{-1}(f(A)) \supseteq A$.
\item[(b)] If $f$ is injective, then $f^{-1}(f(A)) = A$ also holds.
\item[(c)] Is (b) true for all functions?
\end{enumerate}

\subsection*{(a)}

\textbf{Statement:} For every $A \subseteq X$, $f^{-1}(f(A)) \supseteq A$.

\textbf{Proof:}

We need to show that $A \subseteq f^{-1}(f(A))$, which means every element of $A$ is also in $f^{-1}(f(A))$.

Let $x$ be an arbitrary element of $A$. We need to show that $x \in f^{-1}(f(A))$.

\textbf{Step 1:} Since $x \in A$ and $A \subseteq X$, we have $x \in X$.

\textbf{Step 2:} Since $f : X \to Y$ and $x \in X$, we know that $f(x)$ is defined and $f(x) \in Y$.

\textbf{Step 3:} Since $x \in A$, we have $f(x) \in f(A)$ by the definition of the image of a set.

(Recall: $f(A) = \{f(a) : a \in A\}$)

\textbf{Step 4:} Since $f(x) \in f(A)$, by the definition of preimage, we have $x \in f^{-1}(f(A))$.

(Recall: $f^{-1}(f(A)) = \{z \in X : f(z) \in f(A)\}$)

\textbf{Conclusion:} Since $x$ was arbitrary, we have shown that every element of $A$ is in $f^{-1}(f(A))$.

Therefore: $A \subseteq f^{-1}(f(A))$, which means $f^{-1}(f(A)) \supseteq A$.



\subsection*{(b)}

\textbf{Statement:} If $f$ is injective, then $f^{-1}(f(A)) = A$ also holds.

\textbf{Proof:}

From part (a), we already know that $f^{-1}(f(A)) \supseteq A$ for any function $f$.

To prove equality $f^{-1}(f(A)) = A$ when $f$ is injective, we need to show the reverse inclusion: $f^{-1}(f(A)) \subseteq A$.

Let $x$ be an arbitrary element of $f^{-1}(f(A))$. We need to show that $x \in A$.

\textbf{Step 1:} Since $x \in f^{-1}(f(A))$, by definition of preimage, we have:
$$x \in X \text{ and } f(x) \in f(A)$$

\textbf{Step 2:} Since $f(x) \in f(A)$, by definition of image, there exists some element $a \in A$ such that:
$$f(x) = f(a)$$

\textbf{Step 3:} Since $f$ is injective and $f(x) = f(a)$, we can conclude:
$$x = a$$

\textbf{Step 4:} Since $a \in A$ and $x = a$, we have:
$$x \in A$$

\textbf{Conclusion:} Since $x$ was arbitrary, we have shown that every element of $f^{-1}(f(A))$ is in $A$.

Therefore: $f^{-1}(f(A)) \subseteq A$

Combined with the result from part (a) that $f^{-1}(f(A)) \supseteq A$, we get:
$$f^{-1}(f(A)) = A$$

\textbf{Key insight:} The injectivity of $f$ is crucial in Step 3. Without injectivity, we couldn't conclude that $x = a$ from $f(x) = f(a)$.

\subsection*{(c)}

\textbf{Statement:} Is (b) true for all functions?

\textbf{Answer:} No, part (b) is not true for all functions. The equality $f^{-1}(f(A)) = A$ requires $f$ to be injective.

\textbf{Counterexample:}

Let $X = \{1, 2, 3\}$, $Y = \{a, b\}$, and define $f : X \to Y$ by:
\begin{align}
f(1) &= a\\
f(2) &= a\\
f(3) &= b
\end{align}

This function is not injective since $f(1) = f(2) = a$.

Let $A = \{1\}$. Then:

\textbf{Step 1:} Calculate $f(A)$
$$f(A) = f(\{1\}) = \{f(1)\} = \{a\}$$

\textbf{Step 2:} Calculate $f^{-1}(f(A))$
$$f^{-1}(f(A)) = f^{-1}(\{a\}) = \{x \in X : f(x) = a\} = \{1, 2\}$$

\textbf{Step 3:} Compare $A$ and $f^{-1}(f(A))$
\begin{itemize}
\item $A = \{1\}$
\item $f^{-1}(f(A)) = \{1, 2\}$
\end{itemize}

Clearly, $A \neq f^{-1}(f(A))$ since $\{1\} \neq \{1, 2\}$.

We have $f^{-1}(f(A)) \supset A$ (proper superset), but not equality.

\textbf{Explanation:} The element $2$ is in $f^{-1}(f(A))$ because $f(2) = a \in f(A)$, but $2 \notin A$. This happens precisely because $f$ is not injective -- both elements $1$ and $2$ map to the same value $a$, so when we take the preimage of $f(A) = \{a\}$, we get back both $1$ and $2$, not just the original element $1$.

\textbf{Conclusion:} The equality $f^{-1}(f(A)) = A$ holds if and only if $f$ is injective. For non-injective functions, we only have the inclusion $f^{-1}(f(A)) \supseteq A$ from part (a).


\section*{3}

\textbf{Problem:} We consider the function $f : \mathbb{N} \to \mathbb{Q}$:

$$f(n) = \frac{1 - (-1)^n(2n + 1)}{4}$$

\begin{itemize}
\item Investigate whether $f$ is injective.
\item What is $f(\mathbb{N})$?
\end{itemize}

\textbf{Solution:}

Let me first analyze what this function does by computing some values.

\textbf{Step 1: Understanding the function}

The key is the term $(-1)^n$:
\begin{itemize}
\item When $n$ is even: $(-1)^n = 1$
\item When $n$ is odd: $(-1)^n = -1$
\end{itemize}

Let's compute $f(n)$ for both cases:

\textbf{Case 1:} $n$ is even ($(-1)^n = 1$)
\begin{align}
f(n) &= \frac{1 - 1 \cdot (2n + 1)}{4}\\
&= \frac{1 - 2n - 1}{4}\\
&= \frac{-2n}{4}\\
&= -\frac{n}{2}
\end{align}

\textbf{Case 2:} $n$ is odd ($(-1)^n = -1$)
\begin{align}
f(n) &= \frac{1 - (-1) \cdot (2n + 1)}{4}\\
&= \frac{1 + 2n + 1}{4}\\
&= \frac{2n + 2}{4}\\
&= \frac{n + 1}{2}
\end{align}

\textbf{Step 2: Computing specific values}

Let's compute the first few values:
\begin{align}
f(1) &= \frac{1 + 1}{2} = 1 \quad \text{(since 1 is odd)}\\
f(2) &= -\frac{2}{2} = -1 \quad \text{(since 2 is even)}\\
f(3) &= \frac{3 + 1}{2} = 2 \quad \text{(since 3 is odd)}\\
f(4) &= -\frac{4}{2} = -2 \quad \text{(since 4 is even)}\\
f(5) &= \frac{5 + 1}{2} = 3 \quad \text{(since 5 is odd)}\\
f(6) &= -\frac{6}{2} = -3 \quad \text{(since 6 is even)}
\end{align}

\textbf{Part 1: Is $f$ injective?}

From our analysis:
\begin{itemize}
\item For odd $n$: $f(n) = \frac{n + 1}{2}$, which gives values: $1, 2, 3, 4, \ldots$
\item For even $n$: $f(n) = -\frac{n}{2}$, which gives values: $-1, -2, -3, -4, \ldots$
\end{itemize}

Since odd inputs map to positive values and even inputs map to negative values, and within each case the function is strictly monotonic:

\begin{itemize}
\item For odd inputs: If $n_1$ and $n_3$ are odd with $n_1 \neq n_3$, then $\frac{n_1 + 1}{2} \neq \frac{n_3 + 1}{2}$.
\item For even inputs: If $n_2$ and $n_4$ are even with $n_2 \neq n_4$, then $-\frac{n_2}{2} \neq -\frac{n_4}{2}$.
\end{itemize}

Since odd inputs give positive values and even inputs give negative values, and within each case the values are distinct, $f$ is \textbf{injective}.

\textbf{Part 2: What is $f(\mathbb{N})$?}

From our analysis:
\begin{itemize}
\item Odd $n \in \mathbb{N}$: $f(n) = \frac{n + 1}{2}$ gives us $\{1, 2, 3, 4, \ldots\} = \mathbb{N}$
\item Even $n \in \mathbb{N}$: $f(n) = -\frac{n}{2}$ gives us $\{-1, -2, -3, -4, \ldots\} = -\mathbb{N}$
\end{itemize}

Therefore: $f(\mathbb{N}) = \mathbb{N} \cup (-\mathbb{N}) = \mathbb{Z} \setminus \{0\}$

\textbf{Summary:}
\begin{itemize}
\item $f$ is injective
\item $f(\mathbb{N}) = \mathbb{Z} \setminus \{0\}$ (all non-zero integers)
\end{itemize}


\section*{4}

\textbf{Problem:} Let $m, n \in \mathbb{N}$, $X = \{1, 2, \ldots, m\}$ and $Y = \{1, 2, \ldots, n\}$

\begin{enumerate}
\item[(a)] How many different functions $f : X \to Y$ are there?
\item[(b)] How many different injective functions $f : X \to Y$ are there?
\end{enumerate}

\subsection*{(a)}

\textbf{Statement:} How many different functions $f : X \to Y$ are there?

\textbf{Solution:}

To count the number of functions $f : X \to Y$, I need to determine how many choices we have for each element in the domain $X$.

\textbf{Step 1: Understanding what defines a function}

A function $f : X \to Y$ assigns to each element in $X$ exactly one element in $Y$.

Since $X = \{1, 2, \ldots, m\}$, we need to specify:
\begin{itemize}
\item $f(1) \in Y$
\item $f(2) \in Y$
\item $\vdots$
\item $f(m) \in Y$
\end{itemize}

\textbf{Step 2: Counting the choices}

For each element in $X$, we can choose any element from $Y$ as its image:

\begin{itemize}
\item For $f(1)$: We have $n$ choices (any element from $Y = \{1, 2, \ldots, n\}$)
\item For $f(2)$: We have $n$ choices (any element from $Y$)
\item $\vdots$
\item For $f(m)$: We have $n$ choices (any element from $Y$)
\end{itemize}

\textbf{Step 3: Applying the multiplication principle}

Since these choices are independent, we multiply the number of choices:

$$\text{Total number of functions} = \underbrace{n \times n \times \cdots \times n}_{m \text{ times}} = n^m$$

\textbf{Answer:} There are $n^m$ different functions $f : X \to Y$.

\subsection*{(b)}

\textbf{Statement:} How many different injective functions $f : X \to Y$ are there?

\textbf{Solution:}

To count injective functions $f : X \to Y$, I need to ensure that no two elements in $X$ map to the same element in $Y$.

\textbf{Step 1: Understanding injectivity constraint}

A function $f : X \to Y$ is injective if $f(x_1) \neq f(x_2)$ whenever $x_1 \neq x_2$.

This means each element in $Y$ can be the image of at most one element from $X$.

\textbf{Step 2: Consider two cases}

\textbf{Case 1:} $m > n$

If $|X| = m > n = |Y|$, then by the pigeonhole principle, it's impossible to have an injective function from $X$ to $Y$.

We have $m$ elements in $X$ that need distinct images, but only $n < m$ elements available in $Y$.

Therefore, when $m > n$: Number of injective functions $= 0$

\textbf{Case 2:} $m \leq n$

When $m \leq n$, injective functions are possible.

\textbf{Step 3: Counting for $m \leq n$}

We need to assign distinct elements from $Y$ to each element in $X$:

\begin{itemize}
\item For $f(1)$: We have $n$ choices (any element from $Y$)
\item For $f(2)$: We have $(n-1)$ choices (any element from $Y$ except $f(1)$)
\item For $f(3)$: We have $(n-2)$ choices (any element from $Y$ except $f(1)$ and $f(2)$)
\item $\vdots$
\item For $f(m)$: We have $(n-m+1)$ choices
\end{itemize}

\textbf{Step 4: Applying the multiplication principle}

$$\text{Total number of injective functions} = n \times (n-1) \times (n-2) \times \cdots \times (n-m+1)$$

This is: $$\frac{n!}{(n-m)!}$$

\textbf{Final Answer:}

The number of different injective functions $f : X \to Y$ is:

$$\begin{cases}
0 & \text{if } m > n \\
\frac{n!}{(n-m)!} & \text{if } m \leq n
\end{cases}$$

\textbf{Note:} When $m = n$, this gives $\frac{n!}{0!} = n!$, which is the number of bijective functions (permutations) from $X$ to $Y$.


\section*{5}

\textbf{Problem:} Prove that:
$$1 + \frac{1}{\sqrt{2}} + \frac{1}{\sqrt{3}} + \cdots + \frac{1}{\sqrt{n}} > 2(\sqrt{n+1} - 1)$$
for all $n \in \mathbb{N}$.

\textbf{Solution:}

I'll prove this inequality using the key insight that each term $\frac{1}{\sqrt{k}}$ can be bounded below by a telescoping sum.

\textbf{Key Lemma:}

For any $k \geq 1$, we have: $\frac{1}{\sqrt{k}} > 2(\sqrt{k+1} - \sqrt{k})$

\textbf{Proof of Lemma:}

We want to show $\frac{1}{\sqrt{k}} > 2(\sqrt{k+1} - \sqrt{k})$.

Multiply both sides by $\sqrt{k}(\sqrt{k+1} + \sqrt{k})$:

\textbf{Left side:} $\sqrt{k} \cdot (\sqrt{k+1} + \sqrt{k}) = \sqrt{k} \cdot \sqrt{k+1} + k$

\textbf{Right side:} $2\sqrt{k}(\sqrt{k+1} - \sqrt{k})(\sqrt{k+1} + \sqrt{k}) = 2\sqrt{k}((k+1) - k) = 2\sqrt{k}$

So we need to show: $\sqrt{k} \cdot \sqrt{k+1} + k > 2\sqrt{k}$

Dividing by $\sqrt{k}$: $\sqrt{k+1} + \sqrt{k} > 2$

This simplifies to: $\sqrt{k+1} > 2 - \sqrt{k}$

Since $\sqrt{k} \geq 1$ for $k \geq 1$, we have $2 - \sqrt{k} \leq 1$.
Since $\sqrt{k+1} > \sqrt{k} \geq 1$, the inequality holds.

\textbf{Alternative approach:} We can show this more directly.

Multiply both sides of $\frac{1}{\sqrt{k}} > 2(\sqrt{k+1} - \sqrt{k})$ by $\sqrt{k}(\sqrt{k+1} + \sqrt{k})$:

\begin{itemize}
\item Left: $\sqrt{k+1} + \sqrt{k}$
\item Right: $2\sqrt{k}$
\end{itemize}

So we need: $\sqrt{k+1} + \sqrt{k} > 2\sqrt{k}$

This gives: $\sqrt{k+1} > \sqrt{k}$ $\checkmark$

\textbf{Main Proof:}

Using the lemma, for each $k = 1, 2, \ldots, n$:
$$\frac{1}{\sqrt{k}} > 2(\sqrt{k+1} - \sqrt{k})$$

Summing from $k = 1$ to $n$:
$$\sum_{k=1}^n \frac{1}{\sqrt{k}} > \sum_{k=1}^n 2(\sqrt{k+1} - \sqrt{k})$$

The right side is a \textbf{telescoping sum}:
\begin{align}
\sum_{k=1}^n 2(\sqrt{k+1} - \sqrt{k}) &= 2[(\sqrt{2} - \sqrt{1}) + (\sqrt{3} - \sqrt{2}) + \cdots + (\sqrt{n+1} - \sqrt{n})]\\
&= 2(\sqrt{n+1} - \sqrt{1})\\
&= 2(\sqrt{n+1} - 1)
\end{align}

Therefore:
$$1 + \frac{1}{\sqrt{2}} + \frac{1}{\sqrt{3}} + \cdots + \frac{1}{\sqrt{n}} > 2(\sqrt{n+1} - 1)$$

\textbf{Intuitive Explanation:}

The inequality works because:

\begin{enumerate}
\item Each term $\frac{1}{\sqrt{k}}$ on the left is strictly greater than the corresponding "increment" $2(\sqrt{k+1} - \sqrt{k})$ in the telescoping sum on the right

\item When we sum all these strict inequalities, we get the desired result

\item The geometric intuition is that the sum of reciprocals of square roots grows faster than twice the difference of consecutive square roots

\item This reflects the fact that $\frac{1}{\sqrt{x}}$ decreases more slowly than the rate at which $\sqrt{x}$ increases
\end{enumerate}

The key insight is recognizing that $\frac{1}{\sqrt{k}}$ can be bounded below by the increment $2(\sqrt{k+1} - \sqrt{k})$, which allows us to use the powerful telescoping sum technique.

\subsection*{Detailed Explanation: Alternative Proof Algebra}

Let me break down the algebraic step more carefully.

\textbf{Goal:} We want to prove $\frac{1}{\sqrt{k}} > 2(\sqrt{k+1} - \sqrt{k})$

\textbf{Step 1:} Multiply both sides by $\sqrt{k}(\sqrt{k+1} + \sqrt{k})$

This is a clever choice because $(\sqrt{k+1} + \sqrt{k})$ is the \textbf{conjugate} of $(\sqrt{k+1} - \sqrt{k})$.

\textbf{Step 2:} Work out the Left Hand Side (LHS)

\begin{align}
\text{LHS} &= \frac{1}{\sqrt{k}} \times \sqrt{k}(\sqrt{k+1} + \sqrt{k})\\
&= \frac{1}{\sqrt{k}} \times \sqrt{k} \times (\sqrt{k+1} + \sqrt{k})\\
&= 1 \times (\sqrt{k+1} + \sqrt{k}) \quad \text{[since $\frac{1}{\sqrt{k}} \times \sqrt{k} = 1$]}\\
&= \sqrt{k+1} + \sqrt{k}
\end{align}

\textbf{Step 3:} Work out the Right Hand Side (RHS) - The tricky part!

\begin{align}
\text{RHS} &= 2(\sqrt{k+1} - \sqrt{k}) \times \sqrt{k}(\sqrt{k+1} + \sqrt{k})\\
&= 2\sqrt{k} \times (\sqrt{k+1} - \sqrt{k}) \times (\sqrt{k+1} + \sqrt{k})
\end{align}

Now I need to expand $(\sqrt{k+1} - \sqrt{k}) \times (\sqrt{k+1} + \sqrt{k})$

This is a \textbf{difference of squares} pattern: $(a - b)(a + b) = a^2 - b^2$

Where $a = \sqrt{k+1}$ and $b = \sqrt{k}$

\begin{align}
(\sqrt{k+1} - \sqrt{k})(\sqrt{k+1} + \sqrt{k}) &= (\sqrt{k+1})^2 - (\sqrt{k})^2\\
&= (k+1) - k\\
&= 1
\end{align}

\textbf{Step 4:} Complete the RHS calculation

\begin{align}
\text{RHS} &= 2\sqrt{k} \times (\sqrt{k+1} - \sqrt{k}) \times (\sqrt{k+1} + \sqrt{k})\\
&= 2\sqrt{k} \times 1 \quad \text{[from Step 3]}\\
&= 2\sqrt{k}
\end{align}

\textbf{Step 5:} Compare LHS and RHS

We now need to show: $\sqrt{k+1} + \sqrt{k} > 2\sqrt{k}$

Subtract $\sqrt{k}$ from both sides: $\sqrt{k+1} > \sqrt{k}$

This is obviously true since $k+1 > k$.

\textbf{The Key Insight}

The magic was using the \textbf{conjugate multiplication} technique. When we have an expression like $(\sqrt{k+1} - \sqrt{k})$, multiplying by its conjugate $(\sqrt{k+1} + \sqrt{k})$ eliminates the square roots through the difference of squares formula:

$$(\sqrt{k+1} - \sqrt{k})(\sqrt{k+1} + \sqrt{k}) = (k+1) - k = 1$$

This is a standard technique for simplifying expressions involving square root differences!


\section*{6}

\textbf{Problem:} There are $3^n$ golden coins, but exactly one of them is a counterfeit and weighs known to be less than the other coins. A balance scale is available.

Show using the principle of induction that one can find the counterfeit coin with $n$ weighings.

\textbf{Solution:}

This is a classic problem that demonstrates the power of divide-and-conquer strategies and mathematical induction.

\textbf{Strategy Overview}

The key insight is to divide the coins into three equal groups at each step, which allows us to eliminate $\frac{2}{3}$ of the coins with each weighing.

\textbf{Proof by Mathematical Induction}

\textbf{Base Case: $n = 1$}

When $n = 1$, we have $3^1 = 3$ coins, with exactly one counterfeit.

\textbf{Strategy:} Place one coin on each side of the balance scale, keeping the third coin aside.

\textbf{Possible outcomes:}
\begin{itemize}
\item If the scales balance: The counterfeit coin is the one set aside
\item If the scales don't balance: The counterfeit coin is on the lighter side
\end{itemize}

\textbf{Result:} We can identify the counterfeit coin with 1 weighing. $\checkmark$

\textbf{Inductive Hypothesis}

Assume that for some $k \geq 1$, we can find the counterfeit coin among $3^k$ coins using $k$ weighings.

\textbf{Inductive Step: Prove for $n = k+1$}

We now have $3^{k+1} = 3 \times 3^k$ coins.

\textbf{Strategy:} Divide the $3^{k+1}$ coins into three equal groups of $3^k$ coins each.

\textbf{Step 1:} Use the balance scale to compare two of these groups (each containing $3^k$ coins).

\textbf{Possible outcomes:}

\textbf{Case 1:} The scales balance
\begin{itemize}
\item This means both groups on the scale contain only genuine coins
\item The counterfeit coin must be in the third group (the one not weighed)
\item We now have $3^k$ coins containing exactly one counterfeit
\end{itemize}

\textbf{Case 2:} The scales don't balance
\begin{itemize}
\item The counterfeit coin is in the lighter group
\item We now have $3^k$ coins containing exactly one counterfeit
\end{itemize}

\textbf{Step 2:} Apply the inductive hypothesis
\begin{itemize}
\item In both cases, we have reduced the problem to finding the counterfeit among $3^k$ coins
\item By the inductive hypothesis, this can be done in $k$ additional weighings
\end{itemize}

\textbf{Total weighings:} $1$ (for the initial division) $+ k$ (by inductive hypothesis) $= k + 1$ weighings

\textbf{Conclusion}

By mathematical induction, we can find the counterfeit coin among $3^n$ coins using exactly $n$ weighings.

\textbf{Why This Works}

The algorithm works because:

\begin{enumerate}
\item \textbf{Information theory:} Each weighing gives us one of three outcomes (left heavier, right heavier, or balanced), providing $\log_3$ information
\item \textbf{Optimal strategy:} With $3^n$ coins, we need exactly $n$ weighings because $\log_3(3^n) = n$
\item \textbf{Divide and conquer:} By dividing into three equal groups, we eliminate the maximum number of possibilities with each weighing
\end{enumerate}

\textbf{Example for $n = 2$ (9 coins)}

\textbf{Weighing 1:} Divide 9 coins into groups A, B, C (3 coins each). Compare A vs B.
\begin{itemize}
\item If balanced: counterfeit in C. Problem reduces to 3 coins $\rightarrow$ 1 more weighing needed
\item If unbalanced: counterfeit in lighter group. Problem reduces to 3 coins $\rightarrow$ 1 more weighing needed
\end{itemize}

\textbf{Weighing 2:} Among the identified group of 3 coins, compare 2 coins.
\begin{itemize}
\item If balanced: third coin is counterfeit
\item If unbalanced: lighter coin is counterfeit
\end{itemize}

\textbf{Total:} 2 weighings for $3^2 = 9$ coins $\checkmark$

\end{document}