% ============================================================================
% Analysis I & II (English Translation Workspace)
% ----------------------------------------------------------------------------
% This file is a structured, bilingual skeleton prepared for gradually
% translating the original two-semester German Analysis script.
%
% HOW TO USE:
%     so it's easy to cross-reference the source.
%  2. Replace the \todo{...} markers with translated content incrementally.
%  3. If you keep some German text temporarily, wrap it in \textit{...} or a
%     quote environment and tag with a \todo{translate} so you can grep later.
%  4. Add mathematical macros below instead of inlining symbols repeatedly.
%  5. For larger proofs, consider adding a commented block at top of the
%     subsection with the original wording for fidelity, then translate.
%
% NOTE: No original copyrighted body text has been inserted—only an outline
%       inferred from the public table of contents you provided.
%
% WORKFLOW SUGGESTION:
%  (a) Copy the German paragraph as a commented block starting with % DE: ...
%  (b) Write the English translation below it.
%  (c) Remove the German once confident, or keep in comments for traceability.
%  (d) Use consistent terminology: "sequence" vs. "series", "limit superior",
%      "least upper bound (supremum)", etc. Maintain a glossary section later.
%  (e) Add labels (\label{sec:...}) as you begin referencing results.
%
% Build recommendation: lualatex or pdflatex twice (for TOC) when needed.
% ============================================================================
\documentclass[12pt,a4paper]{article}

% -------------------- Packages --------------------
\usepackage[utf8]{inputenc}
\usepackage[T1]{fontenc}
\usepackage[english]{babel}
\usepackage{amsmath,amssymb,amsthm}
\usepackage{mathtools}
\usepackage{geometry}
\usepackage{enumitem}
\usepackage{xcolor}
\usepackage[unicode]{hyperref}
\hypersetup{unicode=true,hypertexnames=false}
% Avoid enumitem negative labelwidth warnings by setting a safe leftmargin
\setlist[enumerate]{leftmargin=*,labelsep=0.5em}
% Soften line-breaking to avoid overfull hboxes in long headings/paragraphs
\emergencystretch=2em
\usepackage{microtype}

\geometry{margin=2.5cm}

% -------------------- Metadata --------------------
\iffalse
		itle{Analysis I \& II \newline\small (English Translation Draft)}
\fi
% (replaced malformed title line below)
\csname title\endcsname{Analysis I \& II \newline\small (English Translation Draft)}
\author{Based on original German script (private use)}
\date{\today}

% -------------------- Macros --------------------
\newcommand{\N}{\mathbb{N}}
\newcommand{\Z}{\mathbb{Z}}
\newcommand{\Q}{\mathbb{Q}}
\newcommand{\R}{\mathbb{R}}
\newcommand{\C}{\mathbb{C}}
\newcommand{\K}{\mathbb{K}} % used to denote either R or C
\newcommand{\eps}{\varepsilon}
\newcommand{\dd}{\,\mathrm{d}}
\newcommand{\todo}[1]{\textcolor{red}{[TODO: #1]}}
% Some earlier notes used \odo by mistake; keep it compiling by aliasing:
\newcommand{\odo}[1]{\todo{#1}}

% -------------------- Manual Numbering Helpers --------------------
% Use these when you want to preserve the original numbering from the German
% script. They deliberately avoid coupling to LaTeX's counters.
% Pattern:
%   \NumberedDefinition{<chapter.section.item>}{<Title>}{<Body>}
% If you later want cross-references, add an explicit \label inside #3 and
% reference it normally with \ref. The visual number won't auto-sync if you
% renumber—update the macro argument manually.
% Translator note: Keep using numbered theorems/remarks/examples exactly as in
% the German script (e.g., 4.1.1, 4.1.2, ...). Prefer these helpers where you
% need to mirror source numbering during translation.
\newcommand{\NumberedDefinition}[3]{% #1 number, #2 title, #3 body
\paragraph*{Definition #1 ( #2 ).} #3\par}
\newcommand{\NumberedTheorem}[3]{% #1 number, #2 title, #3 body
\paragraph*{Theorem #1 ( #2 ).} #3\par}
\newcommand{\NumberedProposition}[3]{% #1 number, #2 title, #3 body
\paragraph*{Proposition #1 ( #2 ).} #3\par}
\newcommand{\NumberedLemma}[3]{% #1 number, #2 title, #3 body
\paragraph*{Lemma #1 ( #2 ).} #3\par}
\newcommand{\NumberedCorollary}[3]{% #1 number, #2 title, #3 body
\paragraph*{Corollary #1 ( #2 ).} #3\par}
\newcommand{\NumberedRemark}[3]{% #1 number, #2 title, #3 body
\paragraph*{Remark #1 ( #2 ).} #3\par}
\newcommand{\NumberedExample}[3]{% #1 number, #2 title, #3 body
\paragraph*{Example #1 ( #2 ).} #3\par}

\begin{document}

\maketitle
% Table of contents on its own page, with subsections included
\setcounter{tocdepth}{2}
\renewcommand{\contentsname}{Table of Contents}
\clearpage
\tableofcontents
\clearpage

\section{Basic Concepts (Grundbegriffe)}
\subsection{Logic (Logik)}\label{subsec:logic}
% Placeholder to preserve numbering; content to be translated/added.
	odo{Translate and insert: propositional calculus basics, quantifiers, basic proof methods.}

\subsection{Sets (Mengen)}\label{subsec:sets}
% Placeholder
	odo{Translate and insert: basic set operations, subsets, Cartesian products, maps.}

\subsection{Functions (Funktionen)}\label{subsec:functions}
% Placeholder
	odo{Translate and insert: definition, image/preimage, injective/surjective/bijective, composition.}

\subsection{Relations (Relationen)}\label{subsec:relations}
% Placeholder
	odo{Translate and insert: equivalence relations, order relations, partitions.}

\section{Numbers (Zahlen)}
\subsection{The Natural Numbers (Die natürlichen Zahlen)}\label{subsec:natural-numbers}
% Placeholder
	odo{Translate and insert: Peano axioms (brief), induction, basic properties.}

\subsection{The Rational Numbers (Die rationalen Zahlen)}
\paragraph{Motivation.} In $\N$ neither addition nor multiplication is invertible in general: given $n,m\in\N$ there need not exist $x\in\N$ with $n+x=m$ nor $x\in\N$ with $n\,x=m$. Passing to integers $\Z$ fixes additivity (every $m\in\Z$ has an additive inverse $-m$), but multiplicative inverses still fail to exist except for $\pm1$. We therefore enlarge $\Z$ to a number system where every nonzero element has a multiplicative inverse—the \emph{rational numbers} $\Q$.

\paragraph{Construction of $\Q$ as fractions of $\Z$.} Consider $\Z\times(\Z\setminus\{0\})$ and define an equivalence relation $\sim$ by
\[
		(p_1,q_1) \sim (p_2,q_2) \quad:\iff\quad p_1 q_2 = p_2 q_1.
\]
Let $[(p,q)]$ denote the equivalence class of $(p,q)$; we write $\dfrac{p}{q}$ for $[(p,q)]$ and call it a \emph{fraction}. The set of rational numbers is
\[
		\Q := \big(\Z\times(\Z\setminus\{0\})\big)/\sim = \{\,[(p,q)] : p\in\Z,\ q\in\Z\setminus\{0\}\,\}.
\]
Two pairs represent the same rational number precisely when they are proportional: $(p_1,q_1)\sim(p_2,q_2)$ iff $p_1 q_2 = p_2 q_1$.

\paragraph{Operations.} On representatives define
\[
	[(p,q)] + [(r,s)] := [(ps+rq,\ qs)],\qquad [(p,q)]\cdot[(r,s)] := [(pr,\ qs)].
\]
Negation and (when $p\ne0$) reciprocals are given by
\[
	-[(p,q)] := [(-p,q)],\qquad \big([(p,q)]\big)^{-1} := [(q,p)].
\]
These operations are \emph{well-defined} on equivalence classes (independent of the chosen representatives) and satisfy the usual field axioms.

\paragraph{Field structure and embedding of $\Z$.}
\begin{itemize}[leftmargin=*]
	\item With the above operations, $\Q$ is a field. The additive identity is $0=[(0,1)]$ and the multiplicative identity is $1=[(1,1)]$. Every nonzero element $[(p,q)]\ne0$ (i.e. $p\ne0$) has inverse $[(q,p)]$.
	\item The map $\iota: \Z\to\Q$, $\iota(p)=[(p,1)]$, is an injective ring homomorphism; we thus identify $\Z$ with a subring of $\Q$.
	\item Two fractions are equal iff they cross-multiply: $\dfrac{p}{q}=\dfrac{r}{s}$ in $\Q$ $\iff$ $ps=rq$ in $\Z$.
\end{itemize}
\emph{Proof.} Routine verification; the key point is well-definedness under the relation $\sim$. \qed

\paragraph{Order on $\Q$.} Each rational number admits a representative with positive denominator. Using such representatives define
\[
	\frac{p}{q} < \frac{r}{s} \quad:\iff\quad p s < r q \qquad (q>0,\ s>0).
\]
This yields a total order compatible with the field operations; in particular $\Q$ becomes an \emph{ordered field}. The sign is determined by the product of numerator and denominator: $\dfrac{p}{q}>0$ iff $pq>0$.

% (Removed earlier duplicate Remark 2.2.3 to align with the original numbering.)

\paragraph{Non-uniqueness and reduced form.} A rational number has many representatives: $\dfrac{p}{q}=\dfrac{kp}{kq}$ for any $k\in\Z\setminus\{0\}$. One may choose a unique \emph{reduced} representative by requiring $q>0$ and $\gcd(p,q)=1$.

\NumberedTheorem{2.2.1}{$\Q$ is an ordered field}{With the operations and the order introduced above, $\Q$ is an ordered field. We do not supply a proof here; the intended meaning is that $\Q$ satisfies the field axioms and the order is total and compatible with the arithmetic in the sense that
\begin{itemize}[leftmargin=*]
	\item if $a\le b$ then $a+c\le b+c$ for all $c\in\Q$;
	\item if $0\le a$ and $0\le b$ then $0\le a\,b$.
\end{itemize}}

\iffalse
\NumberedDefinition{2.2.2}{Field (Körper)}{A set $K$ equipped with an addition $+ : K\times K\to K$ and a multiplication $\cdot : K\times K\to K$ is called a \emph{field} if the following hold:
\begin{enumerate}[label={A\arabic*)}, leftmargin=*]

\fi
\subsection{The Real Numbers (Die reellen Zahlen)}\label{subsec:real-numbers}

% Included from external file to keep structure modular and avoid ordering conflicts.
\input{sections/2.3-real-numbers}

\subsection{Countable and Uncountable Sets (Abzählbar und überabzählbar)}\label{subsec:countable}
We collect basic facts on countability (countable unions), Cantor's theorem on power sets, and the uncountability of the reals.
\paragraph{Proof.} For each $n\in\N$ pick an injection (bijectively identify) $B_n\subset \N$ with $A_n$; that is, choose a bijection $\phi_n: B_n\to A_n$ with $B_n\subset \N$. Now set
\[
	B := \{ (n,m)\in \N\times\N : m\in B_n\}.
\]
Then $B\subset \N\times\N$, so $B$ is countable. Define $\phi: B\to \bigcup_{n\ge1} A_n$ by $\phi(n,m)=\phi_n(m)$. The map $\phi$ is surjective, hence the union is countable by Proposition~2.4.5. \qed

\emph{Application.} Let
\[
	S_0 := \{ (a_i)_{i\in\N} : a_i\in\Z \text{ and } (\exists n\in\N)\ (a_i=0 \ \forall i>n)\}
\]
be the set of integer sequences that are eventually zero (i.e. have only finitely many nonzero terms). We claim that $S_0$ is countable. For $n\ge1$ set
\[
	S^{(n)} := \{ (a_i)_{i\in\N} \in S_0 : a_i=0 \ \forall i>n\}.
\]
Each $S^{(n)}$ is in bijection with $\Z^n$ (ignore the tail of zeros), hence countable. Since
\[
	S_0 = \bigcup_{n=1}^{\infty} S^{(n)},
\]
Theorem~2.4.8 implies that $S_0$ is countable.

% ---------------------------------------------------------------------------
% Uncountability (Cantor)
% ---------------------------------------------------------------------------
\NumberedTheorem{2.4.9}{Cantor}{Let $M$ be a set. Then $M$ and its power set $\mathcal{P}(M)$ are not equipotent.}

\paragraph{Proof.} Suppose for contradiction that $\phi: M\to \mathcal{P}(M)$ is a bijection. Consider the set
\[
  B := \{ x\in M : x\notin \phi(x)\}.
\]
Then for every $y\in M$ we have $\phi(y)\ne B$: if $y\in \phi(y)$ then by definition $y\notin B$, so $\phi(y)\ne B$; if $y\notin \phi(y)$ then $y\in B$, again $\phi(y)\ne B$. Hence $B$ is not in the image of $\phi$, contradicting surjectivity. \qed

\NumberedDefinition{2.4.10}{Uncountable set}{A set that is not countable is called \emph{uncountable}. For example, $\mathcal{P}(\N)$ is uncountable.}

\NumberedProposition{2.4.11}{$\R$ is uncountable}{The set $\R$ is uncountable.}

\paragraph{Proof.} The power set $\mathcal{P}(\N)$ is uncountable. Let $\{0,1\}^{\N}$ denote the set of sequences with entries $0$ or $1$. Define a map $\phi: \mathcal{P}(\N) \to \{0,1\}^{\N}$ by the characteristic sequence: $(\phi(X))_i := 1$ if $i\in X$, and $(\phi(X))_i :=0$ otherwise. This map is a bijection, so $\{0,1\}^{\N}$ is uncountable. Now let $X_0\subset [0,1]$ be the subset of real numbers whose decimal expansion uses only the digits $0$ and $1$. Each $x\in X_0$ corresponds to a sequence $(a_i)_{i\ge1}$ with $a_i\in\{0,1\}$ via $x=\sum_{i\ge1} a_i 10^{-i}$. (Choose the expansion not terminating in repeating $9$'s.) Thus $X_0$ is in bijection with $\{0,1\}^{\N}$, hence uncountable. Since $X_0\subset [0,1]\subset \R$, the set $\R$ is uncountable. \qed

From the above proof it follows that $\R$ and $X_0$ are equipotent.

\paragraph{Question.} Is the cardinality of $\R$ the smallest uncountable cardinality? Equivalently, does there exist a subset of $\R$ that is uncountable but not equipotent with $\R$?

\paragraph{Continuum Hypothesis (Cantor).} Every subset of $\R$ is either countable or equipotent with $\R$.

\paragraph{Independence (P. Cohen, 1960).} It is impossible to prove or disprove the Continuum Hypothesis from the standard Zermelo–Fraenkel axioms with the Axiom of Choice (ZFC); it is independent of these axioms.

	odo{Add further examples (evens, odds, pairs by Cantor pairing).}

\subsection{The Complex Numbers (Die komplexen Zahlen)}\label{subsec:complex-numbers}

We now introduce the complex numbers. Since for every $x\in \R$ we have $x^2\ge 0$, the equation
\[
	x^2=-1
\]
has no real solution. We want to construct a smallest possible superset of $\R$ in which this equation becomes solvable. We proceed as follows: we introduce \emph{formally} the imaginary unit via
\[
	i^2=-1,
\]
and define the set of \emph{complex numbers} by
\[
	\C := \{\, x+iy : x,y\in \R \,\}.
\]

We agree that $0\cdot i:=0$, so that $\R$ is contained in $\C$. Addition and multiplication on $\C$ are defined by
\begin{align*}
	(x+iy)+(u+iv) &:= (x+u) + i\,(y+v),\\
	(x+iy)(u+iv) &:= (xu - yv) + i\,(xv + uy).
\end{align*}
From this one obtains, for $x+iy\ne 0$, the division rule
\[
	\frac{u+iv}{x+iy}
	 = \frac{xu+yv}{x^2+y^2} 
		 + i\,\frac{xv-yu}{x^2+y^2},
\]
which follows from $(x+iy)^{-1}=\dfrac{x-iy}{x^2+y^2}$.

\NumberedRemark{2.5.1}{\(\C\) cannot be ordered}{There is no order $\le$ on $\C$ making it an ordered field. Otherwise one would have $z^2\ge 0$ for every $z\in\C$, which contradicts $i^2=-1$.}

\paragraph{Modulus and $|z|^2$.}
For $z=x+iy\in\C$ the \emph{modulus} is $|z|:=\sqrt{x^2+y^2}$. Writing the \emph{conjugate} as $\overline{z}=x-iy$, one has the fundamental identity
\[
  |z|^2 = z\,\overline{z} = x^2+y^2. 
\]
Consequences used frequently:
\begin{itemize}[leftmargin=*]
  \item $|zw|=|z|\,|w|$ and $\overline{zw}=\overline{z}\,\overline{w}$.
  \item $z\neq0 \iff |z|>0$ and $z^{-1}=\overline{z}/|z|^2$.
  \item $|z+w|^2 = |z|^2 + |w|^2 + 2\,\Re( z\overline{w} )$ (parallelogram law), hence $|z+w|\le |z|+|w|$.
  \item $|\Re z|\le |z|$ and $|\Im z|\le |z|$.
\end{itemize}

% ---------------------------------------------------------------------------
% Fixed-number translation: Definition 2.5.2
% ---------------------------------------------------------------------------
\NumberedDefinition{2.5.2}{Basic notions and rules for complex numbers}{
Let $z=x+iy$ with $x,y\in\R$ be a complex number. Then we define
\begin{itemize}[leftmargin=*]
	\item $\Re z := x$ the real part of $z$,
	\item $\Im z := y$ the imaginary part of $z$,
	\item $|z| := \sqrt{x^2+y^2}$ the modulus of $z$,
	\item $\overline{z} := x - iy$ the conjugate of $z$.
\end{itemize}
}

% [Removed misplaced Section 4 continuity content from here]




\section{Sequences and Series (Folgen und Reihen)}
\subsection{Convergent Sequences (Konvergenz von Folgen)}

\NumberedTheorem{3.1.20}{Order of limits and squeeze principle}{
\begin{enumerate}[label={(\arabic*)}, leftmargin=*]
	\item Let $(x_n)_{n\in\N}$ and $(y_n)_{n\in\N}$ be convergent sequences in $\R$ with
	\[
		a := \lim_{n\to\infty} x_n,\qquad b := \lim_{n\to\infty} y_n.
	\]
	Suppose that $x_n\le y_n$ holds for infinitely many $n\in\N$. Then $a\le b$.

	\item Let $(x_n)_{n\in\N}$ and $(y_n)_{n\in\N}$ be convergent sequences in $\R$ with
	\[
		a := \lim_{n\to\infty} x_n = \lim_{n\to\infty} y_n.
	\]
	Let $(z_n)_{n\in\N}$ be a third sequence in $\R$. Assume there exists $n_0\in\N$ such that
	\[
		x_n \le z_n \le y_n \qquad \forall n\ge n_0.
	\]
	Then $(z_n)$ is convergent and $\lim_{n\to\infty} z_n = a$.
\end{enumerate}}

\paragraph{Proof.}
\begin{enumerate}[label={(\arabic*)}, leftmargin=*]
	\item Let $\eps>0$. Since $x_n\to a$ and $y_n\to b$, there exist indices $n_1,n_2\in\N$ such that
	\[
		|x_n-a|<\eps \quad (n\ge n_1), \qquad |y_n-b|<\eps \quad (n\ge n_2).
	\]
	Hence for all $n\ge n_\ast:=\max\{n_1,n_2\}$ we have
	\[
		a-\eps < x_n, \qquad y_n < b+\eps.
	\]
	If $x_n\le y_n$ for some $n\ge n_\ast$, then
	\[
		a-\eps < x_n \le y_n < b+\eps \quad \Longrightarrow \quad a-b \le 2\eps.
	\]
	By hypothesis, there are infinitely many indices with $x_n\le y_n$, hence (in particular) such indices exist beyond $n_\ast$, so the above estimate holds. Since $\eps>0$ is arbitrary, it follows that $a\le b$.

	\item Let $\eps>0$. There exist $n_1,n_2\in\N$ such that
	\[
		|x_n-a|<\eps \quad (n\ge n_1),\qquad |y_n-a|<\eps \quad (n\ge n_2).
	\]
	For all $n\ge n_\eps:=\max\{n_0,n_1,n_2\}$ we obtain
	\[
		a-\eps < x_n \le z_n \le y_n < a+\eps,
	\]
	which implies $|z_n-a|<\eps$. Thus $z_n\to a$, proving both convergence and identification of the limit.
\end{enumerate}
\qed
\NumberedTheorem{3.4.3}{Necessary condition for convergence}{If the series $\sum x_n$ is convergent, then $(x_n)_{n\in\N}$ is a null sequence.}
\paragraph{Proof.}
$(s_n)_{n\in\N}$ is convergent and hence a Cauchy sequence. Thus, for every $\eps>0$ there exists $n_\eps\in\N$ such that for all $n\ge n_\eps$ one has
\[
	|x_n| = |s_n - s_{n-1}| < \eps.
\]
\qed


\NumberedRemark{3.4.4}{}{Example~3.4.2(3) shows that the converse of Theorem~3.4.3 is, in general, false.}

\NumberedRemark{3.4.4}{}{Example~3.4.2(3) shows that the converse of Theorem~3.4.3 is, in general, false.}

\paragraph{Rules for series.}

\NumberedTheorem{3.4.5}{Linearity}{Let $\sum x_n$ and $\sum y_n$ be two convergent series and let $\alpha\in\K$. Then:
\begin{enumerate}[label={(\arabic*)}, leftmargin=*]
	\item $\sum \alpha x_n$ is convergent with sum $\alpha\sum x_n$.
	\item $\sum (x_n+y_n)$ is convergent with sum $(\sum x_n) + (\sum y_n)$.
\end{enumerate}}

\paragraph{Convergence criteria for series.}

\NumberedTheorem{3.4.6}{Cauchy criterion for series}{$\sum x_n$ is convergent if and only if for every $\eps>0$ there exists $n_\eps\in\N$ such that
\[
	\Bigg|\sum_{k=m+1}^{n} x_k\Bigg| < \eps \qquad \text{for all } n>m\ge n_\eps.
\]}
\paragraph{Proof.}
Note that $|s_n-s_m| = \big|\sum_{k=m+1}^{n} x_k\big|$. The claim is therefore exactly the Cauchy criterion for the sequence of partial sums $(s_n)$ (Theorems~3.2.7 and~3.2.9). \qed

\NumberedTheorem{3.4.7}{Series with nonnegative terms}{Let $\sum x_n$ be a series in $\R$ with $x_n\ge0$ for all $n\in\N$. Then the series is convergent if and only if the sequence $(s_n)$ of partial sums is bounded.}
\paragraph{Proof.}
Since $x_n\ge0$ for all $n\in\N$, the sequence $(s_n)_{n\in\N}$ is monotone increasing. The assertion follows from Theorems~3.1.8 and~3.2.2. \qed

\NumberedTheorem{3.4.8}{Leibniz criterion}{Let $(x_n)_{n\in\N}$ be a monotone decreasing sequence with $x_n\ge0$. Then the series $\sum (-1)^n x_n$ is convergent if and only if $(x_n)_{n\in\N}$ is a null sequence.}
\paragraph{Proof.}
From Theorem~3.4.3 we get: if $\sum (-1)^n x_n$ converges, then $((-1)^n x_n)$ is a null sequence, hence $x_n$ is a null sequence.

For the converse direction, assume $(x_n)$ is monotone decreasing with $x_n\ge0$ and $(x_n)$ is a null sequence. Let $(s_n)$ be the partial sums of $\sum (-1)^n x_n$.
\begin{enumerate}[label=\roman*. , leftmargin=*]
	\item Show that $(s_{2n})_{n\in\N}$ is monotone decreasing:
	\[
		s_{2n+2} - s_{2n} = x_{2n+2} - x_{2n+1} \le 0.
	\]
	\item Show that $(s_{2n+1})_{n\in\N}$ is monotone increasing:
	\[
		s_{2n+3} - s_{2n+1} = -x_{2n+2} + x_{2n+3} \ge 0.
	\]
\end{enumerate}
	\begin{enumerate}[label=\roman*. , leftmargin=*]
		\setcounter{enumi}{2}
		\item We have the chain of inequalities
		\[
			s_1 \;\le\; s_{2n+1} \;\le\; s_{2n} \;\le\; s_0.\footnote{An exclamation mark placed over an equality/inequality in the original text indicates that the equality/estimate is still to be shown.}
		\]
		Indeed, (ii) shows $(s_{2n+1})$ is increasing, hence $s_1\le s_{2n+1}$, and (i) shows $(s_{2n})$ is decreasing, hence $s_{2n}\le s_0$. Moreover,
		\[
			s_{2n+1}-s_{2n} = -x_{2n+1} \le 0,
		\]
		so $s_{2n+1}\le s_{2n}$.

		\item Show that the two limits coincide. From Theorem~3.1.19 (algebra of limits for sequences) we get
		\[
			s-t = \lim_{n\to\infty} s_{2n} - \lim_{n\to\infty} s_{2n+1}
					= \lim_{n\to\infty} (s_{2n}-s_{2n+1})
					= \lim_{n\to\infty} x_{2n+1} = 0.
		\]
		Let $\eps>0$. Then there exist $n_1,n_2\in\N$ such that
		\[
			|s_{2n}-s|<\eps \quad (n\ge n_1), \qquad |s_{2n+1}-s|<\eps \quad (n\ge n_2).
		\]
		With $n_\eps := \max\{2n_1,\,2n_2+1\}$ we obtain $|s_n-s|<\eps$ for all $n\ge n_\eps$. Hence $(s_n)$ converges. \qed
	\end{enumerate}

	\NumberedExample{3.4.9}{Alternating harmonic series}{
	\[
		\sum_{n=1}^{\infty} (-1)^{n+1}\,\frac{1}{n}
			= 1 - \frac12 + \frac13 - \frac14 + \frac15 - \cdots \quad (=\; \log 2),
	\]
	and
	\[
		\sum_{n=0}^{\infty} (-1)^n \, \frac{1}{2n+1}
			= 1 - \frac13 + \frac15 - \frac17 + \cdots \quad\Big(=\; \frac{\pi}{4}\Big).
	\]}

 

\NumberedRemark{3.1.21}{Order of limits is not inherited from pointwise inequalities}{The sequences $x_n:=\tfrac{1}{n}$ and $y_n:=-\tfrac{1}{n}$ show that from $x_n>y_n$ for all $n\in\N$ it does \emph{not} follow that
\[
	\lim_{n\to\infty} x_n > \lim_{n\to\infty} y_n.
\]
Indeed $x_n>y_n$ holds for every $n$, yet $\lim x_n = 0 = \lim y_n$.}



\subsection{Completeness (Vollständigkeit)}

In the proofs of this chapter we will occasionally use the (complete) induction principle to establish statements for natural numbers.

\NumberedDefinition{3.2.1}{Monotone sequences}{A sequence $(x_n)_{n\in\N}$ in $\R$ is called \emph{monotone increasing} (resp. \emph{monotone decreasing}) if for all $n\in\N$ we have
\[
	x_{n+1} \ge x_n \quad (\text{resp. } x_{n+1} \le x_n ).
\]}

\NumberedTheorem{3.2.2}{Monotone bounded sequences converge}{Every monotone increasing (resp. monotone decreasing) bounded sequence $(x_n)_{n\in\N}$ in $\R$ is convergent, and moreover
\[
	\lim_{n\to\infty} x_n = \sup\{x_n : n\in\N\} \quad\big(\text{resp. }\ \lim_{n\to\infty} x_n = \inf\{x_n : n\in\N\}\big).
\]}

\paragraph{Proof.} We prove the claim for monotone increasing sequences; the decreasing case is analogous.

Assume $(x_n)$ is monotone increasing and bounded. Then the set $\{x_n : n\in\N\}$ is bounded above, hence it has a supremum $s:=\sup\{x_n : n\in\N\}$ by the least upper bound property of $\R$. Let $\eps>0$. By the definition of the supremum there exists $n_\eps\in\N$ with
\[
	s-\eps < x_{n_\eps} \le s.
\]
By monotonicity this inequality holds for all $n\ge n_\eps$, i.e. $|x_n-s|<\eps$ for all $n\ge n_\eps$. Hence $x_n\to s$.
\qed

In the following we compile some statements about natural numbers that can be proved by complete induction and that often appear as exercises. We need the notation: for $k\in\N$ the factorial is defined recursively by
\[
	0! := 1, \qquad k! := k\,(k-1)! \quad (k=1,2,\dots).
\]
% (deleted: earlier misplaced, unnumbered "4.4 Functions in R" block)
\[
	|x_{k_{\ell_m}} - (a+ib)| \le \big|\Re(x_{k_{\ell_m}}) - a\big| + \big|\Im(x_{k_{\ell_m}}) - b\big| \xrightarrow[m\to\infty]{} 0.
\]
Hence $x_{k_{\ell_m}}\to a+ib$ and $a+ib\in\C$ is an accumulation point of $(x_k)$. This concludes the proof. \qed

\NumberedDefinition{3.2.6}{Cauchy sequence}{A sequence $(x_n)_{n\in\N}$ in $\K$ is called a \emph{Cauchy sequence} (abbrev. C.S.) if for every $\eps>0$ there exists $n_0\in\N$ such that
\[
	|x_n - x_m| \le \eps \qquad \forall\, n,m\ge n_0.
\]}

\NumberedTheorem{3.2.7}{Every convergent sequence is Cauchy}{Every convergent sequence in $\K$ is a Cauchy sequence.}

\paragraph{Proof.} Let $\eps>0$ and suppose $x_n\to a$ in $\K$. Then there exists $n_\eps\in\N$ such that $|x_n-a|<\eps/2$ for all $n\ge n_\eps$. For $n,m\ge n_\eps$ we have
\[
	|x_n-x_m| = |(x_n-a)-(x_m-a)| \le |x_n-a| + |x_m-a| < \eps/2+\eps/2 = \eps.
\]
Thus $(x_n)$ is Cauchy. \qed

\NumberedTheorem{3.2.8}{Every Cauchy sequence is bounded}{Every Cauchy sequence in $\K$ is bounded.}

\paragraph{Proof.} By assumption there exists $n_1\in\N$ such that $|x_n-x_m|<1$ for all $n,m\ge n_1$. Define
\[
	C := \max\{ |x_n|+1 : n\le n_1\}.
\]
Then for $n\le n_1$ we have $|x_n|\le C$ by definition, and for $n>n_1$ we estimate
\[
	|x_n| = |x_n-x_{n_1}+x_{n_1}| \le |x_n-x_{n_1}| + |x_{n_1}| < 1 + |x_{n_1}| \le C.
\]
Hence $(x_n)$ is bounded. \qed

\NumberedTheorem{3.2.9}{Cauchy implies convergence in $\K$}{Every Cauchy sequence in $\K$ is convergent.}

\paragraph{Proof.} By Theorem~3.2.8 a Cauchy sequence $(x_n)$ is bounded. Therefore, by Bolzano--Weierstrass (Theorem~3.2.5), $(x_n)$ has an accumulation point $a\in\K$. We claim that $x_n\to a$.

Let $\eps>0$. Since $(x_n)$ is Cauchy, there exists $n_\eps\in\N$ such that $|x_n-x_m|<\eps/2$ for all $n,m\ge n_\eps$. Moreover, because $a$ is an accumulation point, there exists an index $k_\eps\ge n_\eps$ with $|x_{k_\eps}-a|<\eps/2$. Then for all $n\ge n_\eps$ we have
\[
	|x_n-a| \le |x_n-x_{k_\eps}| + |x_{k_\eps}-a| < \eps/2 + \eps/2 = \eps.
\]
Thus $x_n\to a$, proving convergence. \qed

\paragraph{Note.} In the proof of Theorem~3.2.9, the crucial external input was the Bolzano--Weierstrass theorem. The proof of Bolzano--Weierstrass itself used that in $\K$ (via $\R$) bounded monotone sequences converge to the supremum/infimum of their ranges. Equivalently: $\R$ is order complete; i.e., every nonempty bounded subset has a supremum/infimum.

\NumberedExample{3.2.10}{Cauchy in $\Q$ without $\Q$-limit}{There exist Cauchy sequences in $\Q$ which do not converge in $\Q$. For instance, construct rational numbers that approximate $\sqrt{2}$ by bisection on $[1,2]$:
\begin{itemize}[leftmargin=*]
	\item Start with the interval $I_0=[1,2]$. Its midpoint $m_0=3/2$ satisfies $m_0^2=2.25>2$, hence the root of $x^2-2=0$ lies in $[1,3/2]$.
	\item Inductively, having $I_n=[a_n,b_n]$ with $a_n,b_n\in\Q$ and $a_n^2\le 2\le b_n^2$, let $m_n:=\tfrac{a_n+b_n}{2}$. If $m_n^2\le 2$ set $I_{n+1}=[m_n,b_n]$, else set $I_{n+1}=[a_n,m_n]$.
\end{itemize}
Then $a_n,b_n\in\Q$, $I_{n+1}\subset I_n$, and $|I_n|=b_n-a_n=2^{-n}(b_0-a_0)=2^{1-n}$, so $(a_n)$ and $(b_n)$ are Cauchy sequences in $\R$. Any choice of a point $x_n\in I_n\cap\Q$ defines a rational sequence $(x_n)$ with $|x_n-x_m|\le |I_{\min\{n,m\}}|\to 0$; hence $(x_n)$ is Cauchy.

However, in $\Q$ the sequence cannot converge, because its (real) limit must be the unique number $c\in[1,2]$ with $c^2=2$, namely $c=\sqrt{2}\notin\Q$ (cf. Remark~2.3.3). Thus $(x_n)$ is Cauchy in $\Q$ but has no limit in $\Q$; viewed in $\R$ it converges to $\sqrt{2}$.}

\NumberedRemark{3.2.11}{Completion viewpoint}{The real numbers $\R$ can be obtained from $\Q$ by \emph{completion}: one considers the set of Cauchy sequences in $\Q$ and identifies two such sequences if their difference is a null sequence; the resulting equivalence classes form an ordered field that is order complete and isomorphic to $\R$ (Theorem~2.3.4). Under this identification, a rational $q\in\Q$ corresponds to the constant sequence $(q,q,\dots)$, while an irrational such as $\sqrt{2}$ is represented by any rational Cauchy sequence that converges to it in $\R$ (e.g. the bisection sequence above).}

\subsection{Improper Convergence (Uneigentliche Konvergenz)}
\NumberedDefinition{3.3.1}{Extended reals and improper limits}{Let $\overline{\R}:=\R\cup\{-\infty,+\infty\}$ be the set of \emph{extended real numbers}, ordered by
\[
	-\infty < x < +\infty \quad (\forall x\in\R).
\]
For a sequence $(x_n)$ in $\R$ we define:
\begin{itemize}[leftmargin=*]
	\item $x_n \to +\infty$ if for every $M\in\R$ there exists $n_0\in\N$ such that $x_n\ge M$ for all $n\ge n_0$.
	\item $x_n \to -\infty$ if for every $M\in\R$ there exists $n_0\in\N$ such that $x_n\le -M$ for all $n\ge n_0$ (equivalently: $(-x_n)\to +\infty$).
	\item We say $x_n$ \emph{diverges to infinity} and write $x_n\to \infty$ if either $x_n\to +\infty$ or $x_n\to -\infty$; when the sign matters we specify $+\infty$ or $-\infty$.
\end{itemize}
Thus a (properly) convergent sequence in $\R$ is one that converges in the usual sense to a point of $\R$; an \emph{improperly convergent} sequence is one that converges in $\overline{\R}$ to $\pm\infty$.
}

\paragraph{Remarks and basic facts.}
\begin{itemize}[leftmargin=*]
	\item If $(x_n)$ is monotone increasing and unbounded above, then $x_n\to +\infty$. If it is monotone decreasing and unbounded below, then $x_n\to -\infty$.
	\item If $x_n\to +\infty$ and $a>0$, then $a\,x_n\to +\infty$. If $x_n\to +\infty$ and $y_n\ge c>0$ eventually, then $x_n y_n\to +\infty$.
	\item If $x_n\to a\in\R$ and $y_n\to +\infty$, then $x_n+y_n\to +\infty$.
	\item Improper limits obey the same order rule as in Theorem~3.1.20(1): if $x_n\le y_n$ for infinitely many $n$ and $x_n\to +\infty$, then $y_n\to +\infty$.
\end{itemize}

\NumberedDefinition{3.3.2}{Arithmetic in $\overline{\R}$}{We extend the usual operations on $\R$ to the extended reals $\overline{\R}=\R\cup\{-\infty,+\infty\}$ by the following rules (whenever the expressions make sense): for $x\in\R$,
\[
	x + (+\infty) := +\infty \ (\forall x> -\infty),\qquad x + (-\infty) := -\infty \ (\forall x< +\infty),
\]
\[
	x - (+\infty) := -\infty \ (\forall x< +\infty),\qquad x - (-\infty) := +\infty \ (\forall x> -\infty),
\]
\[
	x\cdot (+\infty) := \begin{cases}
		+\infty,& x>0,\\[-2pt]
		-\infty,& x<0,
	\end{cases}
	\qquad
	x\cdot (-\infty) := -\,x\cdot(+\infty),
\]
\[
	\frac{x}{+\infty} := 0,\quad \frac{x}{-\infty} := 0 \qquad (\forall x\in\R),
\]
\[
	\frac{+\infty}{x} := \begin{cases}
		+\infty,& x>0,\\[-2pt]
		-\infty,& x<0,
	\end{cases}
	\qquad
	\frac{-\infty}{x} := -\,\frac{+\infty}{x},
\]
\[
	\frac{x}{0} := \begin{cases}
		+\infty,& x>0,\\[-2pt]
		-\infty,& x<0.
	\end{cases}
\]
Undefined expressions (such as $\infty-\infty$, $0\cdot\infty$, $\tfrac{\infty}{\infty}$, $\tfrac{0}{0}$, etc.) are intentionally left without a value.}

\NumberedRemark{3.3.3}{Suprema/infima in $\overline{\R}$}{Every subset of $\R$ has both an infimum and a supremum \emph{in} $\overline{\R}$. For example, if $A\subset\R$ is unbounded above, then $\sup_{\overline{\R}} A=+\infty$; if $A$ is unbounded below, then $\inf_{\overline{\R}} A=-\infty$.}

\NumberedDefinition{3.3.4}{Limits and cluster points at $\pm\infty$}{Let $(x_n)$ be a sequence in $\R$. We say
\begin{itemize}[leftmargin=*]
	\item $x_n\to +\infty$ (resp. $x_n\to -\infty$) if for every $R\in\R_{\ge0}$ there exists $n_R\in\N$ such that $x_n>R$ (resp. $x_n<-R$) for all $n\ge n_R$.
	\item $a=+\infty$ (resp. $a=-\infty$) is a \emph{cluster point} of $(x_n)$ if for every $R\ge 0$ and every $n\in\N$ there exists $n_R\ge n$ with $x_{n_R}>R$ (resp. $x_{n_R}< -R$).
\end{itemize}}

\NumberedExample{3.3.5}{Improper limits and cluster points}{
\begin{enumerate}[label={(\arabic*)}, leftmargin=*]
	\item $\displaystyle \lim_{n\to\infty} n = +\infty$.
	\item $\displaystyle \lim_{n\to\infty} (-2n) = -\infty$.
	\item The sequence $x_n:=(-1)^n n$ does not converge in $\overline{\R}$, but it has the two cluster points $+\infty$ and $-\infty$ (even indices $\to +\infty$, odd indices $\to -\infty$).
\end{enumerate}}

\NumberedTheorem{3.3.6}{Reciprocals and improper limits}{Let $(x_n)$ be a sequence in $\R$. Then:
\begin{enumerate}[label={(\arabic*)}, leftmargin=*]
	\item If $x_n\to +\infty$ or $x_n\to -\infty$, then $\dfrac{1}{x_n}\to 0$.
	\item If $x_n\to 0$ and $x_n>0$ for all but finitely many $n$, then $\dfrac{1}{x_n}\to +\infty$. If $x_n\to 0$ and $x_n<0$ for all but finitely many $n$, then $\dfrac{1}{x_n}\to -\infty$.
\end{enumerate}}

\paragraph{Proof.}
\begin{enumerate}[label={(\arabic*)}, leftmargin=*]
	\item Suppose $x_n\to +\infty$. Let $\eps>0$. Choose $n_\eps$ with $x_n>1/\eps$ for all $n\ge n_\eps$. Then $0<\tfrac{1}{x_n}<\eps$ for all $n\ge n_\eps$, hence $\tfrac{1}{x_n}\to 0$. The case $x_n\to -\infty$ is analogous using $|1/x_n|$.
	\item Assume $x_n\to 0$ and $x_n>0$ for all but finitely many $n$. Let $R>0$. There exists $n_R$ such that $|x_n|<1/R$ for all $n\ge n_R$. Possibly increasing $n_R$ we may assume $x_n>0$ for all $n\ge n_R$. Then $\tfrac{1}{x_n}>R$ for all $n\ge n_R$, proving $\tfrac{1}{x_n}\to +\infty$. The case with $x_n<0$ eventually is analogous and yields $\tfrac{1}{x_n}\to -\infty$.
\end{enumerate}\qed

\NumberedRemark{3.3.7}{Undefined forms in $\overline{\R}$ and caution}{Using the symbols $+\infty$ and $-\infty$ is often practical, but $\overline{\R}$ is not a field, so several algebraic rules from $\R$ no longer hold. In particular, the following combinations are \emph{not defined}:
\[
	\frac{\infty}{\infty},\ \frac{-\infty}{-\infty},\ \frac{-\infty}{\infty},\ \frac{\infty}{-\infty},\quad
	-\infty + \infty,\ \infty + (-\infty),\quad 0\cdot(+\infty),\ 0\cdot(-\infty).
\]
If such expressions arise during manipulations, one has reached a dead end and should reformulate to avoid them. Moreover, some rules for limits fail in the improper setting. For example, with $a_n=n$ and $b_n=-2n$ we have $a_n\to +\infty$ and $b_n\to -\infty$, but the sum sequence $c_n=a_n+b_n=-n$ converges to $-\infty$, while the `sum of limits' $+\infty+(-\infty)$ is not defined. Similar issues occur for quotients of sequences that tend to $\pm\infty$.}

\NumberedTheorem{3.3.8}{Monotone sequences converge in $\overline{\R}$}{Every monotone increasing (resp. monotone decreasing) sequence $(x_n)_{n\in\N}$ in $\R$ converges in $\overline{\R}$, and
\[
	\lim_{n\to\infty} x_n = \sup\{x_n: n\in\N\}\quad \big(\text{resp. }\ \lim_{n\to\infty} x_n = \inf\{x_n: n\in\N\}\big).
\]
\paragraph{Proof.} Consider the increasing case; the decreasing case is analogous. If $x_n=-\infty$ for all $n$, then clearly $\lim x_n=-\infty$ and $\sup\{x_n:n\in\N\}=-\infty$. Otherwise, choose $n_0\in\N$ with $x_{n_0}>-\infty$ and examine the tail $(x_n)_{n\ge n_0}$. If it is bounded above in $\R$, then by Theorem~3.2.2 it converges in $\R$ to its supremum. If it is not bounded above, then for every $R>0$ there exists $n_R\ge n_0$ with $x_{n_R}>R$, and monotonicity gives $x_n>R$ for all $n\ge n_R$, hence $x_n\to +\infty$. In either case the limit in $\overline{\R}$ equals the supremum. \qed}

\paragraph{Tail suprema and infima.} Let $(x_n)_{n\in\N}$ be a sequence in $\R$ and define, for $n\in\N$,
\[
	y_n := \sup\{ x_k : k\ge n\}, \qquad z_n := \inf\{ x_k : k\ge n\}.
\]
Then $(y_n)$ is monotone decreasing in $\overline{\R}$ and $(z_n)$ is monotone increasing in $\overline{\R}$. By Theorem~3.3.8, both $(y_n)$ and $(z_n)$ converge in $\overline{\R}$.

\NumberedDefinition{3.3.9}{Limsup and liminf}{Let $(x_n)_{n\in\N}$ be a sequence in $\R$. Then
\[
	\limsup_{n\to\infty} x_n := \lim_{n\to\infty} \sup\{x_k : k\ge n\},
	\qquad
	\liminf_{n\to\infty} x_n := \lim_{n\to\infty} \inf\{x_k : k\ge n\}.
\]}

\NumberedRemark{3.3.10}{}{Because $\sup\{x_k : k\ge n\} \ge \inf\{x_k : k\ge n\}$ holds for every $n$, we always have
\[
	\limsup_{n\to\infty} x_n \;\ge\; \liminf_{n\to\infty} x_n.
\]}

\NumberedTheorem{3.3.11}{Limsup/liminf and cluster points}{Let $(x_n)_{n\in\N}$ be a sequence in $\R$. Then $\limsup_{n\to\infty} x_n$ (resp. $\liminf_{n\to\infty} x_n$) is the greatest (resp. smallest) cluster point in $\overline{\R}$ of $(x_n)$.}

\paragraph{Proof.} Set $x := \limsup_{n\to\infty} x_n$ and define $y_n := \sup\{x_k : k\ge n\}$. Then $(y_n)$ is monotone decreasing and $y_n\to x$.

\emph{Case 1: $x=+\infty$.} Since $y_n\to +\infty$ and $(y_n)$ is decreasing, we have $\forall R>0\ \forall n\in\N\ \exists k>n:\ x_k>R$ (because $+\infty$ is the least upper bound of the tail sets $\{x_k:k\ge n\}$). Thus $+\infty$ is a cluster point of $(x_n)$ in $\overline{\R}$, and evidently there is no larger cluster point.

\emph{Case 2: $x\in\R$.} First we show that no cluster point $z$ of $(x_n)$ is greater than $x$. Let $\eps>0$. Since $y_n\downarrow x$, there exists $n_\eps$ with $x\le y_n < x+\eps$ for all $n\ge n_\eps$. Hence
\[
	\#\{k\in\N : x_k \ge x+\eps\} \le n_\eps < \infty.
\]
Therefore every cluster point $z$ satisfies $z\le x+\eps$, and since $\eps>0$ is arbitrary, $z\le x$.

Next we show that $x$ is a cluster point of $(x_n)$. Let $\eps>0$ and $n\in\N$. Since $x\le y_n < x+\eps$ and $y_n$ is the least upper bound of $\{x_k:k\ge n\}$, there exists $k\ge n$ with
\[
	x \le x_k < x+\eps.
\]
Thus $x$ is indeed a cluster point. Hence $x$ is the greatest cluster point.

\emph{Case 3: $x=-\infty$.} Let $R\in\R_{\ge0}$. Then there exists $n_R\in\N$ such that $y_n<-R$ for all $n\ge n_R$. It follows that
\[
	\#\{k\in\N : x_k \ge -R\} \le n_R < \infty.
\]
Therefore any cluster point $z$ must satisfy $z\le -\infty$, i.e. $z=-\infty$. Moreover, for every $R\in\R_{\ge0}$ and every $n\in\N$ there exists $k\ge n$ with $x_k<-R$, proving that $-\infty$ is a cluster point. This shows that the smallest cluster point equals $-\infty$ in this case.

The proof for $\liminf_{n\to\infty} x_n$ is analogous. \qed

\NumberedExample{3.3.12}{}{Let $x_n := (-1)^n\,\dfrac{n}{n+1}$. Then
\[
	\limsup_{n\to\infty} x_n = 1
	\qquad\text{and}\qquad
	\liminf_{n\to\infty} x_n = -1.
\]
Indeed, with $y_n := \sup\{x_k:k\ge n\}$ and $z_n := \inf\{x_k:k\ge n\}$ we have $y_n=1$ for all $n\in\N$ and $z_n=-1$ for all $n\in\N$.}

\subsection{Series (Reihen)}

\NumberedDefinition{3.4.1}{Series and partial sums}{Let $(x_n)_{n\in\N}$ be a sequence in $\K$. We define the $n$th partial sum $s_n$ by
\[
	s_n := \sum_{k=0}^{n} x_k.
\]
The sequence $(s_n)_{n\in\N}$ is called the \emph{series} associated to $(x_n)$ and is denoted for short by the symbol
\[
	\sum_{k} x_k.
\]
The series is called \emph{convergent} (resp. \emph{divergent}) if the sequence $(s_n)_{n\in\N}$ is convergent (resp. divergent) in $\K$.\footnote{In this chapter we consider convergent sequences and series in $\K$, not improper convergence.}
If $s_n \to s\in\K$, we write
\[
	s = \sum_{n=0}^{\infty} x_n.
\]}

\NumberedExample{3.4.2}{}{\begin{enumerate}[label={(\arabic*)}, leftmargin=*]
	\item $\displaystyle \sum_{n\ge 1} \frac{1}{n!}$ is convergent; in fact $e = 1 + 1 + \tfrac12 + \tfrac16 + \tfrac{1}{24} + \tfrac{1}{120} + \tfrac{1}{720} + \cdots$.

	\item $\displaystyle \sum_{n\ge 1} \frac{1}{n^2} = \Big(1 + \tfrac{1}{4} + \tfrac{1}{9} + \tfrac{1}{16} + \tfrac{1}{25} + \cdots\Big)$ is convergent (indeed to $\pi^2/6$).

	\item $\displaystyle \sum_{n\ge 1} \frac{1}{n} = \big(1 + \tfrac{1}{2} + \tfrac{1}{3} + \tfrac{1}{4} + \tfrac{1}{5} + \cdots\big)$ is divergent (the harmonic series).

	\item The geometric series $\sum a^n = (1 + a + a^2 + a^3 + \cdots)$ is convergent exactly when $|a|<1$. In that case
	\[
		\sum_{n=0}^{\infty} a^n = \frac{1}{1-a}.
	\]
\end{enumerate}}

\paragraph{Proof.}
\begin{enumerate}[label={(\arabic*)}, leftmargin=*]
	\item See Example~3.2.4(6).

	\item Let $(s_n)_{n\in\N}$ with $s_n := \sum_{k=1}^{n} \tfrac{1}{k^2}$. Then $(s_n)$ is monotone increasing. Moreover, for $n\ge2$,
	\begin{align*}
		s_n &= 1 + \sum_{k=2}^{n} \frac{1}{k^2}
			\;\le\; 1 + \sum_{k=2}^{n} \frac{1}{k(k-1)}
			\;=\; 1 + \sum_{k=2}^{n} \Big( \frac{1}{k-1} - \frac{1}{k} \Big) \\
			&= 1 + \sum_{k=2}^{n} \frac{1}{k-1} - \sum_{k=2}^{n} \frac{1}{k}
			\;=\; 1 + \sum_{\ell=1}^{n-1} \frac{1}{\ell} - \sum_{\ell=2}^{n} \frac{1}{\ell}
			\;=\; 1 + 1 - \frac{1}{n}
			\;\le\; 2.
	\end{align*}
	Thus $(s_n)$ is monotone increasing and bounded, hence convergent.

	\item For $n\in\N$ one has
	\[
		s_{2n} - s_n 
			= \sum_{k=n+1}^{2n} \frac{1}{k}
			\;\ge\; \sum_{k=n+1}^{2n} \frac{1}{2n}
			= n\cdot\frac{1}{2n}
			= \frac{1}{2}.
	\]
	Hence $(s_n)_{n\in\N}$ is not a Cauchy sequence and therefore, by Theorem~3.2.7, not convergent.

	\item For $a=1$ we have $s_n = \sum_{k=0}^{n} 1^k = n+1$, which diverges. For $a\ne1$, by Lemma~3.2.3(3),
	\[
		s_n = \sum_{k=0}^{n} a^k = \frac{1-a^{n+1}}{1-a}.
	\]
	Since for $a\ne1$ the sequence $a^{n+1}$ converges if and only if $|a|<1$ (Example~3.2.4(1)), the claim follows.
\end{enumerate}
\qed
\subsection{Absolute Convergence (Absolute Konvergenz)}

\NumberedDefinition{3.5.1}{Absolute convergence}{A series $\sum x_n$ is called \emph{absolutely convergent} if the series $\sum |x_n|$ is convergent. A convergent series that is not absolutely convergent is called \emph{conditionally convergent}.}

\NumberedTheorem{3.5.2}{Absolutely convergent implies convergent}{Every absolutely convergent series converges.}
\paragraph{Proof.}
Suppose $\sum x_n$ is absolutely convergent and let $\eps>0$. The series $\sum |x_n|$ satisfies the Cauchy criterion: there exists $n_\eps\in\N$ such that for all $n>m\ge n_\eps$,
\[
	\sum_{k=m+1}^{n} |x_k| < \eps.
\]
By the triangle inequality,
\[
	\Bigg|\sum_{k=m+1}^{n} x_k\Bigg| \le \sum_{k=m+1}^{n} |x_k| < \eps.
\]
Thus $\sum x_k$ also satisfies the Cauchy criterion and is therefore convergent. \qed

\NumberedRemark{3.5.3}{}{The converse of Theorem~3.5.2 is, in general, false. Example: $x_n := (-1)^{n+1}\,\tfrac{1}{n}$. Then $\sum x_n$ is the alternating harmonic series and convergent, but $\sum |x_n|$ is the harmonic series and divergent.}

\paragraph{Criteria for absolutely convergent series.}

\NumberedTheorem{3.5.4}{Comparison (majorant) test}{Let $(x_n)_{n\in\N}$ be a sequence in $\K$. Suppose there exists a sequence $(\alpha_n)_{n\in\N}$ in $\R_{\ge0}$ with the properties
\begin{itemize}
	\item $\sum \alpha_n$ is convergent,
	\item there exists $n_0\in\N$ such that $|x_n|\le \alpha_n$ for all $n\ge n_0$.
\end{itemize}
Then the series $\sum x_n$ is absolutely convergent.}
\paragraph{Proof.}
Let $\eps>0$. Then there exists $n_\eps\in\N$ with
\[
	\sum_{k=m+1}^{n} \alpha_k < \eps \qquad (\forall\, n>m\ge n_\eps).
\]
Set $n'_\eps := \max\{n_\eps, n_0\}$. For $n>m\ge n'_\eps$ we obtain
\[
	\sum_{k=m+1}^{n} |x_k| \le \sum_{k=m+1}^{n} \alpha_k < \eps.
\]
Hence $\sum |x_k|$ satisfies the Cauchy criterion and is therefore convergent; i.e. $\sum x_n$ is absolutely convergent. \qed

\NumberedExample{3.5.5}{}{Let $k\ge2$. From Example~3.4.2(2) and the comparison test it follows that $\sum_{n\ge1} \dfrac{1}{n^k}$ is absolutely convergent.}

\NumberedTheorem{3.5.6}{Root test}{Let
\[
	\alpha := \limsup_{n\to\infty} \sqrt[n]{|x_n|}.
\]
If $\alpha<1$, then $\sum x_n$ is absolutely convergent. If $\alpha>1$, then $\sum x_n$ is divergent. If $\alpha=1$, no conclusion is possible in general.}
\paragraph{Proof.}
\begin{enumerate}[label={(\arabic*)}, leftmargin=*]
	\item Assume $\alpha<1$ and choose $q$ with $\alpha<q<1$. By the definition of $\limsup$ there exists $n_q\in\N$ such that $\sqrt[n]{|x_n|}\le q$ for all $n\ge n_q$. Thus $|x_n|\le q^n$ for $n\ge n_q$, so $\sum |x_n|$ has a convergent geometric majorant $\sum q^n$ and is therefore absolutely convergent.
	\item Assume $\alpha>1$. Then for some $\delta>0$ there are infinitely many $n$ with $\sqrt[n]{|x_n|}\ge 1+\delta \ge 1$, hence $|x_n|\ge1$ for infinitely many $n$. Therefore $(x_n)$ is not a null sequence and by Theorem~3.4.3 the series $\sum x_n$ diverges.
\end{enumerate}
\qed

\NumberedTheorem{3.5.7}{Ratio test}{Assume there exists $k_0\in\N$ such that $x_n\ne0$ for all $n\ge k_0$.
\begin{enumerate}[label={(\arabic*)}, leftmargin=*]
	\item If
	\[
		\limsup_{n\to\infty,\, n\ge k_0} \Bigg|\frac{x_{n+1}}{x_n}\Bigg| < 1,
	\]
	then $\sum x_n$ is absolutely convergent.
	\item If there exists $k_1\ge k_0$ with
	\[
		\Bigg|\frac{x_{n+1}}{x_n}\Bigg| > 1 \qquad \forall\, n\ge k_1,
	\]
	then $\sum x_n$ is divergent.
\end{enumerate}}
\paragraph{Proof.}
\begin{enumerate}[label={(\arabic*)}, leftmargin=*]
	\item Since the limsup of $\big|\tfrac{x_{n+1}}{x_n}\big|$ is smaller than $1$, choose $q\in(0,1)$ and $n_q\ge k_0$ with
	\[
		\Bigg|\frac{x_{n+1}}{x_n}\Bigg| \le q \qquad (n\ge n_q)
	\]
	(cf. Theorem~3.3.11 on limsup). Then $|x_{n_q+1}|\le q|x_{n_q}|$, and by induction one shows
	\[
		|x_n| \le q^{\,n-n_q}\,|x_{n_q}| = \underbrace{\frac{|x_{n_q}|}{q^{n_q}}}_{=:c}\, q^{\,n} = c\,q^{\,n} \qquad (n\ge n_q).
	\]
	Since $\sum c\,q^{\,n}$ converges for $q<1$, we have found a convergent majorant; hence $\sum x_n$ is absolutely convergent.
	\item If $\big|\tfrac{x_{n+1}}{x_n}\big|>1$ for all $n\ge k_1$, then $|x_n|\ge |x_{k_1}|>0$ for all $n\ge k_1$. Thus $(x_n)$ is not a null sequence and $\sum x_n$ diverges by Theorem~3.4.3.
\end{enumerate}
\qed

% ============================================================================
% (Removed misplaced standalone Chapter: "Normed Vector Spaces")
% All "Normed Vector Spaces" content belongs as Section 4.1 under
% "Continuous Functions (Stetige Funktionen)", matching the German TOC.


\NumberedExample{3.5.8}{}{\begin{enumerate}[label={(\arabic*)}, leftmargin=*]
	\item $\displaystyle \sum \frac{n^2}{2^n} = 0 + \frac12 + \frac14 + \frac{4}{8} + \frac{9}{16} + \frac{16}{16} + \frac{25}{32} + \frac{36}{64} + \cdots$ is absolutely convergent (sum equal to $6$).
	\item $\displaystyle \sum_{n=0}^{\infty} \Big(\tfrac12\Big)^{n+(-1)^n} = \tfrac12 + 1 + \tfrac18 + \tfrac14 + \tfrac{1}{32} + \tfrac{1}{16} + \cdots$ is absolutely convergent.
	\item $\displaystyle \sum \frac{z^n}{n!}$ is absolutely convergent for every $z\in\C$. The function $\exp:\C\to\C$ that assigns to each $z\in\C$ the value $\sum \tfrac{z^n}{n!}$ is called the exponential function; in particular $\exp: \R\to\R$.
\end{enumerate}}

\paragraph{Proof.}
\begin{enumerate}[label={(\arabic*)}, leftmargin=*]
	\item For $x_n=\tfrac{n^2}{2^n}$,
	\[
		\Bigg|\frac{x_{n+1}}{x_n}\Bigg| = \frac{(n+1)^2}{2^{n+1}}\,\frac{2^n}{n^2} = \frac12\Big(1+\frac{1}{n}\Big)^2 \longrightarrow \frac12.
	\]
	By the ratio test, the series is absolutely convergent.
	\item For $x_n = 2^{-(n+(-1)^n)}$, the successive ratio equals
	\[
		\Bigg|\frac{x_{n+1}}{x_n}\Bigg| = 2^{-1 - (-1)^{n+1} + (-1)^n} = 2^{\,2(-1)^n-1} =
		\begin{cases}
			2, & n \text{ even},\\[2pt]
				\frac{1}{8}, & n \text{ odd}.
		\end{cases}
	\]
	Hence the ratio test is inconclusive. However,
	\[
		\sqrt[n]{|x_n|} = 2^{-(n+(-1)^n)/n} = 2^{-1 - (-1)^n/n} \longrightarrow \tfrac12,
	\]
	so the root test yields absolute convergence.
	\item For $x_n = \tfrac{z^n}{n!}$ we have
	\[
		\Bigg|\frac{x_{n+1}}{x_n}\Bigg| = \frac{|z|}{n+1} \longrightarrow 0,
	\]
	hence the ratio test shows absolute convergence for all $z\in\C$.
\end{enumerate}

	\NumberedDefinition{3.5.9}{Rearrangements of a series}{Let $\sigma:\N\to\N$ be a bijection and let $\sum x_n$ be a series. Then the series $\sum x_{\sigma(n)}$ is called a \emph{rearrangement} of $\sum x_n$.}

	\NumberedExample{3.5.10}{}{Let $s := \sum_{n=1}^{\infty} (-1)^{n+1}\,\tfrac{1}{n} = 1 - \tfrac12 + \tfrac13 - \tfrac14 + \tfrac15 - \cdots$. Consider the rearrangement obtained by listing terms in the order $k,\ 2k,\ 2(k+1)$ for $k=1,3,5,\dots$; this produces the series
	\[
		1 - \frac12 - \frac14 + \frac13 - \frac16 - \frac18 + \frac15 - \frac1{10} - \frac1{12} + \frac17 - \frac1{14} - \frac1{16} + \cdots
	\]
	which can be grouped as
	\[
		\frac12 \Big\{ 1 - \frac12 + \frac13 - \frac14 + \frac15 - \frac16 + \frac17 - \frac18 + \cdots \Big\} = \frac{s}{2}.
	\]
	Thus addition with infinitely many summands is, in general, not commutative.}

	\NumberedTheorem{3.5.11}{Rearrangements of absolutely convergent series}{If $\sum x_n$ is absolutely convergent, then every rearrangement is also absolutely convergent and, for every bijection $\sigma:\N\to\N$, one has
	\[
		\sum_{n=0}^{\infty} x_n = \sum_{n=0}^{\infty} x_{\sigma(n)}.
	\]}
	\paragraph{Proof.}
	Let $\sigma:\N\to\N$ be a bijection and $\eps>0$. Then there exists $n_\eps\in\N$ such that for all $n>m\ge n_\eps$,
	\[
		\sum_{k=m+1}^{n} |x_k| < \frac{\eps}{2}.
	\]
	Choose $r_\eps$ large enough so that
	\[
		\{0,1,2,\dots,n_\eps\} \subset \{ \sigma(1),\sigma(2),\sigma(3),\dots,\sigma(r_\eps) \}.
	\]
	For every $r\ge r_\eps$ and every $n\ge n_\eps$, setting $N := \max\{n,\sigma(0),\sigma(1),\dots,\sigma(r)\}$, we get
	\[
		\Bigg| \sum_{k=0}^{r} x_{\sigma(k)} - \sum_{k=0}^{n} x_k \Bigg| \le \sum_{k=n_\eps+1}^{N} |x_k| < \frac{\eps}{2}.
	\]
	Hence, for all $r\ge r_\eps$,
	\[
		\Bigg| \sum_{k=0}^{r} x_{\sigma(k)} - \sum_{k=0}^{\infty} x_k \Bigg| \le \frac{\eps}{2} < \eps.
	\]
	Therefore $\sum x_{\sigma(k)}$ converges to $\sum_{k=0}^{\infty} x_k$. Applying the same argument to $\sum |x_{\sigma(k)}|$ and $\sum |x_k|$ shows absolute convergence of the rearranged series as well. \qed

	\NumberedRemark{3.5.12}{Riemann rearrangement theorem}{Let $\sum x_n$ be conditionally convergent in $\R$ and let $z\in\R$ be arbitrary. One can show that there exists a rearrangement of $\sum x_n$ with sum equal to $z$, i.e. there is a bijection $\sigma$ of $\N$ such that $\sum x_{\sigma(n)} = z$.}

	Next we turn to the problem of computing the product of two convergent series. For the partial sums we obtain
	\[
		\Bigg( \sum_{k=0}^{n} x_k \Bigg) \Bigg( \sum_{\ell=0}^{n} y_\ell \Bigg)
			= \sum_{k=0}^{n} \sum_{\ell=0}^{n} (x_k y_\ell)
			= \sum_{m=0}^{2n} \Bigg( \sum_{\substack{0\le k,\,\ell\le n\\ k+\ell = m}} x_k y_\ell \Bigg)
	\]
	\[
			= \sum_{m=0}^{n} \Bigg( \sum_{k=0}^{m} x_k y_{m-k} \Bigg)
				+ \sum_{m=n+1}^{2n} \Bigg( \sum_{\substack{0\le k,\,\ell\le n\\ k+\ell = m}} x_k y_\ell \Bigg) \tag{$\ast$}
	\]
	Formally, this suggests
	\[
		\Big( \sum x_n \Big) \Big( \sum y_n \Big) = \sum z_n, \qquad
		z_n := \sum_{\substack{0\le k,\,\ell\le n\\ k+\ell = n}} x_k y_\ell = \sum_{k=0}^{n} x_k y_{n-k} = \sum_{k=0}^{n} x_{n-k} y_k.
	\]
	The sequence $(z_n)$ is called the \emph{Cauchy product}. The following result makes this formal argument precise.

	\NumberedTheorem{3.5.13}{Cauchy product of absolutely convergent series}{If $\sum x_n$ and $\sum y_n$ are absolutely convergent, then the series $\sum z_n$ with
	\[
		z_n := \sum_{k=0}^{n} x_k y_{n-k}
	\]
	is also absolutely convergent and
	\[
		\Big( \sum_{n=0}^{\infty} x_n \Big) \Big( \sum_{n=0}^{\infty} y_n \Big) = \sum_{n=0}^{\infty} z_n.
	\]
}
\paragraph{Proof.}
	Applying ($\ast$) to $\sum |x_n|$ and $\sum |y_n|$ we obtain the bound
	\[
		\sum_{m=0}^{n} |z_m|
			\le \sum_{m=0}^{n} \sum_{\substack{0\le k,\,\ell\le n\\ k+\ell = m}} |x_k|\,|y_\ell|
			\le \sum_{m=0}^{2n} \sum_{\substack{0\le k,\,\ell\le n\\ k+\ell = m}} |x_k|\,|y_\ell|
			= \Big( \sum_{k=0}^{n} |x_k| \Big) \Big( \sum_{\ell=0}^{n} |y_\ell| \Big)
			\le \Big( \sum_{k=0}^{\infty} |x_k| \Big) \Big( \sum_{\ell=0}^{\infty} |y_\ell| \Big).
	\]
	From this and Theorem~3.2.2 it follows that $\sum z_n$ is absolutely convergent. Let $a := \sum_{k=0}^{\infty} |x_k|$ and $b := \sum_{\ell=0}^{\infty} |y_\ell|$ (so $a>0$ and $b>0$). Given $\eps>0$, there exist $n_1,n_2\in\N$ such that
	\[
		\sum_{k=m+1}^{n} |x_k| < \frac{\eps}{2b} \quad (\forall\, n>m\ge n_1),
		\qquad
		\sum_{\ell=m+1}^{n} |y_\ell| < \frac{\eps}{2a} \quad (\forall\, n>m\ge n_2).
	\]
	Set $n_\eps := 2\max\{n_1,n_2\}+2$. Using the auxiliary computation
	\[
		\Big(\sum_{k=0}^{n} x_k\Big)\Big(\sum_{k=0}^{n} y_k\Big)
		\overset{(\ast)}{=} \sum_{k=0}^{n} z_k 
				+ \sum_{m=n+1}^{2n} \sum_{\substack{0\le k,\,\ell\le n\\ k+\ell=m}} x_k y_\ell,
	\]
	we obtain for all $n\ge n_\eps$:
	\begin{align*}
		\Bigg| \sum_{k=0}^{n} z_k - \Big(\sum_{k=0}^{n} x_k\Big)\Big(\sum_{k=0}^{n} y_k\Big) \Bigg|
		 &= \Bigg| \sum_{m=n+1}^{2n} \sum_{\substack{0\le k,\,\ell\le n\\ k+\ell=m}} x_k y_\ell \Bigg|
			\le \sum_{m=n+1}^{2n} \sum_{\substack{0\le k,\,\ell\le n\\ k+\ell=m}} |x_k|\,|y_\ell| \\
		 &= \sum_{k=0}^{n} |x_k| \sum_{\ell=n+1-k}^{n} |y_\ell|
			\le \sum_{k=0}^{n} |x_k| \sum_{\ell=\lceil n/2\rceil}^{n} |y_\ell| 
			 + \sum_{\ell=0}^{n} |y_\ell| \sum_{k=\lceil n/2\rceil}^{n} |x_k| \\
		 &\le a\, \sum_{\ell=n_2+1}^{2n} |y_\ell| 
			 + b\, \sum_{k=n_1+1}^{2n} |x_k| \\
		 &< a\,\frac{\eps}{2a} + b\,\frac{\eps}{2b}
			= \eps.
	\end{align*}
	Hence $\big(\sum_{k=0}^{n} z_k - (\sum_{k=0}^{n} x_k)(\sum_{k=0}^{n} y_k)\big)_{n\in\N}$ is a null sequence. By Lemma~3.1.18 and Theorem~3.1.19 the claim follows. \qed

	\NumberedRemark{3.5.14}{}{If $\sum x_n$ and $\sum y_n$ are only conditionally convergent, the assertion of Theorem~3.5.13 is in general false.}
	\paragraph{Proof.}
	Choose $x_n=y_n:=(-1)^n/\sqrt{n}$, $n\in\N$. By the Leibniz criterion (Theorem~3.4.8), $\sum x_n = \sum y_n$ converges. Since $1/\sqrt{n} \ge 1/n$ for all $n\in\N$ and Example~3.4.2(3) (harmonic series) diverges, $\sum (-1)^n/\sqrt{n}$ is only conditionally convergent. We show that $z_n := \sum_{k=0}^{n} x_k y_{n-k}$ is not a null sequence:
	\[
		|z_n| = \Bigg| \sum_{k=1}^{n-1} x_k y_{n-k} \Bigg| = \sum_{k=1}^{n-1} \frac{1}{\sqrt{k}\,\sqrt{n-k}} 
			\ge \sum_{k=1}^{n-1} \frac{2}{k+(n-k)} = \frac{2}{n}(n-1) = 2\Big(1-\frac{1}{n}\Big) \ge 1 \quad (n\ge2),
	\]
	where we used $\sqrt{a\,b} \le (a+b)/2$. Hence $(z_n)$ is not a null sequence and $\sum z_n$ diverges. One can show that Theorem~3.5.13 remains valid if one of the series $\sum x_n$, $\sum y_n$ is absolutely convergent and the other merely conditionally convergent. \qed

	\NumberedExample{3.5.15}{}{For the exponential function of Example~3.5.8(3) and for $z_1,z_2\in\C$ we have
	\[
		e^{z_1+z_2} := \exp(z_1+z_2) = (\exp(z_1))\,(\exp(z_2)) = e^{z_1}\,e^{z_2}.
	\]}
	\paragraph{Proof.}
	By Theorem~3.5.13 and the binomial theorem,
	\begin{align*}
		e^{z_1}\,e^{z_2}
			&= \Big( \sum_{n=0}^{\infty} \frac{z_1^{n}}{n!} \Big)
				 \Big( \sum_{n=0}^{\infty} \frac{z_2^{n}}{n!} \Big)
			 = \sum_{n=0}^{\infty} \Big( \sum_{k=0}^{n} \frac{z_1^{k}}{k!}\,\frac{z_2^{n-k}}{(n-k)!} \Big) \\
			&= \sum_{n=0}^{\infty} \frac{1}{n!} \Big( \sum_{k=0}^{n} \binom{n}{k} z_1^{k} z_2^{n-k} \Big)
			 = \sum_{n=0}^{\infty} \frac{(z_1+z_2)^n}{n!}
			 = \exp(z_1+z_2) = e^{z_1+z_2}.
	\end{align*}
	In particular, for $z\in\C$ we get $1 = e^{0} = e^{z}\,e^{-z}$, hence $e^{z}\ne0$ and therefore $e^{-z} = 1/e^{z}$. \qed
	
\subsection{Power Series (Potenzreihen)}

\NumberedDefinition{3.6.1}{Power series}{Let $(a_n)_{n\in\N}$ be a sequence in $\K$ and $z_0\in\K$. The series
\[
	\sum a_n\,(z-z_0)^{n}
\]
in the variable $z\in\K$ is called a \emph{power series} with coefficients $a_n$ and center $z_0$.}

\NumberedExample{3.6.2}{}{
\begin{enumerate}[label={(\arabic*)}, leftmargin=*]
	\item $\displaystyle \sum \frac{z^{n}}{n!}$. This power series $(a_n=1/n!,\ z_0=0)$ converges absolutely for all $z\in\C$.
	\item $\displaystyle \sum z^{n}$. This power series $(a_n=1,\ z_0=0)$ converges absolutely for all $z\in\C$ with $|z|<1$.
	\item $\displaystyle \sum (n!)\, z^{n}$ with $a_n=n!,\ z_0=0$. Since
	\[
		\Bigg|\frac{(n+1)\,z^{n+1}}{n!\,z^{n}}\Bigg| = (n+1)|z| \longrightarrow \infty \quad (n\to\infty) \quad \text{for every } z\ne0,
	\]
	the series converges only for $z=0$.
\end{enumerate}}

\NumberedTheorem{3.6.3}{Radius of convergence}{Let $\sum a_n (z-z_0)^n$ be a power series and set
\[
	\rho := \frac{1}{\displaystyle \limsup_{n\to\infty} \sqrt[n]{|a_n|}}\;\in [0,\infty].
\]
Then:
\begin{enumerate}[label={(\arabic*)}, leftmargin=*]
	\item The power series converges absolutely for all $z\in\K$ with $|z-z_0|<\rho$.
	\item The power series diverges for all $z\in\K$ with $|z-z_0|>\rho$.
\end{enumerate}}
The number $\rho$ is called the \emph{radius of convergence} of the power series, and the set $\{z\in\K:\ |z-z_0|<\rho\}$ is called its (open) disc of convergence.

\paragraph{Proof.}
Use the root test:
\[
	\limsup_{n\to\infty} \sqrt[n]{\,|a_n (z-z_0)^n|\,}
	 = |z-z_0|\, \limsup_{n\to\infty} \sqrt[n]{|a_n|}
	 = \frac{|z-z_0|}{\rho}.
\]
Hence the series converges for $|z-z_0|<\rho$. The divergence for $|z-z_0|>\rho$ follows similarly. \qed

\NumberedTheorem{3.6.4}{Ratio form for radius}{Let $\sum a_n (z-z_0)^n$ be a power series and assume
\[
	\lim_{n\to\infty} \frac{|a_n|}{|a_{n+1}|} =: \alpha \in \overline{\R}.
\]
Then the radius of convergence satisfies $\rho = \alpha$.}
\paragraph{Proof.}
Apply the ratio test:
\[
	\Bigg|\frac{a_{n+1}(z-z_0)^{n+1}}{a_n (z-z_0)^n}\Bigg|
	 = |z-z_0|\, \Bigg|\frac{a_{n+1}}{a_n}\Bigg| \longrightarrow \frac{|z-z_0|}{\alpha}.
\]
Thus the series converges for $|z-z_0|<\alpha$ and does not converge for $|z-z_0|>\alpha$, so $\rho=\alpha$. \qed

\NumberedExample{3.6.5}{}{
\begin{enumerate}[label={(\arabic*)}, leftmargin=*]
	\item $\displaystyle \sum \frac{z^{n}}{n!}$: Here $\frac{|a_n|}{|a_{n+1}|} = \frac{(n+1)!}{n!} = n+1 \to \infty$, hence $\rho=\infty$.
	\item $\displaystyle \sum n^{m} z^{n}$ with fixed $m\in\Z$: Then
	\[
		\frac{|a_n|}{|a_{n+1}|} = \frac{n^{m}}{(n+1)^{m}} = \Big(\frac{n}{n+1}\Big)^{m} = \Big( \frac{1}{1+\tfrac{1}{n}} \Big)^{m} \longrightarrow 1,
	\]
	so $\rho=1$.
\end{enumerate}}

\NumberedRemark{3.6.6}{}{No general statement can be made about behavior on the boundary of the disc of convergence. For example:
\begin{enumerate}[label={(\arabic*)}, leftmargin=*]
	\item $\sum z^{n}$ has $\rho=1$ and diverges for every $z$ with $|z|=1$.
	\item $\sum \dfrac{z^{n}}{n}$ has $\rho=1$ and converges for $z=-1$ but diverges for $z=1$.
	\item $\sum \dfrac{z^{n}}{n^{2}}$ has $\rho=1$ and converges for all $|z|=1$ (convergent majorant $\sum \tfrac{1}{n^{2}}$).
\end{enumerate}}

\paragraph{Rules for power series.}

\NumberedTheorem{3.6.7}{Algebra of power series}{Let $\alpha\in\K$ and let $\sum a_n (z-z_0)^n$, $\sum b_n (z-z_0)^n$ be two power series with the same center $z_0$ and radii of convergence $\rho_1,\rho_2$. Then:
\begin{enumerate}[label={(\arabic*)}, leftmargin=*]
	\item $\displaystyle \sum_{n=0}^{\infty} \alpha a_n (z-z_0)^n = \alpha \sum_{n=0}^{\infty} a_n (z-z_0)^n$ for all $|z-z_0|<\rho_1$.
	\item $\displaystyle \sum_{n=0}^{\infty} (a_n+b_n)(z-z_0)^n = \sum_{n=0}^{\infty} a_n (z-z_0)^n + \sum_{n=0}^{\infty} b_n (z-z_0)^n$ for all $|z-z_0|<\min\{\rho_1,\rho_2\}$.
	\item $\displaystyle \sum_{n=0}^{\infty} \Big( \sum_{k=0}^{n} a_k b_{n-k} \Big) (z-z_0)^n 
		= \Big( \sum_{n=0}^{\infty} a_n (z-z_0)^n \Big) \Big( \sum_{n=0}^{\infty} b_n (z-z_0)^n \Big)$ for all $|z-z_0|<\min\{\rho_1,\rho_2\}$.
\end{enumerate}}

\paragraph{Function vs. series representation.}
Consider $f(z) := \dfrac{1}{1-z}$ for $z\in\C\setminus\{1\}$. For $|z|<1$ the function also has the representation as a power series
\[
	g(z) := \sum z^{n}.
\]
For $|z|\ge1$ this representation is no longer valid. However, for $|z|>1$, using
\[
	f(z) = -\frac{1}{z}\,\frac{1}{1-\tfrac{1}{z}} \qquad\text{and}\qquad \Big|\frac{1}{z}\Big|<1,
\]
we obtain the representation
\[
	f(z) = -\frac{1}{z}\sum_{i=0}^{\infty} \Big(\frac{1}{z}\Big)^{i} = -\sum_{i=0}^{\infty} z^{-i-1}.
\]
	
% ============================================================================
% 4 Continuous Functions (Stetige Funktionen) -- Original start p.57
% ============================================================================
\section{Continuous Functions (Stetige Funktionen)}

\subsection{Normed Vector Spaces (Normierte Vektorräume)}

In this section we assume the notion of a vector space from linear algebra is known and briefly recall the basic definitions and properties.

\NumberedDefinition{4.1.1}{$\K$-vector space}{A set $X$ is called a $\K$-vector space if there are two maps
\[
\begin{aligned}
 &+ : X\times X \to X &&\text{(called addition)},\\
 &\cdot : \K\times X \to X &&\text{(called scalar multiplication)},
\end{aligned}
\]
with the following properties:
\begin{description}[leftmargin=*]
	\item[VR1] $(X,+)$ is an abelian group with neutral element $0$.
	\item[VR2] For all $\alpha,\beta\in\K$ and all $x,y\in X$ (distributive/associative laws):
	\[
		\alpha\cdot(x+y)=\alpha\cdot x+\alpha\cdot y,\qquad
		(\alpha+\beta)\cdot x = \alpha\cdot x + \beta\cdot x,\qquad
		\alpha\cdot(\beta\cdot x)=(\alpha\beta)\cdot x.
	\]
	\item[VR3] $1\cdot x = x$ for all $x\in X$.
\end{description}}

\NumberedExample{4.1.2}{}{
\begin{enumerate}[label={(\arabic*)}, leftmargin=*]
	\item $\displaystyle \K^m := \underbrace{\K\times\K\times\cdots\times\K}_{m\text{-fold}} = \{(x_1,\dots,x_m):\ 1\le i\le m:\ x_i\in\K\}$ is a $\K$-vector space.

	VR: \quad “$+$”:
	\[
		(x_1,\dots,x_m) + (y_1,\dots,y_m) = (x_1+y_1,\ x_2+y_2,\ \dots,\ x_m+y_m) \in \K^m,
	\]
	“$\cdot$”:
	\[
		t\cdot(x_1,\dots,x_m) = (t\,x_1,\ \dots,\ t\,x_m) \in \K^m \qquad (t\in\K).
	\]
	Null element: $0=(0,0,\dots,0)$.

	\item $\displaystyle \K^{\N} := \{(a_n)_{n\in\N}:\ \forall n\in\N:\ a_n\in\K\}$ is a $\K$-vector space.

	“$+$”:
	\[
		(a_n)_{n\in\N} + (b_n)_{n\in\N} := (a_n+b_n)_{n\in\N} \in \K^{\N},
	\]
	“$\cdot$”:
	\[
		t\cdot(a_n)_{n\in\N} := (t\cdot a_n)_{n\in\N} \in \K^{\N} \qquad (t\in\K).
	\]
	Null element: $0 := (0,0,\dots)$.

	\item $\ell_\infty$ is the space of all bounded sequences in $\K$,
	\[
		\ell_\infty := \{ (a_n)_{n\in\N} : \sup_{n\in\N} |a_n| < \infty \},
	\]
	and a $\K$-vector space, with “$+$” as in $\K^{\N}$. Let $(a_n)_{n\in\N}, (b_n)_{n\in\N}\in \ell_\infty$. Then there exist $\alpha,\beta\in\R$ with
	\[
		|a_n| \le \alpha,\qquad |b_n| \le \beta, \qquad \forall n\in\N.
	\]
	Then for $(c_n)_{n\in\N} := (a_n+b_n)_{n\in\N}$ we have
	\[
		|c_n| = |a_n+b_n| \le |a_n| + |b_n| \le \alpha + \beta.
	\]

	\item The space
	\[
		\ell_1 := \{ (a_n)_{n\in\N} \subset \K : \sum_{n=0}^{\infty} |a_n| < \infty \}
	\]
	is a $\K$-vector space. Let $(a_n)_{n\in\N}, (b_n)_{n\in\N}\in \ell_1$ and $(c_n)_{n\in\N} := (a_n+b_n)_{n\in\N}$. Then
	\[
		\sum_{n=0}^{\infty} |c_n| = \sum_{n=0}^{\infty} |a_n+b_n| \le \sum_{n=0}^{\infty} (|a_n|+|b_n|) = \sum_{n=0}^{\infty} |a_n| + \sum_{n=0}^{\infty} |b_n|.
	\]
	By the comparison test it follows: $\sum |c_n|$ convergent.

	\item The space $\ell_2$ of all $(x_n)_{n\in\N}\in \K^{\N}$ such that
	\[
		\sqrt{\sum_{n=0}^{\infty} |x_n|^2} < \infty
	\]
	is a $\K$-vector space. “$+$” and “$\cdot$” as in $\K^{\N}$. For “$+\colon \ell_2\times\ell_2\to\ell_2$”: let $x_n, y_n\in \ell_2$. Then
	\[
		\sqrt{\sum_{n=0}^{\infty} |x_n+y_n|^2}
		\;\overset{(*)}{\le}\;
		\sqrt{\sum_{n=0}^{\infty} \big(2|x_n|^2 + 2|y_n|^2\big)}
		= \sqrt{\,2\sum_{n=0}^{\infty} |x_n|^2\, + \,2\sum_{n=0}^{\infty} |y_n|^2\,} 
		< \infty.
	\]
	The estimate $(*)$ follows from
	\[
		|x+y|^2 \le (|x|+|y|)^2 = |x|^2 + 2|x|\,|y| + |y|^2,
	\]
	\[
		0 \le (|x|-|y|)^2 = |x|^2 - 2|x|\,|y| + |y|^2.
	\]
\end{enumerate}}



\NumberedDefinition{4.1.3}{Norm}{Let $X$ be a $\K$-vector space. A map $\|\,\cdot\,\| : X\to\R$ is called a norm if the following hold:
\begin{description}[leftmargin=*]
	\item[N1] $\|x\|\ge 0$ for all $x\in X$, \quad and \quad $\|x\|=0 \iff x=0$,
	\item[N2] $\|\alpha x\| = |\alpha|\,\|x\|$ for all $x\in X$, $\alpha\in\K$,
	\item[N3] $\|x+y\| \le \|x\| + \|y\|$ for all $x,y\in X$ (triangle inequality).
\end{description}
The pair $(X,\|\cdot\|)$ is called a normed vector space.}

\NumberedRemark{4.1.4}{}{From the triangle inequality one obtains the reverse triangle inequality
\[
	\|x\| - \|y\| \le \|x-y\|,
\]
since: $\|x\| = \|y + (x-y)\| \le \|y\| + \|x-y\| \ \Rightarrow\ \|x\| - \|y\| \le \|x-y\|$; analogously $\|y\| - \|x\| \le \|x-y\|$. Hence
\[
	\big|\,\|x\| - \|y\|\,\big| = \max\{\,\|x\| - \|y\|,\ \|y\| - \|x\|\,\} \le \|x-y\|.
\]}

\NumberedExample{4.1.5}{}{\begin{enumerate}[label={(\arabic*)}, leftmargin=*]
	\item $|\,\cdot\,|$ is a norm on $\R$ and on $\C$.

	\item For $x\in\K^{m}$ the following define three norms:
	\[
		\|x\|_{\infty} := \max_{1\le k\le m} |x_k|, \qquad
		\|x\|_{1} := \sum_{k=1}^{m} |x_k|, \qquad
		\|x\|_{2} := \sqrt{\sum_{k=1}^{m} |x_k|^{2}}.
	\]
	Moreover:
	\begin{enumerate}[label=\roman*.), leftmargin=2em]
		\item $\|x\|_{\infty} \le \|x\|_{1} \le m\,\|x\|_{\infty}$,
		\item $\|x\|_{\infty} \le \|x\|_{2} \le \sqrt{m}\,\|x\|_{\infty}$,
		\item $\tfrac{1}{\sqrt{m}}\,\|x\|_{1} \le \|x\|_{2} \le \sqrt{m}\,\|x\|_{1}$.
	\end{enumerate}

	\item $\displaystyle \|x\|_{\infty} := \sup_{n\in\N} |x_n|$ is a norm on $\ell_{\infty}$.

	\item $\displaystyle \|x\|_{1} := \sum_{k=0}^{\infty} |x_k|$ is a norm on $\ell_{1}$.

	\item $\displaystyle \|x\|_{2} := \sqrt{\sum_{k=0}^{\infty} |x_k|^{2}}$ is a norm on $\ell_{2}$.
\end{enumerate}}

	\paragraph{Proof.}
	@2: The norm properties for $\|\cdot\|_1$, $\|\cdot\|_\infty$ follow from the norm properties of $|\cdot|$ on $\R$ and $\C$. For $\|\cdot\|_2$, (N1) and (N2) likewise follow from the norm properties of $|\cdot|$ on $\R$ and $\C$. For (N3): let $(x_n),(y_n)\in \ell_2$. Then
	\[
		\sum_{n=1}^{m} |x_n+y_n|^2 \le \sum_{n=1}^{m} (|x_n|^2 + |y_n|^2 + 2|x_n|\,|y_n|)
		\;\overset{(*)}{\le}\; \|x\|_2^2 + \|y\|_2^2 + 2\,\|x\|_2\,\|y\|_2 = (\|x\|_2 + \|y\|_2)^2.
	\]
	($*$ follows from the Cauchy--Schwarz inequality: $\sum_{i=1}^{n} a_i b_i \le \sqrt{\sum_{i=1}^{n} a_i^2}\,\sqrt{\sum_{i=1}^{n} b_i^2}$.)
	It follows that $\|x+y\|_2 \le \|x\|_2 + \|y\|_2$.

	\noindent i. Follows from the definitions of $\|\cdot\|_\infty := \max_{1\le i\le m} |x_i|$ and $\|\cdot\|_1 := \sum_{i=1}^{m} |x_i|$.

	\noindent ii. Follows likewise from the definition of $\|\cdot\|_2 := \sqrt{\sum_{i=1}^{m} |x_i|^2}$.

	\noindent iii. One has
	\[
		\Big(\sum_{i=1}^{m} |x_i|\Big)^2 = \Big(\sum_{i=1}^{m} |x_i|\Big)\Big(\sum_{j=1}^{m} |x_j|\Big)
		\;\le\; \Big(\sum_{i=1}^{m} |x_i|^2\Big)^{1/2}\,\Big(\sum_{i=1}^{m} 1^2\Big)^{1/2}\,\Big(\sum_{j=1}^{m} |x_j|\Big)
		= \|x\|_2\,\sqrt{m}\,\|x\|_1.
	\]
	Hence $\tfrac{1}{\sqrt{m}}\,\|x\|_1 \le \|x\|_2$. Together with ii. (and i.) this yields the stated inequalities.

	@3: The norm properties for $\|\cdot\|_\infty$ follow from the norm properties of $|\cdot|$ on $\K$.

	@4, 5: (N1) and (N2) follow from property (N1), (N2) on $\K$ and the usual rules for series. For (N3) use the definition $\|x\|_2 := \sqrt{\sum_{n=0}^{\infty} |x_n|^2}$. Since $\sum_{k=0}^{n} |x_k|^2$ is a norm on $\K^{n+1}$ (see (2)), it follows (see (2)) that
	\[
		\sum_{k=0}^{n} |x_k + y_k|^2 \le \sum_{k=0}^{n} |x_k|^2 + \sum_{k=0}^{n} |y_k|^2.
	\]
	The sequences $ (a_n), (b_n)$ with $a_n := \sum_{k=0}^{n} |x_k|^2$ and $b_n := \sum_{k=0}^{n} |x_k|^2 + \sum_{k=0}^{n} |y_k|^2$ are convergent and satisfy $a_n \le b_n$ for all $n\in\N$. Therefore
	\[
		\sum_{k=0}^{\infty} |x_k + y_k|^2 \le \sum_{k=0}^{\infty} |x_k|^2 + \sum_{k=0}^{\infty} |y_k|^2.
	\]
	\qed

	\NumberedRemark{4.1.6}{Convention}{In the following, $(X,\|\cdot\|)$ will always denote a normed vector space.}

	\NumberedDefinition{4.1.7}{Open ball}{For $x_0\in X$ and $r\in\R$ set
	\[
		B(x_0,r) := \{\, x\in X : \|x - x_0\| < r \,\}.
	\]
	This is the (open) ball around $x_0$ with radius $r$.\footnote{The scan shows an illustrative figure of unit balls; omitted here.}}

	\NumberedRemark{4.1.8}{}{In a normed vector space one defines the concepts of sequence, convergent sequence, accumulation point, Cauchy sequence, and series analogously by replacing the absolute value $|\cdot|$ everywhere by the norm $\|\cdot\|$.

	Convergence: Let $(x_n)_{n\in\N}\subset X$ be a sequence with elements $x_n\in X$. Then $(x_n)$ converges to $x\in X$ precisely when
	\[
		\forall\,\eps>0\ \exists\, n_\eps\in\N\ \forall\, n\ge n_\eps:\ \|x - x_n\| < \eps.
	\]
	The sequence $(x_n)$ is a Cauchy sequence precisely when
	\[
		\forall\,\eps>0\ \exists\, n_\eps\in\N\ \forall\, n,m\ge n_\eps:\ \|x_n - x_m\| < \eps.
	\]
	Example~4.1.5 shows that one can define different norms on a given vector space. We will discuss whether a sequence that converges with respect to one norm also converges with respect to all other possible norms. This is certainly the case if the different norms can be estimated against each other.}

	\NumberedDefinition{4.1.9}{Equivalent norms}{Let $X$ be a $\K$-vector space and let $\|\cdot\|_a,\ \|\cdot\|_b$ be two norms on $X$. They are called \emph{equivalent} if there exist positive constants $c$ and $\bar c$ such that
	\[
		c\,\|x\|_a \le \|x\|_b \le \bar c\,\|x\|_a \qquad \forall\, x\in X.
	\]}

	\NumberedRemark{4.1.10}{}{\begin{enumerate}[label={(\arabic*)}, leftmargin=*]
		\item Example~4.1.5(2) shows that the norms $\|\cdot\|_1$, $\|\cdot\|_2$, $\|\cdot\|_\infty$ on $\K^{m}$ are equivalent.
		\item If $\|\cdot\|_a$ and $\|\cdot\|_b$ are equivalent on a $\K$-vector space $X$, then a sequence $(x_n)_{n\in\N}$ in $X$ converges with respect to $\|\cdot\|_a$ if and only if it converges with respect to $\|\cdot\|_b$.
		\item The same holds for Cauchy sequences.
	\end{enumerate}}

	\NumberedTheorem{4.1.11}{On $\K^{n}$ all norms are equivalent}{Let $n\in\N$. On $\K^{n}$ all norms are equivalent.}
	\paragraph{Proof.}
	It suffices to show that every norm $\|\cdot\|$ on $\K^{n}$ is equivalent to the $\|\cdot\|_{1}$-norm. For $1\le i\le n$ let
	\[
		e_i := (0,\dots,0,\underset{\text{$i$-th component}}{1},0,\dots,0)\in\K^{n}
	\]
	be the $i$-th canonical unit vector. Then for $x=(x_1,\dots,x_n)=\sum_{i=1}^{n} x_i e_i$ we have
	\[
		\|x\| = \bigg\|\sum_{i=1}^{n} x_i e_i\bigg\| \le \sum_{i=1}^{n} |x_i|\,\|e_i\|.
	\]
	Define $\|e_i\| =: C_i$ and let $C := \max_{1\le i\le n} C_i$. Then
	\[
		\|x\| \le \sum_{i=1}^{n} |x_i|\,\|e_i\| \le C\, \sum_{i=1}^{n} |x_i| = C\,\|x\|_{1}.
	\]
	Thus we may choose the upper constant $\bar c=C$.

	Now set
	\[
		M := \{\, \|x\| : x\in\K^{n},\ \|x\|_{1}=1\,\}.
	\]
	The set $M$ is nonempty (since $e_i\in M$ for $1\le i\le n$) and bounded below by $0$. Hence
	\[
		\rho := \inf M \ge 0.
	\]
	We show $\rho\ne0$ by contradiction. Suppose $\rho=0$. Then there exists a sequence $(x_k)_{k\in\N}$, $x_k\in\K^{n}$, with
	\[
		\|x_k\| \le \frac{1}{k}, \qquad \|x_k\|_{1} = 1 \quad (\forall k\in\N).
	\]
	For the $i$-th component $x_{k,i}$ of $x_k$ we have $|x_{k,i}| \le \|x_k\|_{1} = 1$. Thus $(x_{k,i})_{k\in\N}$ is a bounded sequence in $\K$, and by Bolzano--Weierstrass it has a convergent subsequence. Applying the same argument successively to the remaining coordinates, we obtain (by a diagonal extraction) a subsequence $(x_{k_\ell})_{\ell\in\N}$ such that each coordinate converges:
	\[
		x_{k_\ell,i} \to x_i^{*} \in \K, \qquad 1\le i\le n.
	\]
	Set $x^{*} := (x_1^{*},\dots,x_n^{*})\in\K^{n}$. Then
	\[
		\|x^{*}\| \le \|x_{k_\ell}\| + \|x^{*}-x_{k_\ell}\| \le \frac{1}{k_\ell} + C\,\|x^{*}-x_{k_\ell}\|_{1} \xrightarrow[\ell\to\infty]{} 0,
	\]
	hence $\|x^{*}\|=0$ and therefore $x^{*}=0$. On the other hand,
	\[
		\|x^{*}\|_{1} = \lim_{\ell\to\infty} \|x_{k_\ell}\|_{1} = 1,
	\]
	which is a contradiction. Thus $\rho>0$.

	Finally, for arbitrary $x\in\K^{n}$ with $\|x\|_{1}\ne0$, set $y:=x/\|x\|_{1}$. Then $\|y\|_{1}=1$ and so $\|y\|\ge\rho$, i.e.
	\[
		\|x\| \ge \rho\,\|x\|_{1}.
	\]
	Together with the estimate $\|x\|\le C\,\|x\|_{1}$ this shows that $\|\cdot\|$ and $\|\cdot\|_{1}$ are equivalent. Since any two norms are both equivalent to $\|\cdot\|_{1}$, all norms on $\K^{n}$ are equivalent. \qed

	\NumberedRemark{4.1.12}{}{Theorem~4.1.11 does not hold in infinite-dimensional vector spaces in general. For example, the norms $\|\cdot\|_\infty$ and $\|\cdot\|_1$ on $\ell_1$ are not equivalent.}
	\paragraph{Proof.}
	Indirect proof: assume there exists $\bar c>0$ with
	\[
		\|x\|_{1} \le \bar c\,\|x\|_{\infty} \qquad \forall x\in\ell_{1}.\tag{$*$}
	\]
	Let $m\in\N$ with $m>\bar c$. Choose $x\in\ell_{1}$ by
	\[
		x_i := \begin{cases} 1,& 0\le i\le m,\\ 0,& \text{otherwise.} \end{cases}
	\]
	Then $\|x\|_{1}=m+1$ and $\|x\|_{\infty}=1$, contradicting $(*)$. \qed

	\paragraph{Recall.} In $\K$ the statements
	\begin{itemize}[leftmargin=*]
		\item $(x_n)$ is convergent,
		\item $(x_n)$ is a Cauchy sequence
	\end{itemize}
	are equivalent.

	\NumberedDefinition{4.1.13}{Complete (Banach) space}{A normed $\K$-vector space $(X,\|\cdot\|)$ is called \emph{complete} or a \emph{Banach space} if every Cauchy sequence in $X$ converges.}

	\NumberedTheorem{4.1.14}{The following normed $\K$-vector spaces are complete}{\begin{enumerate}[label={(\arabic*)}, leftmargin=*]
		\item $\K$,
		\item $\K^{n}$ for $n\in\N$,
		\item $(\ell_{1},\,\|\cdot\|_{1})$,
		\item $(\ell_{\infty},\,\|\cdot\|_{\infty})$,
		\item $(\ell_{2},\,\|\cdot\|_{2})$.
\end{enumerate}}

\paragraph{Proof.}
@ 1: Theorem 3.2.9.

@ 2: By Theorem 4.1.11 it suffices to prove the claim for the norm $\|\cdot\|_\infty$. Let $(x_k)_{k\in\N}$ be a Cauchy sequence in $(\K^{n},\|\cdot\|_{\infty})$, where $x_k\in\K^n$ and $x_{k,i}\in\K$. Then for each $1\le i\le n$ the sequence of coordinates $(x_{k,i})_{k\in\N}$ is a Cauchy sequence in $\K$, hence convergent (Theorem 3.2.9), say to $x_i^{\ast}\in\K$. Set
\[
	\ x^{\ast} := (x_1^{\ast},x_2^{\ast},\dots,x_n^{\ast}) \in \K^{n},
\]
the candidate for the limit of $(x_k)_{k\in\N}$. Let $\eps>0$. Then there exist indices $k_{\eps,i}\in\N$ such that
\[
	\ |x_i^{\ast}-x_{k,i}| < \eps \qquad \forall k\ge k_{\eps,i},\ 1\le i\le n.
\]
Set $k_{\eps} := \max_{1\le i\le n} k_{\eps,i}$. Then for all $k\ge k_{\eps}$ we have
\[
	\ \|x^{\ast}-x_k\|_{\infty} 
		 = \max_{1\le i\le n} |x_i^{\ast}-x_{k,i}| 
		 < \eps.
\]
Thus $x_k\to x^{\ast}$ in $\|\cdot\|_{\infty}$.

@ 3:

i. Construct the limit.

Let $(x_k)_{k\in\N}$ be a Cauchy sequence with respect to $\|\cdot\|_{1}$ in $\ell_{1}$, i.e. $x_k\in\ell_{1}$, $x_{k,i}\in\K$, $i\in\N$, and
\[
	\ \forall\,\eps>0\ \exists\,k_{\eps}\in\N\ \forall\,k,l\ge k_{\eps} :\ \eps > \|x_k-x_l\|_{1} 
		\ = \sum_{i=0}^{\infty} |x_{k,i}-x_{l,i}|.
\]
In particular, for each fixed $i\in\N$ we obtain
\[
	\ \forall\,\eps>0\ \exists\,k_{\eps}\in\N\ \forall\,k,l\ge k_{\eps} :\ |x_{k,i}-x_{l,i}|<\eps.
\]
By Theorem 3.2.9 there exists for each $i$ a limit $x_i^{\ast}\in\K$ with $x_{k,i}\to x_i^{\ast}$ as $k\to\infty$. Set $x^{\ast}:=(x_i^{\ast})_{i\in\N}\in\K^{\N}$.

ii. Show: $x^{\ast}\in \ell_{1}$, i.e. $\sum_{i=0}^{\infty} |x_i^{\ast}| < \infty$.

By Remark 4.1.4 we know that $(\|x_k\|_{1})_{k\in\N}$ is a Cauchy sequence in $\R$. Hence
\[
	\ c := \sup_{k\in\N} \|x_k\|_{1} < \infty.
\]
Fix $N\in\N$. For $\eps=\tfrac{1}{N}$ choose indices $k_{\eps,1},\dots,k_{\eps,N}$ with $|x_i^{\ast}-x_{k,i}|<\eps$ for all $\,k\ge k_{\eps,i}$ and $1\le i\le N$. Let $k_{\eps}:=\max_{1\le i\le N} k_{\eps,i}$. Then
\[
	\ \sum_{i=0}^{N} |x_i^{\ast}|
		\ \le \underbrace{\sum_{i=0}^{N} |x_i^{\ast}-x_{k_{\eps},i}|}_{<\,N\cdot(1/N)=1}
		\ \ + \underbrace{\sum_{i=0}^{N} |x_{k_{\eps},i}|}_{\le\,\|x_{k_{\eps}}\|_{1}\,\le\,c}
		\ < 1 + c.
\]
Thus the sequence of partial sums $\big(S_N:=\sum_{i=0}^{N} |x_i^{\ast}|\big)_{N\in\N}$ is monotone increasing and bounded, hence the series $\sum_{i=0}^{\infty} |x_i^{\ast}|$ is convergent. Therefore $x^{\ast}\in\ell_{1}$.

iii. It remains to show: $\|x^{\ast} - x_{k}\|_{1} \to 0$ as $k\to\infty$.

Assume this is not the case. Then there exists $\varepsilon_{0} > 0$ such that for all $k \in \N$ there is an $m \ge k$ with
\[
	\ \|x^{\ast} - x_{m}\|_{1} \ge \varepsilon_{0}.\tag{$*$}
\]
Since $(x_{k})_{k\in\N}$ is a Cauchy sequence, there exists $k_{0} \in \N$ such that
\[
	\ \|x_{k} - x_{l}\|_{1} < \frac{\varepsilon_{0}}{4} \qquad \forall\, k,l \ge k_{0}.
\]
For this $k_{0}$ there is, by $(*)$, an $m_{0} \ge k_{0}$ with $\|x^{\ast} - x_{m_{0}}\|_{1} \ge \varepsilon_{0}$. The series $\sum_{i=n_{0}}^{\infty} |x^{\ast}_{i} - x_{m_{0},i}|$ is convergent; hence, by Theorem~3.4.6, there exists $n_{0} \in \N$ with
\[
	\ \sum_{i=n_{0}}^{\infty} |x^{\ast}_{i} - x_{m_{0},i}| < \frac{\varepsilon_{0}}{4}.
\]
Since $x_{k,i} \to x^{\ast}_{i}$ as $k \to \infty$, there exists $l_{0} \ge k_{0}$ with
\[
	\ |x_{l_{0},i} - x^{\ast}_{i}| < \frac{\varepsilon_{0}}{4 n_{0}} \qquad \text{for } 1 \le i \le n_{0}.
\]
Therefore
\[
\begin{aligned}
\varepsilon_{0}
&\le \|x^{\ast} - x_{m_{0}}\|_{1}
 = \sum_{i=0}^{\infty} |x^{\ast}_{i} - x_{m_{0},i}|
 \\&\le \sum_{i=0}^{n_{0}-1} |x^{\ast}_{i} - x_{m_{0},i}| + \frac{\varepsilon_{0}}{4}
 \le \sum_{i=0}^{n_{0}-1} |x^{\ast}_{i} - x_{l_{0},i}| + \sum_{i=0}^{n_{0}-1} |x_{l_{0},i} - x_{m_{0},i}| + \frac{\varepsilon_{0}}{4} \\
&< n_{0}\,\frac{\varepsilon_{0}}{4 n_{0}} + \|x_{l_{0}} - x_{m_{0}}\|_{1} + \frac{\varepsilon_{0}}{4}
 < \frac{3}{4}\,\varepsilon_{0},
\end{aligned}
\]
which is a contradiction. Hence $\|x^{\ast}-x_k\|_{1}\to 0$ and $(\ell_1,\|\cdot\|_1)$ is complete.

@ 4 and 5: exercises. \qed

\subsection{Basic Topological Concepts (Topologische Grundbegriffe)}

In the following, $(X,\|\cdot\|)$ is a normed $\K$-vector space.

\NumberedDefinition{4.2.1}{Interior point, open set, neighborhood}{
\begin{enumerate}[label={(\arabic*)}, leftmargin=*]
	\item Let $\emptyset\ne M\subset X$. A point $x\in M$ is called an \emph{interior point} of $M$ if there exists $\eps>0$ with $B(x,\eps)\subset M$.
	\item A set $M\subset X$ is called \emph{open} if every $x\in M$ is an interior point.
	\item Let $x\in X$ and let $U\subset X$ be open with $x\in U$. Then $U$ is called a \emph{neighborhood} of $x$. The collection of all neighborhoods is denoted by $\mathcal U(x)$.
\end{enumerate}}

\NumberedExample{4.2.2}{Open balls are open}{$B(x_0,r)$ with $x_0\in X$ and $r>0$ is open.

\emph{Proof.} Let $x\in B(x_0,r)$. To show: $x$ is an interior point. Set $\eps := \tfrac{1}{2}(r-\|x-x_0\|)$. Let $y\in B(x,\eps)$. We show: $\|y-x_0\|<r$.
\[
	\ \|y-x_0\| \le \|y-x\| + \|x-x_0\| \le \tfrac{1}{2}(r-\|x-x_0\|) + \|x-x_0\|
		 = \tfrac{1}{2}(r+\|x-x_0\|) < \tfrac{1}{2}(2r) = r.
\]
\qed}
	
% (Removed misplaced subsection header; 4.2 continues here. Proper 4.3 Continuity starts later under Section 4.)

\NumberedRemark{4.2.3}{}{\begin{enumerate}[label={(\arabic*)}, leftmargin=*]
	\item The notion of "open" depends on the ambient vector space. For instance, the interval $(0,1)$ is open in $\R$, but the set $(0,1)\times\{0\}$ is not open in $\R^{2}$.
	\item If $\|\cdot\|_{a}$ and $\|\cdot\|_{b}$ are equivalent norms on $X$, then a subset $O\subset X$ is open with respect to $\|\cdot\|_{a}$ if and only if it is open with respect to $\|\cdot\|_{b}$. In general, however, the open balls differ as sets:
	\[
		B(x,r)_{\|\cdot\|_{a}} \ne B(x,r)_{\|\cdot\|_{b}}.
	\]
\end{enumerate}}

\NumberedTheorem{4.2.4}{Basic closure properties of open sets}{Let $\mathcal O\subset \mathcal P(X)$ denote the collection of all open subsets of $X$. Then:
\begin{enumerate}[label={(\arabic*)}, leftmargin=*]
	\item $\emptyset\in\mathcal O$ and $X\in\mathcal O$.
	\item If $\{O_\alpha\}_{\alpha\in A}$ is any family of sets in $\mathcal O$ (with arbitrary index set $A$), then $\displaystyle \bigcup_{\alpha\in A} O_\alpha \in\mathcal O$.
	\item If $O_1,O_2\in\mathcal O$, then $O_1\cap O_2\in\mathcal O$. In particular, finite intersections of open sets are open.
\end{enumerate}}
\paragraph{Proof.}
1) Trivial.

2) Let $x\in \bigcup_{\alpha\in A} O_\alpha$. Then $x\in O_{\alpha_0}$ for some $\alpha_0\in A$. Since $O_{\alpha_0}$ is open, there exists $\eps>0$ with
\[
	B(x,\eps) \subset O_{\alpha_0} \subset \bigcup_{\alpha\in A} O_\alpha.
\]
Thus the union is open.

3) Let $x\in O_1\cap O_2$. Since $O_i$ is open, there exists $\eps_i>0$ with $B(x,\eps_i)\subset O_i$ for $i=1,2$. With $\eps:=\min\{\eps_1,\eps_2\}$ we have $B(x,\eps)\subset O_1\cap O_2$. Hence $O_1\cap O_2$ is open. \qed

\NumberedRemark{4.2.5}{}{Let $X\ne\emptyset$ and let $\mathcal T\subset \mathcal P(X)$ be a family of subsets satisfying properties (1)--(3) of Theorem~4.2.4. Then $\mathcal T$ is called a topology on $X$, and $(X,\mathcal T)$ is a topological space. Many statements proved below for normed vector spaces remain valid verbatim for general topological spaces.}

\NumberedRemark{4.2.6}{}{A countable intersection of open sets need not be open in general. Moreover, singletons $\{x\}$ are never open in a normed vector space.}

\NumberedDefinition{4.2.7}{Closed set}{A subset $M\subset X$ is called \emph{closed} if and only if its complement $M^{c}:=X\setminus M$ is open.}

\NumberedTheorem{4.2.8}{Basic closure properties of closed sets}{\begin{enumerate}[label={(\arabic*)}, leftmargin=*]
	\item $\emptyset$ and $X$ are closed.
	\item Arbitrary intersections of closed sets are closed.
	\item Finite unions of closed sets are closed.
\end{enumerate}}
\paragraph{Proof.}
(1) Since $X^{c}=\emptyset$ and $\emptyset^{c}=X$ are open, by definition both $X$ and $\emptyset$ are closed.

(2),(3) Follow from De Morgan's laws together with Theorem~4.2.4: for a family $\{M_\alpha\}_{\alpha\in A}$ one has
\[
	\Big( \bigcap_{\alpha\in A} M_\alpha \Big)^{c} = \bigcup_{\alpha\in A} (M_\alpha)^{c},
	\qquad
	(M_1\cup M_2)^{c} = M_1^{c}\cap M_2^{c}.
\]
Since arbitrary unions of open sets are open and finite intersections of open sets are open, the complements above are open, hence the original sets are closed. \qed

\NumberedRemark{4.2.9}{}{Arbitrary unions of closed sets need not be closed in general.}

\NumberedDefinition{4.2.10}{Contact and accumulation points; closure}{Let $M\subset X$.
\begin{itemize}[leftmargin=*]
	\item A point $x\in X$ is called a \emph{contact point} of $M$ if for every neighborhood $U\in\mathcal U(x)$ we have $U\cap M\ne\emptyset$.
	\item A point $x\in X$ is called an \emph{accumulation point} of $M$ if for every $U\in\mathcal U(x)$ we have $M\cap (U\setminus\{x\})\ne\emptyset$.
\end{itemize}
The \emph{closure} of $M$ is
\[
	\overline{M} := \{\, x\in X : x \text{ is a contact point of } M \,\}.
\]
}

\NumberedExample{4.2.11}{}{\begin{enumerate}[label={(\arabic*)}, leftmargin=*]
	\item $M=\{\tfrac{1}{n} : n\in\N\}$. Then $0\in X$ is a contact point and is the unique accumulation point of $M$.
	\item $M=B(0,1)$. Then $\overline{M}=\{x\in X : \|x\|\le 1\}$, and every contact point is also an accumulation point.
\end{enumerate}}
\paragraph{Proof.}
(1) Clear: every ball around $0$ contains some $1/n$, and every other point either belongs to $M$ or has a neighborhood disjoint from $M$. Moreover, if $a\ne 0$, then for sufficiently small $\eps>0$ the punctured ball $B(a,\eps)\setminus\{a\}$ contains no points of $M$; hence $0$ is the only accumulation point.

(2) Let $x\in X$.
\begin{enumerate}[label={(\alph*)}, leftmargin=2em]
	\item If $\|x\|<1$, then $x\in M$, so for every $U\in\mathcal U(x)$ we have $U\cap M\ne\emptyset$; thus $x$ is a contact point.
	\item If $\|x\|=1$, fix $0<\eps<1$ and set $x_{\eps}:=(1-\tfrac{\eps}{2})x\in M$. Then $\|x-x_{\eps}\|=\tfrac{\eps}{2}\,\|x\|=\tfrac{\eps}{2}<\eps$, hence $x_{\eps}\in B(x,\eps)\cap M\ne\emptyset$. Thus $x$ is a contact point.
	\item If $\|x\|>1$, let $\eps:=\tfrac{1}{2}(\|x\|-1)>0$ and take any $y\in B(x,\eps)$. Then
	\[
		\|y\| = \|(-x) - (y-x)\| \ge \|x\| - \|y-x\| > \|x\| - \tfrac{1}{2}(\|x\|-1) = \tfrac{1}{2}\,\|x\| + \tfrac{1}{2} > 1,
	\]
	so $y\notin B(0,1)$. Hence $B(x,\eps)\cap M=\emptyset$ and $x$ is not a contact point.
\end{enumerate}
Consequently $\overline{M}=\{x: \|x\|\le 1\}$. At boundary points with $\|x\|=1$ the construction in (b) moreover shows there are infinitely many points of $M$ in every punctured neighborhood of $x$, so every contact point is an accumulation point. \qed

\NumberedTheorem{4.2.12}{Elementary properties of the closure}{Let $M\subset X$. Then:
\begin{enumerate}[label={(\arabic*)}, leftmargin=*]
	\item $M\subset \overline{M}$.
	\item $M=\overline{M}$ if and only if $M$ is closed.
\end{enumerate}}
\paragraph{Proof.}
(1) Immediate from the definitions of contact point and $\overline{M}$.

(2) ("$\Rightarrow$") Suppose $M=\overline{M}$. If $x\in M^{c}=(\overline{M})^{c}$, then $x$ is not a contact point of $M$, hence there exists some $U\in\mathcal U(x)$ with $U\cap M=\emptyset$. Thus $U\subset M^{c}$ and $M^{c}$ is open. Therefore $M$ is closed.

("$\Leftarrow$") If $M$ is closed, then $M^{c}$ is open. Hence for every $x\in M^{c}$ there exists $U\in\mathcal U(x)$ with $U\subset M^{c}$, i.e. $U\cap M=\emptyset$. Thus $x$ is not a contact point of $M$, so $x\notin \overline{M}$, which shows $M^{c}\subset (\overline{M})^{c}$. Taking complements yields $\overline{M}\subset M$. Together with (1) we obtain $M=\overline{M}$. \qed

\NumberedTheorem{4.2.13}{Sequential characterization}{Let $M\subset X$ and $x\in X$.
\begin{enumerate}[label={(\arabic*)}, leftmargin=*]
	\item $x$ is an accumulation point of $M$ if and only if there exists a sequence $(x_n)_{n\in\N}\subset M$ with $x_n\ne x$ for all $n$ and $x_n\to x$.
	\item $x$ is a contact point of $M$ if and only if there exists a sequence $(x_n)_{n\in\N}\subset M$ with $x_n\to x$.
\end{enumerate}}
\paragraph{Proof.}
(1) ("$\Rightarrow$") If $x$ is an accumulation point, then for each $n\in\N$ we can choose $x_n\in B(x,1/n)\cap M\setminus\{x\}$. By construction $x_n\to x$ and $x_n\ne x$.

("$\Leftarrow$") Let $U\in\mathcal U(x)$. Since $x_n\to x$, there exists $k\in\N$ with $x_k\in U$. If additionally $x_k\ne x$ for all $k$, then $U\cap (M\setminus\{x\})\ne\emptyset$, proving that $x$ is an accumulation point.

(2) Analogous to (1), omitting the constraint $x_n\ne x$. \qed

\NumberedTheorem{4.2.14}{Closedness via accumulation points and sequences}{For $M\subset X$ the following are equivalent:
\begin{enumerate}[label={(\arabic*)}, leftmargin=*]
	\item $M$ is closed.
	\item $M$ contains all its accumulation points.
	\item For every convergent sequence $(x_n)_{n\in\N}\subset M$ one has $\displaystyle\lim_{n\to\infty} x_n \in M$.
\end{enumerate}}
\paragraph{Proof.} (1)$\Leftrightarrow$(2) follows from Theorems~4.2.12 and 4.2.13. Equivalence with (3) is standard and left as an exercise. \qed

\NumberedTheorem{4.2.15}{Closure as the smallest closed superset}{The closure is the smallest closed set containing $M$, i.e.
\[
	\overline{M} = \bigcap\{ A : A\supset M,\ A \text{ closed}\}.
\]}
\paragraph{Proof.}
Let $B := \bigcap\{ A : A\supset M,\ A \text{ closed}\}$. By Theorem~4.2.8, $B$ is closed. Also $M\subset B$. If $x\in B^{c}$, then there exists $U\in\mathcal U(x)$ with $U\subset B^{c}$ (since $B^{c}$ is open as the union of the opens $A^{c}$). Thus $U\cap M=\emptyset$, hence $x\notin \overline{M}$. Therefore $B^{c}\subset (\overline{M})^{c}$ and $\overline{M}\subset B$.

Conversely, if $x\in (\overline{M})^{c}$ then there exists $U\in\mathcal U(x)$ with $U\cap M=\emptyset$. The set $U^{c}$ is closed and contains $M$, hence $B\subset U^{c}$ and so $x\notin B$. Thus $(\overline{M})^{c}\subset B^{c}$ and $B\subset \overline{M}$. Hence $B=\overline{M}$. \qed

\NumberedTheorem{4.2.16}{Elementary rules for closures}{For $A,B\subset X$:
\begin{enumerate}[label={(\arabic*)}, leftmargin=*]
	\item If $A\subset B$, then $\overline{A}\subset \overline{B}$.
	\item $\overline{\,\overline{A}\,}=\overline{A}$.
	\item $\overline{A\cup B} = \overline{A}\cup \overline{B}$.
\end{enumerate}}
\paragraph{Proof.} Exercise. \qed

\NumberedDefinition{4.2.17}{Interior}{For $M\subset X$ define the interior of $M$ by
\[
	M^{\circ} := \{\, x\in X : x \text{ is an interior point of } M \,\}.
\]}

From Definition~4.2.1 of “open” it follows immediately:

\NumberedTheorem{4.2.18}{Basic facts about the interior}{\begin{enumerate}[label={(\arabic*)}, leftmargin=*]
	\item $M^{\circ}\subset M$.
	\item $M^{\circ} = M$ if and only if $M$ is open.
\end{enumerate}}

Analogous to 4.2.15 and 4.2.16 we obtain:

\NumberedTheorem{4.2.19}{Interior as the largest open subset}{The interior is the largest open subset of $M$, i.e.
\[
	M^{\circ} = \bigcup\{ O : O\subset M,\ O \text{ open}\}.
\]}
\paragraph{Proof.} Analogous to the proof of Theorem~4.2.15. \qed

\NumberedTheorem{4.2.20}{Elementary rules for interiors}{\begin{enumerate}[label={(\arabic*)}, leftmargin=*]
	\item If $A\subset B$, then $A^{\circ}\subset B^{\circ}$.
	\item $(A^{\circ})^{\circ} = A^{\circ}$.
	\item $(A\cap B)^{\circ} = A^{\circ}\cap B^{\circ}$.
\end{enumerate}}
\paragraph{Proof.} Exercise. \qed

\NumberedDefinition{4.2.21}{Boundary}{The boundary of a set $M$ is defined by
\[
	\partial M := \overline{M} \setminus M^{\circ} .
\]}

\NumberedExample{4.2.22}{Boundaries of basic balls}{
\begin{enumerate}[label={(\arabic*)}, leftmargin=*]
	\item If $M=B(0,1)$, then $M^{\circ}=M$, $\overline{M}=\{x\in X: \|x\|\le 1\}$ and $\partial M = \{x\in X: \|x\|=1\}$.
	\item If $M=B(0,1)\setminus\{0\}$, then $M^{\circ}=M$, $\overline{M}=\{x\in X: \|x\|\le 1\}$ and $\partial M = \{x\in X: \|x\|=1\}\cup\{0\}$.
\end{enumerate}}

The following Hausdorff separation axiom is easy to verify for normed vector spaces. For general topological spaces it must be imposed as a separate axiom; without it many essential theorems are false.

\NumberedTheorem{4.2.23}{Hausdorff property in normed spaces}{Let $x,y\in X$ with $x\ne y$. Then there exist neighborhoods $U_x\in\mathcal U(x)$ and $U_y\in\mathcal U(y)$ with $U_x\cap U_y=\emptyset$.}
\paragraph{Proof.} Let $r:=\|x-y\|>0$. Set $U_x:=B(x,\tfrac r2)$ and $U_y:=B(y,\tfrac r2)$. For $z\in U_y$ we have
\[\|x-z\|\ge \|x-y\|-\|y-z\|> \|x-y\|-\tfrac r2 = \tfrac r2,\]
so $z\notin U_x$. Hence $U_x\cap U_y=\emptyset$. \qed

An easy consequence of Theorem~4.2.14 is:

\NumberedTheorem{4.2.24}{Singletons are closed}{For $x\in X$, the set $\{x\}$ is closed.}

\NumberedDefinition{4.2.25}{Relative topology}{Let $M\subset X$, $M\ne\emptyset$. The relative (subspace) topology on $M$ is defined by
\[
	\mathcal T_M := \{\, U\cap M : U \text{ open in } X \,\}.
\]
We call $A\subset M$ relatively open in $M$ if $A\in\mathcal T_M$. For $x\in M$ define the system of relative neighborhoods by $\mathcal U_M(x):=\{\, U\cap M : U\in\mathcal U(x) \,\}$. A point $x\in A\subset M$ is a relative interior point of $A$ if there exists $V\in\mathcal U_M(x)$ with $x\in V\subset A$. We call $A\subset M$ relatively closed in $M$ if $A\in\{\, B\cap M : B \text{ closed in } X \,\}$.}

Obviously, $A\subset M$ is relatively open (resp. relatively closed) if and only if there exists an open (resp. closed) set $B\subset X$ with $A = B\cap M$.

\NumberedExample{4.2.26}{Relatively open/closed subsets of a half-closed interval}{Let $X=\R$ and $M=[0,1)$. Then $[0,\tfrac12)$ is relatively open in $M$ and $[\tfrac12,1)$ is relatively closed in $M$. Indeed,
\[
	[0,\tfrac12) = \big((-\tfrac12,\tfrac12)\big) \cap M =: B_o\cap M,\qquad
	[\tfrac12,1) = \big([\tfrac12,1]\big) \cap M =: B_a\cap M,
\]
where $B_o\subset\R$ is open and $B_a\subset\R$ is closed.}
	
\subsection{Continuity (Stetigkeit)}

\NumberedDefinition{4.3.1}{Continuity}{Let $X$ and $Y$ be normed vector spaces, $\emptyset\ne D\subset X$, and $f:D\to Y$.
\begin{enumerate}[label={(\arabic*)}, leftmargin=*]
	\item $f$ is said to be continuous at $x_0\in D$ if for every $\eps>0$ there exists $\delta>0$ such that
	\[
		\|x-x_0\|<\delta\ (x\in D) \quad\Longrightarrow\quad \|f(x)-f(x_0)\|_Y<\eps.
	\]
	Equivalently: for every neighborhood $V\in\mathcal U\big(f(x_0)\big)$ there exists $U\in\mathcal U(x_0)$ with $f\big(U\cap D\big)\subset V$.
	\item If $\emptyset\ne D'\subset D$, then $f$ is continuous on $D'$ if it is continuous at every $x\in D'$.
\end{enumerate}}

\NumberedTheorem{4.3.2}{Equivalent characterizations of continuity at a point}{Let $\emptyset\ne D\subset X$, $f:D\to Y$, and $x_0\in D$. The following are equivalent:
\begin{enumerate}[label={(\arabic*)}, leftmargin=*]
	\item $f$ is continuous at $x_0$ in the $\eps$–$\delta$ sense of Definition~4.3.1.
	\item For every neighborhood $V\in\mathcal U\big(f(x_0)\big)$ there exists $U\in\mathcal U(x_0)$ such that $f\big(U\cap D\big)\subset V$.
	\item (Sequential characterization.) For every sequence $(x_n)\subset D$ with $x_n\to x_0$ in $X$, one has $f(x_n)\to f(x_0)$ in $Y$.
\end{enumerate}}
\paragraph{Proof.}
$(1)\Rightarrow(2)$: Given $V\in\mathcal U\big(f(x_0)\big)$, choose $\eps>0$ with $B_Y\big(f(x_0),\eps\big)\subset V$. By (1) there exists $\delta>0$ such that $\|x-x_0\|<\delta$ implies $\|f(x)-f(x_0)\|<\eps$. With $U:=B_X(x_0,\delta)$ we get $f(U\cap D)\subset V$.

$(2)\Rightarrow(3)$: Let $(x_n)\subset D$ with $x_n\to x_0$. For $\eps>0$ set $V:=B_Y\big(f(x_0),\eps\big)$. By (2) there exists $U\in\mathcal U(x_0)$ with $f(U\cap D)\subset V$. Since $x_n\to x_0$, there exists $n_\eps$ with $x_n\in U$ for all $n\ge n_\eps$, hence $f(x_n)\in V$ for $n\ge n_\eps$. Thus $f(x_n)\to f(x_0)$.

$(3)\Rightarrow(1)$: Suppose (1) fails. Then there exists $\eps_0>0$ such that for every $k\in\N$ there is $x_k\in D$ with $\|x_k-x_0\|<1/k$ and $\|f(x_k)-f(x_0)\|\ge\eps_0$. Then $x_k\to x_0$ but $f(x_k)\not\to f(x_0)$, contradicting (3). \qed

\NumberedTheorem{4.3.3}{Preimage criteria for continuity}{Let $f:D\to Y$ with $D\subset X$ and let $\emptyset\ne D'\subset D$. The following are equivalent:
\begin{enumerate}[label={(\arabic*)}, leftmargin=*]
	\item $f$ is continuous on $D'$.
	\item Preimages of open sets in $Y$ are relatively open in $D'$, i.e. for every open $V\subset Y$ the set
		\[
			f^{-1}(V) := \{ x\in D : f(x)\in V\}
		\]
		satisfies that $D'\cap f^{-1}(V)$ is relatively open in $D'$.
	\item Preimages of closed sets in $Y$ are relatively closed in $D'$, i.e. for every closed $A\subset Y$ the set $D'\cap f^{-1}(A)$ is relatively closed in $D'$.
\end{enumerate}}
\paragraph{Proof.}
		extbf{a.} (1)$\Rightarrow$(2). Consider $D'\cap f^{-1}(V)$. If it is empty, it is relatively open in $D'$ by convention. Otherwise, let $x\in D'\cap f^{-1}(V)$, so $f(x)\in V$ with $V$ open in $Y$. By continuity at $x$ there exists $U\in\mathcal U(x)$ with $f(U\cap D)\subset V$. Then
\[
	U\cap D' \subset f^{-1}(V)\cap D',
\]
which shows that for every $x\in D'\cap f^{-1}(V)$ there is a neighborhood $U$ of $x$ in $X$ with $U\cap D'\subset f^{-1}(V)\cap D'$. Hence $D'\cap f^{-1}(V)$ is relatively open in $D'$.

		extbf{b.} (2)$\Rightarrow$(3). Let $A\subset Y$ be closed. Then $Y\setminus A$ is open and, by (2),
\[
	D'\setminus f^{-1}(A) 
		= D'\cap f^{-1}(Y\setminus A)
\]
is relatively open in $D'$. Hence there exists some open $U\subset X$ with $D'\setminus f^{-1}(A) = U\cap D'$. Taking complements relative to $D'$ yields
\[
	f^{-1}(A)\cap D' = U^{c}\cap D',
\]
and since $U^{c}$ is closed in $X$, the set $f^{-1}(A)\cap D'$ is relatively closed in $D'$.

		extbf{c.} (3)$\Rightarrow$(1). Let $x\in D'$ and let $V\in\mathcal U\big(f(x)\big)$. Then $Y\setminus V$ is closed, hence by (3) the set $D'\cap f^{-1}(Y\setminus V)$ is relatively closed in $D'$. Arguing as in the proof of (2)$\Rightarrow$(3), it follows that $f^{-1}(V)$ is relatively open in $D'$, so there exists an open neighborhood $U\in\mathcal U(x)$ with
\[
	f^{-1}(V)\cap D' = U\cap D'.
\]
Consequently
\[
	f(U\cap D') = f\big(f^{-1}(V)\cap D'\big) \subset f\big(f^{-1}(V)\big) = V,
\]
showing that $f$ is continuous at $x$. Since $x\in D'$ was arbitrary, $f$ is continuous on $D'$. \qed

\NumberedExample{4.3.4}{Basic continuity examples}{
\begin{enumerate}[label={(\arabic*)}, leftmargin=*]
	\item $f:X\to\R_{\ge 0}$, $f(x):=\|x\|$, is continuous.
	\item $f: \R_{\ge 0}\to\R_{\ge 0}$, $f(x):=\sqrt{x}$, is continuous.
	\item $f: \R\to\R$, $f(x):=\lfloor x\rfloor := \max\{k\in\Z: k\le x\}$, is continuous on $\R\setminus\Z$ and discontinuous at every $x\in\Z$.
	\item $f: \R\to\R$ given by $f(x):=\begin{cases}1,& x\in\Q,\\ 0,& x\in \R\setminus\Q,\end{cases}$ is nowhere continuous.
	\item For $\K\in\{\R,\C\}$ and $m\in\N$, the coordinate projection $f: \K^{m}\to\K$, $f(x)=x_j$ for a fixed $j\in\{1,\dots,m\}$, is continuous.
	\item The map $f: \C\to\R$, $f(z):=\Re(z)$, is continuous. The same holds for $f(z):=\Im(z)$.
		\item Let $(X,\|\cdot\|_a)$ and $(X,\|\cdot\|_b)$ be two normed vector spaces on the same underlying vector space with equivalent norms. Then the identity $I:(X,\|\cdot\|_a)\to (X,\|\cdot\|_b)$, $I(x)=x$, is continuous.
		\item If $f:X\to\R$ is continuous and $r\in\R$, then the sets $\{x\in X: f(x)>r\}$ and $\{x\in X: f(x)<r\}$ are open, while $\{x\in X: f(x)\ge r\}$ and $\{x\in X: f(x)\le r\}$ are closed.
\end{enumerate}}

\paragraph{Proof.}
\begin{enumerate}[label={(\arabic*)}, leftmargin=*]
	\item Given $\eps>0$ choose $\delta:=\eps$. If $\|x-x_0\|<\delta$, then by the reverse triangle inequality
		\[
			\big|\,\|x\| - \|x_0\|\,\big| \le \|x-x_0\| < \eps.
		\]
	\item Consider $f(x)=\sqrt{x}$ on $\R_{\ge0}$ and fix $x_0\ge0$.
		\begin{enumerate}[label=\alph*.), leftmargin=2em]
			\item If $x_0=0$, let $\eps>0$ and set $\delta:=\eps^2$. For $x\in[0,\delta]$ we have
				$|\sqrt{x}-\sqrt{x_0}|=\sqrt{x}<\sqrt{\delta}=\eps$.
			\item If $x_0>0$, let $\eps>0$ and set $\delta:=\eps\sqrt{x_0}$. Then for all $x\ge0$ with $|x-x_0|<\delta$,
				\[
					|\sqrt{x}-\sqrt{x_0}| = \frac{|x-x_0|}{\sqrt{x}+\sqrt{x_0}}
						\le \frac{|x-x_0|}{\sqrt{x_0}} < \eps.
				\]
		\end{enumerate}
	\item Let $f(x)=\lfloor x\rfloor$.
		If $x_0\notin\Z$, there exists $\delta>0$ with $I:=(x_0-\delta, x_0+\delta)\subset \R\setminus\Z$. On $I$, $f$ is constant and hence continuous. If $x_0\in\Z$, then for $x<x_0$ we have $f(x)\le f(x_0)-1=x_0-1$, and for $x\ge x_0$, $f(x)\ge f(x_0)=x_0$. Thus, for every $\delta>0$ there exists $x\in(x_0-\delta, x_0+\delta)$ with $|f(x)-f(x_0)|\ge1$, so $f$ is discontinuous at $x_0$.
	\item Let $f=\mathbf{1}_{\Q}$, i.e. $f(x)=1$ if $x\in\Q$ and $0$ otherwise. If $x\in\R\setminus\Q$, choose a sequence $(x_n)\subset\Q$ with $x_n\to x$. Then $f(x)=0$ but $f(x_n)=1\to1\ne f(x)$. If $x\in\Q$, choose $(x_n)\subset \R\setminus\Q$ with $x_n\to x$. Then $f(x)=1$ while $f(x_n)=0\to0\ne f(x)$. Hence $f$ is nowhere continuous.
	\item Let $f(x)=x_j$ on $\K^m$ and let $\|\cdot\|$ be any norm on $\K^m$. Since all norms on finite-dimensional spaces are equivalent, there exists $C\ge1$ with $\|y\|_{\infty} \le C\,\|y\|$ for all $y\in\K^m$. Then
		\[
			|f(x)-f(x_0)| = |(x_0)_j - x_j| \le \max_{1\le i\le m} |(x_0)_i - x_i| = \|x-x_0\|_{\infty} \le C\,\|x-x_0\|.
		\]
		Given $\eps>0$, choose $\delta:=\eps/C$.
	\item For $f(z)=\Re(z)$ and any $z,z_0\in\C$,
		\[
			|\Re(z)-\Re(z_0)| = |\Re(z-z_0)| \le |z-z_0|.
		\]
		Hence with $\delta:=\eps$ we obtain $|\Re(z)-\Re(z_0)|<\eps$ whenever $|z-z_0|<\delta$. The same argument applies to $\Im$.
	\item By equivalence of norms there exist constants $c,\,\overline{c}>0$ with
		\[
			c\,\|x\|_a \le \|x\|_b \le \overline{c}\,\|x\|_a \qquad (\forall x\in X).
		\]
		Given $\eps>0$, choose $\delta:=\eps/\overline{c}$. If $\|x-x_0\|_a<\delta$, then
		\[
			\|I x - I x_0\|_b = \|x-x_0\|_b \le \overline{c}\,\|x-x_0\|_a < \eps,
		\]
		showing continuity of $I$.
	\item Let $B:=\{y\in\R: y<r\}$, which is open in $\R$. Then
		\[
			\{x\in X: f(x)<r\} = f^{-1}(B)
		\]
		is open by Theorem~4.3.3. The case $f(x)>r$ is analogous with $\{y: y>r\}$, and the closedness of $\{f\ge r\}$ and $\{f\le r\}$ follows by taking preimages of the closed sets $(-\infty,r]$ and $[r,\infty)$.
\end{enumerate}

\NumberedTheorem{4.3.5}{Algebra of continuous functions}{Let $D_f,D_g\subset X$, $x_0\in D_f\cap D_g$, and let $f:D_f\to Y$, $g:D_g\to Y$ be continuous at $x_0$. Let $\alpha\in\K$. Then:
\begin{enumerate}[label={(\arabic*)}, leftmargin=*]
	\item $\alpha f : x\mapsto \alpha\,f(x)$ is continuous at $x_0$.
	\item $f+g : x\mapsto f(x)+g(x)$ is continuous at $x_0$.
	\item If $Y=\K$, then $f\cdot g : x\mapsto f(x)\,g(x)$ is continuous at $x_0$.
	\item If $Y=\K$ and $g(x_0)\ne0$, then $\dfrac{f}{g} : x\mapsto \dfrac{f(x)}{g(x)}$ is continuous at $x_0$.
\end{enumerate}}
\paragraph{Proof.} Use the sequential characterization of continuity (Theorem~4.3.2(3)): if $x_n\to x_0$ in $X$ with $x_n\in D_f\cap D_g$, then by continuity $f(x_n)\to f(x_0)$ and $g(x_n)\to g(x_0)$ in $Y$ (resp. in $\K$). The claims (1)–(4) follow from the algebra of limits for sequences (Theorem~3.1.19); in (4) also use Theorem~3.1.19(5) to pass to reciprocals since $g(x_0)\ne0$. \qed

\NumberedDefinition{4.3.6}{Space of continuous maps}{Let $M\subset X$, $M\ne\emptyset$. We denote by $C(M,Y)$ the vector space of all continuous functions $f: M\to Y$.}

By Theorem~4.3.5 and Definition~4.3.6, the set $C(M,Y)$ is a vector space under pointwise operations.

\NumberedTheorem{4.3.7}{Polynomials and rational functions are continuous}{Every polynomial, i.e. a function of the form
\[
	f(x) = \sum_{k=0}^{n} a_k x^{k}, \qquad a_k\in\K,\ n\in\N,
\]
and every rational function on its domain,
\[
	f(x) = \frac{p(x)}{q(x)}, \qquad p,q \text{ polynomials},
\]
is continuous on its domain of definition.}
\paragraph{Proof.} Immediate from Theorem~4.3.5 by iteratively applying the rules for sums, products, and quotients (where defined). \qed

\NumberedTheorem{4.3.8}{Composition}{Let $f:D_f\to Y$ and $g:D_g\to Z$ be functions with $D_f\subset X$, $D_g\subset Y$ and $f(D_f)\subset D_g$. Define the composition $g\circ f: D_f\to Z$ by $(g\circ f)(x):=g\bigl(f(x)\bigr)$. If $f$ is continuous at $x_0\in D_f$ and $g$ is continuous at $y_0=f(x_0)\in D_g$, then $g\circ f$ is continuous at $x_0$.}
\paragraph{Proof.} Let $W\in\mathcal U(g(y_0))$. Since $g$ is continuous at $y_0$, there exists $V\in\mathcal U(y_0)$ such that $g\bigl(V\cap D_g\bigr)\subset W$. As $f$ is continuous at $x_0$, there exists $U\in\mathcal U(x_0)$ with $f\bigl(U\cap D_f\bigr)\subset V$. Because $f(D_f)\subset D_g$, we have $f\bigl(U\cap D_f\bigr)\subset V\cap D_g$, hence $(g\circ f)\bigl(U\cap D_f\bigr)\subset W$. Thus $g\circ f$ is continuous at $x_0$. \qed

\NumberedExample{4.3.9}{}{{\leavevmode}
\begin{enumerate}[label={(\arabic*)}, leftmargin=*]
	\item If $f:D\to Y$ is continuous and $D\subset X$, then the map $\|f\|_Y : D\to \R_{\ge0}$, $x\mapsto \|f(x)\|_Y$, is continuous (cf. Example~4.3.4(1), Theorem~4.3.8).
	\item If $f:D\to\K$ is continuous with $D\subset X$, then $f^{2}:D\to\K$, $x\mapsto (f(x))^2$, is continuous (Theorem~4.3.5(3)).
	\item If $f:D\to\R_{\ge0}$ is continuous with $D\subset X$, then $\sqrt{\,\cdot\,}\circ f : D\to\R_{\ge0}$, $x\mapsto \sqrt{f(x)}$, is continuous (cf. Example~4.3.4(2), Theorem~4.3.8).
\end{enumerate}}

\NumberedTheorem{4.3.10}{Continuity via components}{
\begin{enumerate}[label={(\arabic*)}, leftmargin=*]
	\item Let $f=(f_1,\dots,f_n): D\to\K^{n}$ with $D\subset X$ and $x_0\in D$. Then $f$ is continuous at $x_0$ if and only if each component $f_i:D\to\K$ is continuous at $x_0$ for $1\le i\le n$.
	\item A function $f:D\to\C$, $D\subset X$, is continuous at $x_0\in D$ if and only if $\Re f$ and $\Im f$ are continuous at $x_0$.
\end{enumerate}}
\paragraph{Proof.}
\begin{enumerate}[label={(\arabic*)}, leftmargin=*]
	\item ($\Rightarrow$) The coordinate projections $\pi_i: \K^n\to\K$, $\pi_i(y)=y_i$, are continuous (Example~4.3.4(5)). Hence $f_i=\pi_i\circ f$ is continuous at $x_0$ by Theorem~4.3.8.

	($\Leftarrow$) Fix $\eps>0$. For each $1\le j\le n$, continuity of $f_j$ at $x_0$ yields a $\delta_j(\eps)>0$ such that
	\[
		|f_j(x)-f_j(x_0)| < \frac{\eps}{n} \quad \text{whenever } \|x-x_0\|<\delta_j(\eps).
	\]
	Set $\delta := \min_{1\le j\le n} \delta_j(\eps)$. Then for all $x\in B(x_0,\delta)$,
	\[
		\|f(x)-f(x_0)\|_{1} = \sum_{j=1}^{n} |f_j(x)-f_j(x_0)| < n\cdot \frac{\eps}{n} = \eps.
	\]
	Thus $f$ is continuous at $x_0$ with respect to $\|\cdot\|_1$ on $\K^n$. By equivalence of norms on finite-dimensional spaces, continuity holds for any norm on $\K^n$ as well.
\end{enumerate}
\qed

\NumberedRemark{4.3.13}{Limits and continuous extension}{
\begin{enumerate}[label={(\arabic*)}, leftmargin=*]
	\item Let $f:D\to Y$, $D\subset X$, and $x_0\in D$. Then $\lim_{x\to x_0} f(x) = f(x_0)$ is equivalent to “$f$ is continuous at $x_0$” (Theorem~4.3.2).
	\item Let $f:D\to Y$, $D\subset X$, $x_0\notin D$, and suppose $y = \lim_{x\to x_0} f(x)$ exists. Define an extension $\tilde f: D\cup\{x_0\}\to Y$ by
	\[
				ilde f(x) :=
		\begin{cases}
			f(x), & x\in D,\\
			y, & x=x_0.
		\end{cases}
	\]
	Then $\tilde f$ is continuous at $x_0$ and is called the continuous extension of $f$ (at $x_0$).
\end{enumerate}}

\NumberedExample{4.3.14}{}{{\leavevmode}
\begin{enumerate}[label={(\arabic*)}, leftmargin=*]
	\item Let $X=Y=\R$, $D=\R\setminus\{1\}$, and $f(x):=\dfrac{x^{n}-1}{x-1}$ with $n\in\N$. By Lemma~3.2.3,
		\[
			f(x) = \sum_{k=0}^{n-1} x^{k} \qquad (x\in\R\setminus\{1\}),
		\]
		hence $\lim_{x\to 1} f(x)=n$. Thus the function
		\[
					ilde f(x) :=
			\begin{cases}
				\dfrac{x^{n}-1}{x-1}, & x\ne 1,\\
				n, & x=1,
			\end{cases}
		\]
		is a continuous extension of $f$ at $x=1$.
	\item Let $X=Y=\C$, $D=\C\setminus\{0\}$, and $f(z):=\dfrac{e^{z}-1}{z}$. For any sequence $(z_n)$ with $z_n\to 0$ we have $f(z_n)\to 1$; thus defining
		\[
					ilde f(z) :=
			\begin{cases}
				\dfrac{e^{z}-1}{z}, & z\in\C\setminus\{0\},\\
				1, & z=0,
			\end{cases}
		\]
		gives a continuous extension at $z=0$.
\end{enumerate}}

\subsection{Functions on \texorpdfstring{$\R$}{R} (Funktionen in \texorpdfstring{$\R$}{R})}

In the following, let $I\subset\R$ always be a nonempty interval.

\NumberedDefinition{4.4.1}{Monotonicity}{A function $f:I\to\R$ is called
\begin{enumerate}[label={(\arabic*)}, leftmargin=*]
	\item monotone increasing,
	\item monotone decreasing,
	\item strictly increasing,
	\item strictly decreasing,
\end{enumerate}
if for all $x,y\in I$ with $x<y$ we have, respectively,
\begin{enumerate}[label={(\arabic*)}, leftmargin=*]
	\item $f(x)\le f(y)$,
	\item $f(x)\ge f(y)$,
	\item $f(x)< f(y)$,
	\item $f(x)> f(y)$.
\end{enumerate}
A function that is either (strictly) decreasing or (strictly) increasing is called (strictly) monotone.}

\NumberedExample{4.4.2}{}{{\leavevmode}
\begin{enumerate}[label={(\arabic*)}, leftmargin=*]
	\item $f(x):=x^{2}$ is strictly decreasing on $(-\infty,0]$ and strictly increasing on $[0,\infty)$. If $0\in \mathring I$ (i.e. $0$ is an interior point of $I$), then $f$ is not monotone on $I$.
	\item The floor function $f(x):=\lfloor x\rfloor$ is monotone increasing.
	\item The function
		\[
			f(x):=\begin{cases}
				1, & x\in\Q,\\
				0, & x\in\R\setminus\Q,
			\end{cases}
		\]
		is not monotone.
\end{enumerate}}

\NumberedTheorem{4.4.3}{One-sided limits of monotone functions}{Let $f:I\to\R$ be monotone and set $\alpha:=\inf I$, $\beta:=\sup I$ (possibly infinite). Then the one-sided limits
\[
	f(\alpha+0):=\lim_{x\to\alpha+0} f(x)=\lim_{\substack{x\to\alpha\\ x>\alpha}} f(x),
	\qquad
	f(\beta-0):=\lim_{x\to\beta-0} f(x)=\lim_{\substack{x\to\beta\\ x<\beta}} f(x)
\]
exist in $\R$, and we have
\[
	f(\alpha+0)=
	\begin{cases}
		\inf\{f(x): x\in I\setminus\{\alpha\}\}, & \text{if $f$ is increasing},\\[0.3em]
		\sup\{f(x): x\in I\setminus\{\alpha\}\}, & \text{if $f$ is decreasing},
	\end{cases}
\]
and
\[
	f(\beta-0)=
	\begin{cases}
		\sup\{f(x): x\in I\setminus\{\beta\}\}, & \text{if $f$ is increasing},\\[0.3em]
		\inf\{f(x): x\in I\setminus\{\beta\}\}, & \text{if $f$ is decreasing}.
	\end{cases}
\]
}
\paragraph{Proof.} Consider the increasing case and set
\[
	b:=\sup\{f(x): x\in I\setminus\{\beta\}\}\in\overline{\R}.
\]
Let $(x_n)\subset [\alpha,\beta)$ with $x_n\to\beta$ and $x_n<\beta$ for all $n\in\N$. Since $b$ is the supremum, for every $\eta<b$ there exists $\xi\in[\alpha,\beta)$ with $f(\xi)\ge\eta$ and an index $n_{\xi}$ such that $x_n\ge\xi$ for all $n\ge n_{\xi}$. Because $f$ is increasing,
\[
	\eta\le f(\xi)\le f(x_n)\le b \qquad (n\ge n_{\xi}).
\]
As $\eta<b$ was arbitrary, it follows that $\lim_{n\to\infty} f(x_n)=b$. Hence $f(\beta-0)$ exists and equals $b$. The remaining statements are shown analogously. \qed

\NumberedDefinition{4.4.4}{Jump discontinuity}{Let $D\subset\R$, $D\ne\emptyset$, and let $Y$ be a normed vector space. A function $f:D\to Y$ is said to have a jump at a point $x_0\in\R$ with
\[
	x_0\in \overline{D\cap(-\infty,x_0)}\ \cap\ \overline{D\cap(x_0,\infty)}
\]
if the one-sided limits
\[
	\lim_{x\to x_0\pm 0} f(x) = f(x_0\pm 0)
\]
exist and are different.}

\NumberedExample{4.4.5}{}{For the function $f(x):=\lfloor x\rfloor$, every $k\in\Z$ is a jump point, since
\[
	f(k+0)-f(k-0)=1.
\]
}

Theorem~4.4.3 shows that for every monotone function $f:I\to\R$ and every $x_0$ in the interior of $I$, the one-sided limits $f(x_0\pm 0)$ exist. Example~4.4.5 shows that there are functions with countably many jump discontinuities. The next theorem shows that monotone functions can have at most countably many jumps.

\NumberedTheorem{4.4.6}{}{Let $f:I\to\R$ be monotone. Then $f$ is continuous except at at most countably many jump points.}
\paragraph{Proof.} Assume $f$ is increasing. Define the set of jump points
\[
	M := \{\, x\in \mathring I : f(x+0)\ne f(x-0)\,\}.
\]
We claim that $M$ is countable. Idea: construct an injective map $q:M\to\Q$. For $x\in M$ we have $f(x-0)<f(x+0)$, so by density of $\Q$ there exists $z\in\Q$ with
\[
	z\in\bigl( f(x-0),\, f(x+0) \bigr).
\]
Define $q(x):=z$. We show $q$ is injective. If $x,y\in M$ with $x<y$, then by monotonicity
\[
	q(x) < f(x+0) \le f(y-0) < q(y).
\]
Hence $q(x)\ne q(y)$ whenever $x\ne y$, so $q$ is injective. Since $\Q$ is countable, $M$ is countable. \qed

\NumberedTheorem{4.4.7}{Intermediate Value Theorem}{Let $f:[a,b]\to\R$ be continuous with $f(a)<0<f(b)$. Then $f$ has a zero $\xi\in(a,b)$, i.e. $f(\xi)=0$.}
\paragraph{Proof.} (Interval bisection.) Construct a sequence of nested intervals $[a_k,b_k]$ recursively by
\[
	[a_0,b_0]:=[a,b], \qquad
	[a_{k+1},b_{k+1}]:=
	\begin{cases}
		\bigl[a_k,\, \tfrac{a_k+b_k}{2}\bigr], & \text{if } f\bigl( \tfrac{a_k+b_k}{2} \bigr) > 0,\\[0.4em]
		\bigl[ \tfrac{a_k+b_k}{2},\, b_k \bigr], & \text{if } f\bigl( \tfrac{a_k+b_k}{2} \bigr) \le 0,
	\end{cases}
	\qquad k\ge 0.
\]
Then: (1) $(a_k)$ is monotonically increasing and bounded above by $b$; (2) $(b_k)$ is monotonically decreasing and bounded below by $a$; (3) for all $k\in\N$, $b_k-a_k=2^{-k}(b-a)$. By the nested-interval argument (cf. Bolzano–Weierstrass, Theorem~3.2.5), there exists a common limit $\xi\in[a,b]$. By construction, for all $k\ge0$ we have
\[
	f(a_k) \le 0 \le f(b_k).
\]
By continuity of $f$,
\[
	f(\xi) = \lim_{k\to\infty} f(a_k) \le 0 \le \lim_{k\to\infty} f(b_k) = f(\xi),
\]
hence $f(\xi)=0$ and $\xi\in(a,b)$ since $f(a)<0<f(b)$. \qed

We distinguish three types of intervals: closed $I=[a,b]$, open $I=(a,b)$, and half-open $I=(a,b]$ or $I=[a,b)$.

\NumberedTheorem{4.4.8}{}{Let $f:I\to\R$ be continuous and strictly monotone increasing (respectively, strictly monotone decreasing). Then $J:=f(I)$ is an interval of the same type as $I$, and $f$ maps $I$ bijectively onto $J$. The inverse map $f^{-1}:J\to I$ is continuous and strictly monotone increasing (respectively, strictly monotone decreasing).}
\paragraph{Proof.} We consider the strictly increasing case; the decreasing case is analogous. Set $\alpha:=\inf I$, $\beta:=\sup I$ (in $\overline{\R}$). By Theorem~4.4.3, the one-sided limits
\[
	a:=f(\alpha+0), \qquad b:=f(\beta-0)
\]
exist (in $\overline{\R}$). We first show
\[
	(a,b)\subset J:=f(I)\subset [a,b].
\]
The right inclusion is clear by the definition of $a$ and $b$. For the left inclusion, let $a<\eta<b$. Because $b$ is the supremum of $J$, there exist $t_2\in (\alpha,\beta)$ with $f(t_2)>\eta$ and $t_1\in(\alpha,\beta)$ with $f(t_1)<\eta$. Since $f$ is strictly increasing, $t_1<t_2$ and hence $f(t_1)<\eta<f(t_2)$. By continuity (Intermediate Value Theorem, Theorem~4.4.7) there exists $\tau\in[t_1,t_2]\subset(\alpha,\beta)$ with $f(\tau)=\eta$. Thus $\eta\in J$, so $(a,b)\subset J$. Hence $J$ is an interval.

Strict monotonicity implies that $J$ contains its infimum $a$ if and only if $I$ contains $\alpha$, and $J$ contains its supremum $b$ if and only if $I$ contains $\beta$. Therefore $J$ has the same endpoint type as $I$.

For bijectivity: strict monotonicity yields injectivity of $f$ on $I$, and surjectivity holds by definition of $J=f(I)$. Hence $f:I\to J$ is bijective.

For the inverse $g:=f^{-1}:J\to I$ we first show strict monotonicity: let $y_1<y_2$ in $J$ and assume towards a contradiction that $x_1:=g(y_1)\ge g(y_2)=:x_2$. Then
\[
	y_1=f(g(y_1))\ge f(g(y_2))=y_2,
\]
contradiction. Thus $g$ is strictly increasing.

It remains to prove continuity of $g$. Fix $y_0\in J$. Consider a sequence $(y_n)\subset J$ with $y_n\nearrow y_0$; set $x_n:=g(y_n)$. Then $(x_n)$ is increasing and bounded above by $g(y_0)$, hence $x_n\to\bar x\le g(y_0)$. If $\bar x< x_0:=g(y_0)$, then by continuity of $f$,
\[
	y_0 = \lim_{n\to\infty} y_n = \lim_{n\to\infty} f(x_n) = f(\bar x) < f(x_0)=y_0,
\]
which is impossible. Thus $g(y_0-0)=g(y_0)$. An analogous argument with $y_n\searrow y_0$ shows $g(y_0+0)=g(y_0)$. Hence $g$ is continuous at $y_0$. \qed

\NumberedExample{4.4.9}{}{Let $n\ge 2$ and define $f: \R_{\ge0}\to\R_{\ge0}$ by $f(x):=x^{n}$. We show $f$ is strictly increasing. For $0\le x<y$,
\[
	f(y)-f(x)=y^{n}-x^{n}=y^{n}\Bigl(1-\bigl(\tfrac{x}{y}\bigr)^{n}\Bigr)
	= y^{n}\Bigl(1-\tfrac{x}{y}\Bigr)\sum_{j=0}^{n-1} \Bigl(\tfrac{x}{y}\Bigr)^{\,n-1-j} > 0,
\]
since $y>0$, $0\le x/y<1$, and the sum is positive. Moreover, because $0\le f(x)=x^{n}\le x$ for $x<1$, we have $f(0+0)=0$, and since $f(x)=x^{n}\ge x$ for $x>1$, we have $f(+\infty-0)=+\infty$.

Thus $f:\R_{\ge0}\to\R_{\ge0}$ is bijective, continuous, and strictly increasing. By Theorem~4.4.8 it has a continuous, strictly increasing inverse $g: \R_{\ge0}\to\R_{\ge0}$. We write
\[
	\sqrt[n]{\,x\,} := g(x) := f^{-1}(x).
\]
}

\subsection{The Exponential Function (Die Exponentialfunktion)}

\NumberedDefinition{4.5.1}{Power-series definitions}{Define three functions on $\C$ by
\[
	\exp(z):=e^{z} := \sum_{n=0}^{\infty} \frac{z^{n}}{n!},\qquad
	\cos(z) := \sum_{n=0}^{\infty} (-1)^{n} \frac{z^{2n}}{(2n)!},\qquad
	\sin(z) := \sum_{n=0}^{\infty} (-1)^{n} \frac{z^{2n+1}}{(2n+1)!}.
\]
These are called the exponential, cosine, and sine functions.}

\NumberedTheorem{4.5.2}{Basic properties}{
\begin{enumerate}[label={(\arabic*)}, leftmargin=*]
	\item The power series in Definition~4.5.1 all have radius of convergence $+\infty$.
	\item $\exp,\sin,\cos$ are real-valued on $\R$.
	\item $e^{z_1+z_2}=e^{z_1}\cdot e^{z_2}$.
	\item $e^{z}\ne0$ for all $z\in\C$, and $e^{-z}=1/e^{z}$.
	\item $e^{iz}=\cos(z)+i\sin(z)$.
\end{enumerate}}
	
% ============================================================================
% 5 Differential Calculus in One Variable (Differentialrechnung einer Veränderlichen) -- p.95
% ============================================================================
\section{Differential Calculus in One Variable (Differentialrechnung einer Veränderlichen)}

\subsection{Differentiability (Differenzierbarkeit)}
	
\subsection{Mean Value Theorems (Mittelwertsätze)}
	
\subsection{Taylor's Formula (Taylorsche Formel)}
	
% ============================================================================
% 6 Sequences of Functions (Funktionenfolgen) -- Original start p.119
% ============================================================================
\section{Sequences of Functions (Funktionenfolgen)}

\subsection{Uniform Convergence (Gleichmässige Konvergenz)}
	
\subsection{Interchanging Limits (Vertauschen von Grenzwerten)}
	
% ============================================================================
% End matter / future parts (e.g., multivariable calculus) can be appended here.
% ============================================================================


\end{document}